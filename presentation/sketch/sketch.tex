\documentclass[a4paper,10pt]{article}

\usepackage{xltxtra}
\usepackage{gfsdidot}
\usepackage{verse}

\setmainfont[Mapping=tex-text]{GFS Didot}
\setsansfont[Mapping=tex-text]{DejaVu Sans}
\setmonofont[Mapping=tex-text]{DejaVu Sans Mono}
\newfontfamily\athenais{Athenais}
\pagenumbering{gobble}
%%%%%%%%%%%%%%%%%%%%%%%%%%%%%%%%%%%%%%%%%%%%%%%%%%%%%%%%%%%%%%%%%%%%%%%%%%%%%%%%
\begin{document}

Αξιότιμες καθηγήτριες, αξιότιμοι καθηγητές, συνάδελφοι, κυρίες και κύριοι:
ονομάζομαι Αλέξανδρος Φιλοθέου και σήμερα είναι σκοπός μου να σάς εκθέσω τις
συμβολές μου σε λύσεις διαδεδομένων προβλημάτων που αφορούν στο πεδίο της
ρομποτικής κινητής βάσης.

Αποφοίτησα από το τμήμα μας τον Ιούλιο του 2013 και τον Νοέμβριο του ίδιου
έτους έγινα μέλος της ομάδας ρομποτικής PANDORA. Εκεί απέκτησα την πρώτη μου
επαφή με τη ρομποτική και το αντικείμενο μου διέγειρε το ενδιαφέρον τόσο ώστε
αποφάσισα να ειδικευτώ σε αυτό. Τον Σεπτέμβριο του 2015 ξεκίνησα το
μεταπτυχιακό μου πάνω στη ρομποτική και τον έλεγχο στο Βασιλικό Ινστιτούτο
Τεχνολογίας στη Σουηδία, και δύο χρόνια αργότερα επέστρεψα στην Ελλάδα για να
εργαστώ. Τον Σεπτέμβριο του 2018 είχα την τύχη να ξεκινήσω να δουλεύω σε αυτό
που είχα επιλέξει στο πανεπιστήμιό μας υπό την επίβλεψη του κυρίου Συμεωνίδη
και του κυρίου Δημητρίου.

Από τότε η έρευνά μου διαμορφώθηκε με βάση πρακτικά προβλήματα που
συναντήσαμε στα πλαίσια εκτέλεσης των έργων στα οποία εργάστηκα και με βάση
προβλήματα τα οποία εντόπισα στις μεθόδους επίλυσής τους που βρίσκονται στην
τρέχουσα βιβλιογραφία.

Οι συμβολές της διατριβής μου αρθρώνονται σε 5 κεφάλαια.

Το πρώτο κεφάλαιο ξεκινάει με έναν ταπεινό θα έλεγα στόχο, ο οποίος αποβλέπει
στη συμβολή στο έργο των μηχανικών ρομποτικής που ασχολούνται με την αυτόνομη
πλοήγηση επίγειων οχημάτων---για την ακρίβεια ό,τι θα σας παρουσιάσω σήμερα
αφορά σε επίγεια οχήματα. Κατά τη διάρκεια επίλυσης του αυτού του πρώτου
προβλήματος εντόπισα ένα σημαντικό πρόβλημα του οποίου η λύση θα είχε
καθοριστική σημασία για τα αποτελέσματα του πρώτου έργου στο οποίο εργάστηκα,
και την οποία ξεκίνησα να ερευνώ στο δεύτερο κεφάλαιο της διατριβής.

Σε αυτό το κεφάλαιο κατασκεύασα μία σειρά λύσεων. Μία από αυτές βασίζεται σε
μία βασική μέθοδο της βιβλιογραφίας, και κατά τη διάρκεια κατασκευής της λύσης
μου εντόπισα στη μέθοδο μία σημαντική παθογένεια, την οποία προσπάθησα να
εξαλείψω στο τρίτο κεφάλαιο. Το πρόβλημα που εντόπισα ήταν απαιτητικό και προς
έκπληξίν μου δεν ξαναείχε τεθεί στην ερευνητική βιβλιογραφία, ίσως λόγω
αδράνειας, ή κορεσμού, ή έλλειψης δυνατότητας εφαρμογής σε πραγματικές συνθήκες.

Αρχικά προσπάθησα να διασφαλίσω ότι μία λύση είναι εφικτή κατ'αρχήν: έκανα
δηλαδή μία σειρά παραδοχών οι οποίες από τη μία είναι ρεαλιστικές και απο την
άλλη οι λιγότερο απαιτητικές. Στόχος μου ήταν να μου δώσουν τη δυνατότητα να
εξετάσω εάν το πρόβλημα έχει λύση και ταυτόχρονα νόημα στα πλαίσια (γενικότερων
προβλημάτων) της ρομποτικής. Αφού είδα ότι το πρόβλημα αυτό ήταν όντως
επιλύσιμο, αφαίρεσα σταδιακά στα επόμενα δύο κεφάλαια μερικές βοηθητικές
παραδοχές που είχα κάνει, μέχρις ώτου να μην υπάρχει δυνατότητα να αφαιρεθεί
κάτι παραπάνω χωρίς να καταρρεύσει η κύρια συμβολή της μεθόδου μου. Έτσι
κατέληξα στην κατασκευή της πρώτης μεθόδου η οποία αφαιρεί έναν σημαντικό
περιορισμό από τις μεθόδους της βιβλιογραφίας και συμβάλει στη λύση ενός
θεμελιώδους πρόβληματος της ρομποτικής και γενικότερα της αναγνώρισης προτύπων.




Η όλη


\end{document}
