\documentclass[a4paper,10pt]{article}

\usepackage{xltxtra}
\usepackage{gfsdidot}
\usepackage{verse}

\setmainfont[Mapping=tex-text]{GFS Didot} \setsansfont[Mapping=tex-text]{DejaVu Sans}
\setmonofont[Mapping=tex-text]{DejaVu Sans Mono}
\newfontfamily\athenais{Athenais}
\pagenumbering{gobble}


%%%%%%%%%%%%%%%%%%%%%%%%%%%%%%%%%%%%%%%%%%%%%%%%%%%%%%%%%%%%%%%%%%%%%%%%%%%%%%%%
\begin{document}

Αξιότιμες καθηγήτριες, αξιότιμοι καθηγητές, συνάδελφοι, κυρίες και κύριοι:
ονομάζομαι Αλέξανδρος Φιλοθέου και σήμερα είναι σκοπός μου να σάς εκθέσω τις
συμβολές μου σε λύσεις διαδεδομένων προβλημάτων που αφορούν στο πεδίο της
ρομποτικής κινητής βάσης.

Αυτό που βλέπουμε εδώ είναι μία γραφική σύνοψη της δουλειάς μου.  Αρχικά θα σας
παρουσιάσω πώς μπορούμε να σχεδιάσουμε μία μέθοδο αξιολόγησης αυτόνομης
πλοήγησης με τρέχουσες μεθόδους, και από εκεί θα οδηγηθούμε στο κύριο
αντικείμενο της μελέτης μου, δηλαδή το πώς είναι δυνατό να μειώσουμε το σφάλμα
κατάστασης παρατηρητών θέσης και προσανατολισμού για ρομπότ που αισθάνονται το
περιβάλλον μέσω lidar, και συγκεκριμένα εδώ του φίλτρο σωματιδίων. Στη συνέχεια
θα λύσουμε το πρόβλημα του global localisation για οποιοδήποτε περιβάλλον,
χρησιμοποιώντας τον μοχλό στον οποίο βρίσκεται η συμβολή αυτής της διατριβής.
Αυτό θα μας οδηγήσει στην εφεύρεση γενικότερων λύσεων, που μπορούν να
χρησιμοποιηθούν και κατά τη διάρκεια της κίνησης του αισθητήρα σε γνωστό
περιβάλλον. Στο τέλος θα γενικεύσουμε αυτές τις μεθόδους όταν δεν γνωρίζουμε
καν δηλαδή το περιβάλλον του, για παράδειγμα για την εκτίμηση της οδομετρίας
ενός οχήματος.

Η ερευνα μου ξεκιναει με το εργο RELIEF. Ο σκοπος του εργου ηταν ο εξης. Στην
αγορά λιανικών προϊόντων υπάρχουν εταιρείες που πουλάνε τα προϊόντα τους σε
καταστήματα, και των οποίων το συνολικό απόθεμα αποθηκεύεται σε κεντρικές
αποθήκες. Εν γένει αυτές οι εταιρείες θα ήθελαν να γνωρίζουν σε καθημερινή βάση
το απόθεμά που βρίσκεται στις αποθήκες τους, όμως αυτό είναι τόσο κοστοβόρο που
μπορούν να μετρούν το απόθεμά τους μόνο λίγες φορές μέσα σε ένα οικονομικό
έτος. Ύστερα θα ήθελαν επίσης να γνωρίζουν τις θέσεις των προϊόντων μέσα σε ένα
κατάστημα ή μία αποθήκη τόσο για λόγους γρήγορης ανάκτησης όσο και για λόγους
που τους επιβάλλονται από τρίτα μέρη, γιατί πχ υπάρχουν συμφωνίες που
επιβάλλουν την τοποθέτηση των προϊόντων σε συγκεκριμένες θέσεις και ύψη μέσα σε
ένα κατάστημα.

Το έργο RELIEF είχε ως στόχο την κατασκευή μίας σειράς από αυτόνομα ρομπότ τα
οποία θα μπορούσαν να καταγράφουν το απόθεμα και να εκτιμούν τη θέση των
εμπορευμάτων μέσω τεχνολογίας RFID, ώστε αυτές οι ενέργειες να γίνονται ακόμα
και σε καθημερινή βάση, με ελάχιστη εμπλοκή ανθρώπων.

Η πρώτη απαίτηση που τέθηκε για τα επίγεια ρομπότ του έργου ήταν να είναι ικανά
να πλοηγούνται αυτόνομα στο χώρο. Σε πραγματικές συνθήκες μπορείτε να
φανταστείτε πως προτού κλείσει η αποθήκη το βράδυ, ή ακόμα και κατά τη διάρκεια
μίας εργάσιμης μέρας, στο ρομπότ δίνεται η εντολή να περάσει από συγκεκριμένα
σημεία του χώρου ώστε να σαρώσει όλα τα ράφια στα οποία υπάρχουν αντικείμενα.
Η αυτονομία της πλοήγησης αφαιρεί την απαίτηση για ακριβό εξωτερικό εξοπλισμό
πάνω στον οποίο θα μπορούσε να οδηγηθεί το ρομπότ, ενώ ταυτόχρονα το κάνει
ικανό να μπορεί να εκτελεί τις ενέργειές του απρόσκοπτα ενώ γύρω του υπάρχουν
κινούμενα εμπόδια όπως άνθρωποι ή μηχανήματα.

2.20'


%%%%%%%%%%%%%%%%%%%%%%%%%%%%%%%%%%%%%%%%%%%%%%%%%%%%%%%%%%%%%%%%%%%%%%%%%%%%%%%%
% ch3 / 7.30 min
%%%%%%%%%%%%%%%%%%%%%%%%%%%%%%%%%%%%%%%%%%%%%%%%%%%%%%%%%%%%%%%%%%%%%%%%%%%%%%%%
% ch3/0 ------------------------------------------------------------------------
Η αυτόνομη πλοήγηση υποθέτει 5 προαπαιτούμενα. Πρέπει να υπάρχει τουλάχιστον
ένας εξωδεκτικός αισθητήρας,
<CLICK>
ο χάρτης του περιβάλλοντος στο οποίο πλοηγείται το ρομπότ,
<CLICK>
μία μέθοδος εκτίμησης της στάσης του ρομπότ στο σύστημα συντεταγμένων
του χάρτη,
<CLICK>
μία αρχική στάση
<CLICK>
και μία τελική στάση. Ως στάση του ρομπότ ορίζουμε το διάνυσμα κατάστασης που
συνιστάται από τη θέση και τον προσανατολισμό του ρομπότ ως προς το σύστημα
αναφοράς του χάρτη.



% ch3/1 ------------------------------------------------------------------------
Δεδομένων αυτών η αυτόνομη πλοήγηση συνίσταται σε δύο μέρη:
<CLICK>
αφενός χρειάζεται ένας αλγόριθμος που δέχεται τον χάρτη, και την αρχική και
τελική στάση και παράγει ένα μονοπάτι το οποίο συνδέει την αρχική με την τελική
στάση χωρίς να τέμνει εμπόδια του χάρτη,
<CLICK>
και αφετέρου ένας ελεγκτής κίνησης, ο οποίος, δεδομένου του μονοπατιού, της
στιγμιαίας εκτίμησης για τη στάση του ρομπότ, και μετρήσεις από αισθητήρες
παράγει εντολές κίνησης τις οποίες λαμβάνουν ως είσοδο οι κινητήρες του ρομπότ
ώστε αυτό να κινείται πάνω στο μονοπάτι που παρήγαγε ο πρώτος αλγόριθμος χωρίς
να συγκρούεται με εμπόδια του περιβάλλοντος.
<CLICK>
Ο αλγόριθμος κατασκευής μονοπατιών ονομάζεται στην ορολογία global planner, και
ο ελεγκτής κίνησης local planner.

% ch3/2 ------------------------------------------------------------------------
Για να υλοποιήσω το συνολικό σύστημα αυτόνομης πλοήγησης στα πλαίσια του έργου
αρχικά στράφηκα στη διαθέσιμη λογισμικογραφία, όπου ανακάλυψα πως υπήρχαν
πολλαπλοί αλγόριθμοι global και local planners.
<CLICK>
Πουθενα ομως στη διαθεσιμη
βιβλιογραφια δεν υπηρχε συγκριτικη αναλυση της συνδυαστικής επιδοσης τους στο
έργο της αυτόνομης πλοήγησης ώστε να επιλέξω ποιός συνδυασμός θα εκπλήρωνε τους
στόχους μας στα πλαίσια του έργου.
<CLICK>
Συνεπώς αποφασίσαμε πως θα ήταν επωφελές τόσο για το έργο όσο και για άλλους
μηχανικούς ρομποτικής να σχεδιάσουμε μία μέθοδο αξιολόγησης των διαθέσιμων
αλγορίθμων global και local planners και των συνδυασμών τους, την οποία θα
χρησιμοποιούσαμε μέσω πειραματικής διαδικασίας ώστε να καταλήξουμε σε
συμπεράσματα για την επίδοση των τρέχοντων διαθέσιμων αλγορίθμων που υλοποιούν
αυτόνομη πλοήγηση σε επίγειες κινητές ρομποτικές βάσεις.

% ch3/3 ------------------------------------------------------------------------
Αρχικά συγκεντρώσαμε όλα τα διαθέσιμα πακέτα λογισμικού και τα υποβάλλαμε σε
αξιολόγηση με βάση ποιοτικά κριτήρια, δηλαδή τα ίδια κριτήρια που θα
χρησιμοποιούσε ένας μηχανικός λογισμικού προτού φτάσει στο σημείο να εξετάσει
εάν πρακτικά η επίδοση του πακέτου λογισμικού είναι επαρκής.  Στα αριστερά
βρίσκονται τα ονόματα των global και local planners και πάνω στην οριζόντια
γραμμή οι συντομογραφίες των ποιοτικών κριτηρίων, όπως εάν το πακέτο διαθέτει
τεκμηρίωση, εάν είναι παραμετροποιήσιμο, και τι ανάγκες έχει σε υπολογιστικούς
πόρους. Στο τέλος αυτού του φιλτραρίσματος απέμειναν τρεις αλγορίθμοι χάραξης
μονοπατιών και τρεις ελεγκτές κίνησης, οπότε η πειραματική διαδικασία
πραγματοποιήθηκε σε συνολικά εννέα συνδυασμούς αλγορίθμων.

% ch3/4 ------------------------------------------------------------------------
Η αξιολόγηση κάθε συνδυασμού έγινε σε δύο περιβάλλοντα προσομοίωσης και σε ένα
πραγματικό περιβάλλον. Και στις τρεις περιπτώσεις θέσαμε μία αρχική και μία
τελική στάση και ζητήσαμε από κάθε συνδυασμό global και local planners να
πλοηγηθεί αυτόνομα από την αρχική στην τελική στάση με βάση το μονοπάτι που θα
σχεδίαζε ο global planner και τις εντολές κίνησης του local planner. Κάθε
συνδυασμός απο planners επανέλαβε το πείραμα 10 φορές σε κάθε περιβάλλον, και
για κάθε πείραμα καταγράφηκε ένας αριθμός από μετρικές προκειμένου να γίνει η
αξιολόγηση κάθε συνδυασμού και να υπάρχει ένα κοινό σύστημα κρίσης για όλους
ώστε στο τέλος να προκύψει μία ιεραρχία συνδυασμών.

% ch3/5 ------------------------------------------------------------------------
Οι μετρικές αυτές είναι τριών ειδών: αφορούν είτε αποκλειστικά στους global
planners, όπως για παράδειγμα το μήκος των σχεδιασθέντων μονοπατιών ή η μέση
ελάχιστη απόσταση ενός μονοπατιού από εμπόδια (8 μετρικές), είτε αποκλειστικα
στους local planners (8 μετρικές), όπως για παράδειγμα ο αριθμός αποτυχιών
έυρεσης ταχυτήτων προς τον συνολικό αριθμό κλήσεων του ελεγκτή, ή αποκλειστικά
στο συνδυασμό τους, όπως για παράδειγμα ο χρόνος πλοήγησης, ή πραγματική ολικά
ελάχιστη απόσταση από εμπόδια κατά τη διάρκεια μίας πλοήγησης (12 μετρικές).

% ch3/6 ------------------------------------------------------------------------
Με βάση τις τιμές των μετρικών που έχουμε καταγράψει, αυτό που θέλουμε στο
τέλος είναι να μπορέσουμε να αποδώσουμε μία τιμή σε κάθε συνδυασμό από
planners.  Τα προβλήματα εδώ είναι δύο: πώς θα δώσουμε μία τιμή όταν οι
μετρικές εκφράζονται σε διαφορετικές μονάδες μέτρησης, και πώς θα
κατασκευάσουμε μία συνάρτηση απόδοσης αξίας σε κάθε συνδυασμό που να είναι
γνησίως αύξουσα όταν κάποιες μετρικές συνεισφέρουν με τρόπο ανάλογο καθώς
αυξάνονται (όπως η μέση απόσταση από εμπόδια, όσο μεγαλύτερη η απόσταση από
εμπόδια τόσο ασφαλέστερη είναι η πλοήγηση) ενώ άλλες με τρόπο αντιστρόφως
ανάλογο (όπως ο χρόνος πλοήγησης).

% ch3/7 ------------------------------------------------------------------------
Για να αποκτήσουμε ένα κοινό σύστημα αναφοράς αρχικά κανονικοποιούμε τις τιμές
των μετρικών μέσω της συνάρτησης N, δηλαδή εξετάζουμε την τιμή της μετρικής m
για έναν συνδυασμό και τις ακραίες τιμές της για κάθε συνδυασμό, ώστε στο τέλος
η τιμή της εκφράζεται στο διάστημα [0,1].  Κατασκευάζουμε την συνάρτηση
απόδοσης αξίας για ένα περιβάλλον μέσω της συνάρτησης V. Εδώ w είναι ένα βάρος
που αποδίδουμε σε κάθε μετρική ώστε να είναι δυνατή η απόδοση μεγαλύτερης ή
μικρότερης βαρύτητας ανάλογα με τις απαιτήσεις της εκάστοτε αξιολόγησης, και
$I$ είναι η συνάρτηση δείκτης. Η συνάρτηση I(C,m) είναι μηδέν μόνο για μετρικές
που αφορούν σε συνδυασμό από planners όταν η πλοήγηση ήταν αποτυχημένη, και τη
χρησιμοποιούμε για να συνεκτιμήσουμε την αξία των συνιστωσών ενός συνδυασμού
ακόμα και όταν η πλοήγηση ήταν αποτυχημένη.

% ch3/8 ------------------------------------------------------------------------
Μέσω της V και των τιμών όλων των μετρικών που έχουμε καταγράψει για κάθε
πείραμα στο τέλος λαμβάνουμε τα αποτελέσματα του πίνακα, τα οποία εμφανίζουν
την αξία κάθε συνδυασμού για κάθε περιβάλλον, και τη συνολική αξία κάθε
συνδυασμού με βάση τα αποτελέσματα κάθε περιβάλλοντος. Η εκτέλεση πειραμάτων σε
διαφορετικά και διαφορετικής δυσκολίας περιβάλλοντα και με διαφορετικούς
αισθητήρες έφερε στο φως το μοτίβο που βλέπετε. Δηλαδή ο καλύτερος local planner
ειναι ο teb, μετα ο eband, και μετα ο dwa, και δεδομένου ενός local planner
ο καλύτερος global planner είναι για ολους τους local planners o navfn, μετα ο
global planner, και μετα ο sbpl.

% ch3/9 ------------------------------------------------------------------------
Αυτό που είναι σήμερα δυνατον που δεν ήταν πριν αυτή τη μελέτη, είναι
καταρχάς η ύπαρξη μίας περιεκτικής και επεκτάσιμης μεθοδολογίας αξιολόγησης
μεθόδων αυτόνομης πλοήγησης. Έπειτα ένας μηχανικός ρομποτικής που έχει
αντικείμενο την αυτόνομη πλοήγηση μπορεί να προσδιορίσει τα δικά του κριτήρια
με βάση διαφορετικές απαιτήσεις δίνοντας διαφορετικά βάρη στη συνεισφορά της
κάθε μετρικής. Έπειτα, με το βλέμμα στο μέλλον, υπάρχει πλέον το υπόβαθρο για
την αξιολόγηση οποιοασδήποτε νέας μεθόδου που είτε χαράσσει μονοπάτια, είτε
είναι ελεγκτής κίνησης, και σε οποιοδήποτε περιβάλλον.

% ch3/10 -----------------------------------------------------------------------
Κατά τη διάρκεια διεξαγωγής της πειραματικής διαδικασίας αυτό που παρατήρησα
ήταν ότι η εκτίμηση της στάσης εμφάνιζε σφάλματα παρατηρήσιμα ακόμα και με
γυμνό μάτι. Σε αυτή τη διαφάνεια βλέπουμε την εξέλιξη του μέσου σφάλματος
εκτίμησης στάσης στα δύο προσομοιωμένα περιβάλλοντα, το οποίο δεν εμφανίζει
σταθερό μέτρο είτε στην πορεία του χρόνου είτε ανά περιβάλλον.  Το μέτρο του
σφάλματος εδώ μετριέται σε εκατοστά γιατί η στάση του ρομπότ υπόκειται σε
εκτίμηση και όχι μέτρηση, δηλαδή υποθέτουμε ότι στο περιβάλλον που κινείται το
ρομπότ δεν υπάρχει μετρητικός εξοπλισμός, αλλά το ρομπότ εκτίμα το ίδιο τη
στάση του μέσω των αισθητήρων του.
<CLICK>
Σε γενικές γραμμές το φαινόμενο του σφάλματος είναι φυσιολογικό λόγω της χρήσης
παρατηρητή για την εκτίμηση της στάσης, όμως στην περίπτωση του έργου RELIEF
αυτό το σφάλμα κληροδοτείται στο σφαλμα θέσης των προϊόντων, καθώς η εκτίμηση
της θέσης ενός προϊόντος προκύπτει μέσω της εκτίμησης της στάσης των κεραιών
που φέρει το ρομπότ, οι οποίες με τη σειρά τους είναι μετασχηματισμοί
της εκτίμησης της στάσης του ρομπότ.

Έτσι περνάμε στο επόμενο πρόβλημα, το οποίο είναι το πρόβλημα της ελάττωσης του
σφάλματος εκτίμησης στάσης παρατηρητών, όπου συγκεκριμένα εστιάζουμε σε φίλτρα
σωματιδίων που χρησιμοποιούν αισθητήρες lidar.

9.50'


%%%%%%%%%%%%%%%%%%%%%%%%%%%%%%%%%%%%%%%%%%%%%%%%%%%%%%%%%%%%%%%%%%%%%%%%%%%%%%%%
% ch4 / 13.40 min
%%%%%%%%%%%%%%%%%%%%%%%%%%%%%%%%%%%%%%%%%%%%%%%%%%%%%%%%%%%%%%%%%%%%%%%%%%%%%%%%

% ch4/0 ------------------------------------------------------------------------
Εδώ χρησιμοποιούμε το φίλτρο σωματιδίων διότι είναι περισσότερο ευέλικτο και
επιδεκτικό στη βελτίωση. Στη διαθέσιμη βιβλιογραφία υπάρχουν δύο μέθοδοι
ελάττωσης του σφάλματος εκτίμησής του, δηλαδή της προσθετικής ευθυγράμμισης
πραγματικών σαρώσεων με εικονικές σαρώσεις, και της ανατροφοδότησης του
αποτελέσμάτος στον πληθυσμό του φίλτρου. Η συμβολή μου σε αυτό το κεφάλαιο
αφορά στη δεύτερη μέθοδο, ενώ εισάγω και μία δεύτερη, αυτήν της διαλογής
σωματιδίων.

% ch4/1 ------------------------------------------------------------------------
Όσο αφορά στην εκτίμηση της στάσης ενός οχήματος, όλοι οι πιθανοτικοί
παρατηρητές στη ρομποτική εκτιμούν τη στάση του, με βάση το μοντέλο παρατήρησης
του αισθητήρα που φέρει το όχημα, και του κινηματικού μοντέλου του. Το φίλτρο
σωματιδίων ειδικά, σε αντίθεση με το φίλτρο καλμαν, χρησιμοποιεί πολλαπλές
υποθέσεις στάσης, και η τελική του εκτίμηση προκύπτει ως ο μέσος όρος των
στάσεων αυτών των υποθέσεων, βεβαρυμμένος κατά το βάρος της κάθε μίας. Ως βάρος
εδώ νοείται η πιθανότητα παρατήρησης μίας δεδομένης μέτρησης από τη στάση της
κάθε υπόθεσης, και συνεπώς υποθέσεις των οποίων το σφάλμα εκτίμησης είναι
μικρότερο από άλλες εμφανίζουν μεγαλύτερο βάρος, και συνεπώς επηρεάζουν
περισσότερο την τελική εκτίμηση.

% ch4/2 ------------------------------------------------------------------------
Με βάση αυτον τον συλλογισμό τυποποιούμε την πρώτη υπόθεση, η οποία λέει ότι
εάν για την εξαγωγή της εκτίμησης του φίλτρου χρησιμοποιηθεί ένα υποσύνολο των
πιο βαρέων σωματιδίων αντί για όλο τον πληθυσμό, τότε αναμένουμε η εκτίμηση
του φίλτρου να έχει χαμηλότερο σφάλμα εκτίμησης.


% ch4/3 ------------------------------------------------------------------------
Ο δέυτερος τρόπος με τον οποίον μπορεί να ελαττωθεί το σφάλμα ενός
παρατηρητή είναι μέσω ευθυγράμμισης πραγματικών με εικονικές σαρώσεις αισθητήρα
lidar. Αυτή η μέθοδος θα μας ακολουθήσει σε όλη την υπόλοιπη έρευνά μου, και
για λόγους οικονομίας θα αναφέρομαι σε αυτήν συντομογραφικά ως sm2. Η τεχνική
sm2 είναι υποσύνολο της γενικής μεθόδου ευθυγράμμισης πραγματικών σαρώσεων, η
οποία ονομάζεται scan-matching.

Έστω λοιπόν ένας αισθητήρας lidar ο οποίος από δύο διαφορετικές στάσεις
συλλαμβάνει δύο διαφορετικές σαρώσεις. Η ευθυγράμμιση δύο πραγματικών σαρώσεων
στοχεύει στην εύρεση εκείνου του μετασχηματισμού που όταν εφαρμοσθεί στα σημεία
της πρώτης σάρωσης θα τα κάνει να συμπέσουν στα σημεία της δεύτερης με το
ελάχιστο σφάλμα. Το πρόβλημα της ευθυγράμμισης δεν έχει μόνο θεωρητικό
ενδιαφέρον, γιατι ο τελικός μετασχηματισμός ανάμεσα στα σημεία των σαρώσεων
είναι ο ίδιος που εκφράζει τη στάση από την οποία συνελήφθη η δεύτερη σάρωση
στο σύστημα αναφοράς που ορίζει η πρώτη. Για αυτό το λόγο η ευθυγράμμιση
σαρώσεων είναι θεμελιώδες κομμάτι της ρομποτικής, και χρησιμοποιείται κατά
κόρον για την εκτίμηση της οδομετρίας ενός οχήματος.  Στα αριστερά
απεικονίζεται ένα περιβάλλον, το ρομπότ σε δύο διαφορετικές στάσεις, και οι
σαρώσεις τις οποίες συνέλαβε από αυτές. Στη μέση βλέπουμε τη διαδικασία
ευθυγράμμισης των δύο αυτών σαρώσεων, και στα δεξιά σε εστίαση την εξέλιξη της
εκτίμησης της δεύτερης στάσης του ρομπότ στο σύστημα αναφοράς της πρώτης
σάρωσης.

Με λίγα λόγια εάν διαθέτουμε δύο σαρώσεις και γνωρίζουμε τη στάση από την οποία
συνελήφθη μία από τις δύο: η ευθυγράμμιση σαρώσεων μπορεί να εκτιμήσει την
άγνωστη στάση από την οποία συνελήφθη η άλλη.

Σε αυτό το γεγονός κρύβεται μία δεύτερη χρησιμότητα της ευθυγράμμισης σαρώσεων.
Εαν αντικαταστήσουμε τη μία από τις δύο μετρήσεις με μία εικονική σάρωση,
δηλαδή με μία σάρωση που προσομοιώνει την αρχή λειτουργίας του lidar στο χάρτη
αντί για το περιβάλλον, η οποία υπολογίζεται από την εκτίμηση της στάσης του
ρομπότ, τοτε μπορούμε να υπολογίσουμε το μετασχηματισμό ανάμεσα στην εκτίμηση
και την άγνωστη πραγματική στάση του ρομπότ, και αφού γνωρίζουμε την εκτίμηση,
μπορούμε να υπολογίσουμε την πραγματική του στάση.

Με αυτόν τον τρόπο μπορούμε να προσαρμόσουμε τη μέθοδο sm2 στην έξοδο του
φίλτρου σωματιδίων, όπου λόγω θορύβου μέτρησης και ατελούς αναπαράστασης του
χάρτη σε σχέση με το περιβάλλον, η τελική έξοδος της μεθόδου sm2 φέρει και αυτή
σφάλμα, και συνεπώς αποτελεί μία δεύτερη εκτίμηση.

Αυτή η δεύτερη εκτίμηση είναι μία υπόθεση για την οποία το ίδιο το φίλτρο δεν
έχει γνώση, και συνεπώς θα ήταν ωφέλιμο, εάν η υπόθεση φέρει όντως μικρότερο
σφάλμα, να εισαχθεί στον πληθυσμό του, ώστε να το σφάλμα του ίδιου το φίλτρου
να μειωθεί.

Κατά μία μέθοδο της βιβλιογραφίας, η υπόθεση που παράγεται μέσω sm2 εισάγεται
στον πληθυσμό του φίλτρου ως ένα διακριτό σωματίδιο, το οποίο σε έναν πληθυσμό
αρκετών εκατοντάδων σωματιδίων έχει μεν επίδραση, αλλά σχεδόν αμελητέα.

Κατά μία άλλη μέθοδο η εκτίμηση της sm2 αντικαθιστά τον πληθυσμό του φίλτρου
στο σύνολό του.  Η επιτυχία αυτής της μεθόδου εξαρτάται αποκλειστικά από την
επιτυχία της ευθυγράμμισης, η οποία δεν είναι εγγυημένη, και μπορεί να έχει ως
αποτέλεσμα την καταστροφική αποτυχία της συνέχειας της εκτίμησης της στάσης του
ρομπότ, και συνεπώς και της ασφάλειας της πλοήγησης η οποία βασίζεται στην
εκτίμηση της στάσης του οχήματος.

Μπορούμε να ελαττώσουμε το σφάλμα εκτίμησης χωρίς να επιφέρουμε αυτές
τις παρενέργειες εάν εισάγουμε το αποτέλεσμα της ευθυγράμμισης στον πληθυσμό
του φίλτρου ως πολλαπλά σωματίδια αλλά σε πληθικότητα μικρότερη από τον
πληθυσμό του φίλτρου. Τότε αυτό θα έχει θεωρητικά ως αποτέλεσμα πιο γρήγορη
σύγκλιση σε σχέση με την περίπτωση που το εισάγαμε ως ένα σωματίδιο, χωρίς
ταυτόχρονα να διακινδυνεύεται σε περίπτωση αποτυχίας της ευθυγράμμισης η
ανθεκτικότητα του φίλτρου, γιατί το ίδιο θα συνεχίσει να περιέχει υποθέσεις που
εξηγούν καλύτερα τις μετρήσεις από αυτές που έχουν καταλήξει σε λανθασμένες
τοποθεσίες.

Με βάση αυτά συντάσσουμε άλλες δύο υποθέσεις. Η πρώτη είναι ότι το αποτέλεσμα
της ευθυγράμμισης της πραγματικής σάρωσης από το lidar και της εικονικής
σάρωσης από την εκτιμώμενη στάση του φίλτρου έχει μικρότερο σφάλμα από την
εκτίμηση του φίλτρου.  Η δεύτερη αφορά στην ανάδραση αυτής της εκτίμησης, και
λέει ότι εάν το αποτέλεσμα της ευθυγράμμισης εισαχθεί στον πληθυσμό του φίλτρου
ως μία πλειάδα σωματιδίων, τότε, εάν στέκει η προηγούμενη υπόθεση, το σφάλμα
εκτίμησης του φίλτρου θα είναι χαμηλότερο από τον σφάλμα εκτίμησης του
ονομαστικού φίλτρου, χαμηλότερο από το σφάλμα εκτίμησης του φίλτρου εάν η
υπόθεση εισάγετο ως μόνο ένα σωματίδιο, και ο πληθυσμός του φίλτρου είναι πιο
ανθεκτικός σε σχέση με τον πληθυσμό του φίλτρου εάν ο αυτός αρχικοποιείται κάθε
φορά με το αποτέλεσμα της ευθυγράμμισης.


% ch4/4 ------------------------------------------------------------------------
Για να δοκιμάσουμε τις τρεις υποθέσεις χρησιμοποιούμε δύο περιβάλλοντα, ένα
απλό το οποίο ονομάζεται corridor, και ένα που ομοιάζει σε αποθήκη, δηλαδή το
αναμενόμενο περιβάλλον των ρομπότ που κατασκευάσαμε για το έργο RELIEF. Για
κάθε περίπτωση διαλογής σωματιδίων και μεθόδου ανάδρασης από το ρομπότ ζητήθηκε
να πλοηγηθεί αυτόνομα από μία αρχική σε μία τελική στάση 100 διαφορετικές
φορές. Το ρομπότ φέρει έναν αισθητήρα lidar γωνιακού εύρους 260 μοιρών, 640
ακτίνων, με θόρυβο μέτρησης κανονικά κατανεμημένο με τυπική απόκλιση ενός
εκατοστού. Ο πληθυσμός του φίλτρου είναι κυμαινόμενος με ελάχιστη πληθικότητα
200 σωματίδια, και μέγιστη 500.

Σε αυτή τη διαφάνεια βλεπουμε την κατανομές των μέσων σφαλμάτων εκτίμησης ανά
διαδρομή για 100 επαναλήψεις σε κάθε περιβάλλον, ανά μέθοδο διαλογής. Με τη
συντομογραφία 100% εννοούμε την ονομαστική κατάσταση του φίλτρου. Με μεγαλύτερο
από μέσο W εννοούμε την κατάσταση που μονο εκείνα τα σωματίδια των οποίων το
βάρος είναι μεγαλύτερο από το μέσο βάρος του πληθυσμού επιλέγονται να ψηφίσουν,
με 10% την κατάσταση που μόνο το 10% των βαρύτερων σωματιδίων ψηφίζουν, και με
τοπ συμβολίζουμε το σωματίδιο που φέρει το υψηλότερο βάρος του πληθυσμού. Με
βάση τα πειραματικά δεδομένα βλέπουμε πως η διαλογή των σωματιδίων με βάρος
μεγαλύτερο από το μέσο βάρος του συνολικού πληθυσμού εμφανίζει μικρότερο σφάλμα
από την ονομαστική κατάσταση, και πως το σφάλμα μειώνεται ακόμα περισσότερο
όταν επιλέγουμε το τοπ 10% των σωματιδίων. Το αντιδιαισθητικό σε αυτά τα
πειράματα είναι πως το σωματίδιο που εμφανίζει το μεγαλύτερο βάρος, δηλαδή αυτό
που εξηγεί την τρέχουσα μέτρηση στο μεγαλύτερο βαθμό με βάση το μοντέλο
παρατήρησης εμφανίζει το μεγαλύτερο σφάλμα ανάμεσα σε όλες τις διαμορφώσεις. Το
συμπέρασμα που αντλούμε από αυτά τα αποτελέσματα είναι ότι ναι μεν
επιβεβαιώνεται η υπόθεσή μας, αλλά η βελτίωση είναι μικρή, και έχει ένα οριακό
σημείο ως προς τον αριθμό των πιο βαρέων σωματιδίων που επιλέγονται.

Σε αυτή τη διαφάνεια με κόκκινο φαίνονται τα σφάλματα του αποτελέσματος της
μεθόδου sm2 για κάθε μέθοδο διαλογής, τα οποία είναι κατά μέσο όρο στο σύνολό
τους χαμηλότερα από εκείνα των εκτιμήσεων του φίλτρου.

και σε αυτή τη διαφάνεια βλέπουμε τα μέσα σφάλματα ανά μέθοδο
ανάδρασης. Εδώ με open συμβολίζουμε την open-loop κατάσταση, δηλαδή και πάλι
την ονομαστική κατάσταση του φίλτρου, με soft-1 τη διαμόρφωση όπου το
αποτέλεσμα του sm2 εισάγεται στον πληθυσμό του φίλτρου με τη μορφή ενός μόνο
σωματιδίου, με soft-50 τη διαμόρφωση όπου το αποτέλεσμα του sm2 εισάγεται στον
πληθυσμό με τη μορφή τόσων σωματιδίων όσών να αποτελούν το 50% του τελικού
πληθυσμού, και hard τη διαμόρφωση όπου το φίλτρο αρχικοποιείται εκ του μηδενός
γύρω από το αποτέλεσμα του sm2. Εδώ βλέπουμε πως γενικά η μέθοδος που εισάγαμε
εμφανίζει κατά μέσο όρο τα μικρότερα σφάλματα, και πως είναι πιο ανθεκτική από
τη μέθοδο hard.

% ch4/5 ------------------------------------------------------------------------
Τί είναι σήμερα δυνατόν που δεν ήταν πριν από αυτή την έρευνα; Το κύριότερο
είναι ότι σήμερα μπορούμε να εγγυηθούμε τη μείωση των σφαλμάτων ενός φίλτρου
σωματιδίων χωρίς να διακινδυνεύσουμε την ευρωστία του. Εδώ ο τρόπος ανάδρασης
που εισάγαμε είναι καθοριστικός γιατί η ευθυγράμμιση πραγματικών με εικονικές
σαρώσεις δεν είναι πάντα επιτυχής, όπως είδαμε από τα αποτελέσματα της μεθόδου
ανάδρασης που αρχικοποιεί το φίλτρο γύρω από το αποτέλεσμά της.

Οπότε εδώ γεννάται το φυσικό ερώτημα: ποιά είναι τα τρωτά σημεία των μεθόδων
ευθυγράμμισης σαρώσεων? Με δύο λόγια τα τρωτά σημεία είναι η ευαισθησία της
λύσης στη ρύθμιση των παραμέτρων και στον θόρυβο.

Το πρόβλημα με την παραμετροποίηση που αφορά τουλάχιστον στη μέθοδο που
παρήγαγε τα αποτελέσματα μου μόλις είδαμε, η οποία είναι μάλιστα η καλύτερη
μέθοδος στη βιβλιογραφία, είναι ότι δεν είναι διαισθητική διαδικασία, ότι
μικρές μεταβολές των τιμών των παραμέτρων παράγουν δυσανάλογα μεγάλες μεταβολές
στην έξοδο, και ότι για κάποιες παραμέτρους δεν υπάρχουν τιμές που να μπορούν
να καλύψουν όλες τις στάσεις σε ένα περιβάλλον.

Δεν είναι τυχαίο πως αυτά τα προβλήματα εμφανίζονται για τις παραμέτρους που
αφορούν στη διαδικασία εύρεσης αντιστοιχίσεων ανάμεσα στις ακτίνες των σαρώσεων
εισόδου. Αν ψάξει κανείς τη βιβλιογραφία θα ανακαλύψει μάλιστα ότι δεν υπάρχει
μέθοδος ευθυγράμμισης που να μην χρησιμοποιεί κάποιου είδους μηχανισμό
αντιστοίχισης, ο οποίος προσπαθεί να εκτιμήσει την αντιστοίχιση σημείων ή
κατανομών σημείων της μίας σάρωσης προς σημειο, ευθύγραμμο τμήμα, ή κατανομή
της άλλης.

Και εδώ φτάνουμε στο δεύτερο τρωτό σημείο, το οποίο αφορά σε όλες τις μεθόδους
ευθυγράμμισης της βιβλιογραφίας, επειδή ακριβώς όλες χρησιμοποιούν αυτόν το
μηχανισμό: το οποίο είναι το πρόβλημα του θορύβου μέτρησης.

Εδώ το πρόβλημα είναι δεν είναι μόνο ότι δυσχεραίνεται η διαδικασία διάκρισης
αληθών από ψευδείς αντιστοιχίσεις, αλλά ότι αυτή η διάκριση καθίσταται ατελής
λόγω παρουσίας θορύβου.

Ταυτόχρονα, ακόμα και αν υποθέσουμε ιδανικές συνθήκες, η ίδια η διαδικασία
εύρεσης αντιστοιχίσεων καθίσταται πιθανά προβληματική γιατί ακόμα και αν οι
σαρώσεις έχουν εύρος 360 μοίρες, δεν υπάρχει εγγύηση ότι όλα τα σημεία της μίας
σάρωσης θα αντιστοιχούν σε όλα τα σημεία της δεύτερης, όπως βλέπουμε σε αυτή τη
διαφάνεια, όπου στα δεξιά έχουμε αντιστρέψει τη φορά ευθυγράμμισης.

Εν ολίγοις το συμπέρασμά μου είναι ότι το πιο τρωτό σημείο των μεθόδων
ευθυγράμμισης σαρώσεων όλης της βιβλιογραφίας είναι ο ίδιος ο μηχανισμός
εύρεσης αντιστοιχίσεων.

Για αυτό το λόγο ξεκινάω να ερευνώ τρόπους με τους οποίους θα ήταν δυνατή η
ευθυγράμμιση πραγματικών με εικονικές σαρώσεις χωρίς την εύρεση αντιστοιχίσεων
ανάμεσα στις σαρώσεις εισόδου.

22'



%%%%%%%%%%%%%%%%%%%%%%%%%%%%%%%%%%%%%%%%%%%%%%%%%%%%%%%%%%%%%%%%%%%%%%%%%%%%%%%%
% ch5 / 11.5min
%%%%%%%%%%%%%%%%%%%%%%%%%%%%%%%%%%%%%%%%%%%%%%%%%%%%%%%%%%%%%%%%%%%%%%%%%%%%%%%%

% ch5/0 ------------------------------------------------------------------------
Εδώ καταρχάς αναζητώ τις συνθήκες εκείνες κάτω από τις οποίες είναι δυνατόν να
είναι επιτυχής η ευθυγράμμιση, και καταλήγω πως πρώτα από όλα μία αναγκαία
συνθήκη είναι η ελαχιστοποίηση του αριθμού των σημείων που είναι ορατά από μία
σάρωση αλλά όχι από την άλλη, δηλαδή είναι αναγκαίο το γωνιακό εύρος όρασης του
αισθητήρα να είναι 360 μοίρες. Έπειτα, επειδή ακριβώς το πρόβλημα της
ευθυγράμμισης χωρίς την εύρεση αντιστοιχίσων, αν είναι επιλύσιμο, είναι
καινοφανές, στόχος μου είναι να εστιάσω στην επι της αρχής λύση του, δηλαδή
χωρίς χρονικούς περιορισμούς.

% ch5/1 ------------------------------------------------------------------------
Το μεγάλο πρόβλημα εδώ όμως είναι ότι δεν μπορώ να πατήσω στη βιβλιογραφία της
ευθυγράμμισης σαρώσεων γιατί ακριβώς όλες οι μέθοδοι της βιβλιογραφίας λύνουν
το πρόβλημα μέσω εύρεσης αντιστοιχίσεων, και συνεπώς αρχίζω να ερευνώ άλλα
πεδία. Εδώ το ζητούμενο είναι η εύρεση δύο μεθόδων: μίας που θα αναλάβει τη
λύση του προβλήματος της εκτίμησης της περιστροφής του συνόλου σημείων της μίας
σάρωσης ως προς αυτά της δεύτερης, και μίας που αφορά στην εκτίμηση της
μετατόπισης τους.

% ch5/2 ------------------------------------------------------------------------
Μέσα από την αναζήτησή μου βρήκα τη μέθοδο FMI-SPOMF, η οποία χρησιμοποιείται
στον κλάδο της υπολογιστικής όρασης για την εκτίμηση ανάμεσα σε δύο εικόνες
της μετατόπισης τους, της διαφοράς της γωνίας περιστροφής τους, της κλιμάκωσης
τους, και του βαθμού ομοιότητάς τους. Αυτές οι παράμετροι υπολογίζονται με
κλειστό τρόπο και χωρίς να υπολογίζονται αντιστοιχίσεις ανάμεσα
στις δύο εικόνες. Πώς μπορούμε να χρησιμοποιήσουμε αυτή τη μέθοδο στα δικά μας
συμφραζόμενα?

% ch5/3 ------------------------------------------------------------------------
Εάν προβάλλουμε δύο σαρώσεις εισόδου στο καρτεσιανό επίπεδο, εκφρασμένες στο
τοπικό σύστημα συντεταγμένων του κάθε αισθητήρα, δηλαδή στο τοπικό σύστημα του
πραγματικού αισθητήρα αφενός και το σύστημα του εικονικού αισθητήρα ο οποίος
μέσω raycasting υπολογίζει την εικονική σάρωση από την εκτίμηση στάσης του
ρομπότ αφετέρου, και στη συνέχεια τις διακριτοποιήσουμε, το αποτέλεσμα είναι
δύο πίνακες, δηλαδή δύο εικόνες, για τις οποίες μπορούμε να
χρησιμοποιήσουμε τη μέθοδο FMI-SPOMF για να υπολογίσουμε τα ζητούμενά μας.  Από
τη μία δε χρειάζεται να υπολογίσουμε τον συντελεστή κλιμάκωσης ανάμεσα στις δύο
εικόνες γιατί γνωρίζουμε τον συντελεστή κλίμακας του χάρτη ως προς το
περιβάλλον που αναπαριστά, από την άλλη όμως είναι προτιμότερο να μην
χρησιμοποιήσουμε το αποτέλεσμα μετατόπισης του SPOMF ώστε να μην υπάρχει
εξάρτηση της εκτίμησης από την ανάλυση των εικόνων.

% ch5/4 ------------------------------------------------------------------------
Για την εκτίμηση της μετατόπισης ανάμεσα στις δύο σαρώσεις χρησιμοποιούμε μία
μέθοδο η οποία εφαρμόζεται μετά την περιστροφή της δεύτερης σάρωσης ως προς την
πρώτη. Η μέθοδος αυτή υπολογίζει το βαρύκεντρο των δύο σαρώσεων στο
καρτεσιανό επίπεδο και μεταφέρει τη δεύτερη σάρωση ώστε να συμπέσει με την
πρώτη με βάση τη διαφορά των κεντροειδών τους. Για τον υπολογισμό της
μετατόπισης δεν χρησιμοποιούνται πουθενά αντιστοιχίσεις, διότι τα κεντροειδή
υπολογίζονται προφανώς το ένα ανεξάρτητα από το άλλο.

% ch5/5 ------------------------------------------------------------------------
Στην πραγματικότητα ο χρόνος εκτέλεσης αυτής της μεθόδου ευθυγράμμισης sm2
είναι πολύ μεγαλύτερος από τον χρόνο εκτέλεσης που απαιτείται για να εφαρμοσθεί
η μέθοδος σε συνδυασμό με ένα φίλτρο σωματιδίων και, επειδή ο χρόνος εκτέλεσης
δεν είναι σε αυτό το σημείο ζήτημα, τη χρησιμοποιούμε για την επίλυση της
παθητικής έκδοσης του προβλήματος του global localisation, η οποία δεν κάνει
παραδοχές για το χρόνο εκτέλεσης της λύσης του προβλήματος.

Κατά κύριο λόγο τα μεγάλα προβλήματα εκτίμησης στη ρομποτικής κινητής βάσης
διακρίνονται στα προβλήματα της παρατήρησης της στάσης ενός ρομπότ καθώς αυτό
κινείται, και στο πρόβλημα του global localisation, του προσδιορισμού δηλαδή
της στάσης του δεδομένων μόνο του χάρτη και μίας μέτρησης, όταν δεν υπάρχει
καμία πληροφορία για τη θέση και τον προσανατολισμό του ρομπότ.

Το πρόβλημα του global localisation μπορεί να λυθεί μέσω οποιασδήποτε τεχνικής
sm2 ως εξής: δεδομένου του χάρτη του περιβάλλοντος στο οποίο βρίσκεται το
φυσικό ρομπότ, διασπείρονται με τυχαίο τρόπο σε αυτόν ένας αριθμός από
υποθέσεις στάσης, οι οποίες τοποθετούνται σε μία ουρά. Από κάθε υπόθεση
υπολογίζεται η εικονική σάρωση, και στη συνέχεια μέσω sm2 επιχειρείται η
ευθυγράμμιση της με τη σάρωση που συλλαμβάνεται από το φυσικό αισθητήρα. Στο
τέλος κάθε ευθυγράμμισης αποθηκεύονται η τελική εκτίμηση στάσης και η τιμή μίας
μετρικής που αποτυπώνει το βαθμό ομοιότητας ή τελικής ευθυγράμμισης ανάμεσα
στην πραγματική σάρωση και την εικονική σάρωση.  Για τις τεχνικές sm2 που
λειτουργούν με αντιστοιχίσεις αυτό το μέτρο υπολογίζεται εσωτερικά σε κάθε
μέθοδο ως το άθροισμα των αποστάσεων των σημειών της μίας σάρωσης ως προς τα
σημεία, τις γραμμές ή τις κατανομές της δεύτερης, και στο δικό μας σύστημα αυτό
το μέτρο προέρχεται απευθείας από τον FMI-SPOMF. Στο τέλος το σύστημα εξάγει ως
τελική εκτίμηση στάσης εκείνη που σημειώνει τη μεγαλύτερη τιμή ομοιότητας.

% ch5/6 ------------------------------------------------------------------------
Για να δοκιμαστεί εάν το πρόβλημα της ανεύρεσης της στάσης ενός ρομπότ που
είναι εξοπλισμένο με έναν πανοραμικό αισθητήρα lidar είναι επιλύσιμο μέσω sm2
χωρίς αντιστοιχίσεις: δοκιμάζουμε το σύστημα επίλυσης σε πέντε προσομοιωμένα
περιβάλλοντα και ένα πραγματικό, για συνολικά 49 στάσεις, οι οποίες
δημιουργήθηκαν είτε τυχαία είτε έτσι ώστε να δοκιμάσουν την επίδοση των μεθόδων
sm2 που θα δοκιμαστούν σε αυτό το πρόβλημα. Τα πειράματα στα προσομοιωμένα
περιβάλλοντα επαναλήφθηκαν για 100 φορές ανά στάση, και στο πραγματικό
περιβάλλον για 5 φορές ανά στάση.  Στα προσομοιωμένα περιβάλλοντα
χρησιμοποιούμε έναν πανοραμικό αισθητήρα lidar μεγίστου βεληνεκούς δέκα μέτρων
με θόρυβο μέτρησης κανονικά κατανεμημένο, με τιμές τυπικής απόκλισης 1,2,και 5
εκατοστά, ενώ στα πραγματικά πειράματα χρησιμοποιούμε έναν αισθητήρα YDLIDAR
μέγιστου βεληνεκούς τριάντα μέτρων με κατανομή θορύβου μέτρησης που φαίνεται σε
αυτόν τον πίνακα.

Σε όλα τα πειράματα καταγράφουμε το τελικό σφάλμα θέσης και προσανατολισμού και
ονομάζουμε επιτυχημένη ανέυρεση στάσης κάθε περίπτωση όπου το τελικό σφάλμα
θέσης ήταν μικρότερο από ένα μέτρο, διότι μετά την επίλυση του προβήματος
global localisation, ακολουθεί η παρατήρηση της στάσης του ρομπότ, η οποία
γίνεται κατά κόρον με πιθανοτικά μέσα, τα οποία έχουν την ικανότητα να
συγκλινουν γιατί είναι εύρωστα σε τέτοια μεγέθη σφάλματος θέσης.

Σε όλα τα πειράματα δοκιμάζουμε δύο υποκείμενες μεθόδους. Από τη μία την
καλύτερη μέθοδο της βιβλιογραφίας, η οποία ονομάζεται plicp, και τη μέθοδο που
αποτελείται από τον FMI-SPOMF σε συνδυασμό με τη μέθοδο των κεντροειδών. Σε
αυτή τη διαφάνεια βλέπουμε τα αποτελέσματα στο προσομοιωμένο περιβάλλον
corridor. Στα δεξιά στην πάνω σειρά βλέπουμε τα ποσοστά αποτυχιών. Στη δεύτερη
σειρά απεικονίζεται το μέσο σφάλμα στάσης ανά τιμή τυπικής απόκλισης μετρητικού
θορύβου, και στην κάτω σειρά απεικονίζονται οι χρόνοι εκτέλεσης των μεθόδων ανά
υπόθεση. Αυτή η διαφάνεια συνοψίζει αυτά που θα δούμε και στις επόμενες, δηλαδή
πως ο plicp εμφανίζει μεγαλύτερα ποσοστά αποτυχιών ανέυρεσης της στάσης του
αισθητήρα, πως τα σφάλματα του FMI-SPOMF είναι της ίδιας τάξης μεγέθους με αυτά
του plicp, και πως ο χρόνος εκτέλεσής του είναι μεγαλύτερος από αυτόν το plicp.

Στο περιβάλλον HOME βλέπουμε τα ίδια, με τον plicp να μην μπορεί να βρει ποτε
σε καποιες περιπτώσεις τη στάση του αισθητήρα λόγω συμμετριών του
περιβάλλοντος,

σε αυτή τη διαφάνεια βλέπουμε τα αποτελέσματα σε μία αποθήκη, όπου  η έλλειψη
ειδικής παραμετροποίησης για το γεγονός ότι το βεληνεκές του αισθητήρα είναι
μικρό σε σχέση με τη γεωμετρία του περιβάλλοντος κάνει τον plicp να αποτυγχάνει
σχεδόν παντού, ενώ o SPOMF δεν εμφανίζει τέτοια προβλήματα.

Στο περιβάλλον WILLOGARAGE φαίνεται πως ο plicp απαιτεί οι στάσεις από τις
οποίες συλλαμβάνονται οι δύο μετρήσεις να έχουν μικρή απόσταση ώστε να είναι
ικανός να τις ευθυγραμμίσει, και πως το ίδιόν του κριτήριο επιλογής τελικής
στάσης δεν είναι ικανό να ξεδιαλύνει ασάφειες, όπως δηλαδή στο περιβάλλον HOME.

Το περιβάλλον LANDFILL το φτιάξαμε από το μηδέν ως ένα περιβάλλον που δεν
περιλαμβάνει ευθείες γραμμές, ή γωνίες, ώστε να δείξουμε πως σε αντίθεση με μία
κλάση μεθόδων της βιβλιογραφίας, η μέθοδος που οικοδομήσαμε είναι αναλλοίωτη
των περιβαλλόντων και δεν απαιτεί να εμφανίζουν συγκεκριμένα features,

και εδώ βλέπουμε τα αποτελέσματα που διενεργήσαμε στο εργαστήριο υπολογιστών
και αρχιτεκτονικής, στα οποία βλέπουμε την αντιστροφή του μοτίβου που αφορά στο
χρόνο εκτέλεσης λόγω αύξησης των ακτίνων του αισθητήρα.


% ch5/7 ------------------------------------------------------------------------
Εν κατακλείδι, μέσα από τα πειράματα ξαναπαρατηρήσαμε την ανάγκη για
custom-made παραμετροποίηση ανά περιβάλλον από τις μεθόδους ευθυγράμμισης που
βασίζονται στις αντιστοιχίσεις, και παρατηρήσαμε για πρώτη φορά κάτι που
αναφέρεται στη βιβλιογραφία και αφορά γενικά στην κλάση των ICP αλγορίθμων,
δηλαδή πως η σύγκλισή τους απαιτεί τη σύλληψη των δύο εισόδων από μικρή
απόσταση μεταξύ τους. Το μεγαλύτερο αποτέλεσμα θα έλεγα είναι πως η μέθοδος που
σχεδιάσαμε εμφανίζει σφάλματα στάσης τα οποία είναι συγκρίσιμα σε σχέση με την
καλύτερη μέθοδο της βιβλιογραφίας.

Αυτό που δεν ήταν προηγουμένως δυνατόν και είναι σήμερα είναι η πρώτη μέθοδος
που λύνει το πρόβλημα sm2 για πανοραμικά lidar χωρίς να υπολογίζει
αντιστοιχίσεις ανάμεσα στις σαρώσεις εισόδου.

Το πρόβλημα εδώ βέβαιως είναι ότι εάν θέλαμε να αντικαταστήσουμε τη μέθοδο που
χρησιμοποιήσαμε στην έξοδο του φίλτρου σωματιδίων με τη μέθοδο που
κατασκευάσαμε αυτό δεν θα ήταν δυνατόν γιατί οι λύσεις της δεν παράγονται σε
πραγματικό χρόνο σε σχέση με το ρυθμό παραγωγής εκτιμήσεων από το φίλτρο.  Και
συνεπώς ο επόμενός μας στόχος είναι ακριβώς αυτός.


31.20'


%%%%%%%%%%%%%%%%%%%%%%%%%%%%%%%%%%%%%%%%%%%%%%%%%%%%%%%%%%%%%%%%%%%%%%%%%%%%%%%%
% ch6 / 15min
%%%%%%%%%%%%%%%%%%%%%%%%%%%%%%%%%%%%%%%%%%%%%%%%%%%%%%%%%%%%%%%%%%%%%%%%%%%%%%%%

% ch6/0 ------------------------------------------------------------------------
Για την ακρίβεια εδώ θα ήθελα να θέσω το πρόβλημα το οποίο θα επιχειρήσουμε
να λύσουμε με λεπτομέρεια. Το πρόβλημα είναι η κατασκευή μίας συνάρτησης η
οποία λύνει το πρόβλημα sm2 δεδομένων μίας πανοραμικής σάρωσης δύο διαστάσεων,
του χάρτη του περιβάλλοντος, και μίας εκτίμησης της στάσης του αισθητήρα,
της οποίας η θέση βρίσκεται σε μία γειτονιά της πραγματικής του θέσης,
τέτοια ώστε η εκτίμηση που παράγεται από την συνάρτηση να έχει μικρότερο σφάλμα
εκτίμησης από αυτό της εκτίμησης εισόδου, και που να εκτελείται σε χρόνο που
να συμβαδίζει με ρυθμό παραγωγής εκτιμήσεων από τη μέθοδο που τις παράγει,
δηλαδή για παράδειγμα ένα φίλτρο σωματιδίων.


% ch6/1 ------------------------------------------------------------------------
Αρχικά εδώ αποσυνθέτουμε το πρόβλημα σε δύο διακριτά υποπροβλήματα. Το πρώτο
είναι η εκτίμηση της θέσης του αισθητήρα δεδομένου ότι γνωρίζουμε ποιός είναι ο
προσανατολισμός του, και το δεύτερο είναι το ανάποδο, δηλαδή η εκτίμηση του
προσανατολισμού του δεδομένου ότι γνωρίζουμε ποιά είναι η θέση του. Η υπόθεσή
μου εδώ είναι πως εάν λυσουμε επαναληπτικά και σειριακά αυτά τα δύο προβλήματα
τότε θα καταλήξουμε να λύσουμε το ολικό πρόβλημα. Εδώ η ανάγκη για
επαναληπτικότητα προκύπτει από το γεγονός ότι δεδομένης αυτής της αποσύνθεσης
του προβλήματος, ο πραγματικός προσανατολισμός μπορεί να εκτιμηθεί μόνο από την
πραγματική θέση, και η πραγματική θέση μπορεί να εκτιμηθεί μόνο από τον
πραγματικό προσανατολισμό, αλλά επί της αρχής η στάση και η εκτίμησή της είναι
άνισες ως προς και τις δυο παραμέτρους.


% ch6/2 ------------------------------------------------------------------------
Το πρώτο πρόβλημα έχει μία λύση η οποία μας έρχεται από πρώτες αρχές. Αυτή λέει
πως δεδομένων των παραδοχών του προβλήματος και της επιπρόσθετης παραδοχής ότι
γνωρίζουμε τον προσανατολισμό του αισθητήρα, τότε μπορούμε να εκτιμήσουμε τη
θέση του εάν μετατρέψουμε την εκτίμηση θέσης στο διάνυσμα κατάστασης
ενός συστήματος το οποίο ανανεώνουμε επαναληπτικά με το διάνυσμα u, το οποίο
είναι συνάρτηση της διαφοράς των πρώτων όρων του μετασχηματισμού fourier των
σαρώσεων που συλλαμβάνονται από την πραγματική στάση του αισθητήρα και την
εκτιμώμενη στάση του ανά επανάληψη.

Για την ακρίβεια μπορούμε να αποδείξουμε μέσω των εργαλείων που μας κληροδότησε
ο lyapunov ότι στην ιδανική περίπτωση που δεν υπάρχει πουθενά θόρυβος στο
σύστημα, με αυτόν τον τρόπο μπορούμε να εκτιμήσουμε με αυθαίρετη ακρίβεια την
πραγματική θέση του αισθητήρα, και πως στη γενική περίπτωση η εκτίμηση θέσης
φράσσεται ομοιόμορφα σε μία γειτονιά της πραγματικής θέσης, της οποίας η ακτίνα
εξαρτάται από τα μέτρα των θορύβων που επενεργούν στα δύο είδη σαρώσεων.

% ch6/3 ------------------------------------------------------------------------
Το δεύτερο πρόβλημα, δηλαδή της εκτίμησης του προσανατολισμού του αισθητήρα
δεδομένης της θέσης του, παραμένει αναπάντητο στη βιβλιογραφία της ρομποτικής.
Παρ' όλα αυτά κατάφερα να μεταγγίσω ή να εφεύρω μία τριλογία μεθόδων για τη
λύση αυτού του προβλήματος. Η πρώτη προκύπτει από τη μέθοδο εκτίμησης θέσης που
είδαμε μόλις, εάν αντί για γνωστό προσανατολισμό θεωρήσουμε γνωστή τη θέση του
αισθητήρα.  Ανανεώνοντας την εκτίμηση προσανατολισμού με τη διαφορά των
ορισμάτων των πρώτων όρων του μετασχηματισμού fourier της πραγματικής σάρωσης
και της εικονικής σάρωσης που έχει υπολογιστεί από την εκτίμηση στάσης οδηγεί
σε ένα επίλοιπο σφάλμα προσανατολισμού που εξαρτάται από τον αριθμό των ακτίνων
του αισθητήρα.

Η δεύτερη μέθοδος είναι απευθείας μετασχηματισμός της μεθόδου FMI-SPOMF που
είδαμε στο προηγούμενο κεφάλαιο εάν χρησιμοποιήσουμε εξαρχής πολική
αναπαράσταση για τις σαρώσεις και παρακάμψουμε έτσι την ανάγκη για τη
δημιουργία εικόνων των οποίων η επεξεργασία είναι δαπανηρή σε πόρους.
Ανανεώνοντας την εκτίμηση προσανατολισμού με την ποσότητα ξι επί γάμμα, όπου
γάμμα είναι η γωνιακή ανάλυση του αισθητήρα και όπου βλέπετε F σημαίνει το
μετασχηματισμό fourier της αντίστοιχης σάρωσης, έχει ως αποτέλεσμα ένα επίλοιπο
σφάλμα προσανατολισμού το οποίο εξαρτάται και αυτό από τη αριθμό ακτίνων που
εκπέμπει ο αισθητήρας.

Και το ίδιο ισχύει και για την τρίτη μέθοδο, η οποία μας έρχεται από τα πεδία
της κρυσταλλογραφίας και της ψυχομετρικής, η οποία στη βιβλιογραφία ονομάζεται
η μέθοδος του προκρούστη. Εδώ εάν PR και PV είναι οι προβολές των
δύο σαρώσεων στο καρτεσιανό επίπεδο, και αποσυνθέσουμε το γινόμενό των δύο
πινάκων σε ιδιάζουσες τιμές, μπορούμε να λάβουμε τον πίνακα περιστροφής που
εάν εφαρμοσθεί στον πίνακα των σημείων της εικονικής σάρωσης θα τα μετασχηματίσει
έτσι ώστε να ευθυγραμμιστούν με αυτά της πραγματικής σάρωσης με το ελάχιστο
τετραγωνικό σφάλμα. Επειδή όμως δεν γνωρίζουμε τον προσανατολισμό του
αισθητήρα, πρέπει να περιστρέψουμε τα σημεία της εικονικής σάρωσης όσες φορές
όσος είναι ο αριθμός των ακτίνων του, και να καταγράψουμε τον μέτρο
ευθυγράμμισης που προκύπτει από την αποσύνθεση. Με αυτόν τον τρόπο βρίσκουμε
το μέγιστο μέτρο ευθυγράμμισης, το οποίο και αυτό αντιστοιχεί όπως και για την
προηγούμενη μέθοδο σε μία περιστροφή κατά ακέραιο πολλαπλάσιο της γωνιακής
ανάλυσης του αισθητήρα.

Ο κοινός παρονομαστής και των τριών μεθόδων είναι το μέγιστο μειονέκτημα τους,
δηλαδή ότι σε αντίθεση με τη λύση του προβλήματος της εκτίμησης της θέσης του
αισθητήρα δεν είναι δυνατόν να προσεγγίσουμε με αυθαίρετη ακρίβεια τον
προσανατολισμό του αισθητήρα ακόμα και σε ιδανικές συνθήκες γιατί η ακρίβεια
και των τριών μεθόδων εξαρτάται από τον αριθμό των ακτίνων που εκπέμπει ο
αισθητήρας, ο οποίος είναι σταθερά.

Σαν να μην έφτανε αυτό, η ελάττωση του σφάλματος εκτίμησης προσανατολισμού
είναι κρίσιμης σημασίας γιατί το σφάλμα αυτό μεταφέρεται στη μέθοδο εκτίμησης
θέσης, και το σφάλμα θέσης που προκύπτει εμποδίζει το ίδιο την εκτίμηση του
προσανατολισμού γιατί το ελάχιστο σφάλμα προσανατολισμού προκύπτει για το
ελάχιστο σφάλμα θέσης, και έτσι το σφάλμα στάσης παγιδεύεται σε φαύλο κύκλο.

Η μόνη λύση για την περαιτέρω ελάττωση του σφάλματος προσανατολισμού χωρίς τη
χρήση αντιστοιχίσεων είναι η γωνιακή υπερδειγματοληψία. Προφανώς θα μπορούσαμε
να υπερδειγματοληπτήσουμε και την πραγματική και την εικονική σάρωση, και δεν
θα προέκυπτε πρόβλημα σε γραμμικές περιοχές του περιβάλλοντος,

αλλά σε μη γραμμικές περιοχές η υπερδειγματοληψία της πραγματικής σάρωσης θα
επέφερε τα δικά της σφάλματα.

Αυτό που μπορούμε όμως να κάνουμε χωρίς εισαγωγή σφαλμάτων είναι να
υπερδειγματοληπτήσουμε το χάρτη. Αυτό σημαίνει ότι αν θέλουμε να ελαττώσουμε το
σφάλμα κατά 2 εις την ν φορές, αρκεί να υπολογίσουμε 2 εις την ν εικονικές
σαρώσεις με μέγεθος ίσο με την πραγματική σάρωση, να ευθυγραμμίσουμε την
κάθεμία με την πραγματική σάρωση, και να καταλήξουμε στο ποιά από τις 2 εις την
ν εκτιμήσεις εμφανίζει το μικρότερο σφάλμα.  Οπότε τώρα εμφανίζεται μπροστά μας
άλλο πρόβλημα. Δεδομένου ενός συνόλου εκτιμήσεων στάσης οι οποίες όλες έχουν
την ίδια εκτίμηση θέσης αλλά διαφορετικές εκτιμήσεις προσανατολισμού, ποιά απο
αυτές εμφανίζει το χαμηλότερο σφάλμα προσανατολισμού?

Για να λύσουμε αυτό το πρόβλημα σε γενικές συνθήκες, όπου η εκτίμηση θέσης δεν
είναι ίση με την πραγματική θέση, εφηύραμε τη μετρική CAER, τις οποίας προφίλ
βλέπουμε σε αυτή τη διαφάνεια, η οποία για δεδομένο σφάλμα θέσης εμφανίζει
χαμηλότερες τιμές για χαμηλότερα σφάλματα προσανατολισμού και για δεδομένο
σφάλμα προσανατολισμού χαμηλότερες τιμές για χαμηλότερα σφάλματα θέσης.



% ch6/4 ------------------------------------------------------------------------
Με την εισαγωγή της μετρικής CAER μπορούμε τώρα να συνθέσουμε το τελικό σύστημα
που επιλύει το πρόβλημα sm2 χωρίς αντιστοιχίσεις. Δεδομένων των παραδοχών του
προβλήματος, το σύστημα fsm2 ελαττώνει επαναληπτικά το σφάλμα στάσης με τον
εξης τροπο.  Αρχικά πραγματοποιείται εκτίμηση του προσανατολισμού, η οποία
παράγει 2 εις την ν εκτιμήσεις στάσης με την ίδια εκτίμηση θέσης αλλα
διαφορετικές εκτιμήσεις προσανατολισμού.  Στη συνέχεια κάθε εκτίμηση στάσης
οδευεται στο σύστημα εκτίμησης θέσης για μία επανάληψη, και σε αυτό το στάδιο
παράγονται 2 εις την ν εκτιμήσεις στάσης, όλες με διαφορετικό προσανατολισμό
και θέση. Με αυτόν τον τρόπο εκτιμήσεις που έχουν μεγαλύτερο σφάλμα
προσανατολισμού αποκτούν ακόμα μεγαλύτερο σφάλμα θέσης, και έτσι γίνεται
ευκολότερη η διάκριση της στάσης με το μικρότερο σφάλμα από τη μετρική CAER.
Στη συνέχεια υπολογίζεται η τιμή CAER για όλους τους συνδυασμούς της
πραγματικής σάρωσης με την εικονική σάρωση που προκύπτει από τις εκτιμήσεις,
και η έξοδος του συστήματος εκτίμησης προσανατολισμού θεωρείται εκείνη που
παράγει την ελάχιστη τιμή CAER. Στη συνέχεια αυτή η εκτίμηση στάσης οδεύεται
στο σύστημα εκτίμησης θέσης, και η διαδικασία επαναλαμβάνεται με βάση αυτή την
εκτίμηση εως ότου ικανοποιηθεί μια σειρά συνθηκών σύγκλισης και τερματισμού.


% ch6/5 ------------------------------------------------------------------------
Για να δοκιμάσουμε την επίδοση των τριών εκδόσεων του συστήματος fsm2 ως προς
τους στόχους που έχουμε θέσει χρησιμοποιούμε πέντε benchmark datasets συνολικής
πληθικότητας 45 χιλιάδων περιβαλλόντων.

Για να δοκιμάσουμε τις επιδόσεις τους σε πραγματικές συνθήκες χρησιμοποιούμε
πέντε επίπεδα θορύβου μέτρησης τα οποία εμφανίζουν εμπορικά διαθέσιμοι
αισθητήρες, σε συνδυασμό με δύο επίπεδα διαφθοράς του χάρτη σε σχέση με το
περιβάλλον που αντιπροσωπευεί, και τυχαία σφάλματα θέσης και προσανατολισμού
που έρχονται από τη βιβλιογραφία. Ταυτόχρονα δοκιμάζουμε την επίδοση μεθόδων
της τρέχουσας βιβλιογραφίας, δηλαδή μεθόδων που όλες χρησιμοποιούν
αντιστοιχίσεις για να φέρουν εις πέρας το έργο της ευθυγράμμισης. Συνολικά κάθε
αλγόριθμος έτρεξε για δέκα φορές για κάθε διαμόρφωση, δηλαδή συνολικά κάθε
μέθοδος κλήθηκε περίπου 4.5 εκατομμύρια φορές.  Εδώ χρησιμοποιούμε πανοραμικές
σαρώσεις μεγέθους 360 ακτίνων, και θέτουμε τις τρεις παραμέτρους που
χρησιμοποιεί ο fsm2. Αυτό πάει να πει πως για το κομμάτι της εκτίμησης του
προσανατολισμού υπερδειγματοληπτούμε το χάρτη κατ' ελάχιστον 2 στην δευτέρα
φορές και το μέγιστο κατά 2 στην πέμπτη, και πως κάθε φορά που πραγματοποιείται
εκτίμηση της θέσης, ο αριθμός των επαναλήψεων του υποσυστήματος εκτίμησης είναι
ανάλογος της ακρίβειας προσέγγισης του προσανατολισμού.

Και επιτέλους είμαι στην ευτυχή θέση να σας αναφέρω το κύριο αποτέλεσμα της
εργασίας μου: Αυτό το γράφημα δείχνει τα ποσοστά των περιπτώσεων στις οποίες το
σφάλμα στάσης μειώθηκε ως αποτέλεσμα εφαρμογής της κάθε μεθόδου.  Οι τρεις
εκδόσεις του fsm2 φαίνονται στα δεξιά με χρώματα, και οι μέθοδοι της
βιβλιογραφίας φαίνονται στα αριστερά. Η άνω σειρά περιέχει τα αποτελέσματα της
περίπτωσης όπου ο χάρτης και το περιβάλλον συμπίπτουν ακριβώς, ενώ στην κάτω
σειρά κάθε συντεταγμένη του χάρτη έχει διαφθαρθεί με θόρυβο κανονικά
κατανεμημένο με τυπική απόκλιση 5 εκατοστών. Κάθε ράβδος που βλέπετε αφορά σε
περίπου 450 χιλιάδες ευθυγραμμίσεις για ένα συγκεκριμένο επίπεδο θορύβου
μέτρησης. Αν παρατηρήσουμε καταρχάς την επίδοση των εκδόσεων του ICP θα δούμε
πως γενικά αυτή φθίνει καθώς αυξάνει ο θόρυβος μέτρησης. Οι εκδόσεις του NDT
δεν εμφανίζουν ευθεία αναλογία με το θόρυβο μέτρησης, και ο TEASER είναι σχεδόν
το ίδιο έυρωστος όσο οι εκδόσεις του fsm2, αλλά το πρόβλημα του είναι ότι δεν
μπορεί να εκτελεστεί σε πραγματικό χρόνο. Για την ακρίβεια χρησιμοποιώντας
τετραπλάσιους πόρους ο ελάχιστος χρόνος εκτέλεσής του είναι δεκαπλάσιος από
αυτόν του fm. Εδώ ο x1 εμφανίζει λιγο χαμηλότερα ποσοστά μείωσης από τον
teaser, αλλά οι υπόλοιπες δύο εκδόσεις του fsm2 εμφανίζουν τα μεγαλύτερα
ποσοστά και τα πιό εύρωστα.

Πάμε τώρα στο δεύτερο στόχο. Σε αυτή τη διαφάνεια βλέπουμε τους χρόνους
εκτέλεσης των τριών εκδόσεων του fsm2 για κάθε τιμή θορύβου μέτρησης και
διαφθοράς του χάρτη. Στην άνω σειρά βλέπουμε τα ευθεία αποτελέσματα από την
πειραματική διαδικασία και στην κάτω σειρά τους χρόνους εκτέλεσης που θα είχαν
οι μέθοδοι εάν ο χάρτης αναπαρίστατο ως εικόνα. Η διαφορά ανάμεσα στις δύο
αναπαραστάσεις είναι μεγάλη λόγω του χρόνου υπολογισμού εικονικών σαρώσεων που
είναι η πιό δαπανηρή πράξη, και που στην περίπτωση της αναπαράστασης μέσω
εικόνας μπορεί να γίνει στο ένα τρίτο του χρόνου σε σχέση με τον τρόπου που την
έχω υλοποιήσει εδώ. Τώρα: ο στόχος μας εδώ είναι κάθε μέθοδος να εκτελείται σε
μεγαλύτερη συχνότητα από τη συχνότητα παραγωγής εκτιμήσεων από το σύστημα που
εκτελεί το pose tracking, η οποία δεν έχει ακριβή ορισμό.  Στην πράξη αυτό που
θα θέλαμε είναι η συχνότητα εκτέλεσης να είναι τουλάχιστον 8 με 10 Hz, οπότε
μόνο η μέθοδος x1 εδώ υπολείπεται αυτής της συχνότητας όταν ο χάρτης είναι
διεφθαρμένος. Οι υπόλοιπες δύο μέθοδοι όμως λειτουργούν ψηλότερα από αυτά τα
όρια με μέση τιμή περίπου στα 13 με 20 Hz.

Αυτό λοιπόν που είνα σήμερα δυνατόν που δεν ηταν προηγουμένως είναι η ύπαρξη
μεθόδων που λύνουν το πρόβλημα sm2 χωρίς να χρησιμοποιούν αντιστοιχίσεις,
οι οποίες είναι περισσότερο ακριβείς και εύρωστες από αυτές της βιβλιογραφίας
που εκτελούνται σε πραγματικό χρόνο. Έπειτα οποιαδήποτε από αυτές τις τρεις
μεθόδους είναι ικανή να λύσει και το πρόβλημα της παρατήρησης της στάσης ενός
οχήματος καθώς αυτό κινείται, και το πρόβλημα της εκτίμησης της στάσης του εκ
του μηδενός, χωρίς να χρησιμοποιούνται πουθενά ad hoc μεταβλητές ή features.

Κάπου εδώ θα τελείωνε η έρευνά μου, αλλά στο τέλος της έρευνάς μου είχα τη
διαίσθηση πως το γενικότερο πρόβλημα της ευθυγράμμισης πραγματικών σαρώσεων θα
μπορούσε να λυθεί μέσω της λύσης του προβλήματος της ευθυγράμμισης πραγματικών
με εικονικές σαρώσεις.

44.20'


%%%%%%%%%%%%%%%%%%%%%%%%%%%%%%%%%%%%%%%%%%%%%%%%%%%%%%%%%%%%%%%%%%%%%%%%%%%%%%%%
% ch7 / 9 min
%%%%%%%%%%%%%%%%%%%%%%%%%%%%%%%%%%%%%%%%%%%%%%%%%%%%%%%%%%%%%%%%%%%%%%%%%%%%%%%%

% ch7/0 ------------------------------------------------------------------------
Ας ξανακοιτάξουμε το πρόβλημα sm2 όπως το θέσαμε. Στα δεδομένα του προβλήματος
έχουμε μία πραγματική σάρωση, η οποία συλλαμβάνεται από την πραγματική και
άγνωστη στάση του αισθητήρα, και ύστερα το χάρτη του περιβάλλοντος και μία
αρχική εκτίμηση για τη στάση του αισθητήρα. Από αυτά τα δύο δεδομένα υπολογίζουμε
την πρώτη εικονική σάρωση SV. Από τα προηγούμενα πειράματα είδαμε πως ο
στόχος που θέσαμε, δηλαδή η ελάττωση του σφάλματος στάσης επετεύχθη. Εάν έχουμε
καταφέρει να εκτιμήσουμε με μεγάλη ακρίβεια τη θέση του αισθητήρα, δεδομένου
ακριβώς ότι αρχική εκτίμησή του βρίσκεται σε μία γειτονιά της, τότε χωρίς
βλάβη της γενικότητας μπορούμε να υποθέσουμε ότι όλες οι ενδιάμεσες εκτιμήσεις
θέσεις βρίσκονται και αυτές σε μία γειτονιά της πραγματικής θέσης του, και,
πιο σημαντικό, η αρχική εικονική σάρωση αποτελεί μία τοπική προσέγγιση του
χάρτη M και συνεπώς του περιβάλλοντος στη γειτονιά της πραγματικής στάσης του
αισθητήρα. Οπότε το ερώτημα εδώ είναι: εάν αντικαταστήσουμε τον χάρτη από τον
οποίον υπολογίζουμε εικονικές σαρώσεις με μία δεύτερη πραγματική σάρωση, θα ήταν
εφικτό η ήδη υπάρχουσα μέθοδος να λύσει το γενικότερο πρόβλημα sm?

Σας θυμίζω πως το πρόβλημα του sm υποθέτει πως έχουμε στη διάθεσή μας μόνο
δύο πραγματικές σαρώσεις που έχουν συλληφθεί από δύο διαφορετικές στάσεις, και
ο στόχος της λύσης του είναι η εύρεση του μετασχηματισμού εκείνου που εάν
εφαρμοσθεί στην πρώτη στάση μας δίνει τη δεύτερη.

Εάν όντως είναι εφικτή η λύση του προβλήματος sm μέσω κατάλληλων μετατροπών της
μεθόδου fsm2 που έχουμε φτιάξει, τότε όχι μόνο θα έχουμε δώσει λύση στα δύο
κύρια προβλήματα της εκτίμησης στάσης στη ρομποτική, που είδαμε στα δύο
προηγούμενα κεφάλαια, αλλά θα καταφέρουμε να έχουμε παράξει μία μέθοδο
οδομετρίας μέσω lidar. Και αυτό είναι σημαντικό γιατί η οδομετρία που μας
παρέχεται μέσω των τροχών ή γενικά των άκρων που έρχονται σε επαφή με το δάπεδο
αποκλίνει μέσα στο χρόνο, και εξαρτάται βαρέως από τις συνθήκες τριβής ανάμεσα
τους. Αν η λύση είναι λοιπόν εφικτή μέσω της μεθόδου fsm2 τότε αυτή θα είναι
και η πρώτη μέθοδος στη βιβλιογραφία που λύνει το πρόβλημα του scan matching
χωρίς να υπολογίζει αντιστοιχίσεις ανάμεσα στις σαρώσεις εισόδου.

Στο πρόβλημα του scan matching, η τελική μέθοδος, την οποία θα ονομάσουμε fsm,
έχει ξεκάθαρες απαιτήσεις χρόνου εκτέλεσης. Θα πρέπει να εκτελείται σε χρόνο
μικρότερο από τον ελάχιστο χρόνο ανανέωσης μετρήσεων, ο οποίος για τους
εμπορικά διαθέσιμους αισθητήρες είναι πεντήντα ms. Η πρόκληση εδώ είναι η
μείωση του χρόνου εκτέλεσης της πιο γρήγορης έκδοσης του fsm2, η οποία είναι η
fm, η οποία έχει χρόνο εκτέλεσης 100 ms.  Η μεγάλη πρόκληση όμως εδώ είναι η
ευρωστία της μεθόδου ευθυγράμμισης στις κενές αντιστοιχίσεις. Στην εικόνα εδώ
βλέπουμε πως στο πρόβλημα sm2 ακριβώς επειδή έχουμε τον χάρτη από την τελική
εκτίμηση η εικονική σάρωση έχει την ίδια μορφή με την πραγματική. Στο sm όμως,
εάν μετατρέψουμε την πρώτη σάρωση ως τον χάρτη μέσα στον οποίο θα εφαρμόσουμε
την τεχνική sm2 τότε λόγω του γεγονότος ότι από διαφορετικές θέσεις μέσα σε ένα
περιβάλλον είναι ορατά διαφορετικά σημεία του, στο τέλος είναι δυνατόν η
ευθυγράμμιση να αποτύχει λόγω των γεγονότων ότι οι καμία εικονική σάρωση δεν θα
φτάσει να έχει τη μορφή της πραγματικής και ότι ο fsm2 δεν υπολογίζει
αντιστοιχίσεις ώστε να απορρίψει στους υπολογισμούς του περιοχές ή σημεία της
μίας σάρωσης που δεν αντιστοιχούν στην άλλη.

Για να εξετάσουμε εάν είναι δυνατή η μετατροπή της λύσης που δώσαμε στο
πρόβλημα sm2 σε λύση του προβλήματος scan-matching διεξάγουμε πειράματα στα
οποία οι στόχοι μας είναι δύο: πρώτον θέλουμε να ανακαλύψουμε ποιά είναι η
σχέση των σφαλμάτων των μετασχηματισμών του fsm σε σχέση με αυτά των μεθόδων
της βιβλιογραφίας, και δεύτερον ο fsm θα πρέπει να εκτελείται σε πραγματικό
χρόνο. Εδω χρησιμοποιούμε και πάλι τα ίδια πέντε σύνολα περιβαλλόντων όπως και
πριν, και,

οπως και πριν, δοκιμάζουμε πέντε επίπεδα θορύβου μέτρησης, όπως αυτά ορίζονται
από εμπορικά διαθέσιμους αισθητήρες. Οι δύο σαρώσεις που συλλαμβάνουμε σε κάθε
περιβάλλον διαφέρουν στη θέση και τον προσανατολισμό κατά ποσότητες που
εξάγονται όπως και πριν από ομοιόμορφες κατανομές. Εδώ ακολουθούμε την
τυποποίηση του censi που έφτιαξε τον plicp, και δοκιμάζουμε έξι διαφορετικά
επίπεδα μετατόπισης, ξεκινώντας από μικρές τιμές, δηλαδή όταν ο αισθητήρας
κινείται με χαμηλές ταχύτητες, και προοδευτικά τις αυξάνουμε για να
προσομοιώσουμε όλο και μεγαλύτερες ταχύτητες κίνησης. Κάθε πείραμα επαναλήφθηκε
δέκα φορές σε κάθε περιβάλλον με τυχαίες αρχικές συνθήκες στάσης και
μετατόπισης, με αποτέλεσμα κάθε μέθοδος να ευθυγραμμίζει περίπου 13 εκατομμύρια
ζευγάρια σαρώσεων. Όπως και πριν υποθέτουμε πανοραμικές σαρώσεις 360 ακτίνων,
αλλά σε αντίθεση με τον fsm2 εδώ έχουμε ελαττώσει το ρυθμό γωνιακής
δειγματοληψίας για να ελαττώσουμε αντίστοιχα το χρόνο εκτέλεσης.

Σε αυτή τη διαφάνεια βλέπουμε τις κατανομές των τελικών σφαλμάτων θέσης. Στην
άνω σειρά βρίσκονται τα αποτελέσματα για το χαμηλότερο επίπεδο μετατόπισης
μεταξύ σαρώσεων και στην κάτω αυτά για το υψηλότερο επίπεδο μετατόπισης.Κάθε
κουτί αναφέρεται σε ένα επίπεδο θορύβου μέτρησης. Θα δείτε πως σε χαμηλές
ταχύτητες ο fsm είναι ισοδύναμος και λίγο καλύτερος από την καλύτερη μέθοδο,
που εδώ είναι ο plicp, αλλά κυριαρχεί σε υψηλές ταχύτητες ιδιαίτερα όσο αυξάνει
ο θόρυβος μέτρησης.

Η εικόνα είναι διαφορετική ως προς τα σφάλματα προσανατολισμού, όπου ο fsm
κυριαρχεί σε όλα τα επίπεδα εκτός από εκείνο που αναφέρεται στο χαμηλότερο
επίπεδο μετατόπισης και το χαμηλότερο επίπεδο θορύβου, όπου ο NDT-PSO εμφανίζει
χαμηλότερα σφάλματα.

Όμως αυτή η μέθοδος δεν τρέχει σε πραγματικό χρόνο. Εδώ βλέπουμε τις κατανομές
των χρόνων εκτέλεσης, όπου ο NDT-PSO τρέχει σε 140 ms, ενώ ο fsm τρέχει σε
λιγότερο από 25 ms για κάθε επίπεδο αρχικής μετατόπισης και θορύβου μέτρησης.

Εδώ τώρα θέλω να σας δείξω πώς μεταφράζονται στην πράξη αυτά τα σφάλματα. Εδώ
βλέπουμε με διαφορετικά χρώματα την εκτίμηση της τροχίας ενός πραγματικού
αισθητήρα YDLIDAR από τους αλγορίθμους της πειραματικής διαδικασίας που τρέχουν
σε πραγματικό χρόνο. Εδώ παρατηρούμε πως το σφάλμα θέσης κάθε μεθόδου αυξάνει
στις μεταβάσεις του αισθητήρα απο δωμάτιο σε δωμάτιο ακριβώς λόγω των
καινούριων περιοχών που μπαίνουν στο πεδίο όρασης του αισθητήρα, οι οποίες δεν
υπάρχουν στις προηγούμενες σαρώσεις, και οι οποίες συνεπώς παράγουν κενές
αντιστοιχίες ανάμεσα σε δύο διαδοχικές σαρώσεις.  Ο αισθητήρας περνάει από
πέντε τέτοια σημεία για συνολικά εννιά φορές. Η εκτιμώμενη απόσταση που διένυσε
ο αισθητήρας είναι 43 μέτρα. Οι εκτιμώμενες τροχιές του έχουν ευθυγραμμισθεί με
βάση αυτά τα σημεία επειδή δεν έχουμε πρόσβαση στην πραγματική τροχιά του
αισθητήρα, και από ότι βλέπετε ο fsm έχει τη μικρότερη απόκλιση μέσα στο χρόνο
σε σχέση με τις μεθόδους της τρέχουσας βιβλιγραφίας.

Εν τέλει μπορούμε να πούμε πως ο fsm είναι εύρωστος όχι μόνο στο θόρυβο
μέτρησης, αλλά και σε κενές αντιστοιχίσεις, και συνιστώσες υψηλών συχνοτήτων.
Και πως είναι η πρώτη scan matching μεθοδος για πανοραμικούς αισθητήρες
που λειτουργεί χωρις αντιστοιχίσεις και μόνο με τρεις προαιρετικές παραμέτρους.


52'


\end{document}
