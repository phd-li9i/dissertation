\documentclass[a4paper,10pt]{article}

\usepackage{xltxtra}
\usepackage{gfsdidot}
\usepackage{verse}

\setmainfont[Mapping=tex-text]{GFS Didot}
\setsansfont[Mapping=tex-text]{DejaVu Sans}
\setmonofont[Mapping=tex-text]{DejaVu Sans Mono}
\newfontfamily\athenais{Athenais}
\pagenumbering{gobble}
%%%%%%%%%%%%%%%%%%%%%%%%%%%%%%%%%%%%%%%%%%%%%%%%%%%%%%%%%%%%%%%%%%%%%%%%%%%%%%%%
\begin{document}

Αξιότιμες καθηγήτριες, αξιότιμοι καθηγητές, συνάδελφοι, κυρίες και κύριοι:
ονομάζομαι Αλέξανδρος Φιλοθέου και σήμερα είναι σκοπός μου να σάς εκθέσω τις
συμβολές μου σε λύσεις διαδεδομένων προβλημάτων που αφορούν στο πεδίο της
ρομποτικής κινητής βάσης.

Αποφοίτησα από το τμήμα μας τον Ιούλιο του 2013 και τον Νοέμβριο του ίδιου
έτους έγινα μέλος της ομάδας ρομποτικής PANDORA. Εκεί απέκτησα την πρώτη μου
επαφή με τη ρομποτική και το αντικείμενο μου διέγειρε το ενδιαφέρον τόσο ώστε
αποφάσισα να ειδικευτώ σε αυτό. Τον Σεπτέμβριο του 2015 ξεκίνησα το
μεταπτυχιακό μου πάνω στη ρομποτική και τον έλεγχο στο Βασιλικό Ινστιτούτο
Τεχνολογίας στη Σουηδία, και δύο χρόνια αργότερα επέστρεψα στην Ελλάδα για να
εργαστώ. Τον Σεπτέμβριο του 2018 είχα την τύχη να ξεκινήσω να δουλεύω σε αυτό
που είχα επιλέξει στο πανεπιστήμιό μας υπό την επίβλεψη του κυρίου Συμεωνίδη
και του κυρίου Δημητρίου.

Από τότε η έρευνά μου διαμορφώθηκε με βάση πρακτικά προβλήματα που
συναντήσαμε στα πλαίσια εκτέλεσης των έργων στα οποία εργάστηκα και με βάση
προβλήματα τα οποία εντόπισα στις μεθόδους επίλυσής τους που βρίσκονται στην
τρέχουσα βιβλιογραφία.

Οι συμβολές της διατριβής μου αρθρώνονται σε 5 κεφάλαια.

Το πρώτο κεφάλαιο ξεκινάει με έναν ταπεινό θα έλεγα στόχο, ο οποίος αποβλέπει
στη συμβολή στο έργο των μηχανικών ρομποτικής που ασχολούνται με την αυτόνομη
πλοήγηση επίγειων οχημάτων---για την ακρίβεια ό,τι θα σας παρουσιάσω σήμερα
αφορά σε επίγεια οχήματα. Κατά τη διάρκεια επίλυσης του αυτού του πρώτου
προβλήματος παρατήρησα την ύπαρξη ενός σημαντικού ζητήματος του οποίου η λύση
θα είχε καθοριστική σημασία για τα αποτελέσματα του πρώτου έργου στο οποίο
εργάστηκα, και το οποίο έχει από μόνο του ερευνητικό ενδιαφέρον στα πλαίσια της
ρομποτικής.

Οπότε ξεκίνησα να ερευνώ πώς θα μπορούσαμε να λύσουμε αυτό το πρόβλημα στο
δεύτερο κεφάλαιο της διατριβής.  Σε αυτό το κεφάλαιο κατασκεύασα μία σειρά
λύσεων. Μία από αυτές βασίζεται σε μία βασική μέθοδο της βιβλιογραφίας, και
κατά τη διάρκεια κατασκευής της λύσης μου εντόπισα στη μέθοδο της βιβλιογραφίας
μία σημαντική παθογένεια. Το πρόβλημα που εντόπισα ήταν κρίσιμης σημασίας και
προς έκπληξίν μου δεν ξαναείχε τεθεί στην ερευνητική βιβλιογραφία (, ίσως λόγω
αδράνειας, ή κορεσμού, ή έλλειψης δυνατότητας εφαρμογής σε πραγματικές
συνθήκες). Στο επόμενο λοιπόν κεφάλαιο προσπάθησα να ερευνήσω το κατά πόσο ήταν
δυνατόν να εξαλειφθεί αυτή η παθογένεια, και ποιές θα ήταν οι ελάχιστες
παραδοχές για να γινει αυτό.

Επειδή το πρόβλημα που εγείρει αυτήν την παθογένεια ήταν απαιτητικό, αρχικά
προσπάθησα να ερευνήσω εάν μία λύση είναι εφικτή κατ'αρχήν: έκανα δηλαδή μία
σειρά παραδοχών οι οποίες από τη μία είναι ρεαλιστικές και απο την άλλη οι
λιγότερο απαιτητικές. Στόχος μου ήταν να μου δώσουν τη δυνατότητα να εξετάσω
εάν το πρόβλημα έχει λύση και ταυτόχρονα νόημα στα πλαίσια (γενικότερων
προβλημάτων) της ρομποτικής. Αφού είδα ότι το πρόβλημα αυτό ήταν όντως
επιλύσιμο, αφαίρεσα στο επόμενο κεφάλαιο μία βοηθητική παραδοχή που είχα κάνει,
και στο τελευταίο μία συνθήκη, μέχρις ώτου να μην υπάρχει δυνατότητα να
αφαιρεθεί κάτι παραπάνω χωρίς να καταρρεύσει η κύρια συμβολή της μεθόδου μου.
Έτσι κατέληξα στην κατασκευή της πρώτης μεθόδου η οποία αφαιρεί έναν σημαντικό
περιορισμό από τις μεθόδους της βιβλιογραφίας και συμβάλει στη λύση ενός
θεμελιώδους πρόβληματος της ρομποτικής και γενικότερα της αναγνώρισης προτύπων.


(3min Μέχρι εδω)



Προτού μπω σε περισσοτερο βαθος πρεπει να ξεκινησω απο την αρχη. Η ερευνα μου
ξεκιναει με το εργο RELIEF. Ο σκοπος του εργου ηταν ο εξης. Στην αγορά λιανικών
προϊόντων υπάρχουν εταιρείες που πουλάνε τα προϊόντα τους σε καταστήματα, και
των οποίων το συνολικό απόθεμα αποθηκεύεται σε κεντρικές αποθήκες. Εν γένει
αυτές οι εταιρείες θα ήθελαν να γνωρίζουν ανά πάσα στιγμή το απόθεμά που
βρίσκεται στις αποθήκες τους, όμως αυτό είναι τόσο κοστοβόρο από άποψης πόρων
χρήματος και χρόνου που μπορούν να μετρούν το απόθεμά τους μόνο λίγες φορές
μέσα σε ένα οικονομικό έτος. Ύστερα θα ήθελαν επίσης να γνωρίζουν τις θέσεις
των προϊόντων μέσα σε ένα κατάστημα ή μία αποθήκη τόσο για λόγους γρήγορης
ανάκτησης όσο και για λόγους που τους επιβάλλονται από τρίτα μέρη, γιατί πχ
υπάρχουν συμφωνίες που επιβάλλουν την τοποθέτηση των προϊόντων σε συγκεκριμένες
θέσεις και ύψη μέσα σε ένα κατάστημα. Το έργο RELIEF είχε ως στόχο την
κατασκευή μίας σειράς από αυτόνομα ρομπότ τα οποία θα μπορούσαν να
αποδεσμεύσουν ανθρώπινο δυναμικό από το τετριμμένο έργο της καταγραφής του
αποθέματος και της εκτίμησης της θέσης των εμπορευμάτων μέσω τεχνολογίας RFID,
ώστε αυτές οι ενέργειες να γίνονται ακόμα και σε καθημερινή βάση, με ελάχιστη
εμπλοκή ανθρώπων.

Η πρώτη απαίτηση που τέθηκε για τα επίγεια ρομπότ του έργου ήταν να είναι ικανά
να πλοηγούνται αυτόνομα στο χώρο. Σε πραγματικές συνθήκες μπορείτε να
φανταστείτε πως προτού κλείσει η αποθήκη το βράδυ, ή ακόμα και κατά τη διάρκεια
μίας εργάσιμης μέρας, στο ρομπότ δίνεται η εντολή να περάσει από συγκεκριμένα
σημεία του χώρου ώστε να σαρώσει όλα τα ράφια στα οποία υπάρχουν αντικείμενα.
Η αυτονομία της πλοήγησης αφαιρεί την απαίτηση για εξωτερικό εξοπλισμό πάνω
στον οποίο θα μπορούσε να οδηγηθεί το ρομπότ ενώ ταυτόχρονα το κάνει ικανό να
μπορεί να εκτελεί τις ενέργειές του απρόσκοπτα ενώ γύρω του υπάρχουν κινούμενα
εμπόδια όπως άνθρωποι ή μηχανήματα.


%%%%%%%%%%%%%%%%%%%%%%%%%%%%%%%%%%%%%%%%%%%%%%%%%%%%%%%%%%%%%%%%%%%%%%%%%%%%%%%%
% ch3
%%%%%%%%%%%%%%%%%%%%%%%%%%%%%%%%%%%%%%%%%%%%%%%%%%%%%%%%%%%%%%%%%%%%%%%%%%%%%%%%
% ch3/0 ------------------------------------------------------------------------
Η αυτόνομη πλοήγηση υποθέτει 5 προαπαιτούμενα. Πρέπει να υπάρχει ο χάρτης του
περιβάλλοντος στο οποίο πλοηγείται το ρομπότ, τουλάχιστον ένας εξωδεκτικός
αισθητήρας, μία μέθοδος εκτίμησης της στάσης του ρομπότ στο σύστημα
συντεταγμένων του χάρτη, μία αρχική στάση και μία τελική στάση. Ως στάση του
ρομπότ ορίζουμε το διάνυσμα κατάστασης που συνιστάται από τη θέση και τον
προσανατολισμό του ρομπότ ως προς το σύστημα αναφοράς του χάρτη.

% ch3/1 ------------------------------------------------------------------------
Δεδομένων αυτών η αυτόνομη πλοήγηση συνίσταται σε δύο μέρη: αφενός χρειάζεται
ένας αλγόριθμος που δέχεται τον χάρτη, και την αρχική και τελική στάση και
παράγει ένα μονοπάτι το οποίο συνδέει την αρχική με την τελική στάση χωρίς να
τέμνει εμπόδια του χάρτη, και αφετέρου ένας ελεγκτής κίνησης, ο οποίος,
δεδομένου του μονοπατιού, της στιγμιαίας εκτίμησης για τη στάση του ρομπότ, και
μετρήσεις από αισθητήρες παράγει εντολές κίνησης τις οποίες λαμβάνουν ως είσοδο
οι κινητήρες του ρομπότ ώστε αυτό να κινείται πάνω στο μονοπάτι που παρήγαγε ο
πρώτος αλγόριθμος. Ο αλγόριθμος κατασκευής μονοπατιών ονομάζεται στην ορολογία
global planner, και ο ελεγκτής κίνησης local planner.

% ch3/2 ------------------------------------------------------------------------
Για να υλοποιήσω το συνολικό σύστημα αυτόνομης πλοήγησης στα πλαίσια του έργου
αρχικά στράφηκα στη διαθέσιμη λογισμικογραφία, όπου ανακάλυψα πως υπήρχαν
πολλαπλοί αλγόριθμοι global και local planners. Πουθενα ομως στη διαθεσιμη
βιβλιογραφια δεν υπηρχε συγκριτικη αναλυση της συνδυαστικής επιδοσης τους στο
έργο της αυτόνομης πλοήγησης ώστε να επιλέξω ποιός συνδυασμός θα εκπλήρωνε τους
στόχους μας στα πλαίσια του έργου. Συνεπώς αποφασίσαμε πως θα ήταν επωφελές
τόσο για το έργο όσο και για άλλους μηχανικούς ρομποτικής να σχεδιάσουμε μία
μέθοδο αξιολόγησης των διαθέσιμων αλγορίθμων global και local planners και των
συνδυασμών τους, την οποία θα χρησιμοποιούσαμε μέσω πειραματικής διαδικασίας
ώστε να καταλήξουμε σε συμπεράσματα για την επίδοση των τρέχοντων διαθέσιμων
αλγορίθμων που υλοποιούν αυτόνομη πλοήγηση σε επίγειες κινητές ρομποτικές
βάσεις.

% ch3/3 ------------------------------------------------------------------------
Αρχικά συγκεντρώσαμε όλα τα διαθέσιμα πακέτα λογισμικού και τα υποβάλλαμε σε
αξιολόγηση με βάση ποιοτικά κριτήρια, δηλαδή τα ίδια κριτήρια που θα
χρησιμοποιούσε ένας μηχανικός λογισμικού προτού φτάσει στο σημείο
να εξετάσει εάν πρακτικά η επίδοση του πακέτου λογισμικού είναι επαρκής.
Στα αριστερά βρίσκονται τα ονόματα των global και local
planners και πάνω στην οριζόντια γραμμή οι συντομογραφίες των ποιοτικών
κριτηρίων, όπως εάν το πακέτο διαθέτει τεκμηρίωση ή είναι σκέτος κώδικας, εάν
είναι ενημερωμένο, δηλαδή εάν μπορεί να τρέξει στο πιο σύγχρονο λειτουργικό και
ειναι ενσωματώσιμο, εάν είναι αυτοτελές, παραμετροποιήσιμο, πόσο συνεπές
είναι στην εκτέλεσή του, και τι ανάγκες έχει σε υπολογιστικούς πόρους. Στη
δεξιά στήλη σημειώνονται με κενές κουκίδες τα πακέτα εκείνα τα οποία
αποτυγχάνουν σε έστω και ένα κριτήριο. Στο τέλος αυτού του φιλτραρίσματος
απέμειναν τρεις αλγορίθμοι χάραξης μονοπατιών και τρεις ελεγκτές κίνησης,
οπότε η πειραματική διαδικασία πραγματοποιήθηκε σε συνολικά εννέα συνδυασμούς
αλγορίθμων.

% ch3/4 ------------------------------------------------------------------------
Η αξιολόγηση κάθε συνδυασμού έγινε σε δύο περιβάλλοντα προσομοίωσης και σε ένα
πραγματικό περιβάλλον. Και στις τρεις περιπτώσεις θέσαμε μία αρχική και μία
τελική στάση και ζητήσαμε από κάθε συνδυασμό global και local planners να
πλοηγηθεί αυτόνομα από την αρχική στην τελική στάση με βάση το μονοπάτι που θα
σχεδίαζε ο global planner και τις εντολές κίνησης του local planner. Κάθε
συνδυασμός απο planners επανέλαβε το πείραμα 10 φορές σε κάθε περιβάλλον, και
για κάθε πείραμα καταγράφηκε ένας αριθμός από μετρικές προκειμένου να γίνει η
αξιολόγηση κάθε συνδυασμού και να υπάρχει ένα κοινό σύστημα κρίσης για όλους
ώστε στο τέλος να προκύψει μία ιεραρχία συνδυασμών.

% ch3/5 ------------------------------------------------------------------------
Οι μετρικές αυτές είναι τριών ειδών: αφορούν είτε αποκλειστικά στους global
planners, όπως για παράδειγμα το μήκος των σχεδιασθέντων μονοπατιών ή η μέση
ελάχιστη απόσταση ενός μονοπατιού από εμπόδια (8 μετρικές), είτε αποκλειστικα
στους local planners (8 μετρικές), όπως για παράδειγμα ο αριθμός αποτυχιών
έυρεσης ταχυτήτων προς τον συνολικό αριθμό κλήσεων του ελεγκτή, ή αποκλειστικά
στο συνδυασμό τους, όπως για παράδειγμα ο χρόνος πλοήγησης, ή πραγματική ολικά
ελάχιστη απόσταση από εμπόδια κατά τη διάρκεια μίας πλοήγησης (12 μετρικές).

% ch3/6 ------------------------------------------------------------------------
Με βάση τις τιμές των μετρικών που έχουμε καταγράψει, αυτό που θέλουμε στο
τέλος είναι να μπορέσουμε να αποδώσουμε μία τιμή σε κάθε συνδυασμό από
planners.  Τα προβλήματα εδώ είναι δύο: πώς θα δώσουμε μία τιμή όταν οι
μετρικές εκφράζονται σε διαφορετικές μονάδες μέτρησης, και πώς θα
κατασκευάσουμε μία συνάρτηση απόδοσης αξίας σε κάθε συνδυασμό που να είναι
γνησίως αύξουσα όταν κάποιες μετρικές συνεισφέρουν με τρόπο ανάλογο καθώς
αυξάνονται (όπως η μέση απόσταση από εμπόδια, όσο μεγαλύτερη η απόσταση από
εμπόδια τόσο ασφαλέστερη είναι η πλοήγηση) ενώ άλλες με τρόπο αντιστρόφως
ανάλογο (όπως ο χρόνος πλοήγησης).

% ch3/7 ------------------------------------------------------------------------
Για να αποκτήσουμε ένα κοινό σύστημα αναφοράς αρχικά κανονικοποιούμε τις τιμές
των μετρικών μέσω της συνάρτησης N, δηλαδή εξετάζουμε την τιμή της
μετρικής m για έναν συνδυασμό και τις ακραίες τιμές της για κάθε συνδυασμό,
ώστε στο τέλος η τιμή της εκφράζεται στο διάστημα [0,1]. Κατασκευάζουμε την
συνάρτηση απόδοσης αξίας για ένα περιβάλλον μέσω της συνάρτησης V. Εδώ $I_Q$ και
$I_\overline{Q}$ είναι η συνάρτηση δείκτης. Δηλαδή η πρώτη ισούται με 1 για
μετρικές m που συνεισφέρουν ανάλογικά στην τιμή της V και 0 για άλλες, και
η δεύτερη το αντίθετο.  Η συνάρτηση I(C,m) είναι μηδεν όταν ο συνδυασμός C
δεν κατάφερε να πλοηγήσει το ρομπότ και η m ειναι μετρική που αφορά στον
συνδυασμό planners, αλλιώς είναι ένα.  Αυτό το κάνουμε έτσι ώστε να μπορούμε με
μία συνάρτηση να συνεκτιμήσουμε την αξία των συνιστωσών ενός συνδυασμού ακόμα
και όταν η πλοήγηση ήταν αποτυχημένη.

% ch3/8 ------------------------------------------------------------------------
Μέσω της V και των τιμών όλων των μετρικών που έχουμε καταγράψει για κάθε
πείραμα στο τέλος λαμβάνουμε τα αποτελέσματα του πίνακα, τα οποία εμφανίζουν
την αξία κάθε συνδυασμού για κάθε περιβάλλον, και τη συνολική αξία κάθε
συνδυασμού με βάση τα αποτελέσματα κάθε περιβάλλοντος. Η εκτέλεση πειραμάτων σε
διαφορετικά και διαφορετικής δυσκολίας περιβάλλοντα και με διαφορετικούς
αισθητήρες δινει τη δυνατοτητα να εμφανιστει το υποκειμενο μοτιβο της ιεραρχίας
αναμεσα σε global και local planners, αυτό δηλαδή που επιζητησε η ερευνά μας.

% ch3/9 ------------------------------------------------------------------------
Τί είναι σήμερα δυνατον που δεν ήταν πριν αυτή τη μελέτη:
καταρχάς ένας μηχανικός ρομποτικής που έχει αντικείμενο την αυτόνομη πλοήγηση
μπορεί να επιλέξει αμέσως τον συνδυασμό αλγορίθμων που καθιστούν εφικτή
την πλοήγηση χωρίς την ανάγκη για εξονυχιστικά τεστ, και ακόμα να προσδιορίσει
τα δικά του κριτήρια με βάση διαφορετικές απαιτήσεις δίνοντας διαφορετικά βάρη
στη συνεισφορά της κάθε μετρικής. Έπειτα, με το βλέμμα στο μέλλον, υπάρχει
πλέον το υπόβαθρο για την αξιολόγηση οποιοασδήποτε νέας μεθόδου που είτε
χαράσσει μονοπάτια, είτε είναι ελεγκτής κίνησης, και σε οποιοδήποτε
συγκεκριμένο περιβάλλον.

% ch3/10 -----------------------------------------------------------------------
Κατά τη διάρκεια διεξαγωγής της πειραματικής διαδικασίας αυτό που παρατήρησα
ήταν ότι η εκτίμηση της στάσης εμφάνιζε σφάλματα παρατηρήσιμα ακόμα και με
γυμνό μάτι. Σε αυτή τη διαφάνεια βλέπουμε την εξέλιξη του μέσου σφάλματος
εκτίμησης στάσης στα δύο προσομοιωμένα περιβάλλοντα, το οποίο δεν εμφανίζει
σταθερό μέτρο είτε στην πορεία του χρόνου είτε ανά περιβάλλον.  Το μέτρο του
σφάλματος εδώ μετριέται σε εκατοστά γιατί η στάση του ρομπότ υπόκειται σε
εκτίμηση και όχι μέτρηση, δηλαδή υποθέτουμε ότι στο περιβάλλον που κινείται το
ρομπότ δεν υπάρχει μετρητικός εξοπλισμός, αλλά το ρομπότ εκτίμα το ίδιο τη
στάση του μέσω των αισθητήρων του και του χάρτη του περιβάλλοντός του. Σε
γενικές γραμμές το φαινόμενο του σφάλματος είναι φυσιολογικό λόγω της χρήσης
παρατηρητή για την εκτίμηση της στάσης, όμως στην περίπτωση του έργου RELIEF
αυτό το σφάλμα κληροδοτείται στο σφαλμα θέσης των προϊόντων, καθώς η εκτίμηση
της θέσης ενός προϊόντος προκύπτει μέσω της εκτίμησης της στάσης των κεραιών
που φέρει το ρομπότ, οι οποίες με τη σειρά τους είναι στατικοί μετασχηματισμοί
της εκτίμησης της στάσης του ρομπότ.  Έτσι περνάμε στο επόμενο πρόβλημα, το
οποίο είναι το πρόβλημα της ελάττωσης του σφάλματος εκτίμησης στάσης
παρατηρητών, όπου συγκεκριμένα εστιάζουμε σε φίλτρα σωματιδίων που
χρησιμοποιούν αισθητήρες lidar.




%%%%%%%%%%%%%%%%%%%%%%%%%%%%%%%%%%%%%%%%%%%%%%%%%%%%%%%%%%%%%%%%%%%%%%%%%%%%%%%%
% ch4
%%%%%%%%%%%%%%%%%%%%%%%%%%%%%%%%%%%%%%%%%%%%%%%%%%%%%%%%%%%%%%%%%%%%%%%%%%%%%%%%

% ch4/0 ------------------------------------------------------------------------
Και συγκεκριμένα χρησιμοποιούμε το φίλτρο σωματιδίων διότι είναι περισσότερο
ευέλικτο και επιδεκτικό στη βελτίωση. Ερευνώντας τη διαθέσιμη βιβλιογραφία πάνω
στο ζήτημα προέκυψαν δύο μέθοδοι βελτίωσης, δηλαδή της προσθετικής
ευθυγράμμισης πραγματικών σαρώσεων με εικονικές σαρώσεις, και της
ανατροφοδότησης του αποτελέσμάτος στον πληθυσμό του φίλτρου. Η δική μου συμβολή
σε αυτό το κεφάλαιο αφορά στη δεύτερη μέθοδο, ενώ εισάγω και μία δεύτερη, αυτήν
της διαλογής σωματιδίων.

% ch4/1 ------------------------------------------------------------------------
Όσο αφορά στην εκτίμηση της στάσης ενός οχήματος, όλοι οι πιθανοτικοί
παρατηρητές στη ρομποτική εκτιμούν τη στάση του, δηλαδή το διάνυσμα της
θεσης και του προσανατολισμού του ως προς ένα σύστημα αναφοράς, αναδρομικα,
με βάση το μοντέλο παρατήρησης του αισθητήρα που φέρει το όχημα, και του
κινηματικού μοντέλου του. Το φίλτρο σωματιδίων ειδικά, σε αντίθεση με το φίλτρο
καλμαν, χρησιμοποιεί πολλαπλές υποθέσεις στάσης, και η τελική του εκτίμηση
προκύπτει ως ο μέσος όρος των στάσεων αυτών των υποθέσεων, βεβαρυμμένος
κατά το βάρος της κάθε μίας. Ως βάρος εδώ νοείται η πιθανότητα παρατήρησης
μίας δεδομένης μέτρησης από τη στάση της κάθε υπόθεσης, και συνεπώς υποθέσεις
των οποίων το σφάλμα εκτίμησης είναι μικρότερο από άλλες εμφανίζουν μεγαλύτερο
βάρος, και συνεπώς επηρεάζουν περισσότερο την τελική εκτίμηση.

% ch4/2 ------------------------------------------------------------------------
Με βάση αυτον τον συλλογισμό τυποποιούμε την πρώτη υπόθεση, η οποία λέει ότι
εάν για την εξαγωγή της εκτίμησης του φίλτρου χρησιμοποιηθεί ένα υποσύνολο των
πιο βαρέων σωματιδίων αντί για όλο τον πληθυσμό, τότε αναμένουμε η εκτίμηση
του φίλτρου να έχει χαμηλότερο σφάλμα εκτίμησης.


% ch4/3 ------------------------------------------------------------------------
Ο δέυτερος τρόπος με τον οποίον μπορεί να ελαττωθεί το σφάλμα ενός
παρατηρητή---και αυτό δεν αφορά μόνο τα φίλτρα σωματιδίων---είναι μέσω
ευθυγράμμισης πραγματικών με εικονικές σαρώσεις αισθητήρα lidar. Αυτή η μέθοδος
θα μας ακολουθήσει σε όλη την υπόλοιπη έρευνά μου, και για λόγους οικονομίας θα
αναφέρομαι σε αυτήν συντομογραφικά ως sm2. Η τεχνική sm2 είναι παρακλάδι της
γενικής μεθόδου ευθυγράμμισης σαρώσεων, και για να περάσω σε λεπτομέρειες
πρέπει πρώτα αναφερθώ στο αντικείμενο αυτών των μεθόδων, δηλαδή τί εστί μία
μέτρηση αισθητήρα lidar.

Πρώτα από όλα, ο αισθητήρας lidar είναι το κύριο μέσο με το οποίο σήμερα τα
ρομπότ εκτιμούν τη στάση τους και πλοηγούνται στο χώρο. Ένας τέτοιος αισθητήρας
μετράει την απόσταση ανάμεσα στον αισθητήρα και αντικείμενα εντός του πεδίου
όρασής του με μία συχνότητα το πολύ 20Hz.  Σε αυτή τη διαφάνεια φαίνεται στα
αριστερά ένα περιβάλλον μέσα στο οποίο τοποθετείται ένας αισθητήρας lidar που
συλλαμβάνει δισδιάστατες μετρήσεις, όπου με μπλε χρώμα απεικονίζονται οι
διακριτές ακτίνες που εκπέμπει σε ένα γωνιακό πεδίο όρασης 270 μοιρών. Στη μέση
απεικονίζεται η μέτρηση στο τοπικό σύστημα αναφοράς του αισθητήρα, το οποίο
βλέπουμε στα δεξιά σε μεγαλύτερη λεπτομέρεια. Μία μέτρηση του αισθητήρα
ονομάζεται αλλιώς σάρωση και αποτελείται από μία ακολουθία ζευγών όπου κάθε
ζεύγος αποτελείται από μία μέτρηση απόστασης και τη γωνία στην οποία αναφέρεται
στο τοπικό σύστημα αναφοράς.

Η ευθυγράμμιση δύο πραγματικών σαρώσεων αισθητήρα lidar στοχεύει στην εύρεση
εκείνου του μετασχηματισμού που όταν εφαρμοσθεί στα σημεία της πρώτης σάρωσης
θα τα κάνει να συμπέσουν στα σημεία της δεύτερης με το ελάχιστο σφάλμα. Το
πρόβλημα της ευθυγράμμισης δεν έχει μόνο θεωρητικό ενδιαφέρον, γιατι ο τελικός
μετασχηματισμός ανάμεσα στα σημεία της δεύτερης σάρωσης σε σχέση με αυτά της
πρώτης είναι ο ίδιος που εκφράζει τη στάση από την οποία συνελήφθη η δεύτερη
σάρωση στο σύστημα αναφοράς που ορίζει η πρώτη, και συνεπώς η ευθυγράμμιση
μετρήσεων αποτελεί έναν διαφορικό παρατηρητή της στάσης του ρομπότ. Για αυτό το
λόγο η ευθυγράμμιση σαρώσεων είναι θεμελιώδες κομμάτι της ρομποτικής, και
χρησιμοποιείται κατά κόρον στη λύση του προβλήματος της ταυτόχρονης κατασκευής
χάρτη και εύρεσης της στάσης ενός ρομπότ, και ως επιπρόσθετος και πιο
αξιόπιστος τρόπος προσδιορισμού της οδομετρίας ενός οχήματος.  Στα
αριστερά απεικονίζεται ένα περιβάλλον, το ρομπότ σε δύο διαφορετικές στάσεις,
και οι σαρώσεις τις οποίες συνέλαβε από αυτές. Στη μέση βλέπουμε τη διαδικασία
ευθυγράμμισης των δύο αυτών σαρώσεων, και στα δεξιά σε εστίαση την εξέλιξη της
εκτίμησης της δεύτερης στάσης του ρομπότ.

Με λίγα λόγια εάν διαθέτουμε δύο σαρώσεις και γνωρίζουμε τη στάση από την οποία
συνελήφθη μία από τις δύο: η ευθυγράμμιση σαρώσεων μπορεί να εκτιμήσει την
άγνωστη στάση από την οποία συνελήφθη η άλλη. Σε αυτό το γεγονός κρύβεται μία
δεύτερη χρησιμότητα της ευθυγράμμισης σαρώσεων. Ακριβώς επειδή οι παρατηρητές
που βασίζονται σε μετρήσεις αισθητήρα lidar δίνουν ως αποτέλεσμα τη στιγμιαία
εκτίμηση στάσης με βάση το χάρτη του περιβάλλοντος στο οποίο κινείται το
ρομπότ, και μία στιγμιαία μέτρηση από το lidar, οι ίδιοι διαθέτουν τα συστατικά
για τον υπολογισμό της άγνωστης και εκτιμητέας στάσης του ρομπότ μέσω της
ευθυγράμμισης μετρήσεων. Πώς? Αρκεί να αντικαταστήσουμε τη μία από τις δύο
μετρήσεις με μία εικονική σάρωση, η οποία προσομοιώνει την αρχή λειτουργίας του
lidar αυτή τη φορά στο χάρτη αντί για το περιβάλλον, από την εκτίμηση της
στάσης του ρομπότ. Ευθυγραμμίζοντας την πραγματική σάρωση που προέρχεται
από τον την πραγματική στάση του αισθητήρα με μία εικονική σάρωση που έρχεται
από το χάρτη και υπολογίζεται από την τρέχουσα εκτίμηση της στάσης του ρομπότ
μπορούμε να υπολογίσουμε το μετασχηματισμό ανάμεσα στην εκτίμηση και την
άγνωστη πραγματική στάση του ρομπότ, και εφόσον γνωρίζουμε την εκτίμηση,
μπορούμε να υπολογίσουμε την πραγματική του στάση.

Με αυτόν τον τρόπο μπορούμε να προσαρμόσουμε τη μέθοδο sm2 στην έξοδο του
φίλτρου σωματιδίων, όπου λόγω θορύβου μέτρησης και ατελούς αναπαράστασης του
χάρτη σε σχέση με το περιβάλλον, η τελική έξοδος της μεθόδου sm2 φέρει και αυτή
σφάλμα, και συνεπώς αποτελεί μία δεύτερη εκτίμηση. Αυτή η δεύτερη εκτίμηση είναι
μία υπόθεση για την οποία το ίδιο το φίλτρο δεν έχει γνώση, και συνεπώς θα ήταν
ωφέλιμο, εάν η υπόθεση φέρει όντως μικρότερο σφάλμα, να εισαχθεί στον πληθυσμό
του, ώστε να το σφάλμα του ίδιου το φίλτρου να μειωθεί.

Μέχρι εδώ σάς έχω παρουσιάσει το state-of-the-art και δεν έχω προσθέσει τίποτα
δικό μου. Η δική μου συμβολή έχει να κάνει με αυτή την ανάδραση του
αποτελέσματος της ευθυγράμμισης. Κατά μία μέθοδο, η υπόθεση που παράγεται μέσω
sm2 εισάγεται στον πληθυσμό του φίλτρου ως ένα διακριτό σωματίδιο, το οποίο σε
έναν πληθυσμό αρκετών εκατοντάδων σωματιδίων έχει μεν επίδραση, αλλά σχεδόν
αμελητέα, και κατά μία άλλη μέθοδο αντικαθιστά τον πληθυσμό του φίλτρου στο
σύνολό του.  Η επιτυχία αυτής της μεθόδου εξαρτάται αποκλειστικά από την
επιτυχία της ευθυγράμμισης, η οποία δεν είναι εγγυημένη, και μπορεί να έχει ως
αποτέλεσμα την καταστροφική αποτυχία της συνέχειας της εκτίμησης της στάσης του
ρομπότ, και συνεπώς και της πλοήγησης και της ασφάλειας της πλοήγησης η οποία
βασίζεται στην εκτίμηση της στάσης του οχήματος. Για την ελάττωση λοιπόν του
σφάλματος σκέφτηκα πως εάν το αποτέλεσμα της ευθυγράμμισης εισαχθεί στον
πληθυσμό του φίλτρου ως πολλαπλά σωματίδια αλλά σε πληθικότητα μικρότερη από
τον πληθυσμό του φίλτρου, τότε αυτό θα έχει θεωρητικά ως αποτέλεσμα πιο γρήγορη
σύγκλιση σε σχέση με την περίπτωση που το εισάγαμε ως ένα σωματίδιο, χωρίς
ταυτόχρονα να διακινδυνεύεται σε περίπτωση αποτυχίας της ευθυγράμμισης η
ανθεκτικότητα του φίλτρου, γιατί το ίδιο θα συνεχίσει να περιέχει υποθέσεις που
εξηγούν καλύτερα τις μετρήσεις από αυτές που έχουν καταλήξει σε λανθασμένες
τοποθεσίες, και συνεπώς θα αποδίδει μεγαλύτερο βάρος στις πρώτες.

Με βάση αυτά συντάσσουμε άλλες δύο υποθέσεις. Η πρώτη είναι ότι το αποτέλεσμα
της ευθυγράμμισης της σάρωσης από το lidar και της σάρωσης εντός του χάρτη από
την εκτιμώμενη στάση του φίλτρου έχει μικρότερο σφάλμα από την εκτίμηση του
φίλτρου.  Η δεύτερη αφορά στην ανάδραση αυτής της εκτίμησης, και υποθέτει ότι
εάν το αποτέλεσμα της ευθυγράμμισης εισαχθεί στον πληθυσμό του φίλτρου ως
μία πλειάδα σωματιδίων, των οποίων το τελικό μέγεθος είναι μεγαλύτερο του ενός
εκατοστού του, και μικρότερο από το μέγιστο, τότε, εάν στέκει η προηγούμενη
υπόθεση, τότε το σφάλμα εκτίμησης του τελίκού πληθυσμού θα είναι χαμηλότερο
από τον σφάλμα εκτίμησης του κανονικού φίλτρου, χαμηλότερο από το σφάλμα
εκτίμησης του φίλτρου εάν η υπόθεση εισάγετο ως μόνο ένα σωματίδιο, και
ο πληθυσμός του φίλτρου είναι πιο ανθεκτικός σε σχέση με τον πληθυσμό του
φίλτρου εκείνου που σχηματίζεται από την αρχικοποίηση του φίλτρου γύρω από το
αποτέλεσμα της ευθυγράμμισης.


% ch4/4 ------------------------------------------------------------------------
Για να δοκιμάσουμε την πρακτική ισχύ των τριών θεωρητικών υποθέσεων
χρησιμοποιούμε δύο περιβάλλοντα, ένα απλό το οποίο ονομάζεται corridor, και ένα
που ομοιάζει σε αποθήκη, δηλαδή το αναμενόμενο περιβάλλον των ρομπότ που
κατασκευάσαμε για το έργο RELIEF. Για κάθε περίπτωση διαλογής σωματιδίων και
μεθόδου ανάδρασης από το ρομπότ ζητήθηκε να πλοηγηθεί αυτόνομα από μία αρχική
σε μία τελική στάση 100 διαφορετικές φορές. Το ρομπότ φέρει έναν αισθητήρα
lidar γωνιακού εύρους 260 μοιρών, 640 ακτίνων, με θόρυβο μέτρησης κανονικά
κατανεμημένο με τυπική απόκλιση ενός εκατοστού. Ο πληθυσμός του φίλτρου
είναι κυμαινόμενος με ελάχιστη πληθικότητα 200 σωματίδια, και μέγιστη 500.

Σε αυτή τη διαφάνεια βλεπουμε την κατανομές των μέσων σφαλμάτων εκτίμησης ανά
διαδρομή για 100 επαναλήψεις σε κάθε περιβάλλον, ανά μέθοδο διαλογής. Με τη
συντομογραφία 100% εννοούμε την κατάσταση στην οποία όλα τα σωματίδια
επιλέγονται να ψηφίσουν κατά βάρος για την εκτίμηση του φίλτρου, η οποία είναι
η ονομαστική κατάσταση του φίλτρου. Με μεγαλύτερο από μέσο W εννοούμε την
κατάσταση που μονο εκείνα τα σωματίδια των οποίων το βάρος είναι μεγαλύτερο από
το μέσο βάρος του πληθυσμού επιλέγονται να ψηφίσουν, με 10% την κατάσταση που
μόνο το 10% των βαρύτερων σωματιδίων ψηφίζουν, και με τοπ συμβολίζουμε το
σωματίδιο που φέρει το υψηλότερο βάρος του πληθυσμού. Με βάση τα πειραματικά
δεδομένα βλέπουμε πως η διαλογή των σωματιδίων με βάρος μεγαλύτερο από το μέσο
βάρος του συνολικού πληθυσμού εμφανίζει μικρότερο σφάλμα από την ονομαστική
κατάσταση, και πως το σφάλμα μειώνεται ακόμα περισσότερο όταν επιλέγουμε το τοπ
10% των σωματιδίων. Το αντιδιαισθητικό σε αυτά τα πειράματα είναι πως το
σωματίδιο που εμφανίζει το μεγαλύτερο βάρος, δηλαδή αυτό που εξηγεί την
τρέχουσα μέτρηση στο μεγαλύτερο βαθμό με βάση το μοντέλο παρατήρησης εμφανίζει
το μεγαλύτερο σφάλμα ανάμεσα σε όλες τις διαμορφώσεις. Το συμπέρασμα που αντλούμε
από αυτά τα αποτελέσματα είναι ότι ναι μεν επιβεβαιώνεται η υπόθεσή μας, αλλά
η βελτίωση είναι μικρή, και έχει ένα οριακό σημείο ως προς τον αριθμό των
πιο βαρέων σωματιδίων που επιλέγονται.

Σε αυτή τη διαφάνεια με κόκκινο φαίνονται τα σφάλματα του αποτελέσματος της
μεθόδου sm2 για κάθε μέθοδο διαλογής, τα οποία είναι κατά μέσο όρο στο σύνολό
τους χαμηλότερα από εκείνα των εκτιμήσεων του φίλτρου, τα οποία επιβεβαιώνουν
την υπόθεσή μας, και σε αυτή τη διαφάνεια βλέπουμε τα μέσα σφάλματα ανά μέθοδο
ανάδρασης. Εδώ με open συμβολίζουμε την open-loop κατάσταση, δηλαδή και πάλι
την ονομαστική κατάσταση του φίλτρου, με soft-1 τη διαμόρφωση όπου το
αποτέλεσμα του sm2 εισάγεται στον πληθυσμό του φίλτρου με τη μορφή ενός μόνο
σωματιδίου, με soft-50 τη διαμόρφωση όπου το αποτέλεσμα του sm2 εισάγεται στον
πληθυσμό με τη μορφή τόσων σωματιδίων όσών να αποτελούν το 50% του τελικού
πληθυσμού, και hard τη διαμόρφωση όπου το φίλτρο αρχικοποιείται εκ του μηδενός
γύρω από το αποτέλεσμα του sm2. Εδώ βλέπουμε πως γενικά η μέθοδος που εισάγαμε
εμφανίζει κατά μέσο όρο τα μικρότερα σφάλματα, και πως είναι πιο ανθεκτική από
τη μέθοδο hard, η οποία δεν έχει τρόπο να προστατεύσει τη συνέχεια της
εκτίμησης λόγω αποτυχίας της μεθόδου sm2.  Κατά συνέπεια η υπόθεσή μας όσο
αφορά στην ανθεκτικότητα και τα χαμηλότερα σφάλματα της μεθόδου ανάδρασης
επιβεβαιώνεται και αυτή.  Το παράδοξο εδώ είναι πως η μέθοδος soft-1 στο
περιβάλλον warehouse εμφανίζει μεγαλύτερα σφάλματα ακόμα και από την ονομαστική
διαμόρφωση του φίλτρου.

% ch4/5 ------------------------------------------------------------------------
Τί είναι σήμερα δυνατόν που δεν ήταν πριν από αυτή την έρευνα; Το κύριότερο
είναι ότι σήμερα μπορούμε να εγγυηθούμε τη μείωση των σφαλμάτων ενός φίλτρου
σωματιδίων χωρίς να διακινδυνεύσουμε την ευρωστία του. Εδώ ο τρόπος ανάδρασης
που εισάγαμε είναι καθοριστικός γιατί η ευθυγράμμιση πραγματικών με εικονικές
σαρώσεις δεν είναι πάντα επιτυχής, όπως είδαμε από τα αποτελέσματα της μεθόδου
ανάδρασης που αρχικοποιεί το φίλτρο γύρω από το αποτέλεσμά της.

Οπότε εδώ γεννάται το φυσικό ερώτημα: ποιά είναι τα τρωτά σημεία της μεθόδου
ευθυγράμμισης σαρώσεων που χρησιμοποιήσα? Με δύο λόγια τα τρωτά σημεία είναι η
ευαισθησία της λύσης στη ρύθμιση των παραμέτρων και στον θόρυβο. Το πρόβλημα με
την παραμετροποίηση που αφορά τουλάχιστον στη μέθοδο που παρήγαγε τα
αποτελέσματα μου μόλις είδαμε, η οποία είναι μάλιστα η καλύτερη μέθοδος στη
βιβλιογραφία, είναι ότι δεν είναι διαισθητική διαδικασία, ότι μικρές μεταβολές
των τιμών των παραμέτρων παράγουν δυσανάλογα μεγάλες μεταβολές στην έξοδο, και
ότι για κάποιες παραμέτρους δεν υπάρχουν τιμές που να μπορούν να καλύψουν όλες
τις στάσεις σε ένα περιβάλλον. Δεν είναι τυχαίο πως αυτά τα προβλήματα
εμφανίζονται για τις παραμέτρους που αφορούν στη διαδικασία εύρεσης
αντιστοιχίσεων ανάμεσα στις ακτίνες των σαρώσεων εισόδου. Αν ψάξει κανείς τη
βιβλιογραφία θα ανακαλύψει μάλιστα ότι δεν υπάρχει μέθοδος ευθυγράμμισης που να
μην χρησιμοποιεί κάποιου είδους μηχανισμό αντιστοίχισης, ο οποίος προσπαθεί να
εκτιμήσει την αντιστοίχιση σημείων ή κατανομών σημείων της μίας σάρωσης προς
σημειο, ευθύγραμμο τμήμα, ή κατανομή της άλλης. Και εδώ φτάνουμε στο δεύτερο
τρωτό σημείο, το οποίο αφορά σε όλες τις μεθόδους ευθυγράμμισης της
βιβλιογραφίας, επειδή ακριβώς όλες χρησιμοποιούν αυτόν το μηχανισμό: το οποίο
είναι το πρόβλημα του θορύβου μέτρησης για τις μεθόδους sm, και επιπρόσθετα της
ατέλειας αναπαράστασης του χάρτη στις μεθόδους sm2. Εδώ το πρόβλημα είναι δεν
είναι μόνο ότι δυσχεραίνεται η διαδικασία διάκρισης αληθών από ψευδείς
αντιστοιχίσεις, αλλά ότι αυτή η διάκριση καθίσταται αδόκιμη λόγω παρουσίας
θορύβου. Ταυτόχρονα, ακόμα και αν υποθέσουμε ιδανικές συνθήκες, η ίδια η
διαδικασία εύρεσης αντιστοιχίσεων καθίσταται πιθανά προβληματική γιατί ακόμα
και αν οι σαρώσεις έχουν εύρος 360 μοίρες, δεν υπάρχει εγγύηση ότι όλα τα
σημεία της μίας σάρωσης θα αντιστοιχούν σε όλα τα σημεία της δεύτερης, όπως
βλέπουμε σε αυτή τη διαφάνεια, όπου στα δεξιά έχουμε αντιστρέψει τη σειρά
ευθυγράμμισης.

Εν ολίγοις το συμπέρασμά μου είναι ότι το πιο τρωτό σημείο των
μεθόδων ευθυγράμμισης σαρώσεων όλης της βιβλιογραφίας είναι ο ίδιος ο
μηχανισμός εύρεσης αντιστοιχίσεων ανάμεσα στις ακτίνες των σαρώσεων εισόδου.

Για αυτό το λόγο ξεκινάω να ερευνώ τρόπους με τους οποίους θα ήταν δυνατή η
ευθυγράμμιση πραγματικών με εικονικές σαρώσεις χωρίς την εύρεση αντιστοιχίσεων
ανάμεσα σε υποενότητες των σαρώσεων.


%%%%%%%%%%%%%%%%%%%%%%%%%%%%%%%%%%%%%%%%%%%%%%%%%%%%%%%%%%%%%%%%%%%%%%%%%%%%%%%%
% ch5
%%%%%%%%%%%%%%%%%%%%%%%%%%%%%%%%%%%%%%%%%%%%%%%%%%%%%%%%%%%%%%%%%%%%%%%%%%%%%%%%

% ch5/0 ------------------------------------------------------------------------
Εδώ καταρχάς αναζητώ τις συνθήκες εκείνες κάτω από τις οποίες είναι δυνατόν να
είναι επιτυχής η ευθυγράμμιση, και καταλήγω πως πρώτα από όλα μία αναγκαία
συνθήκη είναι η ελαχιστοποίηση του αριθμού των σημείων που είναι ορατά από μία
σάρωση αλλά όχι από την άλλη, δηλαδή είναι αναγκαίο το γωνιακό εύρος όρασης του
αισθητήρα να είναι 360 μοίρες. Έπειτα, επειδή ακριβώς το πρόβλημα της
ευθυγράμμισης χωρίς την εύρεση αντιστοιχίσων, αν είναι επιλύσιμο, είναι
καινοφανές, στόχος μου είναι να εστιάσω στην επι της αρχής λύση του, δηλαδή
στη λύση του προβλήματος χωρίς χρονικούς περιορισμούς.

Το μεγάλο πρόβλημα εδώ όμως είναι ότι δεν μπορώ να πατήσω στη βιβλιογραφία της
ευθυγράμμισης σαρώσεων γιατί ακριβώς όλες οι μέθοδοι της βιβλιογραφίας λύνουν
το πρόβλημα μέσω εύρεσης αντιστοιχίσεων, και συνεπώς αρχίζω να ερευνώ άλλα
πεδία. Εδώ το ζητούμενο είναι η εύρεση δύο μεθόδων: μίας που θα αναλάβει τη
λύση του προβλήματος της εκτίμησης της περιστροφής του συνόλου σημείων της μίας
σάρωσης ως προς αυτά της δεύτερης, και μίας που αφορά στην εκτίμηση της
μετατόπισης τους.

Μέσα από την αναζήτησή μου βρήκα τη μέθοδο FMI-SPOMF, η οποία χρησιμοποιείται
στον κλάδο της υπολογιστικής όραση για την εκτίμηση της γωνίας περιστροφής,
της μετατόπισης, και της κλιμάκωσης ανάμεσα σε δύο εικόνες. Η ιδέα εδώ είναι
ότι αν μία εικόνα είναι το περιεστραμμένο, μετατοπισμένο, και κλιμακωμένο
αντίγραφο μίας δεύτερης, τότε με τους κατάλληλους μετασχηματισμούς είναι δυνατή
η απεμπλοκή αυτών των τριών παραμέτρων μεταξύ τους, και ο υπολογισμός τους,
χωρίς πουθενά να υπολογίζονται αντιστοιχίσεις ανάμεσα στις δύο εικόνες, λόγω
της κλειστής φύσης της μεθόδου επίλυσης, η οποία βασίζεται σε ιδιότητες του
μετασχηματισμού Fourier.
Πώς μπορούμε να χρησιμοποιήσουμε αυτή τη μέθοδο στα δικά μας συμφραζόμενα?
Εάν προβάλλουμε δύο σαρώσεις εισόδου στο καρτεσιανό επίπεδο, εκπεφρασμένες στο
τοπικό σύστημα συντεταγμένων του κάθε αισθητήρα, δηλαδή στο τοπικό σύστημα του
πραγματικού αισθητήρα αφενός και το σύστημα του εικονικού αισθητήρα ο οποίος
μέσω raycasting υπολογίζει την εικονική σάρωση από την εκτίμηση στάσης του
ρομπότ αφετέρου, και στη συνέχεια τις διακριτοποιήσουμε, το αποτέλεσμα είναι
δύο πίνακες, δηλαδή δύο εικόνες, για τις οποίες μπορούμε να
χρησιμοποιήσουμε τη μέθοδο FMI-SPOMF για να υπολογίσουμε τα ζητούμενά μας.  Από
τη μία δε χρειάζεται να υπολογίσουμε τον συντελεστή κλιμάκωσης ανάμεσα στις δύο
εικόνες γιατί γνωρίζουμε τον συντελεστή κλίμακας του χάρτη ως προς το
περιβάλλον που αναπαριστά, από την άλλη όμως είναι προτιμότερο να μην
χρησιμοποιήσουμε το αποτέλεσμα μετατόπισης του SPOMF ώστε να μην υπάρχει
εξάρτηση του σφάλματος από την ανάλυση των εικόνων. Για την εκτίμηση της
μετατόπισης ανάμεσα στις δύο σαρώσεις χρησιμοποιούμε μία θα έλεγα σε εισαγωγικά
πρωτόγονη αλλά παρ' όλα αυτά αποτελεσματική μέθοδο, η οποία εφαρμόζεται μετά
την περιστροφή της δεύτερης σάρωσης ως προς την πρώτη, με τη γωνία που έχει
βρεθεί. Η μέθοδος αυτή υπολογίζει το βαρύκεντρο ή κεντροειδές των δύο νεφών
σημείων στο καρτεσιανό επίπεδο και μεταφέρει τη δεύτερη σάρωση ώστε να συμπέσει
με την πρώτη με βάση τη διαφορά των κεντροειδών τους. Για τον υπολογισμό της
μετατόπισης δεν χρησιμοποιούνται πουθενά αντιστοιχίσεις, διότι τα κεντροειδή
υπολογίζονται προφανώς το ένα ανεξάρτητα από το άλλο.  ?? ανεξαρτητο το
κεντροειδες απο το συστημα αναφορας/επαναληπτικοτητα

Στην πραγματικότητα ο χρόνος εκτέλεσης αυτής της μεθόδου ευθυγράμμισης sm2
είναι πολύ μεγαλύτερος από τον χρόνο εκτέλεσης που απαιτείται για να εφαρμοσθεί
η μέθοδος στα προηγούμενα συμφραζόμενα, δηλαδή ως μία προσθετική μέθοδος σε ένα
φίλτρο σωματιδίων και, επειδή ο χρόνος εκτέλεσης δεν είναι σε αυτό το σημείο
ζήτημα, τη χρησιμοποιούμε για την επίλυση της παθητικής έκδοσης του προβλήματος
του global localisation, η οποία δεν κάνει παραδοχές για το χρόνο εκτέλεσης της
λύσης του προβλήματος.

Κατά κύριο λόγο τα μεγάλα προβλήματα της ρομποτικής κινητής βάσης διακρίνονται
στα προβλήματα του pose tracking, του προβλήματος δηλαδή που είδαμε
προηγουμένως, της παρατήρησης δηλαδή της στάσης ενός ρομπότ καθώς αυτό
κινείται, δεδομένου του χάρτη, της προηγούμενης στάσης του, της οδομετρίας του,
και μίας τρέχουσας μέτρησης, και στο πρόβλημα του global localisation, του
προσδιορισμού δηλαδή της στάσης του εντός ενός χάρτη και μετρήσεων, όταν δεν
υπάρχει καμία πληροφορία για τη θέση και τον προσανατολισμό του ρομπότ.

Το πρόβλημα του global localisation μπορεί να λυθεί μέσω οποιασδήποτε τεχνικής
sm2 ως εξής: δεδομένου του χάρτη του περιβάλλοντος στο οποίο βρίσκεται το
φυσικό ρομπότ, διασπείρονται σε αυτόν ένας αριθμός από υποθέσεις στάσης. Από
κάθε υπόθεση υπολογίζεται η εικονική σάρωση, και στη συνέχεια μέσω sm2
επιχειρείται η ευθυγράμμιση της με τη σάρωση που συλλαμβάνεται από το φυσικό
αισθητήρα. Στο τέλος κάθε ευθυγράμμισης αποθηκεύονται η τελική εκτίμηση στάσης
και η τιμή μίας μετρικής που αποτυπώνει το βαθμό ομοιότητας ή τελικής
ευθυγράμμισης ανάμεσα στην πραγματική σάρωση και την εικονική σάρωση που απέδωσε
τη συγκεκριμένη εκτίμηση εξόδου. Στο τέλος το σύστημα εξάγει ως τελική εκτίμηση
στάσης εκείνη που σημειώνει τη μεγαλύτερη τιμή αυτής της μετρικής.

Για να δοκιμαστεί εάν το πρόβλημα της ανεύρεσης της στάσης ενός ρομπότ που
είναι εξοπλισμένο με έναν πανοραμικό αισθητήρα lidar είναι επιλύσιμο μέσω της
ευθυγράμμισης πραγματικών με εικονικές σαρώσεις με την μέθοδο που κατασκεύασα,
δοκιμάζουμε το σύστημα επίλυσης σε πέντε προσομοιωμένα περιβάλλοντα και ένα
πραγματικό, για συνολικά 49 στάσεις, οι οποίες δημιουργήθηκαν είτε τυχαία είτε
έτσι ώστε να δοκιμάσουν την επίδοση των μεθόδων sm2 που θα δοκιμαστούν σε αυτό
το πρόβλημα. Τα πειράματα στα προσομοιωμένα περιβάλλοντα επαναλήφθηκαν για 100
φορές ανά στάση, και στο πραγματικό περιβάλλον για 5 φορές ανά στάση.  Στα
προσομοιωμένα περιβάλλοντα χρησιμοποιούμε έναν πανοραμικό αισθητήρα lidar
μεγίστου βεληνεκούς δέκα μέτρων με τιμές τυπικής απόκλισης θορύβου μέτρησης
1,2,και 5 εκατοστά, ενώ στα πραγματικά πειράματα χρησιμοποιούμε έναν αισθητήρα
YDLIDAR μέγιστου βεληνεκούς τριάντα μέτρων με κατανομή θορύβου μέτρησης που
φαίνεται σε αυτόν τον πίνακα.

Σε όλα τα πειράματα καταγράφουμε το τελικό σφάλμα θέσης και προσανατολισμού και
ονομάζουμε επιτυχημένη ανέυρεση στάσης κάθε περίπτωση όπου το τελικό σφάλμα
θέσης ήταν μικρότερο από ένα μέτρο, διότι μετά την επίλυση του προβήματος
global localisation, ακολουθεί η παρατήρηση της στάσης του ρομπότ, η οποία
γίνεται κατά κόρον με πιθανοτικά μέσα, τα οποία έχουν την ικανότητα να
συγκλινουν γιατί είναι εύρωστα σε τέτοια μεγέθη σφάλματος θέσης.

Σε όλα τα πειράματα δοκιμάζουμε την sm2 μέθοδο που αποτελείται από τον FMI-SPOMF
σε συνδυασμό με τη μέθοδο των κεντροειδών, και τη μέθοδο που χρησιμοποιήσαμε
στη λύση του προηγούμενου προβλήματος σε συνδυασμό με το φίλτρο σωματιδίων, η
οποία ονομάζεται plicp.



\end{document}
