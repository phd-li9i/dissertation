\documentclass[a4paper,10pt]{article}

\usepackage{xltxtra}
\usepackage{gfsdidot}
\usepackage{verse}

\setmainfont[Mapping=tex-text]{GFS Didot}
\setsansfont[Mapping=tex-text]{DejaVu Sans}
\setmonofont[Mapping=tex-text]{DejaVu Sans Mono}
\newfontfamily\athenais{Athenais}
\pagenumbering{gobble}
%%%%%%%%%%%%%%%%%%%%%%%%%%%%%%%%%%%%%%%%%%%%%%%%%%%%%%%%%%%%%%%%%%%%%%%%%%%%%%%%
\begin{document}

Αξιότιμες καθηγήτριες, αξιότιμοι καθηγητές, συνάδελφοι, κυρίες και κύριοι:
ονομάζομαι Αλέξανδρος Φιλοθέου και σήμερα είναι σκοπός μου να σάς εκθέσω τις
συμβολές μου σε λύσεις διαδεδομένων προβλημάτων που αφορούν στο πεδίο της
ρομποτικής κινητής βάσης.

Αποφοίτησα από το τμήμα μας τον Ιούλιο του 2013 και τον Νοέμβριο του ίδιου
έτους έγινα μέλος της ομάδας ρομποτικής PANDORA. Εκεί απέκτησα την πρώτη μου
επαφή με τη ρομποτική και το αντικείμενο μου διέγειρε το ενδιαφέρον τόσο ώστε
αποφάσισα να ειδικευτώ σε αυτό. Τον Σεπτέμβριο του 2015 ξεκίνησα το
μεταπτυχιακό μου πάνω στη ρομποτική και τον έλεγχο στο Βασιλικό Ινστιτούτο
Τεχνολογίας στη Σουηδία, και δύο χρόνια αργότερα επέστρεψα στην Ελλάδα για να
εργαστώ. Τον Σεπτέμβριο του 2018 είχα την τύχη να ξεκινήσω να δουλεύω σε αυτό
που είχα επιλέξει στο πανεπιστήμιό μας υπό την επίβλεψη του κυρίου Συμεωνίδη
και του κυρίου Δημητρίου.

Από τότε η έρευνά μου διαμορφώθηκε με βάση πρακτικά προβλήματα που
συναντήσαμε στα πλαίσια εκτέλεσης των έργων στα οποία εργάστηκα και με βάση
προβλήματα τα οποία εντόπισα στις μεθόδους επίλυσής τους που βρίσκονται στην
τρέχουσα βιβλιογραφία.

Οι συμβολές της διατριβής μου αρθρώνονται σε 5 κεφάλαια.

Το πρώτο κεφάλαιο ξεκινάει με έναν ταπεινό θα έλεγα στόχο, ο οποίος αποβλέπει
στη συμβολή στο έργο των μηχανικών ρομποτικής που ασχολούνται με την αυτόνομη
πλοήγηση επίγειων οχημάτων---για την ακρίβεια ό,τι θα σας παρουσιάσω σήμερα
αφορά σε επίγεια οχήματα. Κατά τη διάρκεια επίλυσης του αυτού του πρώτου
προβλήματος παρατήρησα την ύπαρξη ενός σημαντικού ζητήματος του οποίου η λύση
θα είχε καθοριστική σημασία για τα αποτελέσματα του πρώτου έργου στο οποίο
εργάστηκα, και το οποίο έχει από μόνο του ερευνητικό ενδιαφέρον στα πλαίσια της
ρομποτικής.

Οπότε ξεκίνησα να ερευνώ πώς θα μπορούσαμε να λύσουμε αυτό το πρόβλημα στο
δεύτερο κεφάλαιο της διατριβής.  Σε αυτό το κεφάλαιο κατασκεύασα μία σειρά
λύσεων. Μία από αυτές βασίζεται σε μία βασική μέθοδο της βιβλιογραφίας, και
κατά τη διάρκεια κατασκευής της λύσης μου εντόπισα στη μέθοδο της βιβλιογραφίας
μία σημαντική παθογένεια. Το πρόβλημα που εντόπισα ήταν κρίσιμης σημασίας και
προς έκπληξίν μου δεν ξαναείχε τεθεί στην ερευνητική βιβλιογραφία (, ίσως λόγω
αδράνειας, ή κορεσμού, ή έλλειψης δυνατότητας εφαρμογής σε πραγματικές
συνθήκες). Στο επόμενο λοιπόν κεφάλαιο προσπάθησα να ερευνήσω το κατά πόσο ήταν
δυνατόν να εξαλειφθεί αυτή η παθογένεια, και ποιές θα ήταν οι ελάχιστες
παραδοχές για να γινει αυτό.

Επειδή το πρόβλημα που εγείρει αυτήν την παθογένεια ήταν απαιτητικό, αρχικά
προσπάθησα να ερευνήσω εάν μία λύση είναι εφικτή κατ'αρχήν: έκανα δηλαδή μία
σειρά παραδοχών οι οποίες από τη μία είναι ρεαλιστικές και απο την άλλη οι
λιγότερο απαιτητικές. Στόχος μου ήταν να μου δώσουν τη δυνατότητα να εξετάσω
εάν το πρόβλημα έχει λύση και ταυτόχρονα νόημα στα πλαίσια (γενικότερων
προβλημάτων) της ρομποτικής. Αφού είδα ότι το πρόβλημα αυτό ήταν όντως
επιλύσιμο, αφαίρεσα στο επόμενο κεφάλαιο μία βοηθητική παραδοχή που είχα κάνει,
και στο τελευταίο μία συνθήκη, μέχρις ώτου να μην υπάρχει δυνατότητα να
αφαιρεθεί κάτι παραπάνω χωρίς να καταρρεύσει η κύρια συμβολή της μεθόδου μου.
Έτσι κατέληξα στην κατασκευή της πρώτης μεθόδου η οποία αφαιρεί έναν σημαντικό
περιορισμό από τις μεθόδους της βιβλιογραφίας και συμβάλει στη λύση ενός
θεμελιώδους πρόβληματος της ρομποτικής και γενικότερα της αναγνώρισης προτύπων.


(3min Μέχρι εδω)



Προτού μπω σε περισσοτερο βαθος πρεπει να ξεκινησω απο την αρχη. Η ερευνα μου
ξεκιναει με το εργο RELIEF. Ο σκοπος του εργου ηταν ο εξης. Στην αγορά λιανικών
προϊόντων υπάρχουν εταιρείες που πουλάνε τα προϊόντα τους σε καταστήματα, και
των οποίων το συνολικό απόθεμα αποθηκεύεται σε κεντρικές αποθήκες. Εν γένει
αυτές οι εταιρείες θα ήθελαν να γνωρίζουν ανά πάσα στιγμή το απόθεμά που
βρίσκεται στις αποθήκες τους, όμως αυτό είναι τόσο κοστοβόρο από άποψης πόρων
χρήματος και χρόνου που μπορούν να μετρούν το απόθεμά τους μόνο λίγες φορές
μέσα σε ένα οικονομικό έτος. Ύστερα θα ήθελαν επίσης να γνωρίζουν τις θέσεις
των προϊόντων μέσα σε ένα κατάστημα ή μία αποθήκη τόσο για λόγους γρήγορης
ανάκτησης όσο και για λόγους που τους επιβάλλονται από τρίτα μέρη, γιατί πχ
υπάρχουν συμφωνίες που επιβάλλουν την τοποθέτηση των προϊόντων σε συγκεκριμένες
θέσεις και ύψη μέσα σε ένα κατάστημα. Το έργο RELIEF είχε ως στόχο την
κατασκευή μίας σειράς από αυτόνομα ρομπότ τα οποία θα μπορούσαν να
αποδεσμεύσουν ανθρώπινο δυναμικό από το τετριμμένο έργο της καταγραφής του
αποθέματος και της εκτίμησης της θέσης των εμπορευμάτων, ώστε αυτές οι
ενέργειες να γίνονται ακόμα και σε καθημερινή βάση, με ελάχιστη εμπλοκή
ανθρώπων.

Η πρώτη απαίτηση που τέθηκε για τα επίγεια ρομπότ του έργου ήταν να είναι ικανά
να πλοηγούνται αυτόνομα στο χώρο. Σε πραγματικές συνθήκες μπορείτε να
φανταστείτε πως προτού κλείσει η αποθήκη το βράδυ, ή ακόμα και κατά τη διάρκεια
μίας εργάσιμης μέρας, στο ρομπότ δίνεται η εντολή να περάσει από συγκεκριμένα
σημεία του χώρου ώστε να σαρώσει όλα τα ράφια στα οποία υπάρχουν αντικείμενα.
Η αυτονομία της πλοήγησης αφαιρεί την απαίτηση για εξωτερικό εξοπλισμό πάνω
στον οποίο θα μπορούσε να οδηγηθεί το ρομπότ ενώ ταυτόχρονα το κάνει ικανό να
μπορεί να εκτελεί τις ενέργειές του απρόσκοπτα ενώ γύρω του υπάρχουν κινούμενα
εμπόδια όπως άνθρωποι ή μηχανήματα.

Η αυτόνομη πλοήγηση υποθέτει 5
προαπαιτούμενα.  Πρέπει να υπάρχει ο χάρτης του περιβάλλοντος στο οποίο
πλοηγείται το ρομπότ, τουλάχιστον ένας εξωδεκτικός αισθητήρας, μία μέθοδος
εκτίμησης της στάσης του ρομπότ στο σύστημα συντεταγμένων του χάρτη, μία αρχική
στάση και μία τελική στάση. Ως στάση του ρομπότ ορίζουμε το διάνυσμα κατάστασης
που συνιστάται από τη θέση και τον προσανατολισμό του ρομπότ ως προς το σύστημα
αναφοράς του χάρτη.

Δεδομένων αυτών η αυτόνομη πλοήγηση συνίσταται σε δύο
μέρη: αφενός χρειάζεται ένας αλγόριθμος που δέχεται τον χάρτη, και την αρχική
και τελική στάση και παράγει ένα μονοπάτι το οποίο συνδέει την αρχική με την
τελική στάση χωρίς να τέμνει εμπόδια του χάρτη, και αφετέρου ένας ελεγκτής
κίνησης, ο οποίος, δεδομένου του μονοπατιού, της στιγμιαίας εκτίμησης για τη
στάση του ρομπότ, και μετρήσεις από αισθητήρες παράγει εντολές κίνησης τις
οποίες λαμβάνουν ως είσοδο οι κινητήρες του ρομπότ ώστε αυτό να κινείται πάνω
στο μονοπάτι που παρήγαγε ο πρώτος αλγόριθμος. Ο αλγόριθμος κατασκευής
μονοπατιών ονομάζεται στην ορολογία global planner, και ο ελεγκτής κίνησης
local planner.

Για να υλοποιήσω το συνολικό σύστημα αυτόνομης πλοήγησης στα πλαίσια του έργου
αρχικά στράφηκα στη διαθέσιμη λογισμικογραφία, όπου ανακάλυψα πως υπήρχαν
πολλαπλοί αλγόριθμοι global και local planners. Πουθενα ομως στη διαθεσιμη
βιβλιογραφια δεν υπηρχε συγκριτικη αναλυση της συνδυαστικής επιδοσης τους στο
έργο της αυτόνομης πλοήγησης ώστε να επιλέξω ποιός συνδυασμός θα εκπλήρωνε τους
στόχους μας στα πλαίσια του έργου. Συνεπώς αποφασίσαμε πως θα ήταν επωφελές
τόσο για το έργο όσο και για άλλους μηχανικούς ρομποτικής να σχεδιάσουμε μία
μέθοδο αξιολόγησης των διαθέσιμων αλγορίθμων global και local planners και των
συνδυασμών τους, την οποία θα χρησιμοποιούσαμε μέσω πειραματικής διαδικασίας
ώστε να καταλήξουμε σε συμπεράσματα για την επίδοση των τρέχοντων διαθέσιμων
αλγορίθμων που υλοποιούν αυτόνομη πλοήγηση σε επίγειες κινητές ρομποτικές
βάσεις.

Αρχικά συγκεντρώσαμε όλα τα διαθέσιμα πακέτα λογισμικού και τα υποβάλλαμε σε
αξιολόγηση με βάση ποιοτικά κριτήρια, δηλαδή τα ίδια κριτήρια που θα
χρησιμοποιούσε ένας μηχανικός λογισμικού προτού φτάσει στο σημείο
να εξετάσει εάν πρακτικά η επίδοση του πακέτου λογισμικού είναι επαρκής.  [Εδω
μπαίνει ο πίνακας]. Στα αριστερά βρίσκονται τα ονόματα των global και local
planners και πάνω στην οριζόντια γραμμή οι συντομογραφίες των ποιοτικών
κριτηρίων, όπως εάν το πακέτο διαθέτει τεκμηρίωση ή είναι σκέτος κώδικας, εάν
είναι ενημερωμένο, δηλαδή εάν μπορεί να τρέξει στο πιο σύγχρονο λειτουργικό και
ειναι ενσωματώσιμο, εάν είναι αυτοτελές, παραμετροποιήσιμο, πόσο συνεπές
είναι στην εκτέλεσή του, και τι ανάγκες έχει σε υπολογιστικούς πόρους. Στη
δεξιά στήλη σημειώνονται με κενές κουκίδες τα πακέτα εκείνα τα οποία
αποτυγχάνουν σε έστω και ένα κριτήριο. Στο τέλος αυτού του φιλτραρίσματος
απέμειναν τρεις αλγορίθμοι χάραξης μονοπατιών και τρεις ελεγκτές κίνησης,
οπότε η πειραματική διαδικασία πραγματοποιήθηκε σε συνολικά εννέα συνδυασμούς
αλγορίθμων.

Η αξιολόγηση κάθε συνδυασμού έγινε σε δύο περιβάλλοντα προσομοίωσης και σε ένα
πραγματικό περιβάλλον. Και στις τρεις περιπτώσεις θέσαμε μία αρχική και μία
τελική στάση και ζητήσαμε από κάθε συνδυασμό global και local planners να
πλοηγηθεί αυτόνομα από την αρχική στην τελική στάση με βάση το μονοπάτι που θα
σχεδίαζε ο global planner και τις εντολές κίνησης του local planner. Κάθε
συνδυασμός απο planners επανέλαβε το πείραμα 10 φορές σε κάθε περιβάλλον, και
για κάθε πείραμα καταγράφηκε ένας αριθμός από μετρικές προκειμένου να γίνει η
αξιολόγηση κάθε συνδυασμού και να υπάρχει ένα κοινό σύστημα κρίσης για όλους
ώστε στο τέλος να προκύψει μία ιεραρχία συνδυασμών. Οι μετρικές αυτές είναι
τριών ειδών: αφορούν είτε αποκλειστικά στους global planners, όπως για
παράδειγμα το μήκος των σχεδιασθέντων μονοπατιών ή η μέση ελάχιστη απόσταση
ενός μονοπατιού από εμπόδια (8 μετρικές), είτε αποκλειστικα στους local
planners (8 μετρικές), όπως για παράδειγμα ο αριθμός αποτυχιών έυρεσης
ταχυτήτων προς τον συνολικό αριθμό κλήσεων του ελεγκτή, ή αποκλειστικά στο
συνδυασμό τους, όπως για παράδειγμα ο χρόνος πλοήγησης, ή πραγματική ολικά
ελάχιστη απόσταση από εμπόδια κατά τη διάρκεια μίας πλοήγησης (12 μετρικές).

Αυτό που θέλουμε στο τέλος είναι να μπορέσουμε να αποδώσουμε μία τιμή σε κάθε
συνδυασμό από planners με βάση τις τιμές των μετρικών που έχουμε καταγράψει. Τα
προβλήματα εδώ είναι δύο: πώς θα δώσουμε μία τιμή όταν οι μετρικές εκφράζονται
σε διαφορετικές μονάδες μέτρησης, και πώς θα κατασκευάσουμε μία συνάρτηση
απόδοσης αξίας σε κάθε συνδυασμό που να είναι γνησίως αύξουσα όταν κάποιες
μετρικές συνεισφέρουν αναλογα καθώς αυξάνονται (όπως η μέση απόσταση από
εμπόδια, όσο μεγαλύτερη η απόσταση από εμπόδια τόσο ασφαλέστερη είναι η
πλοήγηση) ενώ άλλες αντιστρόφως ανάλογα (όπως ο χρόνος πλοήγησης). Για να
αποκτήσουμε ένα κοινό σύστημα αναφοράς αρχικά κανονικοποιούμε τις τιμές των
μετρικών μέσω της συνάρτησης N (3.8), δηλαδή εξετάζουμε την τιμή της μετρικής m
για έναν συνδυασμό και τις ακραίες τιμές της για κάθε συνδυασμό, ώστε στο τέλος
η τιμή της εκφράζεται στο διάστημα [0,1]. Κατασκευάζουμε την συνάρτηση
απόδοσης αξίας μέσω της συνάρτησης V (3.11).  Εδώ $I_Q$ και $I_\overline{Q}$
είναι η συνάρτηση δείκτης, (δειξε οτι Q ενωση Qbar = ενωση m), $V_q$ = (3.9),
$V_\overline{q}$ = (3.10). H πρώτη αφορά σε μετρικές m που συνεισφέρουν
ανάλογικά στην τιμή της V και η δεύτερη αντιστρόφως ανάλογα (εξου και το 1-N).
Η συνάρτηση I(C,m) είναι μηδεν όταν ο συνδυασμός C δεν κατάφερε να πλοηγήσει το
ρομπότ και η m ειναι μετρική που αφορά στον συνδυασμό planners, αλλιώς είναι
ένα.  Αυτό το κάνουμε έτσι ώστε να μπορούμε με μία συνάρτηση να συνεκτιμήσουμε
την αξία των συνιστωσών ενός συνδυασμού ακόμα και όταν η πλοήγηση ήταν
αποτυχημένη. Μέσω της V και των τιμών όλων των μετρικών που έχουμε καταγράψει για
κάθε πείραμα στο τέλος λαμβάνουμε τα αποτελέσματα του πίνακα (3.10), τα οποία
εμφανίζουν την αξία κάθε συνδυασμού για κάθε περιβάλλον, και τη συνολική αξία
κάθε συνδυασμού με βάση τα αποτελέσματα κάθε περιβάλλοντος. Η εκτέλεση πειραμάτων
σε διαφορετικά και διαφορετικής δυσκολίας περιβάλλοντα και με διαφορετικούς
αισθητήρες δινει τη δυνατοτητα να εμφανιστει το υποκειμενο μοτιβο της ιεραρχίας
αναμεσα σε global και local planners, αυτό δηλαδή που επιζητησε η ερευνά μας.

Τί υπάρχει σήμερα που δεν υπήρχε πριν αυτή τη μελέτη: μία επεκτάσιμη και
εξονυχιστική μέθοδος αξιολόγησης της επίδοσης του συνδυασμού αλγορίθμων χάραξης
μονοπατιών και ελεγκτών κίνησης, και αποτελέσματα επίδοσης τους με βάση πακέτα
λογισμικού τα οποία είναι διαθέσιμα για ενσωμάτωση σε επίγεια ρομπότ κινητής
βάσης.


\end{document}
