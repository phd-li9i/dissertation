\begin{frame}{\small Global localisation: επιλύσιμο μέσω \texttt{sm2}}

  \begin{figure}
    \input{./figures/slides/ch5/sm2_system/sm2_gl_system.tikz}
  \end{figure}

\note{\scriptsize
Το πρόβλημα του global localisation μπορεί να λυθεί μέσω οποιασδήποτε τεχνικής
sm2 ως εξής: δεδομένου του χάρτη του περιβάλλοντος στο οποίο βρίσκεται το
φυσικό ρομπότ, διασπείρονται με τυχαίο τρόπο σε αυτόν ένας αριθμός από
υποθέσεις στάσης, οι οποίες τοποθετούνται σε μία ουρά. Από κάθε υπόθεση
υπολογίζεται η εικονική σάρωση, και στη συνέχεια μέσω sm2 επιχειρείται η
ευθυγράμμιση της με τη σάρωση που συλλαμβάνεται από το φυσικό αισθητήρα. Στο
τέλος κάθε ευθυγράμμισης αποθηκεύονται η τελική εκτίμηση στάσης και η τιμή μίας
μετρικής που αποτυπώνει το βαθμό ομοιότητας ή τελικής ευθυγράμμισης ανάμεσα
στην πραγματική σάρωση και την εικονική σάρωση.  Για τις τεχνικές sm2 που
λειτουργούν με αντιστοιχίσεις αυτό το μέτρο υπολογίζεται εσωτερικά σε κάθε
μέθοδο ως το άθροισμα των αποστάσεων των σημειών της μίας σάρωσης ως προς τα
σημεία, τις γραμμές ή τις κατανομές της δεύτερης, και στο δικό μας σύστημα αυτό
το μέτρο προέρχεται απευθείας από τον FMI-SPOMF. Στο τέλος το σύστημα εξάγει ως
τελική εκτίμηση στάσης εκείνη που σημειώνει τη μεγαλύτερη τιμή ομοιότητας.}

\end{frame}
