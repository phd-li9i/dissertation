\begin{frame}{Συμπεράσματα πειραμάτων}

  \begin{itemize}
    \item Σφάλματα εκτίμησης \texttt{sm2}: \ \ μέσω \texttt{FMI-SPOMF} $\sim$ μέσω μεθόδων-με-αντιστοιχίσεις
    \item Ευαισθησία μεθόδων με αντιστοιχίσεις σε:
        \begin{itemize}
          \item παραμέτρους {\footnotesize (π.χ. αποτυχίες @ WAREHOUSE)}
          \item αρχικές συνθήκες μετατόπισης {\footnotesize (π.χ. 1 υπόθεση ανά 4 m$^2$ @ WAREHOUSE, WILLOWGARAGE)}
        \end{itemize}
    %\item $\dfrac{dt_{\text{exec}}^{\texttt{\ FMI-SPOMF}}}{d|\mathcal{S}|} < \dfrac{dt_{\text{exec}}^{\texttt{\ PLICP}}}{d|\mathcal{S}|}$
  \end{itemize}

\note{\footnotesize
  Εν κατακλείδι, μέσα από τις προσομοιώσεις και τα πειράματα παρατηρήσαμε πως η
  μέθοδος που σχεδιάσαμε εμφανίζει σφάλματα στάσης τα οποία είναι συγκρίσιμα σε
  σχέση με την καλύτερη μέθοδο της βιβλιογραφίας,---ύστερα την ανάγκη, από
  τις μεθόδους ευθυγράμμισης που βασίζονται στις αντιστοιχίσεις, για custom-made
  παραμετροποίηση ανά περιβάλλον, και για πρώτη φορά κάτι που αναφέρεται στη
  βιβλιογραφία και αφορά γενικά στην κλάση των ICP αλγορίθμων, δηλαδή πως η
  σύγκλισή τους απαιτεί τη σύλληψη των δύο εισόδων από μικρή απόσταση μεταξύ
  τους.}

\end{frame}
