\begin{frame}{Συμπεράσματα πειραμάτων}

  \begin{itemize}
    \item Μέθοδοι με αντιστοιχίσεις:
      \begin{itemize}
        \item Ευαισθησία λύσης σε παραμέτρους---π.χ. αποτυχίες @ WAREHOUSE
        \item Ευαισθησία λύσης σε αρχικές συνθήκες μετατόπισης---π.χ. 1 υπόθεση ανά 4 m$^2$ @ WAREHOUSE, WILLOWGARAGE
      \end{itemize}
    %\item $\dfrac{dt_{\text{exec}}^{\texttt{\ FMI-SPOMF}}}{d|\mathcal{S}|} < \dfrac{dt_{\text{exec}}^{\texttt{\ PLICP}}}{d|\mathcal{S}|}$
    \item Σφάλματα εκτίμησης \texttt{sm2}: \ \ μέσω \texttt{FMI-SPOMF} $\sim$ μέσω μεθόδων με αντιστοιχίσεις
  \end{itemize}

\note{\footnotesize
  Εν κατακλείδι, μέσα από τα πειράματα ξαναπαρατηρήσαμε την ανάγκη από τις
  μεθόδους ευθυγράμμισης που βασίζονται στις αντιστοιχίσεις για custom-made
  παραμετροποίηση ανά περιβάλλον, και παρατηρήσαμε για πρώτη φορά κάτι που
  αναφέρεται στη βιβλιογραφία και αφορά γενικά στην κλάση των ICP αλγορίθμων,
  δηλαδή πως η σύγκλισή τους απαιτεί τη σύλληψη των δύο εισόδων από μικρή
  απόσταση μεταξύ τους. Το μεγαλύτερο αποτέλεσμα θα έλεγα πως είναι ότι η
  μέθοδος που σχεδιάσαμε εμφανίζει σφάλματα στάσης τα οποία είναι συγκρίσιμα σε
  σχέση με την καλύτερη μέθοδο της βιβλιογραφίας.}

\end{frame}
