\begin{frame}{Ευθυγράμμιση πανοραμικών σαρώσεων από ευθυγράμμιση εικόνων}


  \begin{figure}
    \input{./figures/slides/ch5/scan_to_image/mosaic2_faded_3.tikz}
  \end{figure}


\note{\footnotesize
Ας υποθέσουμε λοιπόν πως έχουμε στη διάθεσή μας μία πανοραμική σάρωση $S_R$,
το χάρτη του περιβάλλοντος, και μία εκτίμηση για τη στάση του αισθητήρα lidar.
Μέσω της εκτίμησης και του χάρτη είναι δυνατός ο υπολογισμός της εικονικής
σάρωσης $S_V$ μέσω raycasting. Εάν προβάλλουμε τις δύο σαρώσεις εισόδου στο
καρτεσιανό επίπεδο, εκφρασμένες στο τοπικό σύστημα συντεταγμένων του κάθε
αισθητήρα, και στη συνέχεια τις διακριτοποιήσουμε, το αποτέλεσμα είναι δύο
πίνακες, δηλαδή δύο εικόνες $I_R$ και $I_V$, αντίστοιχα, για τις οποίες
μπορούμε να χρησιμοποιήσουμε τη μέθοδο FMI-SPOMF για να υπολογίσουμε τα
ζητούμενά μας.}


\end{frame}
