\begin{frame}{Ευθυγράμμιση πανοραμικών σαρώσεων από ευθυγράμμιση εικόνων}


  \begin{figure}
    
\definecolor{r}{RGB}{255 0 0}
\definecolor{g}{RGB}{0 155 0}
\definecolor{b}{RGB}{0 0 255}

\tikzset{every picture/.style={line width=0.75pt}} %set default line width to 0.75pt

\begin{tikzpicture}[x=0.75pt,y=0.75pt,yscale=-1,xscale=1]
%uncomment if require: \path (0,442); %set diagram left start at 0, and has height of 442

%Image [id:dp7345435744663564]
\draw (40.99,137.25) node  {\includegraphics[width=52.5pt,height=40.71pt]{./figures/slides/ch5/scan_to_image/env.png}};
%Image [id:dp030872284892191182]
\draw (40.99,217.99) node  {\includegraphics[width=52.5pt,height=39.8pt]{./figures/slides/ch5/scan_to_image/map.png}};
%Image [id:dp9855156395144604]
\draw (140.99,217.99) node  {\includegraphics[width=52.5pt,height=52.5pt]{./figures/slides/ch5/scan_to_image/sv.png}};
%Image [id:dp0883330060973122]
\draw (140.49,137.25) node  {\includegraphics[width=52.5pt,height=52.5pt]{./figures/slides/ch5/scan_to_image/sr.png}};
%Image [id:dp8079986507529633]
\draw (240.49,137.25) node  {\includegraphics[width=52.5pt,height=52.5pt]{./figures/slides/ch5/scan_to_image/real.png}};
%Image [id:dp04118937468211392]
\draw (238.99,217.99) node  {\includegraphics[width=52.5pt,height=52.5pt]{./figures/slides/ch5/scan_to_image/virtual.png}};
%Straight Lines [id:da3336180290494899]
\draw    (275.25,137.25) -- (322.75,137.25) ;
%Straight Lines [id:da6380620434779032]
\draw    (274.25,217.99) -- (322.75,217.99) ;
%Straight Lines [id:da15384576533302807]
\draw    (322.75,137.25) -- (322.75,161.25) ;
\draw [shift={(322.75,163.25)}, rotate = 270] [color={rgb, 255:red, 0; green, 0; blue, 0 }  ][line width=0.75]    (10.93,-3.29) .. controls (6.95,-1.4) and (3.31,-0.3) .. (0,0) .. controls (3.31,0.3) and (6.95,1.4) .. (10.93,3.29)   ;
%Straight Lines [id:da8338083407403734]
\draw    (322.75,218.99) -- (322.75,193.93) ;
\draw [shift={(322.75,191.93)}, rotate = 90] [color={rgb, 255:red, 0; green, 0; blue, 0 }  ][line width=0.75]    (10.93,-3.29) .. controls (6.95,-1.4) and (3.31,-0.3) .. (0,0) .. controls (3.31,0.3) and (6.95,1.4) .. (10.93,3.29)   ;
%Straight Lines [id:da31631966071327233]
\draw    (375.35,177.29) -- (436.85,177.29) -- (436.85,163) ;
%Straight Lines [id:da7038147916669211]
\draw    (355.39,189.38) -- (355.39,261.38) ;
\draw [shift={(355.39,263.38)}, rotate = 270] [color={rgb, 255:red, 0; green, 0; blue, 0 }  ][line width=0.75]    (10.93,-3.29) .. controls (6.95,-1.4) and (3.31,-0.3) .. (0,0) .. controls (3.31,0.3) and (6.95,1.4) .. (10.93,3.29)   ;
%Image [id:dp4359687414115401]
\draw (438.1,113.14) node  {\includegraphics[width=74.86pt,height=74.86pt]{./figures/slides/ch5/scan_to_image/aligned.png}};
%Straight Lines [id:da4365786762663775]
\draw    (77.39,137.25) -- (101.89,137.25) ;
\draw [shift={(103.89,137.25)}, rotate = 180] [color={rgb, 255:red, 0; green, 0; blue, 0 }  ][line width=0.75]    (10.93,-3.29) .. controls (6.95,-1.4) and (3.31,-0.3) .. (0,0) .. controls (3.31,0.3) and (6.95,1.4) .. (10.93,3.29)   ;
%Straight Lines [id:da9500470672515446]
\draw    (77.39,217.99) -- (101.89,217.99) ;
\draw [shift={(103.89,217.99)}, rotate = 180] [color={rgb, 255:red, 0; green, 0; blue, 0 }  ][line width=0.75]    (10.93,-3.29) .. controls (6.95,-1.4) and (3.31,-0.3) .. (0,0) .. controls (3.31,0.3) and (6.95,1.4) .. (10.93,3.29)   ;
%Straight Lines [id:da18131060303921864]
\draw    (175.89,137.25) -- (200.39,137.25) ;
\draw [shift={(202.39,137.25)}, rotate = 180] [color={rgb, 255:red, 0; green, 0; blue, 0 }  ][line width=0.75]    (10.93,-3.29) .. controls (6.95,-1.4) and (3.31,-0.3) .. (0,0) .. controls (3.31,0.3) and (6.95,1.4) .. (10.93,3.29)   ;
%Straight Lines [id:da16835710408982396]
\draw    (175.89,217.99) -- (200.39,217.99) ;
\draw [shift={(202.39,217.99)}, rotate = 180] [color={rgb, 255:red, 0; green, 0; blue, 0 }  ][line width=0.75]    (10.93,-3.29) .. controls (6.95,-1.4) and (3.31,-0.3) .. (0,0) .. controls (3.31,0.3) and (6.95,1.4) .. (10.93,3.29)   ;
%Image [id:dp9297806524160088]
\draw (438.1,240.49) node  {\includegraphics[width=75.38pt,height=75.38pt]{./figures/slides/ch5/scan_to_image/s_aligned.png}};
%Shape: Rectangle [id:dp41358298385049075]
\draw   (5.99,110.1) -- (75.99,110.1) -- (75.99,164.39) -- (5.99,164.39) -- cycle ;
%Shape: Rectangle [id:dp8289234142790565]
\draw   (5.99,191.46) -- (75.99,191.46) -- (75.99,244.52) -- (5.99,244.52) -- cycle ;
%Shape: Rectangle [id:dp5428699224303526]
\draw   (105.49,102.25) -- (175.89,102.25) -- (175.89,172.25) -- (105.49,172.25) -- cycle ;
%Shape: Rectangle [id:dp42147633315554844]
\draw   (105.99,182.99) -- (175.89,182.99) -- (175.89,252.99) -- (105.99,252.99) -- cycle ;
%Shape: Rectangle [id:dp6049360628407487]
\draw   (205.49,102.25) -- (275.25,102.25) -- (275.25,172.25) -- (205.49,172.25) -- cycle ;
%Shape: Rectangle [id:dp2521020767932107]
\draw   (203.99,182.99) -- (273.99,182.99) -- (273.99,252.99) -- (203.99,252.99) -- cycle ;
%Shape: Rectangle [id:dp6142215604413235]
\draw   (388.19,63.23) -- (488.01,63.23) -- (488.01,163.05) -- (388.19,163.05) -- cycle ;
%Shape: Rectangle [id:dp9231570758043264]
\draw   (388.35,190.66) -- (487.85,190.66) -- (487.85,296.23) -- (388.35,296.23) -- cycle ;
%Straight Lines [id:da1181254794306934]
\draw [color={rgb, 255:red, 126; green, 211; blue, 33 }  ,draw opacity=1, dotted ]   (436.85,177.29) -- (436.85,190.5) ;

% Text Node
\draw    (283.16,165) -- (375.35,165) -- (375.35,190) -- (283.16,190) -- cycle  ;
\draw (329.16,177.5) node   [align=left] {\texttt{FMI\_SPOMF}};
% Text Node
\draw (270,276) node [anchor=north west][inner sep=0.75pt]   [align=left] {\small \sout{$(\Delta x, \Delta_y)$}, $\Delta \theta$, \textcolor{g}{$\sigma$}, $w$};
% Text Node
\draw (3,79.5) node [anchor=north west][inner sep=0.75pt]   [align=left] {Περιβάλλον};
% Text Node
\draw (16,267) node [anchor=north west][inner sep=0.75pt]   [align=left] {Χάρτης};
% Text Node
\draw (121,77.39) node [anchor=north west][inner sep=0.75pt]   [align=left] {$\mathcal{S}_R(\bm{p})$};
% Text Node
\draw (121,266.5) node [anchor=north west][inner sep=0.75pt]   [align=left] {$\mathcal{S}_V(\hat{\bm{p}})$};
% Text Node
\draw (231,78.5) node [anchor=north west][inner sep=0.75pt]   [align=left] {$\texttt{I}_R$};
% Text Node
\draw (231,266.5) node [anchor=north west][inner sep=0.75pt]   [align=left] {$\texttt{I}_V$};
% Text Node
\draw (412.1,44) node [anchor=north west][inner sep=0.75pt]   [align=left] {\textcolor{b}{$\texttt{I}_R$}, \textcolor{r}{$\texttt{I}_V^{\text{rot}}$}};
% Text Node
\draw (385.1,300) node [anchor=north west][inner sep=0.75pt]   [align=left] {\textcolor{b}{$\mathcal{S}_R$}(\textcolor{b}{$\bm{p}$}), \textcolor{r}{$\mathcal{S}_V^{\text{rot}}$}(\textcolor{r}{$\hat{\bm{p}}^{\text{rot}}$})};


\end{tikzpicture}


  \end{figure}


\note{\footnotesize
Ας υποθέσουμε λοιπόν πως έχουμε στη διάθεσή μας μία πανοραμική σάρωση $S_R$,
το χάρτη του περιβάλλοντος, και μία εκτίμηση για τη στάση του αισθητήρα lidar.
Μέσω της εκτίμησης και του χάρτη είναι δυνατός ο υπολογισμός της εικονικής
σάρωσης $S_V$ μέσω raycasting.  Εάν προβάλλουμε τις δύο σαρώσεις εισόδου στο
καρτεσιανό επίπεδο, εκφρασμένες στο τοπικό σύστημα συντεταγμένων του κάθε
αισθητήρα, και στη συνέχεια τις διακριτοποιήσουμε, το αποτέλεσμα είναι δύο
πίνακες, δηλαδή δύο εικόνες $I_R$ και $I_V$, αντίστοιχα, για τις οποίες
μπορούμε να χρησιμοποιήσουμε τη μέθοδο FMI-SPOMF για να υπολογίσουμε τα
ζητούμενά μας. Από τη μία δε χρειάζεται να υπολογίσουμε τον συντελεστή
κλιμάκωσης ανάμεσα στις δύο εικόνες γιατί γνωρίζουμε τον συντελεστή κλίμακας
του χάρτη ως προς το περιβάλλον που αναπαριστά, από την άλλη όμως είναι
προτιμότερο να μην χρησιμοποιήσουμε το αποτέλεσμα μετατόπισης του SPOMF ώστε
να μην υπάρχει εξάρτηση του σφάλματος εκτίμησης θέσης από την ανάλυση των
εικόνων.}


\end{frame}
