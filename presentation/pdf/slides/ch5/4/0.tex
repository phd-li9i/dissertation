\begin{frame}{Ευθυγράμμιση πανοραμικών σαρώσεων από ευθυγράμμιση εικόνων}


  \begin{figure}
    \input{./figures/slides/ch5/scan_to_image/mosaic2_plus_trans.tikz}
  \end{figure}

\note{\footnotesize
Για την εκτίμηση της μετατόπισης ανάμεσα στις δύο σαρώσεις χρησιμοποιούμε μία
μέθοδο η οποία εφαρμόζεται μετά την περιστροφή της δεύτερης σάρωσης ως προς την
πρώτη. Η μέθοδος αυτή υπολογίζει το βαρύκεντρο των δύο σαρώσεων στο
καρτεσιανό επίπεδο και μεταφέρει επαναληπτικά τη δεύτερη σάρωση ώστε να
συμπέσει με την πρώτη με βάση τη διαφορά των κεντροειδών τους.

Υπολογίζοντας με αυτόν τον τρόπο τη διαφορά ως προς τη θέση και
και μέσω FMI-SPOMF τη διαφορά ως προς προσανατολισμό, αποφεύγουμε ρητά την
αντιστοίχιση ακτίνων ανάμεσα σε σαρώσεις.
}

\end{frame}
