\begin{frame}{Πειραματική διαδικασία: μη αποδεκτές λύσεις}

  \begin{itemize}
    \item Καταγραφή σφάλματος εκτίμησης θέσης $\hat{\bm{l}}$ και
          προσανατολισμού $\hat{\bm{\theta}}$ τελικής στάσης $\hat{\bm{p}}$

    \item Επιτυχημένη εκτίμηση στάσης όταν $\hat{\bm{l}} < 1.0$ m \\

          καθώς \\

          global localisation $\Rightarrow$ (probabilistic) pose tracking
  \end{itemize}


\note{\footnotesize
Σε όλες τις δοκιμές καταγράφουμε το τελικό σφάλμα θέσης και προσανατολισμού και
ονομάζουμε επιτυχημένη ανέυρεση στάσης κάθε περίπτωση όπου το τελικό σφάλμα
θέσης είναι μικρότερο από ένα μέτρο, διότι μετά την επίλυση του προβλήματος
global localisation, τυπικά ακολουθεί η παρατήρηση της στάσης του ρομπότ, η
οποία γίνεται κατά κόρον με πιθανοτικά μέσα, τα οποία έχουν την ικανότητα να
συγκλινουν, γιατί είναι εύρωστα σε τέτοια μεγέθη σφάλματος θέσης.}
\end{frame}
