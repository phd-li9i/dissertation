\begin{frame}{Πειραματική διαδικασία: μη αποδεκτές λύσεις}

  Καταγραφή σφάλματος εκτίμησης θέσης $\hat{\bm{l}}$ και
  προσανατολισμού $\hat{\bm{\theta}}$ τελικής στάσης $\hat{\bm{p}}$ \\

  inliers / επιτυχημένη εκτίμηση στάσης: $\hat{\bm{l}} < 1.0$ m \\

  καθώς \\

  global localisation $\Rightarrow$ (probabilistic) pose tracking


\note{\footnotesize
Σε όλα τα πειράματα καταγράφουμε το τελικό σφάλμα θέσης και προσανατολισμού και
ονομάζουμε επιτυχημένη ανέυρεση στάσης κάθε περίπτωση όπου το τελικό σφάλμα
θέσης ήταν μικρότερο από ένα μέτρο, διότι μετά την επίλυση του προβήματος
global localisation, ακολουθεί η παρατήρηση της στάσης του ρομπότ, η οποία
γίνεται κατά κόρον με πιθανοτικά μέσα, τα οποία έχουν την ικανότητα να
συγκλινουν γιατί είναι εύρωστα σε τέτοια μεγέθη σφάλματος θέσης.}
\end{frame}
