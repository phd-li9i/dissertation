\begin{frame}{Πειραματική διαδικασία: διάταξη}

  \begin{itemize}
    \item Πέντε προσομοιωμένα περιβάλλοντα / Ένα πραγματικό
    \item Προσομοιωμένα:
      \begin{itemize}
        \item 38 δοκιμαστικές στάσεις
        \item 100 επαναλήψεις ανά στάση
        \item LIDAR: $W_R \sim \mathcal{N}(0,\sigma_R^2 = \{0.01^2, 0.02^2, 0.05^2\})$ {\footnotesize $[$m, m$^2]$}, $r_{\max} = 10.0$ m
      \end{itemize}
    \item CSAL AUTh:
      \begin{itemize}
        \item 11 δοκιμαστικές στάσεις
        \item 5 επαναλήψεις ανά στάση
        \item LIDAR: YDLIDAR TG30, $r_{\max} = 30.0$ m \\

              \noindent\makebox[0.3\linewidth][c]{%o
              \begin{minipage}{0.3\linewidth}
                \begin{figure}
                  \includegraphics[scale=0.09]{./figures/slides/ch5/tb_yd.jpg}
                \end{figure}
              \end{minipage}
              }
              \hfill
              \noindent\makebox[0.6\linewidth][c]{%
              \scriptsize
              \begin{minipage}{0.6\linewidth}
                \begin{table}[h]\centering \vspace{0.5cm}
                        \begin{tabular}{rc}
                          Απόσταση $d$ [mm]    & Μέσο σφάλμα [mm]   \\ \toprule
                          $50$-$5000$          & $\leq \pm 60$      \\
                          $5000$-$20000$       & $\leq \pm 40$      \\
                          $20000$-$30000$      & $\leq \pm 100$     \\ \bottomrule
                        \end{tabular}
                        \caption{\tiny Μέσο σφάλμα μέτρησης αισθητήρα YDLIDAR TG30 ανά επιστρεφόμενη τιμή απόστασης. Πηγή: datasheet κατασκευαστή}
                \end{table}
              \end{minipage}
    }
    \end{itemize}
  \end{itemize}


\note{\footnotesize Για να δοκιμαστεί εάν το πρόβλημα της ανεύρεσης της στάσης
  ενός ρομπότ που είναι εξοπλισμένο με έναν πανοραμικό αισθητήρα lidar είναι
  επιλύσιμο μέσω sm2 χωρίς αντιστοιχίσεις: δοκιμάζουμε το σύστημα επίλυσης σε
  πέντε προσομοιωμένα περιβάλλοντα και ένα πραγματικό, για συνολικά 49 στάσεις,
  οι οποίες δημιουργήθηκαν είτε τυχαία είτε έτσι ώστε να δοκιμάσουν την επίδοση
  των μεθόδων sm2 που θα δοκιμαστούν σε αυτό το πρόβλημα. Οι δοκιμές στα
  προσομοιωμένα περιβάλλοντα επαναλήφθηκαν για 100 φορές ανά στάση, και στο
  πραγματικό περιβάλλον για 5 φορές ανά στάση.  Στα προσομοιωμένα περιβάλλοντα
  χρησιμοποιούμε έναν πανοραμικό αισθητήρα lidar μεγίστου βεληνεκούς δέκα
  μέτρων με θόρυβο μέτρησης κανονικά κατανεμημένο, με τιμές τυπικής απόκλισης
  ένα, δύο, και πέντε εκατοστά, ενώ στα πραγματικά πειράματα χρησιμοποιούμε
  έναν αισθητήρα YDLIDAR μέγιστου βεληνεκούς τριάντα μέτρων με κατανομή θορύβου
  μέτρησης που φαίνεται σε αυτόν τον πίνακα.}

\end{frame}
