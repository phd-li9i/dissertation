\begin{frame}{}

\begin{figure}\centering
  \resizebox{12cm}{!}{\input{./figures/slides/01/roadmap_5.pdf_tex}}
  \caption{\small Οδικός χάρτης της διατριβής}
  \label{fig:roadmap_5}
\end{figure}

\note{\footnotesize Αυτό που βλέπουμε εδώ είναι μία γραφική σύνοψη της δουλειάς
      μου. Αρχικά θα σας παρουσιάσω πώς μπορούμε να σχεδιάσουμε μία μέθοδο
      αξιολόγησης αυτόνομης πλοήγησης με τρέχουσες μεθόδους, και από εκεί θα
      οδηγηθούμε στο κύριο αντικείμενο της μελέτης μου, δηλαδή το πώς είναι
      δυνατό να μειώσουμε το σφάλμα κατάστασης παρατηρητών θέσης και
      προσανατολισμού για ρομπότ που αισθάνονται το περιβάλλον μέσω lidar, και
      συγκεκριμένα εδώ του φίλτρο σωματιδίων. Στη συνέχεια θα λύσουμε το
      πρόβλημα του global localisation για οποιοδήποτε περιβάλλον,
      χρησιμοποιώντας τον μοχλό στον οποίο βρίσκεται η συμβολή αυτής της
      διατριβής. Αυτό θα μας οδηγήσει στην εφεύρεση γενικότερων λύσεων, που
      μπορούν να χρησιμοποιηθούν και κατά τη διάρκεια της κίνησης του αισθητήρα
      σε γνωστό περιβάλλον. Στο τέλος θα γενικεύσουμε αυτές τις μεθόδους όταν
      δεν γνωρίζουμε καν δηλαδή το περιβάλλον του, για παράδειγμα για την
      εκτίμηση της οδομετρίας ενός οχήματος.}

\end{frame}
