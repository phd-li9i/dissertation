\begin{frame}{Το έργο RELIEF: η λύση}


\noindent\makebox[\linewidth][c]{%
\begin{minipage}{\linewidth}
  \begin{minipage}{0.5\linewidth}
    \begin{figure}
      \includegraphics[height=101pt,width=180pt]{./figures/slides/01/relief_concept.png}
      %\caption{}
    \end{figure}
  \end{minipage}
  \hfill
 \begin{minipage}{0.5\linewidth}
 \begin{itemize}
   \item Τοποθέτηση RFID ετικετών σε προϊνόντα
   \item Αυτόνομα επίγεια οχήματα με RFID αναγνώστες
 \end{itemize}
  \end{minipage}
\end{minipage}
}

\note{\footnotesize Το έργο RELIEF είχε ως στόχο την κατασκευή μίας σειράς από
  αυτόνομα ρομπότ τα οποία θα μπορούσαν να καταγράφουν το απόθεμα και να
  εκτιμούν τη θέση των εμπορευμάτων μέσω τεχνολογίας RFID, ώστε αυτές οι
  ενέργειες να γίνονται ακόμα και σε καθημερινή βάση, με ελάχιστη εμπλοκή
  ανθρώπων.

  Η πρώτη απαίτηση που τέθηκε για τα επίγεια ρομπότ του έργου ήταν να είναι
  ικανά να πλοηγούνται αυτόνομα στο χώρο. Σε πραγματικές συνθήκες μπορείτε να
  φανταστείτε πως προτού κλείσει η αποθήκη το βράδυ, ή ακόμα και κατά τη
  διάρκεια μίας εργάσιμης μέρας, στο ρομπότ δίνεται η εντολή να περάσει από
  συγκεκριμένα σημεία του χώρου ώστε να σαρώσει όλα τα ράφια στα οποία υπάρχουν
  αντικείμενα.  Η αυτονομία της πλοήγησης αφαιρεί την απαίτηση για ακριβό
  εξωτερικό εξοπλισμό πάνω στον οποίο θα μπορούσε να οδηγηθεί το ρομπότ, ενώ
  ταυτόχρονα το κάνει ικανό να μπορεί να εκτελεί τις ενέργειές του απρόσκοπτα
  ενώ γύρω του υπάρχουν κινούμενα εμπόδια όπως άνθρωποι ή μηχανήματα.}

\end{frame}
