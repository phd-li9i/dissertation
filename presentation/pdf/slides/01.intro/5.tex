\begin{frame}{Το έργο RELIEF: motivation}



\noindent\makebox[\linewidth][c]{%
\begin{minipage}{\linewidth}
 \hspace{0.5cm}
 \begin{minipage}{0.4\linewidth}
  Αποθήκες προϊόντων: ανάγκη για

  \begin{itemize}
    \item συνεχή απογραφή
    \item γνώση θέσης προϊόντων
  \end{itemize}
  \end{minipage}
  \hspace{0.5cm}
  \begin{minipage}{0.5\linewidth}
    \begin{figure}
      \includegraphics[height=101pt,width=180pt]{./figures/slides/01/relief_warehouse.jpg}
      \caption{Πηγή: BBC, \textit{Amazon sellers hit by 'extensive' fraud campaign}, \footnotesize\url{https://www.bbc.com/news/technology-48215073}}
    \end{figure}
  \end{minipage}
\end{minipage}
}

\note{\footnotesize Η ερευνα μου ξεκιναει με το εργο RELIEF. Ο σκοπος του εργου
  ηταν ο εξης. Στην αγορά λιανικών προϊόντων υπάρχουν εταιρείες που πουλάνε τα
  προϊόντα τους σε καταστήματα, και των οποίων το συνολικό απόθεμα αποθηκεύεται
  σε κεντρικές αποθήκες. Εν γένει αυτές οι εταιρείες θα ήθελαν να γνωρίζουν σε
  καθημερινή βάση το απόθεμά που βρίσκεται στις αποθήκες τους, όμως αυτό είναι
  τόσο κοστοβόρο που μπορούν να μετρούν το απόθεμά τους μόνο λίγες φορές μέσα
  σε ένα οικονομικό έτος. Ύστερα θα ήθελαν επίσης να γνωρίζουν τις θέσεις των
  προϊόντων μέσα σε ένα κατάστημα ή μία αποθήκη τόσο για λόγους γρήγορης
  ανάκτησης όσο και για λόγους που τους επιβάλλονται από τρίτα μέρη, γιατί πχ
  υπάρχουν συμφωνίες που επιβάλλουν την τοποθέτηση των προϊόντων σε
  συγκεκριμένες θέσεις και ύψη μέσα σε ένα κατάστημα.}

\end{frame}
