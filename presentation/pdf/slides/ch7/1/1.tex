\begin{frame}{Πειραματική διαδικασία}



  \noindent\makebox[\linewidth][c]{%
  \begin{minipage}{\linewidth}
    \begin{minipage}{0.35\linewidth}\scriptsize
      \begin{bw_box}
        Στόχοι:
        \begin{itemize}
          \item[(Σ1)] $\|\hat{\bm{q}}^{\ \texttt{fsm}}-\bm{q}\| \stackrel{?}{\leq} \|\hat{\bm{q}}^{\ \texttt{sota}}-\bm{q}\|$
          \item[(Σ2)] $\overline{t}_{\text{exec}}^{\texttt{\ fsm}} \leq 50.0$ ms
        \end{itemize}
      \end{bw_box}
      \vspace{0.25cm}
      Πέντε benchmark περιβάλλοντα δοκιμής
      \vspace{-0.05cm}
      \begin{table}[h]\centering
        \begin{tabular}{lc}
        Σύνολο δεδομένων $D_d$  & Πληθικότητα         \\  \toprule
        \texttt{aces}           & $7373$              \\
        \texttt{fr079}          & $4933$              \\
        \texttt{intel}          & $13630$             \\
        \texttt{mit\_csail}     & $1987$              \\
        \texttt{mit\_killian}   & $17479$             \\ \midrule
                                & $\sum |D_d| = 45402$      \\  \bottomrule
        \end{tabular}
        \caption{\tiny Σύνολα δεδομένων αξιολόγησης, Department of Computer Science, University of Freiburg, \url{http://ais.informatik.uni-freiburg.de/slamevaluation/datasets.php}}
      \end{table}
    \end{minipage}
    \hfill
    \begin{minipage}{0.56\linewidth}
      \vspace{-0.7cm}
      \begin{bw_box}
      \tiny

        Τυπική απόκλιση θορύβου μέτρησης\vspace{-0.25cm}
      \begin{align}
        W_R \sim \mathcal{N}(0,\sigma_R^2 &= \{0.01^2, 0.03^2, 0.05^2, 0.1^20, 0.20^2\}) \text{ [m, m}^2] \nonumber
      \end{align}
        Παραγωγή τυχαίων αρχικών συνθηκών σφάλμάτων στάσης\vspace{-0.25cm}
      \begin{align}
        \Delta \hat{x}_0,\Delta \hat{y}_0     &\sim U(-\overline{\delta}_{xy},+\overline{\delta}_{xy}) \text{ [m]} \nonumber \\
        \Delta\hat{\theta}_0 &\sim U(-\overline{\delta}_{\theta},+\overline{\delta}_{\theta} )  \text{ [rad]}\nonumber
      \end{align}
        \vspace{-0.8cm}
      \begin{table}\centering
        \begin{tabular}{ll}
        Συντομογραφία         & Διαμόρφωση (m,rad)                                                      \\  \toprule
        $\Delta_0$            & $(\overline{\delta}_{xy}, \overline{\delta}_{\theta}) = (0.05, 0.034)$  \\
        $\Delta_1$            & $(\overline{\delta}_{xy}, \overline{\delta}_{\theta}) = (0.10, 0.070)$  \\
        $\Delta_2$            & $(\overline{\delta}_{xy}, \overline{\delta}_{\theta}) = (0.15, 0.150)$  \\
        $\Delta_3$            & $(\overline{\delta}_{xy}, \overline{\delta}_{\theta}) = (0.20, 0.300)$  \\
        $\Delta_4$            & $(\overline{\delta}_{xy}, \overline{\delta}_{\theta}) = (0.20, 0.560)$  \\
        $\Delta_5$            & $(\overline{\delta}_{xy}, \overline{\delta}_{\theta}) = (0.20, \pi/4)$  \\  \bottomrule
        \end{tabular}
      \end{table}

      Συνολικός αριθμός ευθυγραμμίσεων ανά μέθοδο:

      $10 \times \sum |D_k| \times |\sigma_R| \times |\Delta| \simeq 1.36 \cdot 10^7$\\

      Μέγεθος σαρώσεων: $N_s = 360$\\

        $\nu \in [\nu_{\min}, \nu_{\max}] = [0,3]$ \ ($\nu^{\ \texttt{fsm2}} \in [2,5]$) \\

        $I_T = 5(1+\nu)$

      \end{bw_box}

      \vspace{-0.5cm}

      \scriptsize
      \begin{center}

      \end{center}
    \end{minipage}
  \end{minipage}
  }



\note{\footnotesize
οπως και πριν, δοκιμάζουμε πέντε επίπεδα θορύβου μέτρησης, όπως αυτά ορίζονται
από εμπορικά διαθέσιμους αισθητήρες. Οι δύο σαρώσεις που συλλαμβάνουμε σε κάθε
περιβάλλον διαφέρουν στη θέση και τον προσανατολισμό κατά ποσότητες που
εξάγονται όπως και πριν από ομοιόμορφες κατανομές. Εδώ ακολουθούμε την
τυποποίηση του censi που έφτιαξε τον plicp, και δοκιμάζουμε έξι διαφορετικά
επίπεδα μετατόπισης, ξεκινώντας από μικρές τιμές, δηλαδή όταν ο αισθητήρας
κινείται με χαμηλές ταχύτητες, και προοδευτικά τις αυξάνουμε για να
προσομοιώσουμε όλο και μεγαλύτερες ταχύτητες κίνησης. Κάθε πείραμα επαναλήφθηκε
δέκα φορές σε κάθε περιβάλλον με τυχαίες αρχικές συνθήκες στάσης και
μετατόπισης, με αποτέλεσμα κάθε μέθοδος να ευθυγραμμίζει περίπου 13 εκατομμύρια
ζευγάρια σαρώσεων. Όπως και πριν υποθέτουμε πανοραμικές σαρώσεις 360 ακτίνων,
αλλά σε αντίθεση με τον fsm2 εδώ έχουμε ελαττώσει το ρυθμό γωνιακής
δειγματοληψίας για να ελαττώσουμε αντίστοιχα το χρόνο εκτέλεσης.}

\end{frame}
