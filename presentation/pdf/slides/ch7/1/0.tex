\begin{frame}{Πειραματική διαδικασία}



  \noindent\makebox[\linewidth][c]{%
  \begin{minipage}{\linewidth}
    \begin{minipage}{0.35\linewidth}\scriptsize
      \begin{bw_box}
        Στόχοι:
        \begin{itemize}
          \item[(Σ1)] $\|\hat{\bm{q}}^{\ \texttt{fsm}}-\bm{q}\| \stackrel{?}{\leq} \|\hat{\bm{q}}^{\ \texttt{sota}}-\bm{q}\|$
          \item[(Σ2)] $\overline{t}_{\text{exec}}^{\texttt{\ fsm}} \leq 50.0$ ms
        \end{itemize}
      \end{bw_box}
      \vspace{0.25cm}
      Πέντε benchmark περιβάλλοντα δοκιμής
      \vspace{-0.05cm}
      \begin{table}[h]\centering
        \begin{tabular}{lc}
        Σύνολο δεδομένων D    & Πληθικότητα         \\  \toprule
        \texttt{aces}         & $7373$              \\
        \texttt{fr079}        & $4933$              \\
        \texttt{intel}        & $13630$             \\
        \texttt{mit\_csail}   & $1987$              \\
        \texttt{mit\_killian} & $17479$             \\ \midrule
                              & $\sum |D| = 45402$      \\  \bottomrule
        \end{tabular}
        \caption{\tiny Σύνολα δεδομένων αξιολόγησης, Department of Computer Science, University of Freiburg, \url{http://ais.informatik.uni-freiburg.de/slamevaluation/datasets.php}}
      \end{table}
    \end{minipage}
    \hfill
    \begin{minipage}{0.56\linewidth}
%      \vspace{-0.7cm}
      %\begin{bw_box}
      %\tiny

        %Τυπική απόκλιση θορύβου μέτρησης\vspace{-0.25cm}
      %\begin{align}
        %\sigma_R &= \{0.01, 0.03, 0.05, 0.10, 0.20\} \text{ [m]} \nonumber
      %\end{align}
        %Παραγωγή τυχαίων αρχικών συνθηκών σφάλμάτων στάσης\vspace{-0.25cm}
      %\begin{align}
%        \Delta \hat{x}_0     &\sim U(-\overline{\delta}_{xy},+\overline{\delta}_{xy}) \text{ [m]} \nonumber \\
        %\Delta \hat{y}_0     &\sim U(-\overline{\delta}_{xy},+\overline{\delta}_{xy}) \text{ [m]}\nonumber \\
        %\Delta\hat{\theta}_0 &\sim U(-\overline{\delta}_{\theta},+\overline{\delta}_{\theta} )  \text{ [rad]}\nonumber
      %\end{align}
        %\vspace{-0.8cm}
      %\begin{table}\centering
        %\begin{tabular}{ll}
        %Συντομογραφία         & Διαμόρφωση (m,rad)                                                      \\  \toprule
        %$\Delta_0$            & $(\overline{\delta}_{xy}, \overline{\delta}_{\theta}) = (0.05, 0.035)$  \\
        %$\Delta_1$            & $(\overline{\delta}_{xy}, \overline{\delta}_{\theta}) = (0.10, 0.070)$  \\
        %$\Delta_2$            & $(\overline{\delta}_{xy}, \overline{\delta}_{\theta}) = (0.15, 0.150)$  \\
        %$\Delta_3$            & $(\overline{\delta}_{xy}, \overline{\delta}_{\theta}) = (0.20, 0.300)$  \\
        %$\Delta_4$            & $(\overline{\delta}_{xy}, \overline{\delta}_{\theta}) = (0.20, 0.560)$  \\
        %$\Delta_5$            & $(\overline{\delta}_{xy}, \overline{\delta}_{\theta}) = (0.20, \pi/4)$  \\  \bottomrule
        %\end{tabular}
      %\end{table}

      %Συνολικός αριθμός ευθυγραμμίσεων ανά μέθοδο:

      %$10 \times \sum |D| \times |\sigma_R| \times |\Delta| \simeq 13.6 \cdot 10^6$\\

      %Μέγεθος σαρώσεων: $N_s = 360$\\

      %$\nu \in (\nu_{\min}, \nu_{\max}) = (0,3)$ \\

      %$I_T = 1+5\nu$

      %\end{bw_box}

      %\vspace{-0.5cm}

      %\scriptsize
      %\begin{center}

      %\end{center}
    \end{minipage}
  \end{minipage}
  }



\note{\footnotesize
Για να εξετάσουμε εάν είναι δυνατή η μετατροπή της λύσης που δώσαμε στο
πρόβλημα sm2 σε λύση του προβλήματος scan-matching διεξάγουμε πειράματα στα
οποία οι στόχοι μας είναι δύο: πρώτον θέλουμε να ανακαλύψουμε ποιά είναι η
σχέση των σφαλμάτων των μετασχηματισμών του fsm σε σχέση με αυτά των μεθόδων
της βιβλιογραφίας, και δεύτερον ο fsm θα πρέπει να εκτελείται σε πραγματικό
χρόνο. Εδω χρησιμοποιούμε και πάλι τα ίδια πέντε σύνολα περιβαλλόντων όπως και
πριν, και,}

\end{frame}
