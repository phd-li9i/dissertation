\begin{frame}{}

  \vspace{1cm}
  \begin{figure}\centering
    \input{./figures/slides/ch7/experiments/position_errors.tex}
  \end{figure}

\note{\footnotesize
Σε αυτή τη διαφάνεια βλέπουμε τις κατανομές των τελικών σφαλμάτων θέσης. Στην
άνω σειρά βρίσκονται τα αποτελέσματα για το χαμηλότερο επίπεδο μετατόπισης
μεταξύ σαρώσεων και στην κάτω αυτά για το υψηλότερο επίπεδο μετατόπισης.Κάθε
κουτί αναφέρεται σε ένα επίπεδο θορύβου μέτρησης. Θα δείτε πως σε χαμηλές
ταχύτητες ο fsm είναι ισοδύναμος και λίγο καλύτερος από την καλύτερη μέθοδο,
που εδώ είναι ο plicp, αλλά κυριαρχεί σε υψηλές ταχύτητες ιδιαίτερα όσο αυξάνει
ο θόρυβος μέτρησης.}

\end{frame}
