\begin{frame}{Βαρύτητα μετατροπής \texttt{fsm2} σε \texttt{fsm}}

  \begin{itemize}
    \item \texttt{sm} ως μέσο \texttt{sm2} $\Rightarrow$ λύση pose tracking \& global localisation
    \item \texttt{sm} ως μέσο παραγωγής οδομετρίας μέσω lidar $\Rightarrow$ απεξάρτηση από
    \begin{itemize}
      \item Αποκλίνουσα οδομετρία τροχών / άκρων
      \item Συνθήκες τριβής ως προς επιφάνεια επαφής
    \end{itemize}
    \item Πρώτη μέθοδος \texttt{sm} χωρίς υπολογισμό αντιστοιχίσεων
  \end{itemize}



\note{\footnotesize
Εάν όντως είναι εφικτή η λύση του προβλήματος sm μέσω κατάλληλων μετατροπών της
  μεθόδου fsm2 που έχουμε φτιάξει, τότε όχι μόνο θα έχουμε δώσει λύση στα δύο
  κύρια προβλήματα της εκτίμησης στάσης στη ρομποτική, που είδαμε στα δύο
  προηγούμενα κεφάλαια, αλλά θα καταφέρουμε να έχουμε παράξει μία μέθοδο
  οδομετρίας μέσω lidar. Και αυτό είναι σημαντικό γιατί η οδομετρία που μας
  παρέχεται μέσω των τροχών ή γενικά των άκρων που έρχονται σε επαφή με το
  δάπεδο αποκλίνει μέσα στο χρόνο, και εξαρτάται βαρέως από τις συνθήκες τριβής
  ανάμεσα τους.
  %Αν η λύση είναι λοιπόν εφικτή μέσω της μεθόδου fsm2 τότε αυτή θα είναι και η
  %πρώτη μέθοδος στη βιβλιογραφία που λύνει το πρόβλημα του scan matching χωρίς
  %να υπολογίζει αντιστοιχίσεις ανάμεσα στις σαρώσεις εισόδου.
}

\end{frame}
