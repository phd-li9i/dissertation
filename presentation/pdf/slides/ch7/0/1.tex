\begin{frame}{Ευθυγράμμιση \texttt{sm}}

  \definecolor{g}{RGB}{0 150 0}
  \definecolor{r}{RGB}{180 0 0}

  \noindent\makebox[\linewidth][c]{%
  \begin{minipage}{\linewidth}
    \begin{minipage}{0.5\linewidth}

      Πρόβλημα:\\
      \vspace{0.5cm}

      Κατασκευή $h$: \texttt{sm}, δεδομένων:
      \begin{itemize}
        \item Πραγματική σάρωση $\mathcal{S}_0(\bm{p}_0)$: FOV = $360^\circ$
        \item Πραγματική σάρωση $\mathcal{S}_1(\bm{p}_1)$: FOV = $360^\circ$
        \item $\bm{p}_0$ γνωστή (αυθαίρετη)
      \end{itemize}

      \vspace{0.5cm}
      τέτοιας ώστε
      \begin{itemize}
        \item $\bm{q}^{-1} = h(\mathcal{S}_0, \mathcal{S}_1)$
        \item $\bm{p}_1 = \bm{q}^{-1}\cdot \bm{p}_0$
      \end{itemize}
    \end{minipage}
    %\hfill
    \hspace{0.8cm}
    \begin{minipage}{0.4\linewidth}
      \begin{figure}\centering
        \resizebox{6cm}{!}{\input{./figures/slides/ch7/sm.eps_tex}}
      \end{figure}
    \end{minipage}
  \end{minipage}}






\note{\footnotesize
Σας θυμίζω πως το πρόβλημα του sm υποθέτει πως έχουμε στη διάθεσή μας μόνο
δύο πραγματικές σαρώσεις που έχουν συλληφθεί από δύο διαφορετικές στάσεις, και
ο στόχος της λύσης του είναι η εύρεση του μετασχηματισμού εκείνου που εάν
εφαρμοσθεί στην πρώτη στάση μας δίνει τη δεύτερη σε κάποιο σύστημα αναφοράς.}

\end{frame}
