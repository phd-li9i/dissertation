\begin{frame}{Προκλήσεις μετατροπής \texttt{fsm2} σε \texttt{fsm}}

  \definecolor{r}{RGB}{208,2,27}
  \definecolor{b}{RGB}{51,101,152}
  \definecolor{m}{RGB}{255 0 255}

  \begin{itemize}
    \item $t_{\text{exec}}^{\texttt{\ sm}} \leq \dfrac{1.0}{20\text{ Hz}} = 50 \text{ ms}$ \ \ ($\overline{t}_{\text{exec},\min}^{\texttt{\ fsm2}} = \overline{t}_{\text{exec}}^{\texttt{\ fm}} \simeq 100 \text{ ms}$)
    \item $\mathcal{S}_V(\hat{\bm{p}}_0)$ ατελής προσέγγιση του χάρτη $\bm{M} \Rightarrow$ απαίτηση ευρωστίας σε ``\textcolor{m}{κενές αντιστοιχίσεις}"
          \vspace{0.3cm}
          \begin{figure}\centering
            \resizebox{4cm}{!}{\input{./figures/slides/ch7/sv_m_discrepancy.eps_tex}}
            \begin{textblock}{14}(-1.3,9.8)
              $\textcolor{b}{\mathcal{S}_V(\hat{\bm{p}}_0)}$
            \end{textblock}
            \begin{textblock}{14}(4.1,9.8)
              $\textcolor{b}{\mathcal{S}_V(\hat{\bm{p}}_N)}$
            \end{textblock}
            \begin{textblock}{14}(1.4,7.0)
              $\textcolor{r}{\mathcal{S}_R(\bm{p})}$
            \end{textblock}
            \begin{textblock}{14}(1.6,9.45)
              \tiny \texttt{fsm2}
            \end{textblock}
          \end{figure}
  \end{itemize}



\note{\footnotesize
  Στο πρόβλημα του scan matching, η τελική μέθοδος, την οποία θα ονομάσουμε
  fsm, έχει ξεκάθαρες απαιτήσεις χρόνου εκτέλεσης. Θα πρέπει να εκτελείται σε
  χρόνο μικρότερο από τον ελάχιστο χρόνο ανανέωσης μετρήσεων, ο οποίος για τους
  εμπορικά διαθέσιμους αισθητήρες είναι πεντήντα ms, και η πρόκληση εδώ είναι η
  μείωση του χρόνου εκτέλεσης της πιο γρήγορης έκδοσης του fsm2, η οποία έχει
  χρόνο εκτέλεσης 100 ms. Ακόμα πιο πολύ όμως, η πρόκληση εδώ απευθύνεται στην
  ευρωστία της μεθόδου ευθυγράμμισης στις κενές αντιστοιχίσεις. Στην εικόνα εδώ
  βλέπουμε την αρχική και τελική διαμόρφωση σε μία επιτυχή λύση του προβλήματος
  sm2, όπου από την τελική εκτίμηση, η εικονική σάρωση έχει την ίδια μορφή με
  την πραγματική, ακριβώς επειδή υπάρχει γνώση του χάρτη του περιβάλλοντος.
  Στο sm όμως, εάν μετατρέψουμε την πρώτη σάρωση ως τον χάρτη μέσα στον οποίο
  θα εφαρμόσουμε την τεχνική sm2, τότε στο τέλος είναι δυνατόν η ευθυγράμμιση
  να αποτύχει, λόγω του γεγονότος ότι καμία εικονική σάρωση δεν θα φτάσει να
  έχει τη μορφή της πραγματικής, αφού ο χάρτης τώρα δεν περιγράφει το
  περιβάλλον στο σύνολό του, αλλά το περιβάλλον από μία στάση, και λόγω του
  γεγονότος ότι ο fsm2 δεν υπολογίζει αντιστοιχίσεις ώστε να απορρίψει στους
  υπολογισμούς του περιοχές ή σημεία της μίας σάρωσης που δεν αντιστοιχούν στην
  άλλη.}

\end{frame}
