\begin{frame}{Ευθυγράμμιση \texttt{sm2} δίχως αντιστοιχίσεις, υπό χρονικούς περιορισμούς}

  \definecolor{g}{RGB}{0 150 0}
  \definecolor{r}{RGB}{180 0 0}



  Πρόβλημα:\\
  \vspace{0.5cm}

  Κατασκευή $h$: \texttt{sm2} χωρίς αντιστοιχίσεις, δεδομένων:
  \begin{itemize}
    \item Πραγματική σάρωση $\mathcal{S}_R(\bm{p})$: FOV = $360^\circ$
    \item Χάρτης $\bm{M}$ του περιβάλλοντος
    \item Εκτίμηση $\textcolor{r}{\hat{\bm{p}}}(\hat{\bm{l}}, \hat{\theta})$
    \item Η εκτίμηση θέσης $\hat{\bm{l}} = (\hat{x},\hat{y})$ είναι σε μία γειτονιά της $\bm{l} = (x,y)$
  \end{itemize}

  \vspace{0.5cm}
  τέτοιας ώστε
  \begin{itemize}
    \item[(Σ1)] $\textcolor{g}{\hat{\bm{p}}^\prime} \leftarrow h(\mathcal{S}_R, \bm{M}, \textcolor{r}{\hat{\bm{p}}})$:
          \begin{align}
            \|\textcolor{g}{\hat{\bm{p}}^\prime} - \bm{p}\| < \|\textcolor{r}{\hat{\bm{p}}} - \bm{p}\| \nonumber
          \end{align}
    \item[(ΣΤ)] $f_{\text{exec}}(h) \geq f_{\text{exec}}(\texttt{pf})$
  \end{itemize}



\note{\footnotesize
Για την ακρίβεια εδώ θα ήθελα να θέσω το πρόβλημα το οποίο θα επιχειρήσουμε
να λύσουμε με λεπτομέρεια. Το πρόβλημα είναι η κατασκευή μίας συνάρτησης η
οποία λύνει το πρόβλημα sm2 δεδομένων μίας πανοραμικής σάρωσης δύο διαστάσεων,
του χάρτη του περιβάλλοντος, και μίας εκτίμησης της στάσης του αισθητήρα,
της οποίας η θέση βρίσκεται σε μία γειτονιά της πραγματικής του θέσης,
τέτοια ώστε η εκτίμηση που παράγεται από την συνάρτηση να έχει μικρότερο σφάλμα
εκτίμησης από αυτό της εκτίμησης εισόδου, και που να εκτελείται σε χρόνο που
να συμβαδίζει με ρυθμό παραγωγής εκτιμήσεων από τη μέθοδο που τις παράγει,
δηλαδή για παράδειγμα ένα φίλτρο σωματιδίων.}



\end{frame}
