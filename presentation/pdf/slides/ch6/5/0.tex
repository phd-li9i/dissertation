\begin{frame}{Πειραματική διαδικασία}



  \noindent\makebox[\linewidth][c]{%
  \begin{minipage}{\linewidth}
    \begin{minipage}{0.35\linewidth}\scriptsize
      \begin{bw_box}
        Στόχοι:
        \begin{itemize}
          \item[(Σ1)] $\|\hat{\bm{p}}^{\prime}-\bm{p}\| < \|\hat{\bm{p}}-\bm{p}\|$
          \item[(Σ2)] $f_{\text{exec}}^{\texttt{\ fsm2}} \geq f_{\text{exec}}^{\texttt{\ pf}}$
        \end{itemize}
      \end{bw_box}
      \vspace{0.25cm}
      Πέντε benchmark περιβάλλοντα δοκιμής
      \vspace{-0.05cm}
      \begin{table}[h]\centering
        \begin{tabular}{lc}
        Σύνολο δεδομένων D    & Πληθικότητα         \\  \toprule
        \texttt{aces}         & $7373$              \\
        \texttt{fr079}        & $4933$              \\
        \texttt{intel}        & $13630$             \\
        \texttt{mit\_csail}   & $1987$              \\
        \texttt{mit\_killian} & $17479$             \\ \midrule
                              & $\sum |D| = 45402$      \\  \bottomrule
        \end{tabular}
        \caption{\tiny Πηγή: Σύνολα δεδομένων αξιολόγησης SLAM, Τμήμα Επιστήμης
        των Υπολογιστών, Πανεπιστήμιο του Φράιμπουργκ}
      \end{table}
    \end{minipage}
    \hfill
    \begin{minipage}{0.56\linewidth}
%      \begin{bw_box}
      %\scriptsize
      %Τυπική απόκλιση θορύβου μέτρησης και συντεταγμένων χάρτη
      %\begin{align}
        %\sigma_R &= \{0.01, 0.03, 0.05, 0.10, 0.20\} \text{ [m]} \nonumber \\
        %\sigma_{\bm{M}}  &= \{0.0, 0.05\} \text{ [m]} \nonumber
      %\end{align}

      %Παραγωγή τυχαίων αρχικών συνθηκών σφάλμάτων στάσης
      %\begin{align}
        %\Delta \hat{x}_0     &\sim U(-0.20,+0.20) \text{ [m]} \nonumber \\
        %\Delta \hat{y}_0     &\sim U(-0.20,+0.20)  \text{ [m]}\nonumber \\
        %\Delta\hat{\theta}_0 &\sim U(-\dfrac{\pi}{4}, +\dfrac{\pi}{4})  \text{ [rad]}\nonumber
      %\end{align}

      %\end{bw_box}
    \end{minipage}
  \end{minipage}
  }

\note{\footnotesize
Για να δοκιμάσουμε την επίδοση των τριών εκδόσεων του συστήματος fsm2 ως προς
τους στόχους που έχουμε θέσει χρησιμοποιούμε πέντε benchmark datasets συνολικής
πληθικότητας 45 χιλιάδων περιβαλλόντων.}

\end{frame}
