\begin{frame}{Πειραματική διαδικασία}



  \noindent\makebox[\linewidth][c]{%
  \begin{minipage}{\linewidth}
    \begin{minipage}{0.35\linewidth}\scriptsize
      \begin{bw_box}
        Στόχοι:
        \begin{itemize}
          \item[(Σ1)] $\|\hat{\bm{p}}^{\prime}-\bm{p}\| < \|\hat{\bm{p}}-\bm{p}\|$
          \item[(Σ2)] $f_{\text{exec}}^{\texttt{\ fsm2}} \geq f_{\text{exec}}^{\texttt{\ pf}}$
        \end{itemize}
      \end{bw_box}
      \vspace{0.25cm}
      Πέντε benchmark περιβάλλοντα δοκιμής
      \vspace{-0.05cm}
      \begin{table}[h]\centering
        \begin{tabular}{lc}
        Σύνολο δεδομένων D    & Πληθικότητα         \\  \toprule
        \texttt{aces}         & $7373$              \\
        \texttt{fr079}        & $4933$              \\
        \texttt{intel}        & $13630$             \\
        \texttt{mit\_csail}   & $1987$              \\
        \texttt{mit\_killian} & $17479$             \\ \midrule
                              & $\sum |D| = 45402$      \\  \bottomrule
        \end{tabular}
        \caption{\tiny Σύνολα δεδομένων αξιολόγησης, Department of Computer Science, University of Freiburg, \url{http://ais.informatik.uni-freiburg.de/slamevaluation/datasets.php}}
      \end{table}
    \end{minipage}
    \hfill
    \begin{minipage}{0.56\linewidth}
      \vspace{-0.7cm}
      \begin{bw_box}
      \scriptsize

      Τυπική απόκλιση θορύβου μέτρησης και συντεταγμένων χάρτη
      \begin{align}
        \sigma_R &= \{0.01, 0.03, 0.05, 0.10, 0.20\} \text{ [m]} \nonumber \\
        \sigma_{\bm{M}}  &= \{0.0, 0.05\} \text{ [m]} \nonumber
      \end{align}

      Παραγωγή τυχαίων αρχικών συνθηκών σφάλμάτων στάσης
      \begin{align}
        \Delta \hat{x}_0     &\sim U(-0.20,+0.20) \text{ [m]} \nonumber \\
        \Delta \hat{y}_0     &\sim U(-0.20,+0.20)  \text{ [m]}\nonumber \\
        \Delta\hat{\theta}_0 &\sim U(-\dfrac{\pi}{4}, +\dfrac{\pi}{4})  \text{ [rad]}\nonumber
      \end{align}

      Συνολικός αριθμός ευθυγραμμίσεων ανά μέθοδο:

      $10 \times \sum |D| \times |\sigma_R| \times |\sigma_{\bm{M}}| \simeq 4.5 \cdot 10^6$\\

      Μέγεθος σαρώσεων: $N_s = 360$\\

        $\nu \in [\nu_{\min}, \nu_{\max}] = [2,5]$ \\

      $I_T = 1+\nu$

      \end{bw_box}

      \vspace{-0.5cm}

      \scriptsize
      \begin{center}

      \end{center}
    \end{minipage}
  \end{minipage}
  }



\note{\footnotesize
Για να δοκιμάσουμε τις επιδόσεις τους σε πραγματικές συνθήκες χρησιμοποιούμε
πέντε επίπεδα θορύβου μέτρησης τα οποία εμφανίζουν εμπορικά διαθέσιμοι
αισθητήρες, σε συνδυασμό με δύο επίπεδα διαφθοράς του χάρτη σε σχέση με το
περιβάλλον που αντιπροσωπευεί, και αρχικά σφάλματα θέσης και
προσανατολισμού που έρχονται από τη βιβλιογραφία. Ταυτόχρονα δοκιμάζουμε την
επίδοση μεθόδων της τρέχουσας βιβλιογραφίας, δηλαδή μεθόδων που όλες
χρησιμοποιούν αντιστοιχίσεις για να φέρουν εις πέρας το έργο της ευθυγράμμισης.
Συνολικά κάθε αλγόριθμος έτρεξε για δέκα φορές για κάθε διαμόρφωση, δηλαδή
συνολικά κάθε μέθοδος κλήθηκε περίπου 4.5 εκατομμύρια φορές.  Εδώ
χρησιμοποιούμε πανοραμικές σαρώσεις μεγέθους 360 ακτίνων, και θέτουμε τις τρεις
παραμέτρους που χρησιμοποιεί ο fsm2. Αυτό πάει να πει πως για το κομμάτι της
εκτίμησης του προσανατολισμού υπερδειγματοληπτούμε το χάρτη κατ' ελάχιστον 2
στην δευτέρα φορές και το μέγιστο κατά 2 στην πέμπτη, και πως κάθε φορά που
πραγματοποιείται εκτίμηση της θέσης, ο αριθμός των επαναλήψεων του
υποσυστήματος εκτίμησης είναι ανάλογος της ακρίβειας προσέγγισης του
προσανατολισμού.}

\end{frame}
