\begin{frame}{Εκτίμηση θέσης όταν $\hat{\theta} = \theta$}

  \begin{align}
    \hat{\bm{l}}[k+1] = \hat{\bm{l}}[k] + \bm{u}[k] \nonumber
  \end{align}

  \begin{align}
    \bm{u}[k] = \dfrac{1}{N_s}
    \begin{bmatrix}
      \cos\hat{\theta} & \sin\hat{\theta} \\\
      \sin\hat{\theta} & - \cos\hat{\theta}
    \end{bmatrix}
    \begin{bmatrix}
      X_{1,r}\big(\mathcal{S}_R, \mathcal{S}_V|_{\bm{\hat{p}}[k]}\big) \vspace{0.2cm} \\
      X_{1,i}\big(\mathcal{S}_R, \mathcal{S}_V|_{\bm{\hat{p}}[k]}\big)
    \end{bmatrix} \nonumber
  \end{align}

  \begin{align}
    X_1\big(\mathcal{S}_R, \mathcal{S}_V|_{\bm{\hat{p}}[k]}\big) &= X_{1,r}\big(\mathcal{S}_R, \mathcal{S}_V|_{\bm{\hat{p}}[k]}\big)
      + i \cdot X_{1,i}\big(\mathcal{S}_R, \mathcal{S}_V|_{\bm{\hat{p}}[k]}\big) \nonumber \\
      &= \sum\limits_{n=0}^{N_s-1}(\mathcal{S}_R[n] - \mathcal{S}_V[n]|_{\bm{\hat{p}}[k]}) \cdot e^{-i \frac{2 \pi n}{N_s}} \nonumber
  \end{align}

  \placebottom
  \tiny G. Vasiljević, D. Miklić, I. Draganjac, Z. Kovačić, P. Lista, ``High-accuracy vehicle localization for autonomous warehousing". \textit{Robotics and Computer-Integrated Manufacturing}, 2016

\note{\footnotesize
Το πρώτο υποπρόβλημα έχει μία λύση η οποία μας έρχεται από πρώτες αρχές. Αυτή
λέει πως δεδομένων των παραδοχών του προβλήματος και της επιπρόσθετης
παραδοχής ότι γνωρίζουμε τον προσανατολισμό του αισθητήρα, τότε μπορούμε να
εκτιμήσουμε τη θέση του εάν μετατρέψουμε την εκτίμηση θέσης στο διάνυσμα
κατάστασης ενός συστήματος το οποίο ανανεώνουμε επαναληπτικά με το διάνυσμα
u, το οποίο είναι συνάρτηση της διαφοράς των πρώτων όρων του μετασχηματισμού
fourier των σαρώσεων που συλλαμβάνονται από την πραγματική στάση του
αισθητήρα και την εκτιμώμενη του στάση ανά επανάληψη.}
\end{frame}
