\begin{frame}{Αποσύνθεση προβλήματος}

  \begin{itemize}
    \item Εκτίμηση θέσης $\bm{l}(x,y)$ \hspace{7.5ex} όταν $\hat{\theta} = \theta$
    \item Εκτίμηση προσανατολισμού $\theta$ \hspace{1ex} όταν $\hat{\bm{l}} = \bm{l}$
  \end{itemize}

    \begin{figure}
      

\tikzset{every picture/.style={line width=0.75pt}} %set default line width to 0.75pt

\begin{tikzpicture}[x=0.75pt,y=0.75pt,yscale=-1,xscale=1]
%uncomment if require: \path (0,542); %set diagram left start at 0, and has height of 542

%Straight Lines [id:da7972290663823844]
\draw    (280.83,65.33) -- (280.83,109.33) ;
\draw [shift={(280.83,111.33)}, rotate = 270] [color={rgb, 255:red, 0; green, 0; blue, 0 }  ][line width=0.75]    (10.93,-3.29) .. controls (6.95,-1.4) and (3.31,-0.3) .. (0,0) .. controls (3.31,0.3) and (6.95,1.4) .. (10.93,3.29)   ;
%Straight Lines [id:da2357744220142839]
\draw    (280.5,138.86) -- (280.5,167.86) ;
\draw [shift={(280.5,169.86)}, rotate = 270] [color={rgb, 255:red, 0; green, 0; blue, 0 }  ][line width=0.75]    (10.93,-3.29) .. controls (6.95,-1.4) and (3.31,-0.3) .. (0,0) .. controls (3.31,0.3) and (6.95,1.4) .. (10.93,3.29)   ;
%Straight Lines [id:da8737381615370248]
\draw    (114.5,63) -- (114.5,184.5) ;
%Straight Lines [id:da6971384120103545]
\draw    (156.5,63) -- (156.5,184.4) ;
%Straight Lines [id:da3539392704501798]
\draw    (156.5,184.5) -- (114.5,184.5) ;
%Straight Lines [id:da9268042351726278]
\draw    (280.5,196) -- (280.5,226.6) -- (462.6,226.6) -- (462.6,81.4) -- (280.5,81.4) ;
\draw [shift={(283,81.4)}, rotate = 360] [color={rgb, 255:red, 0; green, 0; blue, 0 }  ][line width=0.75]    (10.93,-3.29) .. controls (6.95,-1.4) and (3.31,-0.3) .. (0,0) .. controls (3.31,0.3) and (6.95,1.4) .. (10.93,3.29)   ;
%Straight Lines [id:da14503031560842916]
\draw    (156.5,184.1) -- (218.67,184.1) ;
\draw [shift={(220.67,184.07)}, rotate = 180] [color={rgb, 255:red, 0; green, 0; blue, 0 }  ][line width=0.75]    (10.93,-3.29) .. controls (6.95,-1.4) and (3.31,-0.3) .. (0,0) .. controls (3.31,0.3) and (6.95,1.4) .. (10.93,3.29)   ;
%Straight Lines [id:da8398648469461081]
\draw    (114.5,126.4) -- (181,126.4) ;
\draw [shift={(183,126.5)}, rotate = 180] [color={rgb, 255:red, 0; green, 0; blue, 0 }  ][line width=0.75]    (10.93,-3.29) .. controls (6.95,-1.4) and (3.31,-0.3) .. (0,0) .. controls (3.31,0.3) and (6.95,1.4) .. (10.93,3.29)   ;

% Text Node
\draw    (184,114) -- (378,114) -- (378,139) -- (184,139) -- cycle  ;
\draw (193,119) node [anchor=north west][inner sep=0.75pt]   [align=left] {Εκτίμηση Προσανατολισμού};
% Text Node
\draw (194,173) node [anchor=north west][inner sep=0.75pt]   [align=left] {$ $};
% Text Node
\draw (103,44) node [anchor=north west][inner sep=0.75pt]   [align=left] {$\displaystyle \mathcal{S}_{R}$};
% Text Node
\draw (276.33,44.57) node [anchor=north west][inner sep=0.75pt]   [align=left] {$\displaystyle \hat{p}$};
% Text Node
\draw (150,44.57) node [anchor=north west][inner sep=0.75pt]   [align=left] {$\bm{M}$};
% Text Node
\draw    (222.5,171) -- (339.5,171) -- (339.5,196) -- (222.5,196) -- cycle  ;
\draw (227.5,175) node [anchor=north west][inner sep=0.75pt]   [align=left] {Εκτίμηση Θέσης};


\end{tikzpicture}

    \end{figure}

\note{\footnotesize
Αρχικά εδώ αποσυνέθεσα ρητά το πρόβλημα σε δύο διακριτά υποπροβλήματα. Το
πρώτο είναι η εκτίμηση της θέσης του αισθητήρα δεδομένου ότι γνωρίζουμε ποιός
είναι ο προσανατολισμός του, και το δεύτερο είναι η εκτίμηση του
προσανατολισμού του δεδομένου ότι γνωρίζουμε ποιά είναι η θέση του. Η υπόθεσή
μου εδώ είναι πως εάν λυσουμε επαναληπτικά και σειριακά αυτά τα δύο
προβλήματα τότε θα καταλήξουμε να λύσουμε το ολικό πρόβλημα. Εδώ η ανάγκη για
επαναληπτικότητα προκύπτει από το γεγονός ότι δεδομένης αυτής της αποσύνθεσης
του προβλήματος, ο πραγματικός προσανατολισμός μπορεί να εκτιμηθεί μόνο από
την πραγματική θέση, και η πραγματική θέση μπορεί να εκτιμηθεί μόνο από τον
πραγματικό προσανατολισμό, αλλά επί της αρχής η στάση και η εκτίμησή της
είναι άνισες ως προς και τις δυο παραμέτρους.}


\end{frame}
