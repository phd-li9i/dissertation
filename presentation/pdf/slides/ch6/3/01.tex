\begin{frame}{Εκτίμηση προσανατολισμού όταν $\hat{\bm{l}} = \bm{l}$ (\texttt{rc\_x1}---1/3)}

  \definecolor{r}{RGB}{250 0 0}

  \begin{align}
    \hat{\theta}[k+1] = \hat{\theta}[k] + \angle \mathcal{F}_1\{\mathcal{S}_R\} - \angle\mathcal{F}_1\{\mathcal{S}|_{\bm{\hat{p}}[k]}\} \nonumber
  \end{align}

  Επίλοιπο σφάλμα:

  \begin{align}
    \phi &= \angle \mathcal{F}_1\{\mathcal{S}_V\} - \tan^{-1}\dfrac{|\mathcal{F}_1\{\mathcal{S}_V\}| \sin(\angle \mathcal{F}_1\{\mathcal{S}_V\})-\textcolor{r}{\mathbf{N_s}} |\delta| \sin(\hat{\theta} + \angle \delta)}
                                                                 {|\mathcal{F}_1\{\mathcal{S}_V\}| \cos(\angle \mathcal{F}_1\{\mathcal{S}_V\})-\textcolor{r}{\mathbf{N_s}} |\delta| \cos(\hat{\theta} + \angle \delta)} \nonumber
  \end{align}

\note{\footnotesize
Η πρώτη προκύπτει από τη μέθοδο εκτίμησης θέσης που είδαμε μόλις, εάν αντί
για γνωστό προσανατολισμό θεωρήσουμε γνωστή τη θέση του αισθητήρα.
Ανανεώνοντας την εκτίμηση προσανατολισμού με τη διαφορά των ορισμάτων των
πρώτων όρων του μετασχηματισμού fourier της πραγματικής σάρωσης και της
εικονικής σάρωσης που έχει υπολογιστεί από την εκτίμηση στάσης οδηγεί σε ένα
επίλοιπο σφάλμα προσανατολισμού $\phi$ που εξαρτάται από τον αριθμό των ακτίνων
του αισθητήρα.}

\end{frame}
