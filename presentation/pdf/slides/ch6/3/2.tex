\begin{frame}{Εκτίμηση προσανατολισμού όταν $\hat{\bm{l}} = \bm{l}$ (\texttt{rc\_uf}---3/3)}

  \definecolor{r}{RGB}{250 0 0}


  Έστω
  \begin{itemize}
    \item $\bm{P}_R, \bm{P}_V$ οι προβολές των $\mathcal{S}_R, \mathcal{S}_V$
          στο οριζόντιο επίπεδο
    \item $\bm{U} \bm{D} \bm{V}^\top = \texttt{svd}(\bm{P}_R \bm{P}_V^\top)$
    \item $\bm{S} = \text{diag}(1,\det{(\bm{U}\bm{V})})$
  \end{itemize}
  Τότε $\text{tr}(\bm{DS})$ είναι μέτρο ευθυγράμμισης ανάμεσα στα σύνολα $\bm{P}_R, \bm{P}_V$
  και
  \begin{align}
    \bm{R}^\star &= \bm{U} \bm{S} \bm{V}^\top =
                  \operatorname*{arg\,min}\limits_{\bm{R}} \|\bm{P}_R - \bm{R} \cdot \bm{P}_V\|_F^2 \nonumber
  \end{align}
   \textit{εάν $\theta$ γνωστή} [1].

   Όμως $\theta$ θεμελιωδώς άγνωστη $\Rightarrow$ περιστροφή $\bm{P}_V$ κατά
   $k\cdot\gamma$, $0 \leq k < N_s$.

   Τότε εάν $\hat{\theta}^\prime = \hat{\theta} + k^\star \gamma$,
   $k^\star = \operatorname*{arg\,min}\limits_k \text{tr}(\bm{DS})$, το επίλοιπο σφάλμα:

  \begin{align}
    \phi \leq \dfrac{\textcolor{r}{\gamma}}{2} \nonumber
  \end{align}

  \placebottom
  \tiny [1] S. Umeyama, ``Least-squares estimation of transformation parameters between two point patterns", \textit{IEEE Transactions on Pattern Analysis and Machine Intelligence}, Apr. 1991

\note{\footnotesize
Και το ίδιο ισχύει και για την τρίτη μέθοδο, η οποία μας έρχεται από τα πεδία
της κρυσταλλογραφίας και της ψυχομετρικής, η οποία στη βιβλιογραφία ονομάζεται
η μέθοδος του προκρούστη. Εδώ εάν PR και PV είναι οι προβολές των
δύο σαρώσεων στο καρτεσιανό επίπεδο, και αποσυνθέσουμε το γινόμενό των δύο
πινάκων σε ιδιάζουσες τιμές, μπορούμε να λάβουμε τον πίνακα περιστροφής που
εάν εφαρμοσθεί στον πίνακα των σημείων της εικονικής σάρωσης θα τα μετασχηματίσει
έτσι ώστε να ευθυγραμμιστούν με αυτά της πραγματικής σάρωσης με το ελάχιστο
τετραγωνικό σφάλμα. Επειδή όμως δεν γνωρίζουμε τον προσανατολισμό του
αισθητήρα, πρέπει να περιστρέψουμε τα σημεία της εικονικής σάρωσης όσες φορές
όσος είναι ο αριθμός των ακτίνων του, και να καταγράψουμε τον μέτρο
ευθυγράμμισης που προκύπτει από την αποσύνθεση. Με αυτόν τον τρόπο βρίσκουμε
το μέγιστο μέτρο ευθυγράμμισης, το οποίο και αυτό αντιστοιχεί όπως και για την
προηγούμενη μέθοδο σε μία περιστροφή κατά ακέραιο πολλαπλάσιο της γωνιακής
ανάλυσης του αισθητήρα.}

\end{frame}
