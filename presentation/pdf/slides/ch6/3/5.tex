\begin{frame}{Υπερδειγματοληψία πραγματικής σάρωσης σε γραμμικές περιοχές \tikzcmark}

  \definecolor{r}{rgb}{1 0 0}
  \definecolor{b}{rgb}{0 0.4470 0.7410}

  \noindent\makebox[\linewidth][c]{%
  \begin{minipage}{\linewidth}
    \begin{minipage}{0.3\linewidth}
      \begin{figure}
        \animategraphics[autoplay,loop]{1}{./figures/slides/ch6/oversampling/false_oversampling_}{1}{2}
      \end{figure}

      \begin{textblock}{14}(1.0,4.1)\scriptsize
        \textcolor{b}{$\mathcal{S}_R^{\text{ - -interp}}(\theta)$} \hspace{0.05cm} \textcolor{r}{$\mathcal{S}_V^{\text{ - -oversamp}}(\hat{\theta})$}
      \end{textblock}
    \end{minipage}
    \begin{minipage}{0.3\linewidth}
      %\begin{figure}
        %\animategraphics{1}[autoplay,loop]{./figures/slides/ch6/oversampling/false_oversampling_}{3}{4}
      %\end{figure}
    \end{minipage}
    \begin{minipage}{0.3\linewidth}
      %\begin{figure}
        %\animategraphics{2}[autoplay,loop]{./figures/slides/ch6/oversampling/false_oversampling_}{1}{2}
      %\end{figure}
    \end{minipage}
  \end{minipage}
  }

\note{\footnotesize

Η μόνη λύση για την περαιτέρω ελάττωση του σφάλματος προσανατολισμού χωρίς τη
χρήση αντιστοιχίσεων είναι η γωνιακή υπερδειγματοληψία. Προφανώς θα μπορούσαμε
να υπερδειγματοληπτήσουμε και την πραγματική σάρωση και το χάρτη, και δεν
θα προέκυπτε πρόβλημα σε γραμμικές περιοχές του περιβάλλοντος,}

\end{frame}
