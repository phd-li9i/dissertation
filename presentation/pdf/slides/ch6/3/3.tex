\begin{frame}{Το πρόβλημα του πεπερασμένου των ακτίνων: $\phi = f(N_s)$}

  \begin{itemize}
    \item \texttt{rc\_x1}
    \item \texttt{rc\_fm}
    \item \texttt{rc\_uf}
      \makebox(0,0){\put(0,4.7\normalbaselineskip){$\left.\rule{0pt}{2.0\normalbaselineskip}\right\}$
      επίλοιπο σφάλμα $\phi = f(N_s)$}}

  \end{itemize}


\note{\footnotesize
Ο κοινός παρονομαστής και των τριών μεθόδων είναι το μέγιστο μειονέκτημα τους,
δηλαδή ότι, σε αντίθεση με τη λύση του προβλήματος της εκτίμησης της θέσης του
αισθητήρα, δεν είναι δυνατόν να προσεγγίσουμε με αυθαίρετη ακρίβεια τον
προσανατολισμό του ακόμα και σε ιδανικές συνθήκες γιατί η ακρίβεια
και των τριών μεθόδων εξαρτάται από τον αριθμό των ακτίνων που εκπέμπει ο
αισθητήρας.}

\end{frame}
