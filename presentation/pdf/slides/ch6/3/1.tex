\begin{frame}{Εκτίμηση προσανατολισμού όταν $\hat{\bm{l}} = \bm{l}$ (\texttt{rc\_fm}---2/3)}

  \definecolor{r}{RGB}{250 0 0}


  \begin{align}
    \hat{\theta}^\prime &= \hat{\theta} + \xi \gamma, \text{ όπου} \nonumber \\
    \xi &\triangleq \operatorname*{arg\,max}\limits \mathcal{F}^{-1} \Bigg\{ \dfrac{\mathcal{F}\{\mathcal{S}_V\}^{\star} \cdot^\ast \mathcal{F}\{\mathcal{S}_R\}}{|\mathcal{F}\{\mathcal{S}_V\}| \cdot |\mathcal{F}\{\mathcal{S}_R\}|} \Bigg\}, \text{ και} \nonumber \\
    \gamma &\triangleq \dfrac{2\pi}{\textcolor{r}{N_s}} \nonumber
  \end{align}

  Επίλοιπο σφάλμα:

  \begin{align}
    \phi \leq \dfrac{\textcolor{r}{\gamma}}{2} \nonumber
  \end{align}

\note{\footnotesize
Η δεύτερη μέθοδος είναι απευθείας μετασχηματισμός της μεθόδου FMI-SPOMF που
είδαμε στο προηγούμενο κεφάλαιο εάν χρησιμοποιήσουμε εξαρχής πολική
αναπαράσταση για τις σαρώσεις και παρακάμψουμε έτσι την ανάγκη για τη
δημιουργία εικόνων των οποίων η επεξεργασία είναι δαπανηρή σε πόρους.
Ανανεώνοντας την εκτίμηση προσανατολισμού με την ποσότητα ξι επί γάμμα, όπου
γάμμα είναι η γωνιακή ανάλυση του αισθητήρα και όπου βλέπετε F σημαίνει το
μετασχηματισμό fourier της αντίστοιχης σάρωσης, έχει ως αποτέλεσμα ένα επίλοιπο
σφάλμα προσανατολισμού το οποίο εξαρτάται και αυτό από τη αριθμό ακτίνων που
εκπέμπει ο αισθητήρας.
}
\end{frame}
