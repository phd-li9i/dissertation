\begin{frame}{Ελάττωση σφάλματος εκτίμησης μέσω διαλογής σωματιδίων}

\definecolor{g}{RGB}{0 100 0}

  Εκτίμηση στάσης $\hat{\bm{p}} = [\hat{x}\ \hat{y}\ \hat{\theta}]^\top$

  \begin{align}
    p(\hat{\bm{p}}_t | \bm{z}_{1:t}, \bm{u}_{1:t}, \bm{M}) \propto
      \textcolor{g}{\underbrace{p(\bm{z}_t | \hat{\bm{p}}_t)}_\text{Μοντέλο παρατήρησης}} \cdot \int\limits_{} p(\hat{\bm{p}}_t | \hat{\bm{p}}^{\prime}, \bm{u}_{t})& \cdot p(\hat{\bm{p}}^{\prime} | \bm{z}_{1:t-1}, \bm{u}_{1:t-1}, \bm{M}) d\hat{\bm{p}}^{\prime} \nonumber
  \end{align}
  \vspace{0.5cm}

  \begin{itemize}
    \item Πληθυσμός υποθέσεων $\mathcal{P} = \{(\hat{\bm{p}}^i, w^i)\}$                                   \vspace{0.35cm}
    \item Βάρος σωματιδίου $i$: $w^i = \textcolor{g}{p(\bm{z}| \hat{\bm{p}}^i)}$                          \vspace{0.35cm}
    \item Τελική εκτίμηση: $\hat{\bm{p}}(\mathcal{P}) = \dfrac{\sum w^i \cdot \hat{\bm{p}}^i}{\sum w^i}$  \vspace{0.35cm}
  \end{itemize}


\note{\footnotesize Όσο αφορά στην εκτίμηση της στάσης ενός οχήματος, όλοι οι
  πιθανοτικοί παρατηρητές στη ρομποτική εκτιμούν τη στάση του, με βάση το
  μοντέλο παρατήρησης του αισθητήρα που φέρει το όχημα, και του κινηματικού
  μοντέλου του. Το φίλτρο σωματιδίων ειδικά, σε αντίθεση με το φίλτρο καλμαν,
  χρησιμοποιεί πολλαπλές υποθέσεις στάσης, και η τελική του εκτίμηση προκύπτει
  ως ο μέσος όρος των στάσεων αυτών των υποθέσεων, βεβαρυμμένος κατά το βάρος
  της κάθε μίας. Ως βάρος εδώ νοείται η πιθανότητα παρατήρησης μίας δεδομένης
  μέτρησης από τη στάση της κάθε υπόθεσης, και συνεπώς υποθέσεις των οποίων το
  σφάλμα εκτίμησης είναι μικρότερο από άλλες εμφανίζουν μεγαλύτερο βάρος, και
  συνεπώς επηρεάζουν περισσότερο την τελική εκτίμηση.}

\end{frame}
