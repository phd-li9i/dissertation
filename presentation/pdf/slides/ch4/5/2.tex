\begin{frame}{Τρωτά σημεία μεθόδων ευθυγράμμισης: παραμετροποίηση}

  \begin{figure}\vspace{1.75cm}
    


\tikzset{every picture/.style={line width=0.75pt}} %set default line width to 0.75pt

\begin{tikzpicture}[x=0.75pt,y=0.75pt,yscale=-1,xscale=1]
%uncomment if require: \path (0,300); %set diagram left start at 0, and has height of 300

%Straight Lines [id:da323479033101316]
%\draw    (220.25,136.5) -- (266.25,136.98) ;
%\draw [shift={(268.25,137)}, rotate = 180.6] [color={rgb, 255:red, 0; green, 0; blue, 0 }  ][line width=0.75]    (10.93,-3.29) .. controls (6.95,-1.4) and (3.31,-0.3) .. (0,0) .. controls (3.31,0.3) and (6.95,1.4) .. (10.93,3.29)   ;
%Straight Lines [id:da9037457194473959]
%\draw    (419.25,139) -- (381.75,138.53) ;
%\draw [shift={(379.75,138.5)}, rotate = 0.73] [color={rgb, 255:red, 0; green, 0; blue, 0 }  ][line width=0.75]    (10.93,-3.29) .. controls (6.95,-1.4) and (3.31,-0.3) .. (0,0) .. controls (3.31,0.3) and (6.95,1.4) .. (10.93,3.29)   ;
%Straight Lines [id:da4809821225070685]
\draw    (116.75,146) -- (116.75,267) ;
%Straight Lines [id:da7719317849229219]
\draw    (143.25,228.5) -- (117.25,228.5) ;
%Straight Lines [id:da20180846005955]
\draw    (143.25,247.5) -- (117.25,247.5) ;
\draw    (143.25,266.5) -- (117.25,266.5) ;

% Text Node
\draw    (162.09, 136.5) circle [x radius= 57.98, y radius= 14.85]   ;
\draw (162.09,136.5) node   [align=left] {Παράμετροι};
% Text Node
%\draw    (271.22,115) -- (377.22,115) -- (377.22,161) -- (271.22,161) -- cycle  ;
%\draw (324.22,138) node   [align=left] {Μηχανισμός \\αντιστοιχίσεων};
% Text Node
\draw    (465.98, 139.5) circle [x radius= 45.96, y radius= 14.85]   ;
\draw (465.98,139.5) node   [align=left] {Θόρυβος};
% Text Node
\draw (152,221) node [anchor=north west][inner sep=0.75pt]   [align=left] {\footnotesize Αντιδιαισθητική ρύθμιση};
% Text Node
\draw (152,240) node [anchor=north west][inner sep=0.75pt]   [align=left] {\footnotesize Μικρές μεταβολές $\rightarrow$ μεγάλες διαφορές λύσης};
\draw (152,259) node [anchor=north west][inner sep=0.75pt]   [align=left] {\footnotesize Τοπική ισχύς};


\end{tikzpicture}

  \end{figure}

\note{\footnotesize
Το πρόβλημα με την παραμετροποίηση που αφορά τουλάχιστον στη μέθοδο που
παρήγαγε τα αποτελέσματα μου μόλις είδαμε, η οποία είναι μάλιστα η καλύτερη
μέθοδος στη βιβλιογραφία, είναι ότι δεν είναι διαισθητική διαδικασία, ότι
μικρές μεταβολές των τιμών των παραμέτρων παράγουν δυσανάλογα μεγάλες μεταβολές
στην έξοδο, και ότι για κάποιες παραμέτρους δεν υπάρχουν τιμές που να μπορούν
να καλύψουν όλες τις στάσεις σε ένα περιβάλλον.}
\end{frame}
