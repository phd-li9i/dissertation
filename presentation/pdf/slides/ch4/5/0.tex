\begin{frame}{Τι ήταν προηγουμένως αδύνατον και τώρα είναι εφικτό}

  \vspace{2cm}
  Ελάττωση σφαλμάτων εκτίμησης στάσης φίλτρου σωματιδίων μέσω

  \begin{itemize}
    \item διαλογής σωματιδίων
    \item ανάδρασης αποτελέσματος \texttt{sm2} χωρίς απώλεια ανθεκτικότητας \texttt{pf}
  \end{itemize}



  \placebottom
  \tiny A. Filotheou, E. Tsardoulias, A. Dimitriou, A. Symeonidis, L. Petrou, ``Pose Selection and Feedback Methods in Tandem Combinations of Particle Filters with Scan-Matching for 2D Mobile Robot Localisation". \textit{Journal of Intelligent and Robotic Systems}, 2020

\note{\footnotesize Τί είναι σήμερα δυνατόν που δεν ήταν πριν από αυτή την
  έρευνα; Το κύριότερο είναι ότι σήμερα μπορούμε να εγγυηθούμε τη μείωση των
  σφαλμάτων ενός φίλτρου σωματιδίων με δύο διαφορετικούς τρόπους, και χωρίς να
  διακινδυνεύσουμε την ανθεκτικότητά του. Εδώ ο τρόπος ανάδρασης που εισάγαμε
  είναι καθοριστικός γιατί η ευθυγράμμιση πραγματικών με εικονικές σαρώσεις δεν
  είναι πάντα επιτυχής, όπως είδαμε από τα αποτελέσματα της μεθόδου ανάδρασης
  hard.  Το οποίο εγείρει το φυσικό ερώτημα: }
\end{frame}
