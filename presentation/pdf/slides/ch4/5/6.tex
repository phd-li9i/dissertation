\begin{frame}{\small Τρωτά σημεία μεθόδων ευθυγράμμισης: κενές αντιστοιχίσεις}

  \definecolor{r}{RGB}{255 69 0}
  \definecolor{b}{RGB}{51 102 153}

  \vspace{0.5cm}
  \begin{center}
  \footnotesize Ευθυγράμμιση σάρωσης \textcolor{b}{$\mathcal{S}_{k+1}$} ως προς \textcolor{r}{$\mathcal{S}_{k}$}
  \end{center}

  \noindent\makebox[\linewidth][c]{%
  \begin{minipage}{\linewidth}

    \begin{minipage}{0.45\linewidth}
      \begin{figure}
        \animategraphics[scale=0.3,autoplay]{10}{./figures/slides/ch4/sm_gif_reverse_side_by_side/icp_sigma_0/pic}{0}{50}
      \end{figure}
    \end{minipage}
    \hfill
    \begin{minipage}{0.45\linewidth}
      \begin{figure}
        \animategraphics[scale=0.3,autoplay]{10}{./figures/slides/ch4/sm_gif_reverse_side_by_side/icp_sigma_0_reverse/pic}{0}{50}
      \end{figure}
    \end{minipage}

  \end{minipage}
  }

\note{\footnotesize
Ταυτόχρονα, ακόμα και αν υποθέσουμε ιδανικές συνθήκες, η ίδια η διαδικασία
εύρεσης αντιστοιχίσεων καθίσταται πιθανά προβληματική γιατί ακόμα και αν οι
σαρώσεις έχουν εύρος 360 μοίρες, δεν υπάρχει εγγύηση ότι όλα τα σημεία της
μίας σάρωσης θα αντιστοιχούν σε όλα τα σημεία της δεύτερης, όπως βλέπουμε σε
αυτή τη διαφάνεια. Εδώ και στις δύο εικόνες ευθυγραμμίζουμε τη σάρωση με μπλε
στη σάρωση με κόκκινο, και ευθυγραμμίζουμε τις ίδιες δύο σαρώσεις, αλλά
μεταβάλουμε το ποιά σάρωση ευθυγραμμίζεται σε ποιά.  Στα δεξιά, σε αντίθεση
με την ευθυγράμμιση στα αριστερά, η σάρωση με μπλε αποτελείται από σημεία που
λείπουν απο την άλλη, και συνεπώς υπάρχουν κενές αντιστοιχίσεις, οι οποίες
οδηγούν στη αποτυχία της ευθυγραμισης}
\end{frame}
