\begin{frame}{\small Τρωτά σημεία μεθόδων ευθυγράμμισης: θόρυβος (ατελείς αντιστοιχίσεις)}

  \begin{figure}
    \animategraphics[scale=0.3,autoplay,loop]{1}{./figures/slides/ch4/correspondence_is_the_culprit/icp/icp_cor_s}{0}{1}
      \begin{textblock}{4}(3.5,3.2)
        \scriptsize $\sigma_R = 0.0$ $|$ $\sigma_R = 0.1$ m \\
      \end{textblock}
      \begin{textblock}{4}(8.7,3.2)
        \scriptsize Χώρος αντιστοιχίσεων
      \end{textblock}
  \end{figure}

\note{\footnotesize
Εδώ το πρόβλημα είναι δεν είναι μόνο ότι δυσχεραίνεται η διαδικασία διάκρισης
αληθών από ψευδείς αντιστοιχίσεις, αλλά ότι αυτή η διάκριση καθίσταται ατελής
λόγω παρουσίας θορύβου.}
\end{frame}
