\begin{frame}[fragile]{Ελάττωση σφάλματος εκτίμησης μέσω διαλογής σωματιδίων}

  Υπόθεση Υ1:
  \begin{bw_box}
    Έστω

    \vspace{0.25cm}
    $\mathcal{P} = \{(\hat{\bm{p}^i}, w^i)\}$ o συνολικός πληθυσμός.

    \vspace{0.25cm}
    $\mathcal{Q} = \texttt{sort}(\mathcal{P} | w) = \{(\hat{\bm{p}}^j, w^j)\} : w^0 \geq w^1 \geq \dots$

    \vspace{0.25cm}
    $\overline{\mathcal{Q}} \subset \mathcal{Q} :
    \overline{\mathcal{Q}} = \{(\hat{\bm{p}}^0, w^0), (\hat{\bm{p}}^1, w^1), \dots\}$
    και $|\overline{\mathcal{Q}}| < |\mathcal{Q}|$.

    \vspace{0.75cm}
    Τότε

    \vspace{0.25cm}
    $\|\bm{p}-\hat{\bm{p}}(\overline{\mathcal{Q}})\| < \|\bm{p}-\hat{\bm{p}}(\mathcal{P})\|$

    \vspace{0.05cm}
    (\rotatebox[origin=c]{180}{$\Lsh$} \scriptsize ψηφοφορία βαρύτερων σωματιδίων $\overline{\mathcal{Q}}$ αντί για ${\mathcal{P}}$ $\Rightarrow$ μικρότερο σφάλμα)
  \end{bw_box}

\note{\footnotesize
Με βάση αυτον τον συλλογισμό τυποποιούμε την πρώτη υπόθεση, η οποία εν ολίγοις
αναφέρει ότι εάν για την εξαγωγή της εκτίμησης του φίλτρου χρησιμοποιηθεί ένα
υποσύνολο των πιο βαρέων σωματιδίων αντί για όλο τον πληθυσμό, τότε
αναμένουμε η εκτίμηση του φίλτρου να έχει χαμηλότερο σφάλμα εκτίμησης.}

\end{frame}
