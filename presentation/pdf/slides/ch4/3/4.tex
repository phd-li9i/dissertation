\begin{frame}{\texttt{sm}\{$\cdot$,\texttt{2}\}: \texttt{I/O}}

  \noindent\makebox[\linewidth][c]{%
  \begin{minipage}{\linewidth}
    \begin{minipage}{0.3\linewidth}
      \begin{figure}
        
\definecolor{r}{RGB}{255 69 0}
\definecolor{b}{RGB}{51 102 153}


\tikzset{every picture/.style={line width=0.75pt}} %set default line width to 0.75pt

\begin{tikzpicture}[x=0.75pt,y=0.75pt,yscale=-1,xscale=1]
%uncomment if require: \path (0,300); %set diagram left start at 0, and has height of 300

%Straight Lines [id:da4529423263739871]
\draw   [color={rgb, 255:red, 255; green, 69; blue, 0 }  ] (61.5,52) -- (100.94,83.75) ;
\draw [shift={(102.5,85)}, rotate = 218.83] [color={rgb, 255:red, 255; green, 69; blue, 0 }  ][line width=0.75]    (10.93,-3.29) .. controls (6.95,-1.4) and (3.31,-0.3) .. (0,0) .. controls (3.31,0.3) and (6.95,1.4) .. (10.93,3.29)   ;
%Straight Lines [id:da6488911304441738]
\draw   [color={rgb, 255:red, 51; green, 102; blue, 153 }  ] (58.5,126) -- (100.71,104.89) ;
\draw [shift={(102.5,104)}, rotate = 153.43] [color={rgb, 255:red, 51; green, 102; blue, 153 }  ][line width=0.75]    (10.93,-3.29) .. controls (6.95,-1.4) and (3.31,-0.3) .. (0,0) .. controls (3.31,0.3) and (6.95,1.4) .. (10.93,3.29)   ;
%Straight Lines [id:da5938727374819561]
\draw  [color={rgb, 255:red, 255; green, 69; blue, 0 }  ]  (121.5,37) -- (122.45,75) ;
\draw [shift={(122.5,77)}, rotate = 270] [color={rgb, 255:red, 255; green, 69; blue, 0 }  ][line width=0.75]    (10.93,-3.29) .. controls (6.95,-1.4) and (3.31,-0.3) .. (0,0) .. controls (3.31,0.3) and (6.95,1.4) .. (10.93,3.29)   ;
%Straight Lines [id:da5792126615284177]
\draw    (135.5,94) -- (173.5,94) ;
\draw [shift={(175.5,94)}, rotate = 180] [color={rgb, 255:red, 0; green, 0; blue, 0 }  ][line width=0.75]    (10.93,-3.29) .. controls (6.95,-1.4) and (3.31,-0.3) .. (0,0) .. controls (3.31,0.3) and (6.95,1.4) .. (10.93,3.29)   ;

% Text Node
\draw    (107,82) -- (135,82) -- (135,107) -- (107,107) -- cycle  ;
\draw (121,94.5) node   [align=left] {\texttt{sm}};
% Text Node
\draw (35,32) node [anchor=north west][inner sep=0.75pt]   [align=left] {$\textcolor{r}{\mathcal{S}(\bm{p}_{k})}$};
% Text Node
\draw (34,127) node [anchor=north west][inner sep=0.75pt]   [align=left] {$\textcolor{b}{\mathcal{S}(\bm{p}_{k+1})}$};
% Text Node
\draw (113,17) node [anchor=north west][inner sep=0.75pt]   [align=left] {$\textcolor{r}{\bm{p}_{k}}$};
% Text Node
\draw (179,88) node [anchor=north west][inner sep=0.75pt]   [align=left] {$\textcolor{b}{\bm{p}_{k+1}}$};
\draw  [color={rgb, 255:red, 0; green, 255; blue, 0 }  ]  (195,94) circle [x radius= 16.26, y radius= 16.26]   ;


\end{tikzpicture}


        \caption{\scriptsize \texttt{sm}: ευθυγράμμιση πραγματικών σαρώσεων}
      \end{figure}
    \end{minipage}
    \hspace{1.2cm}
    \begin{minipage}{0.6\linewidth}
      \begin{figure}
        
\definecolor{r}{RGB}{255 69 0}
\definecolor{b}{RGB}{51 102 153}



\tikzset{every picture/.style={line width=0.75pt}} %set default line width to 0.75pt

\begin{tikzpicture}[x=0.75pt,y=0.75pt,yscale=-1,xscale=1]
%uncomment if require: \path (0,300); %set diagram left start at 0, and has height of 300

%Shape: Rectangle [id:dp6131376592433888]
\draw  [dash pattern={on 0.84pt off 2.51pt}] (322,78) -- (519.5,78) -- (519.5,157) -- (322,157) -- cycle ;
%Straight Lines [id:da45736430417167373]
\draw    (294.5,111) -- (325.5,111) ;
\draw [shift={(327.5,111)}, rotate = 180] [color={rgb, 255:red, 0; green, 0; blue, 0 }  ][line width=0.75]    (10.93,-3.29) .. controls (6.95,-1.4) and (3.31,-0.3) .. (0,0) .. controls (3.31,0.3) and (6.95,1.4) .. (10.93,3.29)   ;
%Straight Lines [id:da7019178260598848]
\draw   [color={rgb, 255:red, 51; green, 102; blue, 153 }  ] (293.5,172) -- (470.53,141.34) ;
\draw [shift={(472.5,141)}, rotate = 170.17] [color={rgb, 255:red, 51; green, 102; blue, 153 }  ][line width=0.75]    (10.93,-3.29) .. controls (6.95,-1.4) and (3.31,-0.3) .. (0,0) .. controls (3.31,0.3) and (6.95,1.4) .. (10.93,3.29)   ;
%Straight Lines [id:da9869567554869965]
\draw   [color={rgb, 255:red, 255; green, 69; blue, 0 }  ] (409.5,109) -- (470.56,123.51) ;
\draw [shift={(472.5,124)}, rotate = 194.04] [color={rgb, 255:red, 255; green, 69; blue, 0 }  ][line width=0.75]    (10.93,-3.29) .. controls (6.95,-1.4) and (3.31,-0.3) .. (0,0) .. controls (3.31,0.3) and (6.95,1.4) .. (10.93,3.29)   ;
%Straight Lines [id:da23755426444121874]
\draw    (397.5,37) -- (397.5,89) ;
\draw [shift={(397.5,91)}, rotate = 270] [color={rgb, 255:red, 0; green, 0; blue, 0 }  ][line width=0.75]    (10.93,-3.29) .. controls (6.95,-1.4) and (3.31,-0.3) .. (0,0) .. controls (3.31,0.3) and (6.95,1.4) .. (10.93,3.29)   ;
%Straight Lines [id:da07285087411586999]
\draw   [color={rgb, 255:red, 255; green, 69; blue, 0 }  ] (397.5,37) -- (487.97,112.72) ;
\draw [shift={(489.5,114)}, rotate = 219.93] [color={rgb, 255:red, 255; green, 69; blue, 0 }  ][line width=0.75]    (10.93,-3.29) .. controls (6.95,-1.4) and (3.31,-0.3) .. (0,0) .. controls (3.31,0.3) and (6.95,1.4) .. (10.93,3.29)   ;
%Straight Lines [id:da25729544560038]
\draw    (504.5,132) -- (543.5,132) ;
\draw [shift={(545.5,132)}, rotate = 180] [color={rgb, 255:red, 0; green, 0; blue, 0 }  ][line width=0.75]    (10.93,-3.29) .. controls (6.95,-1.4) and (3.31,-0.3) .. (0,0) .. controls (3.31,0.3) and (6.95,1.4) .. (10.93,3.29)   ;

% Text Node
\draw    (477,119) -- (505,119) -- (505,144) -- (477,144) -- cycle  ;
\draw (491,131.5) node   [align=left] {\texttt{sm}};
% Text Node
\draw (489,58) node [anchor=north west][inner sep=0.75pt]   [align=left] {\texttt{sm2}};
% Text Node
\draw (259,165) node [anchor=north west][inner sep=0.75pt]   [align=left] {$\textcolor{b}{\mathcal{S}(\bm{p})}$};
% Text Node
\draw (224,103) node [anchor=north west][inner sep=0.75pt]   [align=left] {Χάρτης $\bm{M}$};
% Text Node
\draw    (332.1,96) -- (409.1,96) -- (409.1,121) -- (332.1,121) -- cycle  ;
\draw (370.6,111) node   [align=left] {\texttt{scan\_map}};
% Text Node
\draw (393,16) node [anchor=north west][inner sep=0.75pt]   [align=left] {$\textcolor{r}{\hat{\bm{p}}}$};
% Text Node
\draw (422,95) node [anchor=north west][inner sep=0.75pt]   [align=left] {\footnotesize$\textcolor{r}{\mathcal{S}(\hat{\bm{p}})}$};
% Text Node
\draw (549,127) node [anchor=north west][inner sep=0.75pt]   [align=left] {$\textcolor{b}{\bm{p}}$};
\draw  [color={rgb, 255:red, 0; green, 255; blue, 0 }  ]  (556, 133) circle [x radius= 12.26, y radius= 12.26]   ;

\end{tikzpicture}

        \caption{\scriptsize \texttt{sm2}: ευθυγράμμιση πραγματικής σάρωσης με εικονική σάρωση χάρτη}
      \end{figure}
    \end{minipage}

  \end{minipage}
  }


\note{\footnotesize
Σε αυτό το γεγονός κρύβεται μία δεύτερη χρησιμότητα της ευθυγράμμισης σαρώσεων.
Εαν αντικαταστήσουμε τη μία από τις δύο μετρήσεις με μία εικονική σάρωση,
δηλαδή με μία σάρωση που προσομοιώνει την αρχή λειτουργίας του lidar στο χάρτη
αντί για το περιβάλλον, η οποία υπολογίζεται από την εκτίμηση της στάσης του
ρομπότ, τοτε μπορούμε να υπολογίσουμε το μετασχηματισμό ανάμεσα στην εκτίμηση
και την άγνωστη πραγματική στάση του ρομπότ, και αφού γνωρίζουμε την εκτίμηση,
μπορούμε να υπολογίσουμε την πραγματική του στάση.}


\end{frame}
