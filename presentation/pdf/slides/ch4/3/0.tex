\begin{frame}{Ελάττωση σφάλματος εκτίμησης μέσω \texttt{sm2}}

  \definecolor{b}{RGB}{51 102 153}

  \vspace{2cm}

  Ευθυγράμμιση πραγματικών σαρώσεων \textcolor{b}{με εικονικές σαρώσεις χάρτη} (\texttt{sm2}\textsuperscript{\textdaggerdbl})

  \hspace{5cm} $\subset$

  Ευθυγράμμιση πραγματικών σαρώσεων (\texttt{sm}\textsuperscript{\textdagger})

  \vspace{2cm}

  \begin{flalign}\hspace{-20cm}
    \scriptsize
    \text{\textsuperscript{\textdaggerdbl}\underline{s}can--to--\underline{m}ap-\underline{s}can \underline{m}atching} \rightarrow \texttt{smsm} & \rightarrow \scriptsize \texttt{sm2} \nonumber \\
    \scriptsize
    \text{\textsuperscript{\textdagger}\underline{s}can-\underline{m}atching} & \rightarrow  \scriptsize \texttt{sm} \nonumber
  \end{flalign}

\note{\footnotesize Ένας δέυτερος τρόπος με τον οποίον μπορεί να ελαττωθεί το
  σφάλμα ενός παρατηρητή είναι μέσω ευθυγράμμισης πραγματικών με εικονικές
  σαρώσεις αισθητήρα lidar. Αυτή η μέθοδος θα μας ακολουθήσει σε όλη την
  υπόλοιπη έρευνά μου, και για λόγους οικονομίας θα αναφέρομαι σε αυτήν
  συντομογραφικά ως sm2. Η τεχνική sm2 είναι υποσύνολο της γενικής μεθόδου
  ευθυγράμμισης πραγματικών σαρώσεων, στην οποία θα αναφέρομαι ως σκέτο sm,
  εκ της φράσεως scan-matching.}

\end{frame}
