\begin{frame}{Ελάττωση σφάλματος εκτίμησης μέσω ανάδρασης εξόδου \texttt{sm2}}

\definecolor{r}{RGB}{215 0 0}
\definecolor{g}{RGB}{0 155 0}

  \begin{itemize}
    \item<1-> Ανάδραση με τη μορφή μοναδικής υπόθεσης \\$\rightarrow$ \textcolor{r}{αργή σύγκλιση / αμελητέα συμβολή}
    \item<2-> Εξ ολοκλήρου αρχικοποίηση του φίλτρου \\$\rightarrow$ \textcolor{r}{απώλεια ανθεκτικότητας σε περίπτωση αποτυχίας \texttt{sm2}}
    \item<3-> Ανάδραση με τη μορφή πολλαπλών υποθέσεων \\$\rightarrow$ \textcolor{g}{γρήγορη σύγκλιση και διατήρηση ανθεκτικότητας}
  \end{itemize}

\note{\footnotesize
Κατά μία μέθοδο της βιβλιογραφίας, η υπόθεση που παράγεται μέσω sm2 εισάγεται
  στον πληθυσμό του φίλτρου ως \textbf{ένα}, \textbf{διακριτό} σωματίδιο, το
  οποίο σε έναν πληθυσμό αρκετών εκατοντάδων σωματιδίων έχει μεν επίδραση, αλλά
  σχεδόν αμελητέα. <CLICK>\\}

\note{\footnotesize
Κατά μία άλλη μέθοδο η εκτίμηση της sm2 αντικαθιστά τον πληθυσμό του φίλτρου
στο σύνολό του.  Η επιτυχία αυτής της μεθόδου εξαρτάται αποκλειστικά από την
επιτυχία της ευθυγράμμισης, η οποία δεν είναι εγγυημένη, και μπορεί να έχει ως
αποτέλεσμα την καταστροφική αποτυχία της συνέχειας της εκτίμησης της στάσης του
ρομπότ, και συνεπώς και της ασφάλειας της πλοήγησης. <CLICK>\\}

\note{\footnotesize
Μπορούμε να ελαττώσουμε το σφάλμα εκτίμησης χωρίς να επιφέρουμε αυτές
τις παρενέργειες εάν εισάγουμε το αποτέλεσμα της ευθυγράμμισης στον πληθυσμό
του φίλτρου ως πολλαπλά σωματίδια αλλά σε πληθικότητα μικρότερη από τον
πληθυσμό του φίλτρου. Τότε αυτό θα έχει θεωρητικά ως αποτέλεσμα πιο γρήγορη
σύγκλιση σε σχέση με την περίπτωση που το εισάγαμε ως ένα σωματίδιο, χωρίς
ταυτόχρονα να διακινδυνεύεται σε περίπτωση αποτυχίας της ευθυγράμμισης η
ανθεκτικότητα του φίλτρου, γιατί το ίδιο θα συνεχίσει να περιέχει υποθέσεις που
εξηγούν καλύτερα τις μετρήσεις από αυτές που έχουν καταλήξει σε λανθασμένες
τοποθεσίες. <CLICK>\\}


\end{frame}
