\begin{frame}[fragile]{Ελάττωση σφάλματος εκτίμησης μέσω ανάδρασης εξόδου \texttt{sm2}}

  Έστω $\mathcal{P} = \{(\hat{\bm{p}_i}, w_i)\}$ o συνολικός πληθυσμός του φίλτρου και

  $\hat{\bm{p}}^\star \leftarrow \texttt{sm2}(\mathcal{S}(\bm{p}), \bm{M}, \hat{\bm{p}}(\mathcal{P}))$. Τότε \vspace{0.3cm}

  Υπόθεση Υ2: \vspace{-0.6cm}
  \begin{bw_box}
    $\|\bm{p} - \hat{\bm{p}}^\star\| < \|\bm{p} - \hat{\bm{p}}(\mathcal{P})\|$
  \end{bw_box} \vspace{0.4cm}

  Υπόθεση Υ3: \vspace{-0.6cm}
  \begin{bw_box}

    Εάν $\mathcal{P}^\circlearrowleft = \mathcal{P} \circlearrowleft \{\hat{\bm{p}}^\star\}_{q \cdot |\mathcal{P}^\circlearrowleft|}$,
    όπου $q \gg 0.01$ και $q \ll 1.0$:

    \begin{itemize}
      \item $\|\bm{p} - \hat{\bm{p}}(\mathcal{P}^\circlearrowleft)\|                          < \|\bm{p} - \hat{\bm{p}}(\mathcal{P})\| $
      \item $\|\bm{p} - \hat{\bm{p}}(\mathcal{P}^\circlearrowleft)\| < \|\bm{p} - \hat{\bm{p}}(\mathcal{P} \circlearrowleft \{\hat{\bm{p}}^\star\}_1)\|$
      \item $\mathcal{P} \circlearrowleft \{\hat{\bm{p}}^\star\}_{q \cdot |\mathcal{P}^\circlearrowleft|}$ πιο ανθεκτικός από $\mathcal{P} \circlearrowleft \{\hat{\bm{p}}^\star\}_{|\mathcal{P}^\circlearrowleft|}$
    \end{itemize}

  \end{bw_box}


\note{\footnotesize
  Με βάση αυτά συντάσσουμε άλλες δύο υποθέσεις. Η πρώτη είναι ότι το αποτέλεσμα
  της ευθυγράμμισης της πραγματικής σάρωσης από το lidar και της εικονικής
  σάρωσης από την εκτιμώμενη στάση του φίλτρου έχει μικρότερο σφάλμα από την
  εκτίμηση του φίλτρου.  Η δεύτερη αφορά στην ανάδραση αυτής της εκτίμησης, και
  λέει ότι εάν το αποτέλεσμα της ευθυγράμμισης εισαχθεί στον πληθυσμό του
  φίλτρου ως μία πλειάδα σωματιδίων, τότε, εάν στέκει η προηγούμενη υπόθεση, το
  σφάλμα εκτίμησης θα είναι χαμηλότερο από τον σφάλμα του ονομαστικού φίλτρου,
  χαμηλότερο από το σφάλμα του φίλτρου εάν η υπόθεση εισάγετο ως μόνο ένα
  σωματίδιο, και ο πληθυσμός του φίλτρου θα είναι πιο ανθεκτικός σε σχέση με τον
  πληθυσμό του εάν ο αυτός αρχικοποιείτo κάθε φορά με το αποτέλεσμα της
  ευθυγράμμισης.}

\end{frame}
