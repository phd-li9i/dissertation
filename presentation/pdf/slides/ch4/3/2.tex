\begin{frame}{Ευθυγράμμιση πραγματικών σαρώσεων (\texttt{sm})}


\definecolor{r}{RGB}{255 69 0}
\definecolor{b}{RGB}{51 102 153}


  \noindent\makebox[\linewidth][c]{%
  \begin{minipage}{\linewidth}
    \begin{minipage}{0.45\linewidth}
      \begin{figure}
        \animategraphics[scale=0.1,autoplay,loop]{2}{./figures/slides/ch4/sm}{1}{2}
        \caption{\tiny Περιβάλλον, δύο διαδοχικές στάσεις ρομπότ, και οι αντίστοιχες
                 σαρώσεις από την κάθεμία}
      \end{figure}
    \end{minipage}
    \hfill
    \begin{minipage}{0.45\linewidth}
      \begin{figure}
        \animategraphics[scale=0.22,autoplay,loop]{10}{./figures/slides/ch4/sm_gif/imgs/pic}{0}{19}
        \caption{\tiny Ευθυγράμμιση σαρώσεων $\rightarrow$ εκτίμηση μετασχηματισμού στάσεων από τις οποίες συνελήφθησαν οι σαρώσεις}

        \begin{textblock}{4}(7.3,2.4)
          \scriptsize $\textcolor{r}{\mathcal{S}(\bm{p}_k)}$
        \end{textblock}
        \begin{textblock}{4}(8.8,2.4)
          \scriptsize $\textcolor{b}{\mathcal{S}(\bm{p}_{k+1})}$
        \end{textblock}
        \begin{textblock}{11}(8.4,2.4)
          \scriptsize $\Delta \bm{p} = \texttt{sm}(\textcolor{r}{\mathcal{S}(\bm{p}_k)},\textcolor{b}{\mathcal{S}(\bm{p}_{k+1})})$
        \end{textblock}
      \end{figure}


    \end{minipage}

  \end{minipage}
  }


\note{\footnotesize
Αυτή η τεχνική υποθέτει έναν αισθητήρα lidar ο οποίος συλλαμβάνει από δύο
διαφορετικές στάσεις δύο διαφορετικές σαρώσεις, όπως στο σχήμα που απεικονίζεται
στα αριστερά. Στη μέση βλέπουμε τη διαδικασία ευθυγράμμισης των δύο αυτών
σαρώσεων, και στα δεξιά σε εστίαση την εξέλιξη της εκτίμησης της δεύτερης
στάσης του ρομπότ στο σύστημα αναφοράς της πρώτης σάρωσης.  Αυτή η εκτίμηση
είναι δυνατή διότι η ευθυγράμμιση των δύο πραγματικών σαρώσεων έχει ως στόχο
τον υπολογισμό εκείνου του μετασχηματισμού που όταν εφαρμοσθεί στα σημεία της
πρώτης σάρωσης θα τα κάνει να συμπέσουν στα σημεία της δεύτερης με το
ελάχιστο σφάλμα, ο οποίος μετασχηματισμός είναι ο ίδιος που εκφράζει τη στάση
από την οποία συνελήφθη η δεύτερη σάρωση στο σύστημα αναφοράς που ορίζει η
πρώτη. Για αυτό το λόγο η ευθυγράμμιση σαρώσεων είναι θεμελιώδες κομμάτι της
ρομποτικής, και χρησιμοποιείται κατά κόρον για την εκτίμηση της οδομετρίας
ενός οχήματος.}




\end{frame}
