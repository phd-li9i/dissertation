\begin{frame}{Ευθυγράμμιση πραγματικών σαρώσεων (\texttt{sm})}


\definecolor{r}{RGB}{255 69 0}
\definecolor{b}{RGB}{51 102 153}


  \noindent\makebox[\linewidth][c]{%
  \begin{minipage}{\linewidth}
    \begin{minipage}{0.45\linewidth}
      \begin{figure}
        \animategraphics[scale=0.1,autoplay,loop]{2}{./figures/slides/ch4/sm}{1}{2}
        \caption{\scriptsize Περιβάλλον, διαδοχικές στάσεις ρομπότ, και οι αντίστοιχες
                 σαρώσεις από την κάθεμία}
      \end{figure}
    \end{minipage}
    \hfill
    \begin{minipage}{0.45\linewidth}
      \begin{figure}
        \animategraphics[scale=0.22,autoplay,loop]{10}{./figures/slides/ch4/sm_gif/imgs/pic}{0}{19}
        \caption{\scriptsize Ευθυγράμμιση σαρώσεων $\rightarrow$ εκτίμηση μετασχηματισμού στάσεων από τις οποίες συνελήφθησαν οι σαρώσεις}

        \begin{textblock}{4}(7.3,2.4)
          \scriptsize $\textcolor{r}{\mathcal{S}(\bm{p}_k)}$
        \end{textblock}
        \begin{textblock}{4}(8.8,2.4)
          \scriptsize $\textcolor{b}{\mathcal{S}(\bm{p}_{k+1})}$
        \end{textblock}
        \begin{textblock}{11}(8.4,2.4)
          \scriptsize $\Delta \bm{p} = \texttt{sm}(\textcolor{r}{\mathcal{S}(\bm{p}_k)},\textcolor{b}{\mathcal{S}(\bm{p}_{k+1})})$
        \end{textblock}
      \end{figure}


    \end{minipage}

  \end{minipage}
  }


\end{frame}
