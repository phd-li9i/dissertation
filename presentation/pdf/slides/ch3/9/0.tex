\begin{frame}{Τι ήταν προηγουμένως αδύνατον και τώρα είναι εφικτό}

  \vspace{2cm}
  \begin{itemize}
    \item Επεκτάσιμη και περιεκτική μεθοδολογία αξιολόγησης μεθόδων αυτόνομης πλοήγησης
    \item Προσαρμογή μεθόδου αξιολόγησης με βάση ειδικές απαιτήσεις ($w_m \neq 1.0$)
    \item Ενσωμάτωση οποιωνδήποτε μελλοντικών μεθόδων
    \item Ενσωμάτωση ad hoc περιβάλλοντος
  \end{itemize}

\placebottom
  \tiny A. Filotheou, E. Tsardoulias, A. Dimitriou, A. Symeonidis, L. Petrou, ``Quantitative and qualitative evaluation of ROS-enabled local and global planners in 2D static environments",  \textit{Journal of Intelligent and Robotic Systems}, 2020


\note{\footnotesize Αυτό που είναι σήμερα δυνατον που δεν ήταν πριν αυτή τη
  μελέτη, είναι καταρχάς η ύπαρξη μίας περιεκτικής και επεκτάσιμης μεθοδολογίας
  αξιολόγησης μεθόδων αυτόνομης πλοήγησης. Έπειτα ένας μηχανικός ρομποτικής που
  έχει αντικείμενο την αυτόνομη πλοήγηση μπορεί να προσδιορίσει τα δικά του
  κριτήρια με βάση διαφορετικές απαιτήσεις δίνοντας διαφορετικά βάρη στη
  συνεισφορά της κάθε μετρικής. Και τέλος, με το βλέμμα στο μέλλον, υπάρχει
  πλέον το υπόβαθρο για την αξιολόγηση οποιοασδήποτε νέας μεθόδου που είτε
  χαράσσει μονοπάτια, είτε είναι ελεγκτής κίνησης, και σε οποιοδήποτε
  περιβάλλον.}
\end{frame}
