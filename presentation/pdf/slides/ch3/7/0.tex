\begin{frame}{Κατασκευή συνάρτησης απόδοσης αξίας V}

  \definecolor{r}{RGB}{205 0 0}
  \definecolor{g}{RGB}{0 155 0}
  \definecolor{b}{RGB}{22 38 252}

  \begin{itemize}
    \item Κανονικοποίηση τιμών μετρικής $m$:

          \begin{align}
            \textcolor{b}{N(m)} &\triangleq \dfrac{m - \min m}{\max m - \min m} \in [0,1] \nonumber
          \end{align}

    \item $V$ ανά περιβάλλον/χάρτη $\bm{M}$

          \begin{align}
            V_{\bm{M}}(c) &\triangleq \sum\limits_{m} I_{\textcolor{g}{Q}}(m) \cdot \textcolor{g}{V_q}(c,m) + I_{\textcolor{r}{\overline{Q}}}(m) \cdot \textcolor{r}{V_{\overline{q}}}(c,m) \nonumber \\
            \textcolor{g}{V_q}(c,m) &\triangleq  w_m \cdot I(c,m) \cdot \textcolor{b}{N(m)}, \hspace{1.15cm} m \in \textcolor{g}{Q}\nonumber \\
            \textcolor{r}{V_{\overline{q}}}(c,m) &\triangleq  w_m \cdot I(c,m) \cdot( \textcolor{b}{1 - N(m)}), \hspace{0.2cm} m \in \textcolor{r}{\overline{Q}} \nonumber \\
            I(c,m) &\triangleq I_S(c)\ ||\ I_D(m) \nonumber
          \end{align}

  \end{itemize}

\note{\footnotesize Για να αποκτήσουμε ένα κοινό σύστημα αναφοράς αρχικά
  κανονικοποιούμε τις τιμές των μετρικών μέσω της συνάρτησης N, δηλαδή
  εξετάζουμε την τιμή της μετρικής m για έναν συνδυασμό και τις ακραίες τιμές
  της για κάθε συνδυασμό, ώστε στο τέλος η τιμή της εκφράζεται στο διάστημα
  [0,1].  Κατασκευάζουμε την συνάρτηση απόδοσης αξίας για ένα περιβάλλον μέσω
  της συνάρτησης V. Εδώ w είναι ένα βάρος που αποδίδουμε σε κάθε μετρική ώστε
  να είναι δυνατή η απόδοση μεγαλύτερης ή μικρότερης βαρύτητας ανάλογα με τις
  απαιτήσεις της εκάστοτε αξιολόγησης, και $I$ είναι η συνάρτηση δείκτης. Η
  συνάρτηση I(C,m) είναι μηδέν μόνο για μετρικές που αφορούν σε συνδυασμό από
  planners όταν η πλοήγηση ήταν αποτυχημένη, και τη χρησιμοποιούμε για να
  συνεκτιμήσουμε την αξία των συνιστωσών ενός συνδυασμού ακόμα και όταν η
  πλοήγηση ήταν αποτυχημένη.}


\end{frame}
