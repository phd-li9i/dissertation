\begin{frame}{}

\noindent\makebox[\linewidth][c]{%
\begin{minipage}{\linewidth}

  \begin{minipage}{0.45\linewidth}
    \scriptsize
    \begin{figure}
      \includegraphics[height=120pt,width=159.54pt]{./figures/slides/ch3/antenna_frame}
      \vspace{1cm}
      \caption{\tiny Συστήματα αναφοράς ρομπότ, κεραίας, και τυχαίου προϊόντος}
      \begin{textblock}{4}(1.1,8.3)
        \texttt{base\_footprint}
      \end{textblock}
      \begin{textblock}{4}(2.1,4.5)
        \texttt{antenna\_frame}
      \end{textblock}
      \begin{textblock}{4}(4.8,3.6)
        \texttt{product\_n\_frame}
      \end{textblock}
    \end{figure}
  \end{minipage}
  \hspace{0.1cm}
  \begin{minipage}{0.45\linewidth}\vspace{2cm}
    \begin{figure}\centering
      \scalebox{0.5}{

\tikzset{every picture/.style={line width=0.75pt}} %set default line width to 0.75pt

\begin{tikzpicture}[x=0.75pt,y=0.75pt,yscale=-1,xscale=1]
%uncomment if require: \path (0,300); %set diagram left start at 0, and has height of 300
\path(100,100);

%Straight Lines [id:da535044070379217]
\draw    (154.3,52) -- (154.3,80) ;
\draw [shift={(154.3,82)}, rotate = 270] [color={rgb, 255:red, 0; green, 0; blue, 0 }  ][line width=0.75]    (10.93,-3.29) .. controls (6.95,-1.4) and (3.31,-0.3) .. (0,0) .. controls (3.31,0.3) and (6.95,1.4) .. (10.93,3.29)   ;
%Straight Lines [id:da03696969507445824]
\draw    (153.3,111) -- (153.3,139) ;
\draw [shift={(153.3,141)}, rotate = 270] [color={rgb, 255:red, 0; green, 0; blue, 0 }  ][line width=0.75]    (10.93,-3.29) .. controls (6.95,-1.4) and (3.31,-0.3) .. (0,0) .. controls (3.31,0.3) and (6.95,1.4) .. (10.93,3.29)   ;
%Straight Lines [id:da46122910611053514]
\draw    (462.3,56) -- (462.3,145) ;
\draw [shift={(462.3,54)}, rotate = 90] [color={rgb, 255:red, 0; green, 0; blue, 0 }  ][line width=0.75]    (10.93,-3.29) .. controls (6.95,-1.4) and (3.31,-0.3) .. (0,0) .. controls (3.31,0.3) and (6.95,1.4) .. (10.93,3.29)   ;
%Right Arrow [id:dp5085649971129671]
\draw   (257,147) -- (299,147) -- (299,137) -- (327,157) -- (299,177) -- (299,167) -- (257,167) -- cycle ;

% Text Node
\draw    (60.94,27) -- (249.94,27) -- (249.94,52) -- (60.94,52) -- cycle  ;
\draw (155.44,39.5) node   [align=left] {Εκτίμηση θέσης προϊόντων};
% Text Node
\draw    (64.05,86) -- (244.05,86) -- (244.05,111) -- (64.05,111) -- cycle  ;
\draw (154.05,98.5) node   [align=left] {Εκτίμηση στάσης κεραιών};
% Text Node
\draw    (65.04,145) -- (241.04,145) -- (241.04,170) -- (65.04,170) -- cycle  ;
\draw (153.04,157.5) node   [align=left] {Εκτίμηση στάσης ρομπότ};
% Text Node
\draw    (344.21,145) -- (581.21,145) -- (581.21,170) -- (344.21,170) -- cycle  ;
\draw (462.71,157.5) node   [align=left] {Σφάλμα εκτίμησης στάσης ρομπότ};
% Text Node
\draw    (334.03,25) -- (592.03,25) -- (592.03,50) -- (334.03,50) -- cycle  ;
\draw (463.03,37.5) node   [align=left] {Σφάλμα εκτίμησης στάσης προϊόντων};


\end{tikzpicture}

}
      \vspace{0.8cm}
      \caption{\tiny Η εκτίμηση της θέσης των προϊόντων εξαρτάται από το σφάλμα της
               εκτίμησης της στάσης του ρομπότ}
    \end{figure}
  \end{minipage}
\end{minipage}
}


\note{\footnotesize
  Σε γενικές γραμμές το φαινόμενο του σφάλματος στάσης είναι φυσιολογικό λόγω
  της χρήσης παρατηρητή για την εκτίμηση της, όμως στην περίπτωση του έργου
  RELIEF αυτό το σφάλμα κληροδοτείται στο σφαλμα θέσης των προϊόντων, καθώς η
  εκτίμηση της θέσης τους προκύπτει μέσω της εκτίμησης της στάσης των κεραιών
  που φέρει το ρομπότ, οι οποίες με τη σειρά τους είναι μετασχηματισμοί της
  εκτίμησης της στάσης του.}


\end{frame}
