\begin{frame}{Αυτόνομη πλοήγηση: Προαπαιτούμενα}

\noindent\makebox[\linewidth][c]{%
\begin{minipage}{\linewidth}
  \begin{minipage}{0.4\linewidth}
    \begin{enumerate}
      \item Εξωδεκτικός αισθητήρας \\ (lidar, rgb(d), sonar)
      \item Χάρτης $\bm{M}$ του περιβάλλοντος
      \item Εκτίμηση στάσης $\hat{\bm{p}}_t$ \\ (μέσω EKF/PF)
      \item Αρχική συνθήκη στάσης $\bm{p}_0^{\bm{M}}$
      \item Τελική συνθήκη στάσης $\bm{p}_G^{\bm{M}}$
    \end{enumerate}
  \end{minipage}
  \hfill
  \begin{minipage}{0.5\linewidth}
    \begin{figure}
      \animategraphics[height=101pt,width=180pt,autoplay]{10}{./figures/slides/ch3/relief_video_0_imgs/relief_video_0_-}{0}{100}

        %\transduration<0-100>{0.01}
        %\multiinclude[<+->][format=png,start=0,end=4, graphics={height=101pt,width=180pt}]{./figures/slides/ch3/relief_video_0_imgs/relief_video_0_}

        %\movie[height=101pt,width=180pt]{./figures/slides/ch3/relief_video_0_imgs/relief_video_0.mp4}
      \caption{\tiny  Εφαρμογή έργου RELIEF, βιβλιοθήκη ΤΗΜΜΥ, ΑΠΘ. \\ Πηγή: \url{https://relief.web.auth.gr/}}
    \end{figure}
  \end{minipage}
\end{minipage}
}

\note{\footnotesize
Η αυτόνομη πλοήγηση υποθέτει 5 προαπαιτούμενα. \\
Πρέπει να υπάρχει τουλάχιστον ένας εξωδεκτικός αισθητήρας,\\
ο χάρτης του περιβάλλοντος στο οποίο πλοηγείται το ρομπότ,\\
μία μέθοδος εκτίμησης της στάσης του ρομπότ στο σύστημα συντεταγμένων του χάρτη,\\
μία αρχική στάση και μία τελική στάση. \\
Ως στάση ορίζουμε το διάνυσμα κατάστασης που συνίσταται από τη θέση και τον
προσανατολισμό του ρομπότ ως προς το σύστημα αναφοράς του χάρτη.}

\end{frame}
