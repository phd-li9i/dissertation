\begin{frame}{Πειραματική αξιολόγηση: Στόχος}

  \begin{bw_box}
    Απόδοση μίας τιμής-αξίας $V(c_{i,j})$ σε κάθε συνδυασμό $c_{i,j} = g_i \circ l_j$ για όλα τα περιβάλλοντα με βάση όλες τις μετρικές $m_k$
  \end{bw_box}

  \vspace{1cm}

  Προβλήματα:
  \begin{itemize}
    \item Διαφορετικές μονάδες μέτρησης μετρικών
    \item Κατασκευή $V(c) \uparrow$ όταν $c(m_q) \uparrow$ και $c(m_{\overline{q}})\downarrow$, \hspace{0.5cm} $m_q \in Q, m_{\overline{q}} \in \overline{Q}, Q \cup \overline{Q} = \cup m$
  \end{itemize}

\note{\footnotesize Με βάση τις τιμές των μετρικών που έχουμε καταγράψει,
  αυτό που θέλουμε στο τέλος είναι να μπορέσουμε να αποδώσουμε μία τιμή σε κάθε
  συνδυασμό από planners.  Τα προβλήματα εδώ είναι δύο: πώς θα δώσουμε μία τιμή
  όταν οι μετρικές εκφράζονται σε διαφορετικές μονάδες μέτρησης, και πώς θα
  κατασκευάσουμε μία συνάρτηση απόδοσης αξίας σε κάθε συνδυασμό που να είναι
  γνησίως αύξουσα όταν κάποιες μετρικές συνεισφέρουν με τρόπο ανάλογο καθώς
  αυξάνονται (όπως η μέση απόσταση από εμπόδια, όσο μεγαλύτερη η απόσταση από
  εμπόδια τόσο ασφαλέστερη είναι η πλοήγηση) ενώ άλλες με τρόπο αντιστρόφως
  ανάλογο (όπως ο χρόνος πλοήγησης).}

\end{frame}
