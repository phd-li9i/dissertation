\documentclass[11pt,a4paper,twoside]{book}
\usepackage[top=30mm, bottom=30mm, inner=30mm, outer=30mm, footskip = 15mm]{geometry}
%\usepackage[top=1in, bottom=1in, left=1in, right=1in]{geometry}
%\usepackage[utf8]{inputenc}
%\usepackage[greek,english]{babel}
%\usepackage{alphabeta}

\usepackage{fontspec,xgreek,polyglossia}
\defaultfontfeatures{Mapping=tex-text}
\setmainfont[Mapping=tex-text]{CMU Serif} % choose a font that supports greek characters
\setdefaultlanguage{english}
\setotherlanguage[variant=modern]{greek}
\setmonofont{CMU Typewriter Text}


\usepackage{setspace}
\doublespacing
% Colors:
% Blue-ish: 8FC2FF
% Yellow-ish: FFE599
% Red-ish: FF9999
% Brown-ish: E5C07B
% Blue/Green-ish: 397880

%% greek in text but english in bibliography
%% https://tex.stackexchange.com/questions/161934/while-writing-in-greek-i-need-to-keep-citations-and-bibliography-in-english/542844
%\usepackage{biblatex}
%\declarerobustcommand{\lcitep}[1]{%
  %\textlatin{\citep{#1}}%
%}
%\makeatletter
%\g@addto@macro{\bibsection}{\selectlanguage{english}}
%\makeatother

\usepackage[backend=biber,style=alphabetic,sorting=ynt,babel=other,bibencoding=utf8, language=autobib]{biblatex}
\addbibresource{./parts/bibliography/bibliography.bib}
\makeatletter
\apptocmd\select@@language{\blx@langstrings}
\makeatother


\setcounter{secnumdepth}{3}
\usepackage{listings}
\usepackage{color}
\usepackage[hidelinks,unicode]{hyperref}
\def\UrlBreaks{\do\/\do-}
\usepackage{amsmath}
\usepackage{amssymb}
\usepackage{amsthm}
\usepackage{subcaption}
\usepackage{balance}
\usepackage{graphicx}
\usepackage{epstopdf}
\usepackage{bm}
\usepackage{booktabs}
\usepackage{array}
\usepackage{stfloats}
\usepackage{float}
\usepackage{pdflscape}
\usepackage{pgfplots}
\usepackage{algorithm}
\usepackage{algorithmic}
\usepackage{setspace}
\usepackage{tikz}
\usepackage{multirow}
\usepackage{dashrule}
\usepackage{framed}
\pgfplotsset{compat=1.8}
\usepgfplotslibrary{statistics}
\usetikzlibrary{calc}
\usepackage{gnuplot-lua-tikz}
\newcommand{\ra}[1]{\renewcommand{\arraystretch}{#1}}

\newcommand*{\addheight}[2][.5ex]{%
  \raisebox{0pt}[\dimexpr\height+(#1)\relax]{#2}%
}

\definecolor{transfertoserver}{HTML}{D7191C}
\definecolor{database}{HTML}{FDAE61}
\definecolor{transfertoclient}{HTML}{ABDDA4}
\definecolor{rendering}{HTML}{2B83BA}


\renewcommand{\algorithmicrequire}{\textbf{Input:}}
\renewcommand{\algorithmicensure}{\textbf{Output:}}
\renewcommand\thealgorithm{\Roman{algorithm}}
\floatname{algorithm}{Αλγόριθμος}

% Italics for theorems
\theoremstyle{plain}
\newtheorem{theorem}{Theorem}
\renewcommand\thetheorem{\Roman{theorem}}


% Normal for all the rest
\theoremstyle{definition}
\newtheorem{lemma}{Lemma}
\renewcommand\thelemma{\Roman{lemma}}
\newtheorem{corollary}{Επακόλουθο}
\renewcommand\thecorollary{\Roman{corollary}}
\newtheorem{problem}{Problem}
\renewcommand\theproblem{\Roman{problem}}
\newtheorem{definition}{Ορισμός}
\renewcommand\thedefinition{\Roman{definition}}
\newtheorem{remark}{Παρατήρηση}
\renewcommand\theremark{\Roman{remark}}
\newtheorem{proposition}{Proposition}
\newtheorem{assumption}{Παραδοχή}
\renewcommand\theassumption{\Roman{assumption}}
\newtheorem{hypothesis}{Υπόθεση}
\renewcommand\thehypothesis{\Roman{hypothesis}}
\newtheorem{claim}{Ισχυρισμός}
\renewcommand\theclaim{\Roman{claim}}
\newtheorem{scope}{Πεδίο Εφαρμογής}
\renewcommand\thescope{\Roman{scope}}
\newtheorem{innercustomscope}{Πεδίο Εφαρμογής}
\newenvironment{customscope}[1]
  {\renewcommand\theinnercustomscope{#1}\innercustomscope}
  {\endinnercustomscope}

\newtheorem{innercustomproblem}{Πρόβλημα}
\newenvironment{customproblem}[1]
  {\renewcommand\theinnercustomproblem{#1}\innercustomproblem}
  {\endinnercustomproblem}

\newtheorem{innercustomhypothesis}{Υπόθεση}
\newenvironment{customhypothesis}[1]
  {\renewcommand\theinnercustomhypothesis{#1}\innercustomhypothesis}
  {\endinnercustomhypothesis}

\renewcommand\theproposition{\Roman{proposition}}
\newtheorem{objective}{Objective}
\renewcommand\theobjective{\Roman{objective}}
\newtheorem{operation}{Operation}
\renewcommand\theoperation{\Roman{operation}}

\renewcommand{\qedsymbol}{$\blacksquare$}

% Squares and their colours for global localisation tables
\definecolor{glca}{RGB}{235,172,35}
\definecolor{glcb}{RGB}{184,0,88}
\definecolor{glcc}{RGB}{0,140,249}
\definecolor{glcd}{RGB}{0,110,0}
\definecolor{glce}{RGB}{0,187,173}
\definecolor{glcf}{RGB}{209,99,230}
\definecolor{glcg}{RGB}{178,69,2}
\definecolor{glch}{RGB}{255,146,135}
\definecolor{glci}{RGB}{89,84,214}
\definecolor{glcj}{RGB}{135,133,0}
\definecolor{glck}{RGB}{0,198,248}
\newcommand*{\squarea}{\textcolor{glca}{\blacksquare}}
\newcommand*{\squareb}{\textcolor{glcb}{\blacksquare}}
\newcommand*{\squarec}{\textcolor{glcc}{\blacksquare}}
\newcommand*{\squared}{\textcolor{glcd}{\blacksquare}}
\newcommand*{\squaree}{\textcolor{glce}{\blacksquare}}
\newcommand*{\squaref}{\textcolor{glcf}{\blacksquare}}
\newcommand*{\squareg}{\textcolor{glcg}{\blacksquare}}
\newcommand*{\squareh}{\textcolor{glch}{\blacksquare}}
\newcommand*{\squarei}{\textcolor{glci}{\blacksquare}}
\newcommand*{\squarej}{\textcolor{glcj}{\blacksquare}}
\newcommand*{\squarek}{\textcolor{glck}{\blacksquare}}

\definecolor{mygray}{rgb}{0.5,0.5,0.5}
\definecolor{keyword}{rgb}{0.5,0.5,0.5}
\definecolor{greenCode}{rgb}{0, 0.6, 0}

\lstdefinelanguage{HTTP}{
  keywords={GET},
  ndkeywords={PUT},
  comment=[s]{PO}{T},
  morecomment=[s]{D}{LETE}
}

\lstdefinestyle{customc}{
  belowcaptionskip=1\baselineskip,
  language={HTTP},
  breaklines=true,
  frame=tb,
  captionpos=b,
  keywordstyle=\bfseries\color{greenCode},
  ndkeywordstyle=\bfseries\color{red},
  commentstyle=\bfseries\color{magenta},
  stringstyle=\bfseries\color{black},
  xleftmargin={0.75cm},
  showstringspaces=false,
  basicstyle=\footnotesize\ttfamily,
  numbers=left,
  numberstyle=\small\color{black},
}

\lstset{escapechar=@,style=customc}

%\ifCLASSINFOpdf
   %\usepackage[pdftex]{graphicx}
%\else
%\fi

\usepackage{flushend} % Equalize last page
\usepackage[colorinlistoftodos]{todonotes}

% BOXES
\usepackage{environ}
\usepackage[many]{tcolorbox}
\NewEnviron{gg_box}{%
  \begin{tcolorbox}[colback=black!5, colframe=white!80!, breakable]
  %\begin{tcolorbox}[colback=white!10, colframe=black!30!, breakable]
    \BODY
  \end{tcolorbox}
}

\definecolor{sticky_note_yellow}{cmyk}{0.0, 0.0, 0.06, 0.0, 1.00}
\NewEnviron{sticky_note_box}{%
  \begin{tcolorbox}[colback = sticky_note_yellow, colframe=sticky_note_yellow, breakable]
    \BODY
  \end{tcolorbox}
}

\NewEnviron{bw_box}{%
  \begin{tcolorbox}[colback=white!10, colframe=black!80!, arc=0pt,outer arc=0pt, breakable]
    \BODY
  \end{tcolorbox}
}

\NewEnviron{gw_box}{%
  \begin{tcolorbox}[colback=white!10, colframe=black!10!, breakable]
    \BODY
  \end{tcolorbox}
}

\NewEnviron{bg_box}{%
  \begin{tcolorbox}[colback=black!10, colframe=black!80!, arc=0pt,outer arc=0pt, breakable]
    \BODY
  \end{tcolorbox}
}

\widowpenalty=9000
\clubpenalty=9000
\interfootnotelinepenalty=10000

% Redefines cleardoublepage so as to remove headers/footers
\makeatletter
\def\cleardoublepage{\clearpage\if@twoside \ifodd\c@page\else
\hbox{}
\vspace*{\fill}
\begin{center}
%This page intentionally left
\end{center}
\vspace{\fill}
\thispagestyle{empty}
\newpage
\if@twocolumn\hbox{}\newpage\fi\fi\fi}
\makeatother

\raggedbottom

% allows a break to occur anywhere
\allowdisplaybreaks

% insert blank page . . .
\usepackage{afterpage}

\newcommand\blankpage{%
    \null
    \thispagestyle{empty}%
    \addtocounter{page}{-1}%
    \newpage}

%. . . by calling \afterpage{\blankpage}

% Refresh symbol
\def\Circlearrowleft{\ensuremath{%
  \rotatebox[origin=c]{180}{$\circlearrowleft$}}}
