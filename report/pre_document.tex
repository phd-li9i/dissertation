\documentclass[11pt,a4paper,twoside]{book}
\usepackage[top=30mm, bottom=30mm, inner=30mm, outer=30mm, footskip = 15mm]{geometry}
%\usepackage[top=1in, bottom=1in, left=1in, right=1in]{geometry}
%\usepackage[utf8]{inputenc}
%\usepackage[greek,english]{babel}
%\usepackage{alphabeta}

\usepackage[many]{tcolorbox}
\usepackage{amsmath}
\usepackage{amssymb}
\usepackage{amsthm}
\usepackage{mathtools}
\usepackage{dirtree}
\usepackage{footmisc}


% Change math fonts refs.:
% http://mirrors.ibiblio.org/CTAN/macros/xetex/latex/mathspec/mathspec.pdf
% https://tex.stackexchange.com/questions/412231/xelatex-and-math-fonts-from-main-font
% https://tex.stackexchange.com/questions/483722/how-to-set-a-font-everywhere-including-math-mode-in-xelatex
%\usepackage{mathspec}

\usepackage{fontspec,xgreek,polyglossia}
\defaultfontfeatures{Mapping=tex-text}
\setmainfont[Mapping=tex-text]{CMU Serif} % choose a font that supports greek characters
\setdefaultlanguage{english}
\setotherlanguage[variant=modern]{greek}
\setmonofont{CMU Typewriter Text}

\addto{\captionsenglish}{\renewcommand{\appendixname}{Παράρτημα}}

% Change math fonts #1
%\setmathsfont(Digits,Latin)[Scale=MatchLowercase]{CMU Serif}
%\setmathsfont(Greek)[Scale=MatchLowercase]{CMU Serif}
%\setmathrm{CMU Serif}
%\exchangeforms{phi}
%\setminwhitespace[750]

% Change math fonts #2
%\setmathfont{CMU Serif}
%\setmathrm{CMU Serif}
%\setmathfont[range=it]{CMU Serif-Italic}
%\setmathfont[range=bfit]{CMU Serif-BoldItalic}
%\setmathfont[range={}]{CMU Serif} % empty range to get the correct metrics

\usepackage{setspace}
\doublespacing
% Colors:
% Blue-ish: 8FC2FF
% Yellow-ish: FFE599
% Red-ish: FF9999
% Brown-ish: E5C07B
% Blue/Green-ish: 397880

%% greek in text but english in bibliography
%% https://tex.stackexchange.com/questions/161934/while-writing-in-greek-i-need-to-keep-citations-and-bibliography-in-english/542844
%\usepackage{biblatex}
%\declarerobustcommand{\lcitep}[1]{%
  %\textlatin{\citep{#1}}%
%}
%\makeatletter
%\g@addto@macro{\bibsection}{\selectlanguage{english}}
%\makeatother

\usepackage[backend=biber,style=alphabetic,sorting=ynt,babel=other,bibencoding=utf8, language=autobib]{biblatex}
\addbibresource{./parts/bibliography/bibliography.bib}
\makeatletter
\apptocmd\select@@language{\blx@langstrings}
\makeatother


\setcounter{secnumdepth}{3}
\usepackage{listings}
\usepackage{color}
\usepackage[hidelinks,unicode]{hyperref}
\def\UrlBreaks{\do\/\do-}
\usepackage{subcaption}
\usepackage{balance}
\usepackage{graphicx}
\usepackage{epstopdf}
\usepackage{bm}
\usepackage{booktabs}
\usepackage{array}
\usepackage{stfloats}
\usepackage{float}
\usepackage{pdflscape}
\usepackage{pgfplots}
\usepackage{algorithm}
\usepackage{algorithmic}
\usepackage{setspace}
\usepackage{tikz}
\usepackage{multirow}
\usepackage{dashrule}
\usepackage{framed}
\pgfplotsset{compat=1.8}
\usepgfplotslibrary{statistics}
\usetikzlibrary{fit,calc}
\usepackage{gnuplot-lua-tikz}
\newcommand{\ra}[1]{\renewcommand{\arraystretch}{#1}}
\usepackage{changepage}

\newcommand*{\addheight}[2][.5ex]{%
  \raisebox{0pt}[\dimexpr\height+(#1)\relax]{#2}%
}

\definecolor{transfertoserver}{HTML}{D7191C}
\definecolor{database}{HTML}{FDAE61}
\definecolor{transfertoclient}{HTML}{ABDDA4}
\definecolor{rendering}{HTML}{2B83BA}


\renewcommand{\algorithmicrequire}{\textbf{Input:}}
\renewcommand{\algorithmicensure}{\textbf{Output:}}
\renewcommand\thealgorithm{\Roman{algorithm}}
\floatname{algorithm}{Αλγόριθμος}

% Italics for theorems
%\theoremstyle{plain}

\theoremstyle{definition}
\newtheorem{theorem}{Θεώρημα}
\renewcommand\thetheorem{\Roman{theorem}}


% Normal for all the rest
\theoremstyle{definition}
\newtheorem{lemma}{Λήμμα}
\renewcommand\thelemma{\Roman{lemma}}
\newtheorem{corollary}{Επακόλουθο}
\renewcommand\thecorollary{\Roman{corollary}}
\newtheorem{problem}{Problem}
\renewcommand\theproblem{\Roman{problem}}
\newtheorem{definition}{Ορισμός}
\renewcommand\thedefinition{\Roman{definition}}
\newtheorem{remark}{Παρατήρηση}
\renewcommand\theremark{\Roman{remark}}
\newtheorem{proposition}{Proposition}
\newtheorem{assumption}{Παραδοχή}
\renewcommand\theassumption{\Roman{assumption}}
\newtheorem{hypothesis}{Υπόθεση}
\renewcommand\thehypothesis{\Roman{hypothesis}}
\newtheorem{claim}{Ισχυρισμός}
\renewcommand\theclaim{\Roman{claim}}
\newtheorem{scope}{Πεδίο Εφαρμογής}
\renewcommand\thescope{\Roman{scope}}
\newtheorem{innercustomscope}{Πεδίο Εφαρμογής}
\newenvironment{customscope}[1]
  {\renewcommand\theinnercustomscope{#1}\innercustomscope}
  {\endinnercustomscope}

\newtheorem{innercustomproblem}{Πρόβλημα}
\newenvironment{customproblem}[1]
  {\renewcommand\theinnercustomproblem{#1}\innercustomproblem}
  {\endinnercustomproblem}

\newtheorem{innercustomhypothesis}{Υπόθεση}
\newenvironment{customhypothesis}[1]
  {\renewcommand\theinnercustomhypothesis{#1}\innercustomhypothesis}
  {\endinnercustomhypothesis}

\newtheorem{innercustomcontribution}{Συμβολή}
\newenvironment{customcontribution}[1]
  {\renewcommand\theinnercustomcontribution{#1}\innercustomcontribution}
  {\endinnercustomcontribution}

\renewcommand\theproposition{\Roman{proposition}}
\newtheorem{objective}{Objective}
\renewcommand\theobjective{\Roman{objective}}
\newtheorem{operation}{Operation}
\renewcommand\theoperation{\Roman{operation}}

%\renewcommand{\qedsymbol}{$\blacksquare$}
\renewcommand{\qedsymbol}{\hfill$\blacksquare$}

% Squares and their colours for global localisation tables
\definecolor{glca}{RGB}{235,172,35}
\definecolor{glcb}{RGB}{184,0,88}
\definecolor{glcc}{RGB}{0,140,249}
\definecolor{glcd}{RGB}{0,110,0}
\definecolor{glce}{RGB}{0,187,173}
\definecolor{glcf}{RGB}{209,99,230}
\definecolor{glcg}{RGB}{178,69,2}
\definecolor{glch}{RGB}{255,146,135}
\definecolor{glci}{RGB}{89,84,214}
\definecolor{glcj}{RGB}{135,133,0}
\definecolor{glck}{RGB}{0,198,248}
\newcommand*{\squarea}{\textcolor{glca}{\blacksquare}}
\newcommand*{\squareb}{\textcolor{glcb}{\blacksquare}}
\newcommand*{\squarec}{\textcolor{glcc}{\blacksquare}}
\newcommand*{\squared}{\textcolor{glcd}{\blacksquare}}
\newcommand*{\squaree}{\textcolor{glce}{\blacksquare}}
\newcommand*{\squaref}{\textcolor{glcf}{\blacksquare}}
\newcommand*{\squareg}{\textcolor{glcg}{\blacksquare}}
\newcommand*{\squareh}{\textcolor{glch}{\blacksquare}}
\newcommand*{\squarei}{\textcolor{glci}{\blacksquare}}
\newcommand*{\squarej}{\textcolor{glcj}{\blacksquare}}
\newcommand*{\squarek}{\textcolor{glck}{\blacksquare}}

\definecolor{sinisn0}{RGB}{255,255,255}
\definecolor{sinisn1}{RGB}{255,247,251}
\definecolor{sinisn2}{RGB}{236,226,240}
\definecolor{sinisn3}{RGB}{208,209,230}


\definecolor{m}{RGB}{255,0,255}
\definecolor{k}{RGB}{0,0,0}
\newcommand*{\bulletmagenta}{\textcolor{m}{\bullet}}
\newcommand*{\bulletblack}{\textcolor{k}{\bullet}}

\definecolor{mygray}{rgb}{0.5,0.5,0.5}
\definecolor{litegray}{rgb}{0.8,0.8,0.8}
\definecolor{keyword}{rgb}{0.5,0.5,0.5}
\definecolor{greenCode}{rgb}{0, 0.6, 0}

\lstdefinelanguage{HTTP}{
  keywords={GET},
  ndkeywords={PUT},
  comment=[s]{PO}{T},
  morecomment=[s]{D}{LETE}
}

\lstdefinestyle{customc}{
  belowcaptionskip=1\baselineskip,
  language={HTTP},
  breaklines=true,
  frame=tb,
  captionpos=b,
  keywordstyle=\bfseries\color{greenCode},
  ndkeywordstyle=\bfseries\color{red},
  commentstyle=\bfseries\color{magenta},
  stringstyle=\bfseries\color{black},
  xleftmargin={0.75cm},
  showstringspaces=false,
  basicstyle=\footnotesize\ttfamily,
  numbers=left,
  numberstyle=\small\color{black},
}

\lstset{escapechar=@,style=customc}

%\ifCLASSINFOpdf
   %\usepackage[pdftex]{graphicx}
%\else
%\fi

\usepackage{flushend} % Equalize last page
\usepackage[colorinlistoftodos]{todonotes}

% BOXES
\usepackage{environ}
\NewEnviron{gg_box}{%
  \vspace{0.5cm}
  \begin{tcolorbox}[colback=black!5, colframe=white!80!, breakable]
  %\begin{tcolorbox}[colback=white!10, colframe=black!30!, breakable]
    \BODY
  \end{tcolorbox}
}

\definecolor{sticky_note_yellow}{cmyk}{0.0, 0.0, 0.06, 0.0, 1.00}
\NewEnviron{sticky_note_box}{%
  \vspace{0.5cm}
  \begin{tcolorbox}[colback = sticky_note_yellow, colframe=sticky_note_yellow, breakable]
    \BODY
  \end{tcolorbox}
}

\NewEnviron{bw_box}{%
  \vspace{0.5cm}
  \begin{tcolorbox}[colback=white!10, colframe=black!80!, arc=0pt,outer arc=0pt, breakable]
    \BODY
  \end{tcolorbox}
}

\NewEnviron{gw_box}{%
  \vspace{0.5cm}
  \begin{tcolorbox}[colback=white!10, colframe=black!10!, breakable]
    \BODY
  \end{tcolorbox}
}

\NewEnviron{bg_box}{%
  \vspace{0.5cm}
  \begin{tcolorbox}[colback=black!10, colframe=black!80!, arc=0pt,outer arc=0pt, breakable]
    \BODY
  \end{tcolorbox}
}

\widowpenalty=9000
\clubpenalty=9000
\interfootnotelinepenalty=10000

% Redefines cleardoublepage so as to remove headers/footers
\makeatletter
\def\cleardoublepage{\clearpage\if@twoside \ifodd\c@page\else
\hbox{}
\vspace*{\fill}
\begin{center}
%This page intentionally left
\end{center}
\vspace{\fill}
\thispagestyle{empty}
\newpage
\if@twocolumn\hbox{}\newpage\fi\fi\fi}
\makeatother

\raggedbottom

% allows a break to occur anywhere
\allowdisplaybreaks

% insert blank page . . .
\usepackage{afterpage}

\newcommand\blankpage{%
    \null
    \thispagestyle{empty}%
    \addtocounter{page}{-1}%
    \newpage}

%. . . by calling \afterpage{\blankpage}

% Refresh symbol
\def\Circlearrowleft{\ensuremath{%
  \rotatebox[origin=c]{180}{$\circlearrowleft$}}}

% Better stackrel: \leftstackrel
\newlength{\leftstackrelawd}
\newlength{\leftstackrelbwd}
\def\leftstackrel#1#2{\settowidth{\leftstackrelawd}%
{${{}^{#1}}$}\settowidth{\leftstackrelbwd}{$#2$}%
\addtolength{\leftstackrelawd}{-\leftstackrelbwd}%
\leavevmode\ifthenelse{\lengthtest{\leftstackrelawd>0pt}}%
{\kern-.5\leftstackrelawd}{}\mathrel{\mathop{#2}\limits^{#1}}}


% Annulus plot
\usepackage{xcolor}
\usepackage{xparse}
\usepackage{tikz}
\usetikzlibrary{calc,positioning}
\usetikzlibrary{decorations, decorations.text}

\newlength{\layerwd}
\newcounter{outermost}

\NewDocumentEnvironment{onion}{sm}{%*= draw axes; #1: thickness of each annulus
    \begin{tikzpicture}
        \setlength{\layerwd}{#2}%
        \setcounter{outermost}{0}
        \IfBooleanT{#1}{%
            \draw[<->] (-4,0) -- (4,0);
            \draw[<->] (0,4) -- (0,-4);
        }%
}{%
    %\foreach \A in {0,...,\theoutermost}{\draw[thick] (0,0) circle (\A*\layerwd+\layerwd);}
    \end{tikzpicture}
}

% *=text on a circular path; [optional fill color]; #3 layer;
% #4 start angle; #5 stop angle; #6 [optional text]
%% angles are counterclockwise in degrees
\NewDocumentCommand{\annulus}{sO{lightgray}mmmo}{%

    \ifthenelse{#3=0}
    {
      %\filldraw[thick,fill=#2] (#4:#3*\layerwd) %% start here
        %arc [radius=#3*\layerwd, start angle=#4, delta angle=#5-#4] %% inner arc
        %-- (#5:#3*\layerwd+\layerwd) %% move out
        %arc [radius=#3*\layerwd+\layerwd, start angle=#5, delta angle=#4-#5] %% outer arc
        %-- cycle; %% Back to the beginnning
    }
    {
      \filldraw[thick,fill=#2] (#4:#3*\layerwd) %% start here
      arc [radius=#3*\layerwd, start angle=#4, delta angle=#5-#4] %% inner arc
      -- (#5:#3*\layerwd+\layerwd) %% move out
      arc [radius=#3*\layerwd+\layerwd, start angle=#5, delta angle=#4-#5] %% outer arc
      -- cycle; %% Back to the beginnning
    }
    \pgfmathsetmacro{\tmp}{(#5-#4)/2 +#4} %% Locate the middle of the arc
    \IfNoValueF{#6}{%
        \IfBooleanTF{#1}
        {%
            \begingroup
                %% text is always rightside-up:
                \pgfmathsetmacro{\rpTF}{ifthenelse(\tmp>180,"false","true")}
                \def\\{\space} %% A safety precaution, \\ = space on decorated text
                \path[rotate=\tmp-180,postaction={
                    decorate,
                    decoration={
                        text along path,
                        raise=-3pt,
                        text align={align=center},
                        reverse path=\rpTF,
                        text=#6
                    }
                }] (0,0) circle (#3*\layerwd+0.5*\layerwd);
            \endgroup
        }%%
        {%
            \pgfmathsetmacro{\rpTF}{ifthenelse(\tmp>180,\tmp+90,\tmp-90)}
            \pgfmathsetmacro{\rpTF}{ifthenelse(#3==0,\rpTF-90,\rpTF)}
            \node[inner sep=0pt, %%% If there is text, print it
            text width=#3*\layerwd*3+\layerwd,
            align=center,
            rotate=\rpTF,
            font=\footnotesize] at (\tmp:#3*\layerwd+0.5*\layerwd)
            {#6};
        }%
    }%
    \ifnum\theoutermost<#3\setcounter{outermost}{#3}\fi
}


% Double \ast = \astt
\DeclareRobustCommand\astt{%
  \leavevmode
  {\sbox0{\ddag}%
   \ooalign{\raisebox{\ht0-\height}{*}\cr
            \raisebox{\depth-\dp0}{\scalebox{1}[-1]{*}}\cr}%
  }%
}



% Highlight algorithm parts with color
\newcommand*{\tikzmk}[1]{\tikz[remember picture,overlay,] \node (#1) {};\ignorespaces}
%define a boxing command, argument = colour of box
\newcommand{\boxit}[1]{\tikz[remember picture,overlay]{\node[yshift=3pt,fill=#1,opacity=.25,fit={($(A)+(-0.038\linewidth,.3\baselineskip)$)($(B)+(.9\linewidth,-.3\baselineskip)$)}] {};}\ignorespaces}

\definecolor{alg1}{RGB}{251,180,174}
\definecolor{alg2}{RGB}{255,255,204}
\definecolor{alg3}{RGB}{179,205,227}

\definecolor{alg4}{RGB}{229,245,224}
\definecolor{alg5}{RGB}{161,217,155}
\definecolor{alg6}{RGB}{49,163,84}

% Manipulate vspace for algorithm
\makeatletter
\newcommand\fs@spaceruled{\def\@fs@cfont{\bfseries}\let\@fs@capt\floatc@ruled
  \def\@fs@pre{\vspace{0.5\baselineskip}\hrule height.8pt depth0pt \kern2pt}%
  \def\@fs@post{\kern2pt\hrule\relax}%
  \def\@fs@mid{\kern2pt\hrule\kern2pt}%
  \let\@fs@iftopcapt\iftrue}
\makeatother

% Do not number parts
\usepackage{nonumonpart}

\let\oldcontentsline\contentsline
\newcommand{\nopagecontentsline}[3]{\oldcontentsline{#1}{#2}{}}
