Σχετικά με την τιμή-αξία ενός συνδυασμού αλγορίθμου χάραξης μονοπατιών
και ελεγκτή κίνησης που αφορά στις μετρικές αξιολόγησης ενός
\textbf{global planner} του κάνουμε τις εξής παραδοχές.
Η αξία ενός συνδυασμού είναι:

\begin{itemize}
  \item υψηλότερη όσο πιο σύντομο σε μήκος είναι ένα σχέδιο μονοπατιού---ένα
        ρομπότ που το διασχίζει σε σταθερή ταχύτητα χρειάζεται λιγότερο χρόνο
        για να φτάσει από την αρχική στάση στην τελική
  \item υψηλότερη όσο υψηλότερη είναι η ανάλυση του σχεδιασθέντος
        μονοπατιού---όσο πιο λεπτομερής είναι η ανάλυση ενός μονοπατιού τόσο
        περισσότερο πιο πιθανό είναι να υπάρχει ένας (υπο)στόχος εντός του
        ορίζοντα του τοπικού χάρτη κόστους, και τόσο πιο ομαλή μπορεί να είναι
        η διαδρομή
  \item υψηλότερη όσο πιο ομαλό είναι το σχεδιασθέν μονοπάτι---όσο πιο ομαλό
        είναι το μονοπάτι τόσο πιο πιθανό είναι ότι το ρομπότ χρειάζεται
        λιγότερο χρόνο για να διασχίσει τη διαδρομή από την αρχική στάση μέχρι
        τη στάση-στόχο (η πιστή τήρηση του συνολικού σχεδίου είναι θέμα του
        ελεγκτή κίνησης ως προς το πόσο κατάλληλο και εφικτό θεωρεί το μονοπάτι
        προς ακολούθηση)
  \item τόσο υψηλότερη όσο μεγαλύτερη είναι η μέση ελάχιστη απόστασή των στάσεων
        που το απαρτίζουν από τα εμπόδια του χάρτη---ώστε οι συγκρούσεις με
        εμπόδια να είναι λιγότερο πιθανό να συμβούν
  \item υψηλότερη όσο μεγαλύτερη είναι η ολική ελάχιστη απόστασή του
        μονοπατιού από τα εμπόδια σε ένα χάρτη σε όλα τα πειράματα και τις
        προσομοιώσεις, και
  \item χαμηλότερη όσο πιο μεγάλη είναι η διακύμανση της τιμή κάθε
        μετρικής---έτσι ώστε ένας μηχανικός ρομποτικής να μπορεί να υπολογίζει
        στην προβλεψιμότητά της
\end{itemize}

Σχετικά με την τιμή-αξία ενός συνδυασμού αλγορίθμου χάραξης μονοπατιών και
ελεγκτή κίνησης που αφορά στις μετρικές αξιολόγησης ενός \textbf{local planner}
του κάνουμε τις εξής παραδοχές. Η αξία ενός συνδυασμού είναι:

\begin{itemize}
  \item χαμηλότερη όσο υψηλότερος είναι ο μέσος αριθμός των ματαιωμένων
        αποστολών κατά το σύνολο των πειραμάτων και προσομοιώσεων
  \item χαμηλότερη όσο υψηλότερος είναι ο μέσος αριθμός ανακτήσεων με περιστροφή
        που εκτελέστηκαν
  \item χαμηλότερη όσο υψηλότερος είναι ο μέσος αριθμός εκτελούμενων
        εκκαθαρίσεων χαρτών κόστους
  \item χαμηλότερη όσο υψηλότερος είναι ο μέσος αριθμός αποτυχιών διαδρομής
  \item χαμηλότερη όσο υψηλότερος είναι ο σχετικός αριθμός αποτυχιών διαδρομής,
        και
  \item χαμηλότερη όσο πιο μεγάλη είναι η διακύμανση των τιμών της κάθε μετρικής
\end{itemize}

Όπως είναι προφανές όλες οι παραπάνω μετρικές είναι ανεξάρτητες από την
επιτυχία ή την αποτυχία των συνδυασμών των αλγορίθμων κατασκευής μονοπατιών και
ελεγκτών κίνησης στην επίτευξη της πλοήγησης στην στάση-στόχο $\bm{p}_G$ από
την αρχική $\bm{p}_0$. Συνεπώς οι τιμές τους περιλαμβάνονται στον υπολογισμό
της αξίας κάθε συνδυασμού ανεξάρτητα από το αν ο εν λόγω συνδυασμός απέτυχε
να ολοκληρώσει όλες τις αποστολές.

Σχετικά με την τιμή-αξία ενός συνδυασμού αλγορίθμου χάραξης μονοπατιών και
ελεγκτή κίνησης που αφορά στις μετρικές αξιολόγησης του \textbf{συνδυασμού}
τους κάνουμε τις εξής παραδοχές. Η αξία ενός συνδυασμού είναι:

\begin{itemize}
  \item χαμηλότερη όσο μεγαλύτερη είναι η μέση απόκλιση των πραγματικών
        διαδρομών που ακολούθησε το ρομπότ ως αποτέλεσμα της δράσης του ελεγκτή
        κίνησης από τα μονοπάτια που ο αλγόριθμος κατασκευής μονοπατιών σχεδίασε
        για να ακολουθήσει
  \item χαμηλότερη όσο μεγαλύτερη είναι η μέση συνολική απόκλιση των πρώτων
        από τα δεύτερα
  \item χαμηλότερη όσο μεγαλύτερη είναι η μέση απόσταση Frechet των πρώτων
        από τα δεύτερα
  \item υψηλότερη όσο χαμηλότερος είναι ο χρόνος διαδρομής από την αρχική
        προς την τελική-επιθυμητή στάση
  \item υψηλότερη όσο μικρότερη σε μήκος είναι η πραγματική διαδρομή που
        ακολούθησε το ρομπότ
  \item χαμηλότερη όσο λιγότερο ομαλές είναι οι πραγματικές διαδρομές που
        ακολούθησε το ρομπότ
  \item υψηλότερη όσο μεγαλύτερη είναι η μέση ελάχιστη απόσταση του ρομπότ από
        τα εμπόδια του χάρτη
  \item υψηλότερη όσο μεγαλύτερη είναι η ολική ελάχιστη απόσταση του ρομπότ
        από τα εμπόδια σε έναν χάρτη σε όλες τις προσομοιώσεις και τα πειράματα,
        και
  \item χαμηλότερη όσο πιο μεγάλη η διακύμανση της τιμής της κάθε μετρικής
\end{itemize}

Οι παραπάνω μετρικές εξαρτώνται από την επιτυχία ή την αποτυχία του συνδυασμού
των αλγορίθμων κατασκευής μονοπατιών και ελεγκτών κίνησης στην επίτευξη της
πλοήγησης στην στάση-στόχο και, επομένως, δεν συμπεριλαμβάνονται στον υπολογισμό
της τιμής-αξίας ενός συνδυασμού εάν ο συνδυασμός αυτός απέτυχε να πλοηγηθεί
μέχρι την επιθυμητή στάση για κάθε προσομοίωση ή πείραμα που συνέβη αυτό.
