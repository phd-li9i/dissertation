\begin{figure}\centering
    \begin{subfigure}[t]{0.475\linewidth} \centering
        \hspace{0.5cm}
        \input{./figures/parts/appendix/chapters/03/sections/03/real1.tex}\vspace{1em}
        \caption{\small The connected endpoints $\mathcal{P}_r^a$ of the range
                 scan captured from the robot's true pose $\bm{p}_a$, as seen
                 from the local reference frame of the range scan sensor
                 (effectively $(0,0)$)}
        \label{fig:pr1}
    \end{subfigure}
    \hfill
    \begin{subfigure}[t]{0.475\linewidth} \centering
        \hspace{0.5cm}
        \input{./figures/parts/appendix/chapters/03/sections/03/virtual1.tex}\vspace{1em}
        \caption{\small The connected endpoints $\mathcal{P}_\text{v}^c$ of the
                 range scan captured from the robot's hypothesised pose
                 $\bm{p}_c$, as seen from the local reference frame of the
                 virtual range scan sensor $(0,0)$}
        \label{fig:pv1}
    \end{subfigure}
    \begin{subfigure}[t]{0.475\linewidth} \centering
        \hspace{0.5cm}
        % GNUPLOT: LaTeX picture with Postscript
\begingroup
  \makeatletter
  \providecommand\color[2][]{%
    \GenericError{(gnuplot) \space\space\space\@spaces}{%
      Package color not loaded in conjunction with
      terminal option `colourtext'%
    }{See the gnuplot documentation for explanation.%
    }{Either use 'blacktext' in gnuplot or load the package
      color.sty in LaTeX.}%
    \renewcommand\color[2][]{}%
  }%
  \providecommand\includegraphics[2][]{%
    \GenericError{(gnuplot) \space\space\space\@spaces}{%
      Package graphicx or graphics not loaded%
    }{See the gnuplot documentation for explanation.%
    }{The gnuplot epslatex terminal needs graphicx.sty or graphics.sty.}%
    \renewcommand\includegraphics[2][]{}%
  }%
  \providecommand\rotatebox[2]{#2}%
  \@ifundefined{ifGPcolor}{%
    \newif\ifGPcolor
    \GPcolorfalse
  }{}%
  \@ifundefined{ifGPblacktext}{%
    \newif\ifGPblacktext
    \GPblacktexttrue
  }{}%
  % define a \g@addto@macro without @ in the name:
  \let\gplgaddtomacro\g@addto@macro
  % define empty templates for all commands taking text:
  \gdef\gplfronttext{}%
  \gdef\gplfronttext{}%
  \makeatother
  \ifGPblacktext
    % no textcolor at all
    \def\colorrgb#1{}%
    \def\colorgray#1{}%
  \else
    % gray or color?
    \ifGPcolor
      \def\colorrgb#1{\color[rgb]{#1}}%
      \def\colorgray#1{\color[gray]{#1}}%
      \expandafter\def\csname LTw\endcsname{\color{white}}%
      \expandafter\def\csname LTb\endcsname{\color{black}}%
      \expandafter\def\csname LTa\endcsname{\color{black}}%
      \expandafter\def\csname LT0\endcsname{\color[rgb]{1,0,0}}%
      \expandafter\def\csname LT1\endcsname{\color[rgb]{0,1,0}}%
      \expandafter\def\csname LT2\endcsname{\color[rgb]{0,0,1}}%
      \expandafter\def\csname LT3\endcsname{\color[rgb]{1,0,1}}%
      \expandafter\def\csname LT4\endcsname{\color[rgb]{0,1,1}}%
      \expandafter\def\csname LT5\endcsname{\color[rgb]{1,1,0}}%
      \expandafter\def\csname LT6\endcsname{\color[rgb]{0,0,0}}%
      \expandafter\def\csname LT7\endcsname{\color[rgb]{1,0.3,0}}%
      \expandafter\def\csname LT8\endcsname{\color[rgb]{0.5,0.5,0.5}}%
    \else
      % gray
      \def\colorrgb#1{\color{black}}%
      \def\colorgray#1{\color[gray]{#1}}%
      \expandafter\def\csname LTw\endcsname{\color{white}}%
      \expandafter\def\csname LTb\endcsname{\color{black}}%
      \expandafter\def\csname LTa\endcsname{\color{black}}%
      \expandafter\def\csname LT0\endcsname{\color{black}}%
      \expandafter\def\csname LT1\endcsname{\color{black}}%
      \expandafter\def\csname LT2\endcsname{\color{black}}%
      \expandafter\def\csname LT3\endcsname{\color{black}}%
      \expandafter\def\csname LT4\endcsname{\color{black}}%
      \expandafter\def\csname LT5\endcsname{\color{black}}%
      \expandafter\def\csname LT6\endcsname{\color{black}}%
      \expandafter\def\csname LT7\endcsname{\color{black}}%
      \expandafter\def\csname LT8\endcsname{\color{black}}%
    \fi
  \fi
    \setlength{\unitlength}{0.0500bp}%
    \ifx\gptboxheight\undefined%
      \newlength{\gptboxheight}%
      \newlength{\gptboxwidth}%
      \newsavebox{\gptboxtext}%
    \fi%
    \setlength{\fboxrule}{0.5pt}%
    \setlength{\fboxsep}{1pt}%
\begin{picture}(2000.00,2000.00)%
    \gplgaddtomacro\gplfronttext{%
      \colorrgb{0.00,0.00,0.00}%
      \put(128,647){\makebox(0,0)[r]{\strut{}$-4.0$}}%
      \colorrgb{0.00,0.00,0.00}%
      \put(128,929){\makebox(0,0)[r]{\strut{}$-2.0$}}%
      \colorrgb{0.00,0.00,0.00}%
      \put(128,1210){\makebox(0,0)[r]{\strut{}$0.0$}}%
      \colorrgb{0.00,0.00,0.00}%
      \put(128,1492){\makebox(0,0)[r]{\strut{}$2.0$}}%
      \colorrgb{0.00,0.00,0.00}%
      \put(401,195){\makebox(0,0){\strut{}$-8.0$}}%
      \colorrgb{0.00,0.00,0.00}%
      \put(964,195){\makebox(0,0){\strut{}$-4.0$}}%
      \colorrgb{0.00,0.00,0.00}%
      \put(1527,195){\makebox(0,0){\strut{}$0.0$}}%
    }%
    \gplgaddtomacro\gplfronttext{%
      \colorrgb{0.00,0.00,0.00}%
      \put(-428,1034){\rotatebox{90}{\makebox(0,0){\strut{}y [m]}}}%
      \colorrgb{0.00,0.00,0.00}%
      \put(1034,-135){\makebox(0,0){\strut{}x [m]}}%
    }%
    \gplfronttext
    \put(0,0){\includegraphics{./figures/parts/appendix/chapters/03/sections/03/real2}}%
    \gplfronttext
  \end{picture}%
\endgroup
\vspace{1em}
        \caption {\small The connected endpoints $\mathcal{P}_r^a$ of the range
                  scan captured from the robot's true pose $\bm{p}_a$, as seen
                  from the local reference frame of the range scan sensor, and
                  the centroid of the polygon formed by them}
        \label{fig:pr2}
    \end{subfigure}
    \hfill
    \begin{subfigure}[t]{0.475\linewidth} \centering
        \hspace{0.5cm}
        \input{./figures/parts/appendix/chapters/03/sections/03/virtual2.tex}\vspace{1em}
        \caption{\small The connected endpoints $\mathcal{P}_\text{v}^{c\prime}$
                 of the range scan captured from the robot's angularly-aligned
                 hypothesised pose $\bm{p}_c^\prime$, as seen from the local
                 reference frame of the virtual range scan sensor, and the
                 centroid of the polygon formed by them. Notice that the pose
                 has been aligned angularly, but not positionally}
        \label{fig:pv2}
    \end{subfigure}
    \begin{subfigure}[t]{0.475\linewidth} \centering
        \hspace{0.5cm}
        \input{./figures/parts/appendix/chapters/03/sections/03/virtual3.tex}\vspace{1em}
        \caption{\small The connected endpoints $\mathcal{P}_\text{v}^{c\prime}$
                 of the range scan captured from the robot's positionally- and
                 angularly-aligned hypothesised pose $\bm{p}_c^{\prime}$, as
                 seen from the local reference frame of the virtual range scan
                 sensor, and the centroid of the polygon formed by them}
        \label{fig:pv3}
    \end{subfigure}
    \hfill
    \begin{subfigure}[t]{0.475\linewidth} \centering
        \hspace{0.5cm}
        \input{./figures/parts/appendix/chapters/03/sections/03/aligned.tex}\vspace{1em}
        \caption{\small The connected endpoints $\mathcal{P}_r^a$ (black) and
                 $\mathcal{P}_\text{v}^{c\prime}$ (red), and their corresponding
                 centroids, all as seen from the local reference frames of the
                 range scan sensors that captured them}
        \label{fig:aligned}
    \end{subfigure}
    \caption{\small Illustration of orientational and then positional alignment
             of candidate pose $\bm{p}_c$ with respect to true pose $\bm{p}_a$
             in environment CORRIDOR (figure \ref{fig:map_corridor})}
    \label{fig:illustration}
\end{figure}
