\begin{figure}[!h]\centering
  \begin{subfigure}{\linewidth}
    \vspace{2cm}
    \input{./figures/parts/appendix/chapters/04/sections/05/boxplots_position_errors_sm0.tex}
    \vspace{1cm}
  \end{subfigure}\\%
  \begin{subfigure}{\linewidth}
    \vspace{1cm}
    \definecolor{c1}{rgb}{0 0.4470 0.7410}
\definecolor{c2}{rgb}{0.8500 0.3250 0.0980}
\definecolor{c3}{rgb}{0.9290 0.6940 0.1250}
\definecolor{c4}{rgb}{0.4940 0.1840 0.5560}
\definecolor{c5}{rgb}{0.4660 0.6740 0.1880}
\definecolor{c6}{rgb}{0.3010 0.7450 0.9330}
\definecolor{c7}{RGB}{251,180,185}
\definecolor{c8}{RGB}{247,104,161}
\definecolor{c9}{RGB}{255,0,255}

% GNUPLOT: LaTeX picture with Postscript
\begingroup
  \makeatletter
  \providecommand\color[2][]{%
    \GenericError{(gnuplot) \space\space\space\@spaces}{%
      Package color not loaded in conjunction with
      terminal option `colourtext'%
    }{See the gnuplot documentation for explanation.%
    }{Either use 'blacktext' in gnuplot or load the package
      color.sty in LaTeX.}%
    \renewcommand\color[2][]{}%
  }%
  \providecommand\includegraphics[2][]{%
    \GenericError{(gnuplot) \space\space\space\@spaces}{%
      Package graphicx or graphics not loaded%
    }{See the gnuplot documentation for explanation.%
    }{The gnuplot epslatex terminal needs graphicx.sty or graphics.sty.}%
    \renewcommand\includegraphics[2][]{}%
  }%
  \providecommand\rotatebox[2]{#2}%
  \@ifundefined{ifGPcolor}{%
    \newif\ifGPcolor
    \GPcolorfalse
  }{}%
  \@ifundefined{ifGPblacktext}{%
    \newif\ifGPblacktext
    \GPblacktexttrue
  }{}%
  % define a \g@addto@macro without @ in the name:
  \let\gplgaddtomacro\g@addto@macro
  % define empty templates for all commands taking text:
  \gdef\gplfronttext{}%
  \gdef\gplfronttext{}%
  \makeatother
  \ifGPblacktext
    % no textcolor at all
    \def\colorrgb#1{}%
    \def\colorgray#1{}%
  \else
    % gray or color?
    \ifGPcolor
      \def\colorrgb#1{\color[rgb]{#1}}%
      \def\colorgray#1{\color[gray]{#1}}%
      \expandafter\def\csname LTw\endcsname{\color{white}}%
      \expandafter\def\csname LTb\endcsname{\color{black}}%
      \expandafter\def\csname LTa\endcsname{\color{black}}%
      \expandafter\def\csname LT0\endcsname{\color[rgb]{1,0,0}}%
      \expandafter\def\csname LT1\endcsname{\color[rgb]{0,1,0}}%
      \expandafter\def\csname LT2\endcsname{\color[rgb]{0,0,1}}%
      \expandafter\def\csname LT3\endcsname{\color[rgb]{1,0,1}}%
      \expandafter\def\csname LT4\endcsname{\color[rgb]{0,1,1}}%
      \expandafter\def\csname LT5\endcsname{\color[rgb]{1,1,0}}%
      \expandafter\def\csname LT6\endcsname{\color[rgb]{0,0,0}}%
      \expandafter\def\csname LT7\endcsname{\color[rgb]{1,0.3,0}}%
      \expandafter\def\csname LT8\endcsname{\color[rgb]{0.5,0.5,0.5}}%
    \else
      % gray
      \def\colorrgb#1{\color{black}}%
      \def\colorgray#1{\color[gray]{#1}}%
      \expandafter\def\csname LTw\endcsname{\color{white}}%
      \expandafter\def\csname LTb\endcsname{\color{black}}%
      \expandafter\def\csname LTa\endcsname{\color{black}}%
      \expandafter\def\csname LT0\endcsname{\color{black}}%
      \expandafter\def\csname LT1\endcsname{\color{black}}%
      \expandafter\def\csname LT2\endcsname{\color{black}}%
      \expandafter\def\csname LT3\endcsname{\color{black}}%
      \expandafter\def\csname LT4\endcsname{\color{black}}%
      \expandafter\def\csname LT5\endcsname{\color{black}}%
      \expandafter\def\csname LT6\endcsname{\color{black}}%
      \expandafter\def\csname LT7\endcsname{\color{black}}%
      \expandafter\def\csname LT8\endcsname{\color{black}}%
    \fi
  \fi
    \setlength{\unitlength}{0.0400bp}%
    \ifx\gptboxheight\undefined%
      \newlength{\gptboxheight}%
      \newlength{\gptboxwidth}%
      \newsavebox{\gptboxtext}%
    \fi%
    \setlength{\fboxrule}{0.5pt}%
    \setlength{\fboxsep}{1pt}%
\begin{picture}(9600.00,4000.00)%
    \gplgaddtomacro\gplfronttext{%
      \colorrgb{0.15,0.15,0.15}%
      \put(828,2380){\makebox(0,0)[r]{\strut{}\scriptsize $0.0$}}%
      \colorrgb{0.15,0.15,0.15}%
      \put(828,2606){\makebox(0,0)[r]{\strut{}\scriptsize $0.10$}}%
      \colorrgb{0.15,0.15,0.15}%
      \put(828,2831){\makebox(0,0)[r]{\strut{}\scriptsize $0.20$}}%
      \colorrgb{0.15,0.15,0.15}%
      \put(828,3057){\makebox(0,0)[r]{\strut{}\scriptsize $0.30$}}%
      \colorrgb{0.15,0.15,0.15}%
      \put(828,3282){\makebox(0,0)[r]{\strut{}\scriptsize $0.40$}}%
      \colorrgb{0.15,0.15,0.15}%
      \put(828,3508){\makebox(0,0)[r]{\strut{}\scriptsize $0.50$}}%
      \colorrgb{0.15,0.15,0.15}%
      \put(828,3733){\makebox(0,0)[r]{\strut{}\scriptsize $0.60$}}%
      \colorrgb{0.15,0.15,0.15}%
      \put(828,3959){\makebox(0,0)[r]{\strut{}\scriptsize $0.70$}}%
    }%
    \gplgaddtomacro\gplfronttext{%
      \colorrgb{0.00,0.00,0.00}%
%      %\put(4799,4179){\makebox(0,0){\strut{}$\sigma_{\bm{M}} = 0.0$ m}}%
      %\put(600,4700){\makebox(0,0){\strut{}{\color{c1}{\rule[0.6mm]{0.5cm}{0.5mm}}}\scriptsize PLICP}}
      %\put(1600,4700){\makebox(0,0){\strut{}{\color{c2}{\rule[0.6mm]{0.5cm}{0.5mm}}}\scriptsize NDT}}
      %\put(2600,4700){\makebox(0,0){\strut{}{\color{c3}{\rule[0.6mm]{0.5cm}{0.5mm}}}\scriptsize GICP}}
      %\put(3700,4700){\makebox(0,0){\strut{}{\color{c4}{\rule[0.6mm]{0.5cm}{0.5mm}}}\scriptsize VGICP}}
      %\put(5000,4700){\makebox(0,0){\strut{}{\color{c5}{\rule[0.6mm]{0.5cm}{0.5mm}}}\scriptsize NDT-PSO}}
      %\put(6400,4700){\makebox(0,0){\strut{}{\color{c6}{\rule[0.6mm]{0.5cm}{0.5mm}}}\scriptsize TEASER}}
      %\put(7600,4700){\makebox(0,0){\strut{}{\color{c7}{\rule[0.6mm]{0.5cm}{0.5mm}}}\scriptsize \texttt{x1}}}
      %\put(8400,4700){\makebox(0,0){\strut{}{\color{c8}{\rule[0.6mm]{0.5cm}{0.5mm}}}\scriptsize \texttt{uf}}}
      %\put(9300,4700){\makebox(0,0){\strut{}{\color{c9}{\rule[0.6mm]{0.5cm}{0.5mm}}}\scriptsize \texttt{fm}}}
      %\put(4799,5200){\makebox(0,0){\strut{}Κατανομές σφαλμάτων στάσης}}%
    }%
    \gplgaddtomacro\gplfronttext{%
      \colorrgb{0.15,0.15,0.15}%
%      \put(828,40){\makebox(0,0)[r]{\strut{}$0.0$}}%
      %\colorrgb{0.15,0.15,0.15}%
      %\put(828,356){\makebox(0,0)[r]{\strut{}$0.02$}}%
      %\colorrgb{0.15,0.15,0.15}%
      %\put(828,672){\makebox(0,0)[r]{\strut{}$0.04$}}%
      %\colorrgb{0.15,0.15,0.15}%
      %\put(828,987){\makebox(0,0)[r]{\strut{}$0.06$}}%
      %\colorrgb{0.15,0.15,0.15}%
      %\put(828,1303){\makebox(0,0)[r]{\strut{}$0.08$}}%
      %\colorrgb{0.15,0.15,0.15}%
      %\put(828,1619){\makebox(0,0)[r]{\strut{}$0.10$}}%
      \colorrgb{0.15,0.15,0.15}%
      \put(1728,2100){\makebox(0,0){\strut{}\footnotesize $0.01$}}%
      \colorrgb{0.15,0.15,0.15}%
      \put(3264,2100){\makebox(0,0){\strut{}\footnotesize $0.03$}}%
      \colorrgb{0.15,0.15,0.15}%
      \put(4800,2100){\makebox(0,0){\strut{}\footnotesize $0.05$}}%
      \colorrgb{0.15,0.15,0.15}%
      \put(6335,2100){\makebox(0,0){\strut{}\footnotesize $0.10$}}%
      \colorrgb{0.15,0.15,0.15}%
      \put(7871,2100){\makebox(0,0){\strut{}\footnotesize $0.20$}}%
    }%
    \gplgaddtomacro\gplfronttext{%
      \colorrgb{0.15,0.15,0.15}%
      \put(4799,1710){\makebox(0,0){\strut{}\footnotesize Τυπική απόκλιση διαταραχών $\sigma_R$ [m]}}%
      \colorrgb{0.00,0.00,0.00}%
      \put(4799,4150){\makebox(0,0){\strut{}$\sigma_{\bm{M}} = 0.05$ m}}%
    }%
    \put(0,1980){\includegraphics[scale=0.8,clip=true, trim = 0 100 0 0]{./figures/slides/ch6/experiments/boxplots_position_errors_sm5}}%
    \gplfronttext
  \end{picture}%
\endgroup

    \vspace{1cm}
  \end{subfigure}%
\caption{\small Οι κατανομές των σφαλμάτων θέσης για κάθε αλγόριθμο με βάση
         την πειραματική διάταξη της ενότητας \ref{subsection:02_04_05:01}.
         Κάθε ορθογώνιο αντιπροσωπεύει τα στατιστικά $\simeq 4.5\cdot10^5$
         ευθυγραμμίσεων, με την εξαίρεση των μεθόδων NDT-PSO και TEASER οι
         οποίες δοκιμάστηκαν $\simeq 4.5 \cdot 10^4$ φορές}
\label{fig:appendix:02_04_05:01}
\end{figure}

\begin{figure}[!h]\centering
  \begin{subfigure}{\linewidth}
    \vspace{1cm}
    \input{./figures/parts/appendix/chapters/04/sections/05/boxplots_orientation_errors_sm0.tex}
    \vspace{1cm}
  \end{subfigure}\\%
  \begin{subfigure}{\linewidth}
    \vspace{1cm}
    \input{./figures/parts/appendix/chapters/04/sections/05/boxplots_orientation_errors_sm5.tex}
    \vspace{1cm}
  \end{subfigure}%
\caption{\small Οι κατανομές των σφαλμάτων προσανατολισμού για κάθε αλγόριθμο με
         βάση την πειραματική διάταξη της ενότητας
         \ref{subsection:02_04_05:01}.  Κάθε ορθογώνιο αντιπροσωπεύει τα
         στατιστικά $\simeq 4.5\cdot10^5$ ευθυγραμμίσεων, με την εξαίρεση των
         μεθόδων NDT-PSO και TEASER οι οποίες δοκιμάστηκαν $\simeq 4.5 \cdot
         10^4$ φορές}
\label{fig:appendix:02_04_05:02}
\end{figure}
