%%%%%%%%%%%%%%%%%%%%%%%%%%%%%%%%%%%%%%%%%%%%%%%%%%%%%%%%%%%%%%%%%%%%%%%%%%%%%%%%
\subsection{Στοιχεία αξιολόγησης στο περιβάλλον CORRIDOR}
\label{appendix:evaluation_corridor}

\subsubsection{Σχετικά με τους αλγορίθμους κατασκευής μονοπατιών}

Οι πίνακες \ref{tbl:info_global_plan_corridor}
και \ref{tbl:info_global_plan_map_corridor} καταγράφουν τις τιμές
των ποσοτικών μετρικών που αφορούν στους αλγορίθμους κατασκευής μονοπατιών που
ορίζονται στον πίνακα \ref{tbl:metrics_and_proportionality_global_planners} και
που προέκυψαν κατά τις $N = 10$ προσομοιώσεις στο περιβάλλον CORRIDOR.

Όσον αφορά στα παραγόμενα μονοπάτια ο \texttt{global\_planner} παρήγαγε
διαδρομές με το μικρότερο μήκος (πίνακας \ref{tbl:info_global_plan_corridor}),
ο \texttt{sbpl\_lattice\_planner} εκείνα με το μεγαλύτερο μήκος και τη
μικρότερη ανάλυση αλλά με τη μεγαλύτερη ομαλότητα (μικρότεροι αριθμοί
υποδηλώνουν υψηλότερη ομαλότητα), και ο \texttt{navfn} παρήγαγε τα λιγότερο
πυκνά μονοπάτια αλλά με τη χαμηλότερη ομαλότητα. Οι επιδόσεις του
\texttt{sbpl\_lattice\_planner} σε σχέση με το μήκος είναι λογικές, δεδομένου
ότι λαμβάνει υπόψη το κινηματικό μοντέλο του ρομπότ, το οποίο, όντας
διαφορικής κίνησης, και επομένως μη ολόνομικό (non-holonomic), περιορίζεται
στην κίνησή του. Αντίθετα,  οι \texttt{navfn} και \texttt{global\_planner} δεν
λαμβάνουν υπόψη τέτοιους περιορισμούς και, καθώς ο τελευταίος είναι ο διάδοχος
του πρώτου, παράγουν ελαφρώς παρόμοια μονοπάτια (αυτό παρατηρείται επίσης όταν
εξετάζονται τα στοιχεία των δύο παραγόμενων μονοπατιών: φαίνονται σχεδόν
πανομοιότυπα με γυμνό μάτι, σε πλήρη αντίθεση με εκείνα του
\texttt{sbpl\_lattice\_planner}). Μια άλλη παρατηρήσιμη διαφορά στο σχήμα
\ref{fig:global_plans:corridor} είναι ότι τα μονοπάτια που χαράζει ο
\texttt{navfn} και τα περισσότερα του \texttt{sbpl\_lattice\_planner} είναι
ντετερμινιστικά: δεδομένης μιας αρχικής στάσης $\bm{p}_0$, μιας θέσης στόχου
$\bm{p}_G$, και ενός χάρτη, αυτά παράγουν το ίδιο μονοπάτι κάθε φορά, ενώ ο
\texttt{global\_planner} εισάγει έναν μικρό βαθμό τυχαιότητας, το οποίο εξηγεί
γιατί η τυπική απόκλιση των σχεδίων του είναι μη μηδενική σε σύγκριση με τους
άλλους δύο αλγορίθμους.

\begin{table}[h]
\renewcommand{\arraystretch}{1.3}
\begin{tabular}{llccccc}
  & & \multicolumn{5}{c}{Μετρικές επίδοσης αλγορίθμων χάραξης μονοπατιών} \\
  \cline{3-7}
  GP & LP & $\mu_{l}(\bm{\mathcal{G}})$ [m] & $\sigma_{l}(\bm{\mathcal{G}})$ [m] & $\mu_r(\bm{\mathcal{G}})$ [στάσεις/m] & $\mu_{s}(\bm{\mathcal{G}})$ [rad] & $\sigma_{s}(\bm{\mathcal{G}})$ [rad] \\ \toprule
  \texttt{navfn} & \texttt{dwa} & $19.63$ & $0.00$ & $76.18$ & $2.42$ & $0.00$ \\
  \texttt{navfn} & \texttt{eband} & $19.63$ & $0.00$ & $76.18$ & $2.42$ & $0.00$ \\
  \texttt{navfn} & \texttt{teb} & $19.61$ & $0.02$ & $76.20$ & $2.42$ & $0.00$ \\
  \texttt{global\_planner} & \texttt{dwa} & $19.60$ & $0.01$ & $74.43$ & $2.40$ & $0.00$ \\
  \texttt{global\_planner} & \texttt{eband} & $19.59$ & $0.01$ & $74.70$ & $2.40$ & $0.00$ \\
  \texttt{global\_planner} & \texttt{teb} & $19.60$ & $0.01$ & $74.70$ & $2.40$ & $0.00$ \\
  \texttt{sbpl} & \texttt{dwa} & $22.92$ & $0.00$ & $53.25$ & $2.39$ & $0.00$ \\
  \texttt{sbpl} & \texttt{eband} & $22.92$ & $0.00$ & $53.41$ & $2.39$ & $0.00$ \\
  \texttt{sbpl} & \texttt{teb} & $22.92$ & $0.00$ & $53.33$ & $2.39$ & $0.00$ \\ \bottomrule
\end{tabular}
\caption{\small Μέσο συνολικό μήκος μονοπατιών $\mu_{l}(\bm{\mathcal{G}})$ και
         τυπική απόκλιση $\sigma_{l}(\bm{\mathcal{G}})$, μέση ανάλυση
         μονοπατιών $\mu_r(\bm{\mathcal{G}})$, μέση τιμή ομαλότητας
         $\mu_{s}(\bm{\mathcal{G}})$, και τυπική απόκλιση
         $\sigma_{s}(\bm{\mathcal{G}})$, για $N=10$ προσομοιώσεις στο
         περιβάλλον CORRIDOR}
\label{tbl:info_global_plan_corridor}
\end{table}

Όσον αφορά στην κρίσιμη ικανότητα ενός αλγορίθμου κατασκευής μονοπατιών να
σχεδιάζει γύρω από εμπόδια (πίνακας
\ref{tbl:info_global_plan_map_corridor}), ο \texttt{global\_planner} παρήγαγε
διαδρομές που δεν λαμβάνουν πλήρως υπόψη τους το αποτύπωμα του ρομπότ στο
οριζόντιο επίπεδο (η αφαίρεση της ακτίνας του ρομπότ από την ολικά ελάχιστη
απόσταση των μονοπατιών του από το πλησιέστερο εμπόδιο δίνει $-0.02$ m), και
επομένως ένας ελεγκτής κίνησης πλήρους πιστότητας στο σχεδιασθέν μονοπάτι θα
ανάγκαζε, με βεβαιότητα, το ρομπότ να ματαιώσει την αποστολή του (μέχρι να
τεθεί ίσως ένας νέος στόχος), ή ακόμη και να συγκρουστεί με εμπόδια στο
περιβάλλον του. Οι δύο εναπομείναντες αλγόριθμοι παρήγαγαν διαδρομές που θα
ανάγκαζαν το ρομπότ να συγκρουστεί με εμπόδια τουλάχιστον μία φορά. Επιπλέον, ο
\texttt{sbpl\_lattice\_planner} θέτει το ρομπότ να κινηθεί παράλληλα με
τοίχους, μια συμπεριφορά που μπορεί στην πραγματικότητα να υπαγορευτεί στον
αλγόριθμο (ο οποίος ρυθμίστηκε έτσι ώστε το ρομπότ να προτιμά να κινείται
σε ευθείες γραμμές), το οποίο μπορεί να θεωρηθεί πλεονέκτημα, δεδομένου ότι
υπάρχει πάντα ένα εμπόδιο αρκετά κοντά ώστε να μπορεί να αξιοποιηθεί ως σημείο
αναφοράς κατά τη διάρκεια χαρτογράφησης ή εντοπισμού της στάσης ενός ρομπότ.


\begin{table}[h]
\renewcommand{\arraystretch}{1.3}
\begin{tabular}{llccc}
  & & \multicolumn{3}{c}{Μετρικές επίδοσης αλγορίθμων χάραξης μονοπατιών σχετικές με εμπόδια} \\
  \cline{3-5}
  GP & LP & $\inf(d(\bm{\mathcal{G}},\bm{M}_C))$ [m] & $\mu(d(\bm{\mathcal{G}}, \bm{M}_C))$ [m] & $\sigma(d(\bm{\mathcal{G}},\bm{M}_C))$ [m] \\ \toprule
  \texttt{navfn} & \texttt{dwa} & $0.00$ & $0.52$ & $0.32$ \\
  \texttt{navfn} & \texttt{eband} & $0.00$ & $0.52$ & $0.32$ \\
  \texttt{navfn} & \texttt{teb} & $0.00$ & $0.52$ & $0.32$ \\
  \texttt{global\_planner} & \texttt{dwa} & \hspace{1.1cm} $0.00$ (-$0.02$) & $0.48$ & $0.31$ \\
  \texttt{global\_planner} & \texttt{eband} & \hspace{1.1cm} $0.00$ (-$0.02$) & $0.48$ & $0.31$ \\
  \texttt{global\_planner} & \texttt{teb} & \hspace{1.1cm} $0.00$ (-$0.02$) & $0.48$ & $0.32$ \\
  \texttt{sbpl} & \texttt{dwa} & $0.00$ & $0.29$ & $0.20$ \\
  \texttt{sbpl} & \texttt{eband} & $0.00$ & $0.29$ & $0.20$ \\
  \texttt{sbpl} & \texttt{teb} & $0.00$ & $0.29$ & $0.20$ \\ \bottomrule
\end{tabular}
\caption{\small Ολικά ελάχιστη απόσταση μονοπατιών $\bm{\mathcal{G}}$ από
         οποιοδήποτε εμπόδιο $\inf(d(\bm{\mathcal{G}},\bm{M}_C))$, μέση
         ελάχιστη απόσταση $\mu(d(\bm{\mathcal{G}},\bm{M}_C))$ και τυπική
         απόκλιση $\sigma(d(\bm{\mathcal{G}},\bm{M}_C))$ από όλα τα εμπόδια,
         για $N=10$ προσομοιώσεις στο περιβάλλον CORRIDOR}
\label{tbl:info_global_plan_map_corridor}
\end{table}

\subsubsection{Σχετικά με τους ελεγκτές κίνησης}

O πίνακας \ref{tbl:info_failures_corridor} καταγράφει τις τιμές των ποσοτικών
μετρικών που αφορούν στους ελεγκτές κίνησης που ορίζονται στον πίνακα
\ref{tbl:metrics_and_proportionality_local_planners} και που προέκυψαν κατά τις
$N = 10$ προσομοιώσεις στο περιβάλλον CORRIDOR.

Κανένας από τους συνδυασμούς του ελεγκτή κίνησης \texttt{dwa\_local\_planner}
με αλγορίθμους χάραξης μονοπατιών δεν ολοκλήρωσε αποστολή, και αυτό οφείλεται
στο γεγονός ότι ο ελεγκτής ξόδεψε τον περισσότερο χρόνο του εκτελώντας
συμπεριφορές ανάκτησης (έχει τον υψηλότερο μέσο όρο ανακτήσεων με περιστροφή
και εκκαθαρίσεων χαρτών κόστους μεταξύ των τριών ελεγκτών). Αυτό το γεγονός
είχε ως αποτέλεσμα είτε τη ματαίωση των αποστολών, είτε την αποτυχία λόγω
χρονικού time-out. Εν τέλει αυτό οφείλεται στο γεγονός ότι ο
\texttt{dwa\_local\_planner} ακολουθεί τα σχεδιασθέντα μονοπάτια με υψηλή
πιστότητα, τα οποία όμως είναι στην πραγματικότητα ανέφικτα, αφού η ολικά
ελάχιστη απόσταση από τα εμπόδια είναι το πολύ μηδέν (πίνακας
\ref{tbl:info_global_plan_map_corridor}). Επιπλέον, διαθέτει την υψηλότερη
αναλογία αποτυχιών διαδρομής ανά σύνολο κλήσεών του.

Ο ελεγκτής \texttt{eband\_local\_planner} είχε καλύτερες επιδόσεις από τον
\texttt{dwa\_local\_planner}: δεν διέκοψε ποτέ αποστολή, και δεν εκτέλεσε
συμπεριφορές ανάκτησης. Η τελική αποτυχία του είναι ότι δεν προκαλεί
κινήσεις σε εύλογα χρονικά διαστήματα (αυτό μπορεί να παρατηρηθεί στους μέσους
χρόνους διαδρομής που παρουσιάζονται στον πίνακα
\ref{tbl:info_pose_corridor}---υπενθυμίζουμε ότι $t_C^{max} = 120$ sec), δηλαδή
η προσέγγισή του είναι υπερβολικά ασφαλής. Ο συνδυασμός του με τον αλγόριθμο
\texttt{sbpl\_lattice\_planner} ήταν ο χειρότερος, κάτι που θα μπορούσε
θεωρητικά να αποδοθεί εν μέρει στο γεγονός ότι ο τελευταίος παράγει τα πιο
πυκνά και μακρύτερα σχέδια, αλλά στην πραγματικότητα οφείλεται σε ένα
άγνωστο ζήτημα που προκαλεί τον ελεγκτή να ανακηρύξει ότι το ρομπότ έφτασε στο
στόχο του ενώ στην πραγματικότητα εξακολουθεί να βρίσκεται στη μέση της
διαδρομής σε ορισμένες προσομοιώσεις (αυτός είναι ο δεύτερος λόγος για τον
οποίο στον \texttt{sbpl\_lattice\_planner} δόθηκε κατάσταση ανεπάρκειας στον
πίνακα \ref{tbl:qualitative_metrics}---ο πρώτος είναι το σφάλμα που βρέθηκε και
επιλύθηκε που αναφέρθηκε στην ενότητα \ref{subsection:02_01_04:01}).

Αντίθετα, ο ελεγκτής κίνησης \texttt{teb\_local\_planner} είχε την καλύτερη
επίδοση ανά μετρική: δεν ματαίωσε ποτέ αποστολή, δεν εκτέλεσε ούτε μία
συμπεριφορά ανάκτησης, δεν απέτυχε ποτέ να υπολογίσει έγκυρες ταχύτητες
κινητήρων, και ποτέ δεν απέτυχε στο να οδηγήσει το ρομπότ στη στάση-στόχο μέσα
στο προκαθορισμένο χρονικό όριο.

\begin{table}[h]
\renewcommand{\arraystretch}{1.3}
\begin{tabular}{llcccccccc}
  & & \multicolumn{8}{c}{Μετρικές επίδοσης ελεγκτών κίνησης} \\
  \cline{3-10}
    GP & LP & $\mu_{A} / N$ & $\mu_{RR}$ & $\sigma_{RR}$ & $\mu_{CC}$ & $\sigma_{CC}$ & $\mu_{PF}$ & $\sigma_{PF}$ & $\mu_{PF} / \mu_{LPC}$ \\ \toprule
    \texttt{navfn} & \texttt{dwa} & $0.90$ & $2.90$ & $0.57$ & $3.30$ & $0.67$ & $53.50$ & $17.35$ & $0.11$ \\
    \texttt{global\_planner} & \texttt{dwa} & $0.90$ & $3.30$ & $1.16$ & $2.70$ & $0.95$ & $58.90$ & $22.29$ & $0.10$ \\
    \texttt{sbpl} & \texttt{dwa} & $0.50$ & $3.30$ & $0.67$ & $3.00$ & $1.41$ & $8.50$ & $5.58$ & $0.02$ \\
    \texttt{navfn} & \texttt{eband} & $0.00$ & $0.00$ & $0.00$ & $0.00$ & $0.00$ & $1.10$ & $1.66$ & N/A \\
    \texttt{global\_planner} & \texttt{eband} & $0.00$ & $0.00$ & $0.00$ & $0.00$ & $0.00$ & $1.60$ & $1.84$ & N/A \\
    \texttt{sbpl} & \texttt{eband} & $0.00$ & $0.00$ & $0.00$ & $0.00$ & $0.00$ & $0.20$ & $0.42$ & N/A \\
    \texttt{navfn} & \texttt{teb} & $0.00$ & $0.00$ & $0.00$ & $0.00$ & $0.00$ & $0.00$ & $0.00$ & $0.00$ \\
    \texttt{global\_planner} & \texttt{teb} & $0.00$ & $0.00$ & $0.00$ & $0.00$ & $0.00$ & $0.00$ & $0.00$ & $0.00$ \\
    \texttt{sbpl} & \texttt{teb} & $0.00$ & $0.00$ & $0.00$ & $0.00$ & $0.00$ & $0.00$ & $0.00$ & $0.00$ \\ \bottomrule
\end{tabular}
\caption{\small Μέσος αριθμός ματαιωμένων αποστολών επί του αριθμού των
      προσομοιώσεων που πραγματοποιήθηκαν $\mu_A / N$, μέσος αριθμός ανακτήσεων
      με περιστροφή $\mu_{RR}$ και η τυπική τους απόκλιση $\sigma_{RR}$, μέσος
      αριθμός εκκαθαρίσεων χαρτών κόστους $\mu_{CC}$ και η τυπική τους απόκλιση
      $\sigma_{CC}$, μέσος αριθμός αποτυχιών διαδρομής $\mu_{PF}$ και η τυπική
      τους απόκλιση $\sigma_{PF}$, και μέσος αριθμός αποτυχιών διαδρομής επί
      του μέσου αριθμού των κλήσεων του ελεγκτή κίνησης $\mu_{PF} / \mu_{LPC}$,
      για όλους τους συνδυασμούς αλγορίθμων χάραξης μονοπατιών και ελεγκτών
      κίνησης που παρουσιάζονται στον πίνακα \ref{tbl:planners_sifted_list},
      για $N=10$ προσομοιώσεις στο περιβάλλον CORRIDOR}
\label{tbl:info_failures_corridor}
\end{table}



\subsubsection{Σχετικά με το συνδυασμό τους}

Οι πίνακες \ref{tbl:info_pose_corridor}, \ref{tbl:info_ground_truth_corridor},
\ref{tbl:info_ground_truth_map_corridor}, και
\ref{tbl:info_deviation_from_global_plan_corridor} καταγράφουν τις τιμές των
ποσοτικών μετρικών που αφορούν στους συνδυασμούς αλγορίθμων χάραξης μονοπατιών
και ελεγκτών κίνησης που ορίζονται στον πίνακα
\ref{tbl:metrics_and_proportionality_combination_planners}, και που προέκυψαν
κατά τις $N = 10$ προσομοιώσεις στο περιβάλλον CORRIDOR.

Σε όρους χρόνου που απαιτείται για την επίτευξη πλοήγησης από την αρχική στάση
στην στάση-στόχο (πίνακας \ref{tbl:info_pose_corridor}), όλοι οι συνδυασμοί των
αλγορίθμων χάραξης μονοπατιών με τον \texttt{dwa\_local\_planner} αποκλείονται
από αξιολόγηση (αφού αποτελεί προϋπόθεση το ρομπότ να φτάσει στο στόχο του),
και το ίδιο ισχύει και για το συνδυασμό των \texttt{sbpl\_lattice\_planner} και
\texttt{eband\_local\_planner}. Για τους υπόλοιπους συνδυασμούς (α) η χρήση του
\texttt{teb\_local\_planner} επιφέρει τις χαμηλότερου χρόνου διαδρομές (πράγμα
αναμενόμενο, αφού προσεγγίζει το πρόβλημα της πλοήγησης με όρους
βελτιστοποίησης σε σχέση με το χρόνο), (β) ο \texttt{eband\_local\_planner}
είναι ο πιο αργός μεταξύ των δύο---με σημαντική διαφορά, αφού χρειάζεται
περισσότερο από το διπλάσιο χρόνο για να ολοκληρώσει μια αποστολή, και (γ)
οι διαδρομές του πρώτου είναι οι πιο συνεπείς μεταξύ τους. Το γεγονός ότι ο
\texttt{sbpl\_lattice\_planner} παράγει μονοπάτια μεγαλύτερου μήκους---περίπου
$17\%$ μακρύτερα από εκείνα των άλλων δύο αλγορίθμων (πίνακας
\ref{tbl:info_global_plan_corridor})---έκανε τον συνδυασμό του με τον
\texttt{teb\_local\_planner} να εμφανίσει χρόνους πλοήγησης με διαφορά άνω των
δύο δευτερολέπτων, το οποίο μεταφράζεται σε περίπου $10\%$ αύξηση του χρόνου
διαδρομής σε σύγκριση με εκείνους του \texttt{teb\_local\_planner} με τον
\texttt{navfn} και τον \texttt{global\_planner}.

\begin{table}[h]
\renewcommand{\arraystretch}{1.3}
\begin{tabular}{llcc}
  & & \multicolumn{2}{c}{Μετρικές που αφορούν στον χρόνο πραγματικής διαδρομής} \\
  \cline{3-4}
  GP & LP & $\mu_{t}$ [sec] & $\sigma_{t}$ [sec] \\ \toprule
  \texttt{navfn} & \texttt{dwa} & $47.53$ & $14.85$ \\
  \texttt{global\_planner} & \texttt{dwa} & $55.98$ & $24.87$ \\
  \texttt{sbpl} & \texttt{dwa} & $78.72$ & $25.80$ \\
  \textbf{\texttt{navfn}} & \textbf{\texttt{eband}} & $\bm{107.52}$ & $\bm{0.81}$ \\
  \textbf{\texttt{global\_planner}} & \textbf{\texttt{eband}} & $\bm{106.86}$ & $\bm{1.00}$ \\
  \texttt{sbpl} & \texttt{eband} & $70.80$ & $17.93$ \\
  \textbf{\texttt{navfn}} & \textbf{\texttt{teb}} & $\bm{44.89}$ & $\bm{0.44}$ \\
  \textbf{\texttt{global\_planner}} & \textbf{\texttt{teb}} & $\bm{44.83}$ & $\bm{0.44}$ \\
  \textbf{\texttt{sbpl}} & \textbf{\texttt{teb}} & $\bm{46.61}$ & $\bm{0.24}$ \\ \bottomrule
\end{tabular}
\caption{\small Μέσος χρόνος διαδρομής $\mu_{t}$ από την αρχική στην τελική
         στάση και τυπική απόκλιση $\sigma_{t}$ για $N=10$ προσομοιώσεις στο
         περιβάλλον CORRIDOR. Οι συνδυασμοί που ολοκλήρωσαν όλες τις αποστολές
         και οι τιμές των αντίστοιχων μετρικών σημειώνονται με έντονη γραφή}
\label{tbl:info_pose_corridor}
\end{table}


Όσον αφορά στο μέσο μήκος των διανυόμενων διαδρομών
(πίνακας \ref{tbl:info_ground_truth_corridor}), οι ίδιοι συνδυασμοί global
planners με τους \texttt{eband\_local\_planner} και
\texttt{teb\_local\_planner} έκαναν το ρομπότ να διανύσει μεγαλύτερα μήκη σε
σύγκριση με τα σχεδιασθέντα μονοπάτια τους: και οι δύο προσεγγίσεις
παραμορφώνουν το παγκόσμιο σχέδιο προκειμένου να κερδίσουν μεγαλύτερη απόσταση
από εμπόδια, και αυτός είναι ο λόγος για τον οποίο ο
\texttt{dwa\_local\_planner} αποτυγχάνει σε κάθε προσομοίωση. Επιπλέον, οι
διαδρομές που ο \texttt{teb\_local\_planner} υπαγόρευσε στο ρομπότ ήταν οι
μακρύτερες αλλά οι πιο συνεπείς, και η πιο συνεπής από όλες παρατηρήθηκε όταν
χρησιμοποιήθηκε ο \texttt{navfn} ως αλγόριθμος χάραξης μονοπατιών, κάτι που
είναι αναμενόμενο, αφού τα σχεδιασθέντα μονοπάτια του είναι ντετερμινιστικά.
Όσον αφορά στην ομαλότητα των διαδρομών, ο συνδυασμός του
\texttt{sbpl\_lattice\_planner} με τον \texttt{teb\_local\_planner} παρουσίασε
την υψηλότερη τιμή και τη μικρότερη διακύμανση.


\begin{table}[h]
\renewcommand{\arraystretch}{1.3}
\begin{tabular}{llcccc}
  & & \multicolumn{4}{c}{Μετρικές σχετικές με τις διανυθείσες διαδρομές} \\
  \cline{3-6}
  GP & LP & $\mu_{l}(\bm{\mathcal{P}})$ [m] & $\sigma_{l}(\bm{\mathcal{P}})$ [m] & $\mu_{s}(\bm{\mathcal{P}})$ [rad] & $\sigma_{s}(\bm{\mathcal{P}})$ [rad] \\ \toprule
  \texttt{navfn} & \texttt{dwa} & $9.24$ & $3.37$ & $1.56$ & $0.16$ \\
  \texttt{global\_planner} & \texttt{dwa} & $8.62$ & $3.23$ & $1.66$ & $0.15$ \\
  \texttt{sbpl} & \texttt{dwa} & $9.12$ & $3.01$ & $1.60$ & $0.23$ \\
  \textbf{\texttt{navfn}} & \textbf{\texttt{eband}} & $\bm{20.15}$ & $\bm{0.09}$ & $\bm{2.36}$ & $\bm{0.01}$ \\
  \textbf{\texttt{global\_planner}} & \textbf{\texttt{eband}} & $\bm{20.04}$ & $\bm{0.07}$ & $\bm{2.36}$ & $\bm{0.01}$ \\
  \texttt{sbpl} & \texttt{eband} & $12.79$ & $3.52$ & $1.78$ & $0.27$ \\
  \textbf{\texttt{navfn}} & \textbf{\texttt{teb}} & $\bm{20.87}$ & $\bm{0.03}$ & $\bm{1.66}$ & $\bm{0.06}$ \\
  \textbf{\texttt{global\_planner}} & \textbf{\texttt{teb}} & $\bm{20.88}$ & $\bm{0.04}$ & $\bm{1.69}$ & $\bm{0.09}$ \\
  \textbf{\texttt{sbpl}} & \textbf{\texttt{teb}} & $\bm{22.99}$ & $\bm{0.06}$ & $\bm{1.65}$ & $\bm{0.02}$ \\ \bottomrule
\end{tabular}
\caption{\small Μέσο μήκος διαδρομής $\mu_{l}(\bm{\mathcal{P}})$ και τυπική
         απόκλιση $\sigma_{l}(\bm{\mathcal{P}})$, και μέση τιμή ομαλότητας
         διαδρομής $\mu_{s}(\bm{\mathcal{P}})$ και τυπική απόκλιση
         $\sigma_{s}(\bm{\mathcal{P}})$ για $N=10$ προσομοιώσεις στο περιβάλλον
         CORRIDOR. Συνδυασμοί που ολοκλήρωσαν όλες τις αποστολές  σημειώνονται
         με έντονη γραφή}
\label{tbl:info_ground_truth_corridor}
\end{table}

Όσον αφορά στην απόσταση από τα εμπόδια στο χάρτη $\bm{M}_C$ (πίνακας
\ref{tbl:info_ground_truth_map_corridor}), ο συνδυασμός του
\texttt{eband\_local\_planner} με τον \texttt{sbpl\_lattice\_planner} δεν
επέφερε συγκρούσεις με εμπόδια, ενώ οι αποστάσεις του από αυτά ήταν χαμηλότερες
από εκείνες του \texttt{teb\_local\_planner}. Η δυνατότητα παραμετροποίησης του
τελευταίου όσον αφορά στην ελάχιστη απόσταση από τα εμπόδια (που ορίστηκε σε
$0.10$m) έπαιξε σαφώς σημαντικό ρόλο στην απόσταση του ρομπότ από τα εμπόδια:
όντας $0.18$ m σε όλα τα πειράματα έδωσε στο ρομπότ τη μεγαλύτερη ελάχιστη
απόσταση από εμπόδια σε σύγκριση με του άλλους ελεγκτές (αυτός ήταν ένας ακόμη
λόγος για τον οποίο σημείωσε τέτοια βαθμολογία όσον αφορά στην ποιοτική μετρική
της παραμετροποιησιμότητας στον πίνακα \ref{tbl:qualitative_metrics}).
Επιπλέον, το ίδιο παρατηρείται όσον αφορά στη μέση απόσταση κάθε στάσης του
ρομπότ από το πλησιέστερο εμπόδιο στο χάρτη του κόσμου CORRIDOR, ενώ η
διακύμανσή της είναι η μικρότερη (σε σύγκριση με τους συνδυασμούς που
ολοκλήρωσαν την αποστολή).  Παρεμπιπτόντως, ο ελεγκτής
\texttt{dwa\_local\_planner} απέτυχε να αποφύγει εμπόδιο τουλάχιστον μία φορά
σε $N$ προσομοιώσεις όταν ο αντίστοιχος αλγόριθμος χάραξης μονοπατιών που
χρησιμοποιήθηκε λειτουργεί με άγνοια του κινηματικού μοντέλου του ρομπότ. Από
την άλλη πλευρά, όταν χρησιμοποιήθηκε ως αλγόριθμος χάραξης μονοπατιών ο
\texttt{sbpl\_lattice\_planner} το ρομπότ δεν συγκρούστηκε ούτε μία φορά με
εμπόδιο. Επιπλέον, η μέση απόστασή του από εμπόδια ήταν η υψηλότερη μεταξύ των
αλγορίθμων χάραξης μονοπατιών που χρησιμοποιήθηκαν, ενώ η τυπική της απόκλιση
ήταν η χαμηλότερη.



\begin{table}[h]
\renewcommand{\arraystretch}{1.3}
\begin{tabular}{llccc}
  & & \multicolumn{3}{c}{Μετρικές που αφορούν στα εμπόδια και τις πραγματικές διαδρομές} \\
  \cline{3-5}
  GP & LP & $\inf(d(\bm{\mathcal{P}},\bm{M}_C))$ [m] & $\mu(d(\bm{\mathcal{P}},\bm{M}_C))$ [m] & $\sigma(d(\bm{\mathcal{P}},\bm{M}_C))$ [m] \\ \toprule
  \texttt{navfn} & \texttt{dwa} & \hspace{0.82cm} $0.00$ (-$0.02$) & $0.17$ & $0.19$ \\
  \texttt{global\_planner} & \texttt{dwa} & \hspace{0.82cm} $0.00$ (-$0.02$) & $0.15$ & $0.18$ \\
  \texttt{sbpl} & \texttt{dwa} & $0.06$ & $0.21$ & $0.16$ \\
  \textbf{\texttt{navfn}} & \textbf{\texttt{eband}} & $\bm{0.07}$ & $\bm{0.55}$ & $\bm{0.27}$ \\
  \textbf{\texttt{global\_planner}} & \textbf{\texttt{eband}} & $\bm{0.09}$ & $\bm{0.53}$ & $\bm{0.27}$ \\
  \texttt{sbpl} & \texttt{eband} & $0.10$ & $0.40$ & $0.17$ \\
  \textbf{\texttt{navfn}} & \textbf{\texttt{teb}} & $\bm{0.18}$ & $\bm{0.64}$ & $\bm{0.19}$ \\
  \textbf{\texttt{global\_planner}} & \textbf{\texttt{teb}} & $\bm{0.18}$ & $\bm{0.64}$ & $\bm{0.20}$ \\
  \textbf{\texttt{sbpl}} & \textbf{\texttt{teb}} & $\bm{0.18}$ & $\bm{0.49}$ & $\bm{0.16}$ \\ \bottomrule
\end{tabular}
\caption{\small Ολικά ελάχιστη απόσταση των πραγματικών διαδρομών
         $\bm{\mathcal{P}}$ που διήνυσε το ρομπότ από οποιοδήποτε εμπόδιο σε
         όλες τις προσομοιώσεις $\inf(d(\bm{\mathcal{P}},\bm{M}_C))$, μέση
         ελάχιστη απόσταση $\mu(d(\bm{\mathcal{P}},\bm{M}_C))$ από όλα τα
         εμπόδια για $N=10$ προσομοιώσεις στο χάρτη CORRIDOR $\bm{M}_C$, και μέση τυπική απόκλιση
         $\sigma(d(\bm{\mathcal{P}},\bm{M}_C))$. Συνδυασμοί που
         ολοκλήρωσαν όλες τις αποστολές σημειώνονται με έντονη γραφή}
\label{tbl:info_ground_truth_map_corridor}
\end{table}

Όσον αφορά στην απόκλιση των διανυθέντων μονοπατιών από τα αντίστοιχα
σχεδιασθέντα μονοπάτια (πίνακας
\ref{tbl:info_deviation_from_global_plan_corridor}), ο ελεγκτής κίνησης
\texttt{dwa\_local\_planner} παρουσίασε τη χαμηλότερη μέση απόκλιση θέσης στους
συνδυασμούς του με αλγορίθμους χάραξης μονοπατιών που δεν λαμβάνουν υπόψη τους
το κινηματικό μοντέλο του ρομπότ, κάτι που είναι αναμενόμενο, καθώς, όπως
συζητήθηκε προηγουμένως, είναι ο ελεγκτής με τη μεγαλύτερη πιστότητα στο
σχεδιασθέν μονοπάτι μεταξύ των τριών ελεγκτών. Ωστόσο, λόγω της αδυναμίας του
να ολοκληρώσει έστω και μία αποστολή, όλοι οι συνδυασμοί του με αλγορίθμους
χάραξης μονοπατιών αποκλείονται από την αξιολόγηση, και το ίδιο ισχύει και για
τον συνδυασμό του \texttt{sbpl\_lattice\_planner} με τον
\texttt{eband\_local\_planner}. Από τους υπόλοιπους συνδυασμούς, εκείνοι που
χρησιμοποιούν τον \texttt{teb\_local\_planner} παρουσιάζουν τη μικρότερη μέση
απόκλιση κάθε στάσης σε σχέση με το συνολικό σχέδιο, και, ειδικότερα, ο
συνδυασμός του με τον \texttt{sbpl\_lattice\_planner} παρουσιάζει τη μικρότερη
συνολική απόκλιση μεταξύ όλων των άλλων συνδυασμών. Επιπλέον, η μέση διακριτή
απόσταση Frechet ήταν σταθερά χαμηλότερη από εκείνη του
\texttt{eband\_local\_planner}.

\begin{table}[h]
\renewcommand{\arraystretch}{1.3}
\begin{tabular}{llccc}
  & & \multicolumn{3}{c}{Μετρικές σχετικές με την απόκλιση διαδρομών από σχεδιασθέντα μονοπάτια} \\
  \cline{3-5}
  GP & LP & $\mu_{\delta}(\bm{\mathcal{P}},\bm{\mathcal{G}})$ [m] & $\mu_{\delta}(\bm{\mathcal{P}},\bm{\mathcal{G}})$ [m] & $\mu_{\delta}^F(\bm{\mathcal{P}},\bm{\mathcal{G}})$ [m] \\ \toprule
  \texttt{navfn} & \texttt{dwa} & $0.04$ & $45.98$ & $5.58$ \\
  \texttt{global\_planner} & \texttt{dwa} & $0.04$ & $39.39$ & $5.28$ \\
  \texttt{sbpl} & \texttt{dwa} & $0.09$ & $101.28$ & $5.08$ \\
  \textbf{\texttt{navfn}} & \textbf{\texttt{eband}} & $\bm{0.10}$ & $\bm{104.96}$ & $\bm{0.38}$ \\
  \textbf{\texttt{global\_planner}} & \textbf{\texttt{eband}} & $\bm{0.13}$ & $\bm{139.69}$ & $\bm{0.47}$ \\
  \texttt{sbpl} & \texttt{eband} & $0.13$ & $145.06$ & $5.06$ \\
  \textbf{\texttt{navfn}} & \textbf{\texttt{teb}} & $\bm{0.07}$ & $\bm{81.96}$ & $\bm{0.26}$ \\
  \textbf{\texttt{global\_planner}} & \textbf{\texttt{teb}} & $\bm{0.08}$ & $\bm{89.89}$ & $\bm{0.29}$ \\
  \textbf{\texttt{sbpl}} & \textbf{\texttt{teb}} & $\bm{0.07}$ & $\bm{73.59}$ & $\bm{0.26}$ \\ \bottomrule
\end{tabular}
\caption{\small Μέση απόκλιση $\mu_{\delta}(\bm{\mathcal{P}},\bm{\mathcal{G}})$,
         μέση συνολική απόκλιση
         $\mu_{\Delta}(\bm{\mathcal{P}},\bm{\mathcal{G}})$, και μέση απόσταση
         Frechet $\mu_{\delta}^F(\bm{\mathcal{P}},\bm{\mathcal{G}})$ μεταξύ των
         πραγματικών διαδρομών $\bm{\mathcal{P}}$ που ακολούθησε το ρομπότ και
         των αντίστοιχων σχεδιασθέντων μονοπατιών $\bm{\mathcal{G}}$ για $N=10$
         προσομοιώσεις στο χάρτη CORRIDOR $\bm{M}_C$. Συνδυασμοί που
         ολοκλήρωσαν τουλάχιστον μία αποστολή σημειώνονται με έντονη γραφή}
\label{tbl:info_deviation_from_global_plan_corridor}
\end{table}




%%%%%%%%%%%%%%%%%%%%%%%%%%%%%%%%%%%%%%%%%%%%%%%%%%%%%%%%%%%%%%%%%%%%%%%%%%%%%%%%
\subsection{Στοιχεία αξιολόγησης στο περιβάλλον WILLOWGARAGE}
\label{appendix:evaluation_willowgarage}

\subsubsection{Σχετικά με τους αλγορίθμους κατασκευής μονοπατιών}

Οι πίνακες \ref{tbl:info_global_plan_willowgarage}
και \ref{tbl:info_global_plan_map_willowgarage} καταγράφουν τις τιμές
των ποσοτικών μετρικών που αφορούν στους αλγορίθμους κατασκευής μονοπατιών που
ορίζονται στον πίνακα \ref{tbl:metrics_and_proportionality_global_planners} και
που προέκυψαν κατά τις $N = 10$ προσομοιώσεις στο περιβάλλον WILLOWGARAGE.

Όσον αφορά στα παραγόμενα μονοπάτια (πίνακας
\ref{tbl:info_global_plan_willowgarage}), και σε σχέση με τις μετρικές που
αφορούν την αξιολόγηση των αλγορίθμων χάραξης μονοπατιών, τίποτα δεν άλλαξε σε
σύγκριση με εκείνες που αφορούν στο περιβάλλον CORRIDOR: o
\texttt{global\_planner} παρήγαγε και πάλι μονοπάτια με το μικρότερο μήκος, o
\texttt{sbpl\_lattice\_planner} εκείνα με το μεγαλύτερο μήκος και τη χαμηλότερη
ανάλυση αλλά με τη μεγαλύτερη ομαλότητα, και ο \texttt{navfn} παρήγαγε τα
λιγότερο πυκνά μονοπάτια, αλλά με τη χαμηλότερη ομαλότητα.

\begin{table}[h]
\renewcommand{\arraystretch}{1.3}
\begin{tabular}{llccccc}
  & & \multicolumn{5}{c}{Μετρικές επίδοσης αλγορίθμων χάραξης μονοπατιών} \\
  \cline{3-7}
  GP & LP & $\mu_{l}(\bm{\mathcal{G}})$ [m] & $\sigma_{l}(\bm{\mathcal{G}})$ [m] & $\mu_r(\bm{\mathcal{G}})$ [στάσεις/m] & $\mu_{s}(\bm{\mathcal{G}})$ [rad] & $\sigma_{s}(\bm{\mathcal{G}})$ [rad] \\ \toprule
  \texttt{navfn} & \texttt{dwa} & $44.50$ & $0.02$ & $37.15$ & $1.99$ & $0.00$ \\
  \texttt{navfn} & \texttt{eband} & $44.50$ & $0.02$ & $37.15$ & $1.99$ & $0.00$ \\
  \texttt{navfn} & \texttt{teb} & $44.53$ & $0.04$ & $37.10$ & $1.99$ & $0.00$ \\
  \texttt{global\_planner} & \texttt{dwa} & $44.48$ & $0.00$ & $36.76$ & $1.97$ & $0.00$ \\
  \texttt{global\_planner} & \texttt{eband} & $44.49$ & $0.01$ & $36.61$ & $1.97$ & $0.00$ \\
  \texttt{global\_planner} & \texttt{teb} & $44.49$ & $0.01$ & $36.64$ & $1.97$ & $0.00$ \\
  \texttt{sbpl} & \texttt{dwa} & $48.01$ & $0.00$ & $30.93$ & $2.02$ & $0.00$ \\
  \texttt{sbpl} & \texttt{eband} & $48.01$ & $0.00$ & $30.93$ & $2.02$ & $0.00$ \\
  \texttt{sbpl} & \texttt{teb} & $48.01$ & $0.01$ & $30.95$ & $2.02$ & $0.00$ \\ \bottomrule
\end{tabular}
\caption{\small Μέσο συνολικό μήκος μονοπατιών $\mu_{l}(\bm{\mathcal{G}})$ και τυπική
         απόκλιση $\sigma_{l}(\bm{\mathcal{G}})$, μέση ανάλυση μονοπατιών
         $\mu_r(\bm{\mathcal{G}})$, μέση τιμή ομαλότητας
         $\mu_{s}(\bm{\mathcal{G}})$, και τυπική απόκλιση
         $\sigma_{s}(\bm{\mathcal{G}})$, για $N=10$ προσομοιώσεις στο περιβάλλον
         WILLOWGARAGE}
\label{tbl:info_global_plan_willowgarage}
\end{table}

Αυτό που άλλαξε ήταν η ολικά ελάχιστη απόσταση των μονοπατιών του
\texttt{sbpl\_lattice\_planner} από τα εμπόδια (πίνακας
\ref{tbl:info_global_plan_map_willowgarage}): ενώ στο χάρτη του κόσμου CORRIDOR
δεν σχεδίασε ούτε μία φορά \textit{μέσα} από εμπόδια, στο χάρτη του κόσμου
WILLOWGARAGE το έκανε---όπως έκανε και πάλι ο \texttt{global\_planner}. Εκτός
από αυτό, παρουσίασε και πάλι τη χαμηλότερη μέση ελάχιστη απόσταση από εμπόδια
και τη μεγαλύτερη συνέπεια απόστασης γύρω από τη μέση τιμή. Από την άλλη
πλευρά, ο \texttt{navfn} κέρδισε, κατά μέσο όρο, ένα εκατοστό απόστασης, και η
απόδοσή του σε σχέση με τη μέση ελάχιστη απόσταση κάθε στάσης από εμπόδια
ήταν ισοδύναμη με εκείνη του \texttt{global\_planner}, όπως ήταν επίσης και η
τυπική του απόκλιση.

\begin{table}[h]
\renewcommand{\arraystretch}{1.3}
\begin{tabular}{llccc}
  & & \multicolumn{3}{c}{Μετρικές επίδοσης αλγορίθμων χάραξης μονοπατιών σχετικές με εμπόδια} \\
  \cline{3-5}
  GP & LP & $\inf(d(\bm{\mathcal{G}},\bm{M}_W))$ [m] & $\mu(d(\bm{\mathcal{G}},\bm{M}_W))$ [m] & $\sigma(d(\bm{\mathcal{G}},\bm{M}_W))$ [m] \\ \toprule
  \texttt{navfn} & \texttt{dwa} & $0.01$ & $0.51$ & $0.52$ \\
  \texttt{navfn} & \texttt{eband} & $0.01$ & $0.51$ & $0.52$ \\
  \texttt{navfn} & \texttt{teb} & $0.01$ & $0.51$ & $0.52$ \\
  \texttt{global\_planner} & \texttt{dwa} & \hspace{1.1cm} $0.00$ (-$0.02$) & $0.51$ & $0.53$ \\
  \texttt{global\_planner} & \texttt{eband} & \hspace{1.1cm} $0.00$ (-$0.02$) & $0.51$ & $0.53$ \\
  \texttt{global\_planner} & \texttt{teb} & \hspace{1.1cm} $0.00$ (-$0.02$) & $0.51$ & $0.53$ \\
  \texttt{sbpl} & \texttt{dwa} & \hspace{1.1cm} $0.00$ (-$0.02$) & $0.35$ & $0.43$ \\
  \texttt{sbpl} & \texttt{eband} & \hspace{1.1cm} $0.00$ (-$0.02$) & $0.35$ & $0.43$ \\
  \texttt{sbpl} & \texttt{teb} & \hspace{1.1cm} $0.00$ (-$0.02$) & $0.35$ & $0.43$ \\ \bottomrule
\end{tabular}
\caption{\small Ολικά ελάχιστη απόσταση μονοπατιών $\bm{\mathcal{G}}$ από
         οποιοδήποτε εμπόδιο $\inf(d(\bm{\mathcal{G}},\bm{M}_W))$, μέση
         ελάχιστη απόσταση $\mu(d(\bm{\mathcal{G}},\bm{M}_W))$ και τυπική
         απόκλιση $\sigma(d(\bm{\mathcal{G}},\bm{M}_W))$ από όλα τα εμπόδια,
         για $N=10$ προσομοιώσεις στο περιβάλλον WILLOWGARAGE}
\label{tbl:info_global_plan_map_willowgarage}
\end{table}


\subsubsection{Σχετικά με τους ελεγκτές κίνησης}

O πίνακας \ref{tbl:info_failures_willowgarage} καταγράφει τις τιμές των
ποσοτικών μετρικών που αφορούν στους ελεγκτές κίνησης που ορίζονται στον πίνακα
\ref{tbl:metrics_and_proportionality_local_planners} και που προέκυψαν κατά τις
$N = 10$ προσομοιώσεις στο περιβάλλον WILLOWGARAGE.

Το αυξημένο επίπεδο δυσκολίας πλοήγησης του κόσμου WILLOWGARAGE εξέθεσε τις
περισσότερες αδυναμίες των ελεγκτών κίνησης. Αυτό που είναι εντυπωσιακό είναι
ότι όλοι οι συνδυασμοί αλγορίθμων χάραξης μονοπατιών με τους
\texttt{dwa\_local\_planner} και \texttt{eband\_local\_planner} απέτυχαν να
μεταφέρουν το ρομπότ από την αρχική στην τελική στάση σε όλες τις
προσομοιώσεις που πραγματοποιήθηκαν.  Όταν χρησιμοποιήθηκαν αλγόριθμοι χάραξης
μονοπατιών που δεν λαμβάνουν υπόψη το κινηματικό μοντέλο του ρομπότ, ο πρώτος
ματαίωσε όλες τις αποστολές, ενώ ματαίωσε τις περισσότερες από αυτές (7 στις
10) στην αντίθετη περίπτωση (---είναι σαφές ότι η χρήση ενός αλγορίθμου χάραξης
μονοπατιών που λαμβάνει υπόψη τους περιορισμούς κίνησης του ρομπότ είναι
πλεονέκτημα στην περίπτωση ενός ``άκαμπτου" ελεγκτή κίνησης). Ο πρώτος
παρουσίασε και πάλι τον υψηλότερο αριθμό αποτυχιών ελέγχου στην πρώτη περίπτωση
και, στη δεύτερη περίπτωση, τον υψηλότερο μέσο αριθμό εκκαθαρίσεων χαρτών
κόστους.

Όσον αφορά στην επίδοση του \texttt{eband\_local\_planner} στο χάρτη
$\bm{M}_W$, ισχύει το ίδιο που ισχύει και στο χάρτη $\bm{M}_C$ στην περίπτωση
του συνδυασμού του με τον \texttt{sbpl\_lattice\_planner}: παρουσίασε τον
χαμηλότερο αριθμό αποτυχιών διαδρομής (τουλάχιστον τρεις φορές μικρότερο από
τον αμέσως επόμενο χαμηλότερο συνδυασμό). Παρόλο που ο
\texttt{eband\_local\_planner} κατάφερε να κάνει το ρομπότ να διανύσει
σημαντικά μεγαλύτερες αποστάσεις σε σύγκριση με τον
\texttt{dwa\_local\_planner} (εικόνα \ref{fig:ground_truths:willowgarage}), και
παρόλο που ήταν συνεπής στην πλοήγησή του (το σφάλμα λογισμικού που αναφέρθηκε
στην προηγούμενη ενότητα σχετικά με το συνδυασμό του με
\texttt{sbpl\_lattice\_planner} δεν εμφανίστηκε στο περιβάλλον WILLOWGARAGE),
χρειάστηκε και πάλι περισσότερο από τον προκαθορισμένο χρόνο σε κάθε
προσομοίωση, δείχνοντας την υπερβολικά ασφαλή προσέγγισή του (οι μέσοι χρόνοι
του ήταν σταθερά αργοί, όπως παρατηρήθηκε και στις προσομοιώσεις του στο
περιβάλλον CORRIDOR). Οι μέσοι χρόνοι πλοήγησης απεικονίζονται στον πίνακα
\ref{tbl:info_pose_willowgarage} (υπενθυμίζεται ότι $t_W^{max} = 180$ sec).

Σε αντίθεση με όλους τους συνδυασμούς των αλγορίθμων χάραξης μονοπατιών με τους
\texttt{dwa\_local\_planner} και \texttt{eband\_local\_planner}, όλοι οι
συνδυασμοί τους με τον \texttt{teb\_local\_planner} κατάφεραν να διανύσουν τη
διαδρομή από την αρχική στην τελική στάση. Και πάλι ήταν πρώτος στη μη
ματαίωση αποστολής, στην επίτευξη της στάσης στόχου σε όλες τις προσομοιώσεις,
στη μη εκτέλεση έστω και μιας συμπεριφορά ανάκαμψης. Μόνο ο συνδυασμός του με
τον αλγόριθμο χάραξης μονοπατιών \texttt{sbpl\_lattice\_planner} απέτυχε να
εξασφαλίσει έγκυρες εισόδους κινητήρων, αλλά μόνο ελάχιστα (---η χρήση ενός
global planner που λαμβάνει υπόψη του το κινηματικό μοντέλο της βάσης του
ρομπότ δεν φαίνεται να είναι ιδιαίτερα επωφελής στην περίπτωση ενός ευέλικτου
ελεγκτή κίνησης όπως στην περίπτωση ενός άκαμπτου ελεγκτή).

\begin{table}[h]
\renewcommand{\arraystretch}{1.3}
\begin{tabular}{llcccccccc}
  & & \multicolumn{8}{c}{Μετρικές επίδοσης ελεγκτών κίνησης} \\
  \cline{3-10}
    GP & LP & $\mu_{A} / N$ & $\mu_{RR}$ & $\sigma_{RR}$ & $\mu_{CC}$ & $\sigma_{CC}$ & $\mu_{PF}$ & $\sigma_{PF}$ & $\mu_{PF} / \mu_{LPC}$ \\ \toprule
    \texttt{navfn} & \texttt{dwa} & $1.00$ & $2.00$ & $0.00$ & $3.00$ & $0.00$ & $45.80$ & $10.97$ & $0.09$ \\
    \texttt{global\_planner} & \texttt{dwa} & $1.00$ & $2.30$ & $0.48$ & $3.00$ & $0.00$ & $35.40$ & $10.94$ & $0.08$ \\
    \texttt{sbpl} & \texttt{dwa} & $0.70$ & $3.20$ & $1.48$ & $3.60$ & $1.07$ & $12.00$ & $4.85$ & $0.03$ \\
    \texttt{navfn} & \texttt{eband} & $0.00$ & $0.00$ & $0.00$ & $0.00$ & $0.00$ & $20.10$ & $25.78$ & N/A \\
    \texttt{global\_planner} & \texttt{eband} & $0.00$ & $0.00$ & $0.00$ & $0.00$ & $0.00$ & $8.00$ & $10.87$ & N/A \\
    \texttt{sbpl} & \texttt{eband} & $0.00$ & $0.00$ & $0.00$ & $0.00$ & $0.00$ & $3.20$ & $2.15$ & N/A \\
    \texttt{navfn} & \texttt{teb} & $0.00$ & $0.00$ & $0.00$ & $0.00$ & $0.00$ & $0.00$ & $0.00$ & $0.00$ \\
    \texttt{global\_planner} & \texttt{teb} & $0.00$ & $0.00$ & $0.00$ & $0.00$ & $0.00$ & $0.00$ & $0.00$ & $0.00$ \\
    \texttt{sbpl} & \texttt{teb} & $0.00$ & $0.00$ & $0.00$ & $0.00$ & $0.00$ & $0.10$ & $0.32$ & $0.00$ \\ \bottomrule
\end{tabular}
\caption{\small Μέσος αριθμός ματαιωμένων αποστολών επί του αριθμού των
      προσομοιώσεων που πραγματοποιήθηκαν $\mu_A / N$, μέσος αριθμός ανακτήσεων
      με περιστροφή $\mu_{RR}$ και η τυπική τους απόκλιση $\sigma_{RR}$, μέσος
      αριθμός εκκαθαρίσεων χαρτών κόστους $\mu_{CC}$ και η τυπική τους απόκλιση
      $\sigma_{CC}$, μέσος αριθμός αποτυχιών διαδρομής $\mu_{PF}$ και η τυπική
      τους απόκλιση $\sigma_{PF}$, και μέσος αριθμός αποτυχιών διαδρομής επί
      του μέσου αριθμού των κλήσεων του ελεγκτή κίνησης $\mu_{PF} / \mu_{LPC}$,
      για όλους τους συνδυασμούς αλγορίθμων χάραξης μονοπατιών και ελεγκτών
      κίνησης που παρουσιάζονται στον πίνακα \ref{tbl:planners_sifted_list},
      για $N=10$ προσομοιώσεις στο περιβάλλον WILLOWGARAGE}
\label{tbl:info_failures_willowgarage}
\end{table}


\subsubsection{Σχετικά με το συνδυασμό τους}

Οι πίνακες \ref{tbl:info_pose_willowgarage},
\ref{tbl:info_ground_truth_willowgarage},
\ref{tbl:info_ground_truth_map_willowgarage}, και
\ref{tbl:info_deviation_from_global_plan_willowgarage} καταγράφουν τις τιμές
των ποσοτικών μετρικών που αφορούν στους συνδυασμούς αλγορίθμων χάραξης
μονοπατιών και ελεγκτών κίνησης που ορίζονται στον πίνακα
\ref{tbl:metrics_and_proportionality_combination_planners}, και που προέκυψαν
κατά τις $N = 10$ προσομοιώσεις στο περιβάλλον WILLOWGARAGE.

Όσον αφορά στο χρόνο που απαιτείται για την επίτευξη της πλοήγησης από την
αρχική στην επιθυμητή στάση (πίνακας \ref{tbl:info_pose_willowgarage}), όλοι οι
συνδυασμοί που περιλαμβάνουν τους \texttt{dwa\_local\_planner} και
\texttt{eband\_local\_planner} αποκλείονται από την αξιολόγηση λόγω της
αδυναμίας τους να οδηγήσουν το ρομπότ στο στόχο του. Ωστόσο, πρέπει να
σημειωθεί ότι ενώ οι συνδυασμοί του \texttt{dwa\_local\_planner} με αλγορίθμους
χάραξης μονοπατιών που αγνοούν το κινηματικό μοντέλο του ρομπότ δεν μπόρεσαν να
το περάσουν ούτε από το πρώτο άνοιγμα (μια πόρτα), ο συνδυασμός του με τον
\texttt{sbpl\_lattice\_planner} κατάφερε να το κάνει, και μάλιστα για τα
επόμενα τέσσερα ανοίγματα, προτού κολλήσει στο έκτο. Οι υπόλοιποι
συνδυασμοί---όλοι εκείνοι που διαθέτουν τον \texttt{teb\_local\_planner} ως
ελεγκτή κίνησής τους---έκαναν το ρομπότ να χρειαστεί λίγο περισσότερο από το
μισό του μέγιστου χρόνου για να διανύσει το σύνολο της διαδρομής από τη
στάση $\bm{p}_0^W$ στην $\bm{p}_G^W$. Το γεγονός ότι ο
\texttt{sbpl\_lattice\_planner} παράγει σχέδια μεγαλύτερου μήκους---περίπου
$4.5\%$ μακρύτερα από αυτά των άλλων δύο αντίστοιχων αλγορίθμων (πίνακας
\ref{tbl:info_global_plan_willowgarage})---είχε αντίκτυπο στο συνολικό χρόνο
πλοήγησης του ρομπότ στον συνδυασμό του με τον \texttt{teb\_local\_planner}, με
αύξηση λίγο πάνω από $5$ δευτερόλεπτα, η οποία μεταφράζεται σε περίπου την
ίδια ($5\%$) αύξηση στους χρόνους διαδρομής σε σύγκριση με εκείνους τους
συνδυασμού του \texttt{teb\_local\_planner} με τον \texttt{navfn} και τον
\texttt{global\_planner}.

\begin{table}[h]
\renewcommand{\arraystretch}{1.3}
\begin{tabular}{llcc}
  & & \multicolumn{2}{c}{Μετρικές που αφορούν στον χρόνο πραγματικής διαδρομής} \\
  \cline{3-4}
  GP & LP & $\mu_{t}$ [sec] & $\sigma_{t}$ [sec] \\  \toprule
  \texttt{navfn} & \texttt{dwa} & $20.46$ & $19.24$ \\
  \texttt{global\_planner} & \texttt{dwa} & $22.76$ & $18.85$ \\
  \texttt{sbpl} & \texttt{dwa} & $78.91$ & $48.12$ \\
  \texttt{navfn} & \texttt{eband} & $158.14$ & $5.25$ \\
  \texttt{global\_planner} & \texttt{eband} & $151.23$ & $2.00$ \\
  \texttt{sbpl} & \texttt{eband} & $147.53$ & $2.46$ \\
  \textbf{\texttt{navfn}} & \textbf{\texttt{teb}} & $\bm{95.45}$ & $\bm{0.34}$ \\
  \textbf{\texttt{global\_planner}} & \textbf{\texttt{teb}} & $\bm{95.50}$ & $\bm{0.41}$ \\
  \textbf{\texttt{sbpl}} & \textbf{\texttt{teb}} & $\bm{100.55}$ & $\bm{1.56}$ \\ \bottomrule
\end{tabular}
\caption{\small Μέσος χρόνος διαδρομής $\mu_{t}$ από την αρχική στην τελική
         στάση και τυπική απόκλιση $\sigma_{t}$ για $N=10$ προσομοιώσεις στο
         περιβάλλον WILLOWGARAGE. Οι συνδυασμοί που ολοκλήρωσαν όλες τις
         αποστολές και οι τιμές των αντίστοιχων μετρικών σημειώνονται με έντονη
         γραφή}
\label{tbl:info_pose_willowgarage}
\end{table}

Όσον αφορά στο μέσο μήκος των διανυόμενων διαδρομών
(πίνακας\ref{tbl:info_ground_truth_willowgarage}), παρατηρείται και πάλι ότι,
λόγω της παραμόρφωσης του σχεδιασθέντος μονοπατιού από το
\texttt{teb\_local\_planner} έτσι ώστε να επιτυγχάνεται η καθορισμένη ελάχιστη
απόσταση από εμπόδια, οι πραγματικές διαδρομές είναι μεγαλύτερες από τα
αντίστοιχα σχεδιασθέντα μονοπάτια---και αυτό σε βαθμό περίπου $4.5\%$ όσον
αφορά στον \texttt{navfn} και τον \texttt{global\_planner}, και $1.5\%$ όσον
αφορά στο \texttt{sbpl\_lattice\_planner}. Όσον αφορά στην ομαλότητα της
διαδρομής, ο συνδυασμός του με τον \texttt{navfn} έδωσε τα πιο ομαλά μονοπάτια,
με τους άλλους δύο συνδυασμούς να ακολουθούν σε κοντινή απόσταση.

\begin{table}[h]
\renewcommand{\arraystretch}{1.3}
\begin{tabular}{llcccc}
  & & \multicolumn{4}{c}{Μετρικές σχετικές με τις διανυθείσες διαδρομές} \\
  \cline{3-6}
  GP & LP & $\mu_{l}(\bm{\mathcal{P}})$ [m] & $\sigma_{l}(\bm{\mathcal{P}})$ [m] & $\mu_{s}(\bm{\mathcal{P}})$ [rad] & $\sigma_{s}(\bm{\mathcal{P}})$ [rad] \\ \toprule
  \texttt{navfn} & \texttt{dwa} & $2.35$ & $0.03$ & $0.67$ & $0.13$ \\
  \texttt{global\_planner} & \texttt{dwa} & $2.35$ & $0.03$ & $0.63$ & $0.17$ \\
  \texttt{sbpl} & \texttt{dwa} & $7.94$ & $7.62$ & $0.81$ & $0.35$ \\
  \texttt{navfn} & \texttt{eband} & $29.80$ & $1.26$ & $1.83$ & $0.02$ \\
  \texttt{global\_planner} & \texttt{eband} & $28.83$ & $0.43$ & $1.84$ & $0.01$ \\
  \texttt{sbpl} & \texttt{eband} & $26.98$ & $0.53$ & $1.78$ & $0.02$ \\
  \textbf{\texttt{navfn}} & \textbf{\texttt{teb}} & $\bm{46.53}$ & $\bm{0.08}$ & $\bm{1.57}$ & $\bm{0.04}$ \\
  \textbf{\texttt{global\_planner}} & \textbf{\texttt{teb}} & $\bm{46.55}$ & $\bm{0.04}$ & $\bm{1.61}$ & $\bm{0.02}$ \\
  \textbf{\texttt{sbpl}} & \textbf{\texttt{teb}} & $\bm{48.73}$ & $\bm{0.09}$ & $\bm{1.61}$ & $\bm{0.01}$ \\ \bottomrule
\end{tabular}
\caption{\small Μέσο μήκος διαδρομής $\mu_{l}(\bm{\mathcal{P}})$ και τυπική
         απόκλιση $\sigma_{l}(\bm{\mathcal{P}})$, και μέση τιμή ομαλότητας
         διαδρομής $\mu_{s}(\bm{\mathcal{P}})$ και τυπική απόκλιση
         $\sigma_{s}(\bm{\mathcal{P}})$ για $N=10$ προσομοιώσεις στο περιβάλλον
         WILLOWGARAGE. Συνδυασμοί που ολοκλήρωσαν όλες τις αποστολές
         σημειώνονται με έντονη γραφή}
\label{tbl:info_ground_truth_willowgarage}
\end{table}

Όσον αφορά στην απόσταση σε σχέση με τα εμπόδια του χάρτη $\bm{M}_W$ (πίνακας
\ref{tbl:info_ground_truth_map_willowgarage}), ο \texttt{teb\_local\_planner}
κατάφερε να επιτύχει την ελάχιστη τεθειμένη απόσταση ρομπότ-εμποδίων
(που έχει οριστεί όπως προηγουμένως σε $0.10$ m) όταν συνδυάστηκε με τους
προεπιλεγμένους αλγορίθμους χάραξης μονοπατιών του ROS. Όταν συνδυάστηκε με τον
\texttt{sbpl\_lattice\_planner}, ωστόσο, απέτυχε. Ο συνδυασμός του με
τον \texttt{navfn} έδωσε στο ρομπότ τη μεγαλύτερη μέση απόσταση από τα εμπόδια
σε όλες τις προσομοιώσεις. Ο συνδυασμός του με τον
\texttt{sbpl\_lattice\_planner} του έδωσε το μικρότερο μέσο όρο απόστασης και
τη μικρότερη διακύμανση γύρω από αυτήν ($5$ cm λιγότερο), μια συμπεριφορά που
συνάδει με αυτή που παρουσιάζεται στο χάρτη του κόσμου CORRIDOR.

\begin{table}[h]
\renewcommand{\arraystretch}{1.3}
\begin{tabular}{llccc}
  & & \multicolumn{3}{c}{Μετρικές που αφορούν στα εμπόδια και τις πραγματικές διαδρομές} \\
  \cline{3-5}
  GP & LP & $\inf(d(\bm{\mathcal{P}},\bm{M}_W))$ [m] & $\mu(d(\bm{\mathcal{P}},\bm{M}_W))$ [m] & $\sigma(d(\bm{\mathcal{P}},\bm{M}_W))$ [m] \\ \toprule
  \texttt{navfn} & \texttt{dwa} & $0.33$ & $0.37$ & $0.14$ \\
  \texttt{global\_planner} & \texttt{dwa} & $0.30$ & $0.38$ & $0.14$ \\
  \texttt{sbpl} & \texttt{dwa} & $0.01$ & $0.36$ & $0.23$ \\
  \texttt{navfn} & \texttt{eband} & $0.03$ & $0.61$ & $0.51$ \\
  \texttt{global\_planner} & \texttt{eband} & $0.01$ & $0.61$ & $0.52$ \\
  \texttt{sbpl} & \texttt{eband} & $0.00$ & $0.65$ & $0.53$ \\
  \textbf{\texttt{navfn}} & \textbf{\texttt{teb}} & $\bm{0.10}$ & $\bm{0.82}$ & $\bm{0.47}$ \\
  \textbf{\texttt{global\_planner}} & \textbf{\texttt{teb}} & $\bm{0.10}$ & $\bm{0.79}$ & $\bm{0.48}$ \\
  \textbf{\texttt{sbpl}} & \textbf{\texttt{teb}} & $\bm{0.09}$ & $\bm{0.67}$ & $\bm{0.42}$ \\ \bottomrule
\end{tabular}
\caption{\small Ολικά ελάχιστη απόσταση των πραγματικών διαδρομών
         $\bm{\mathcal{P}}$ που διήνυσε το ρομπότ από οποιοδήποτε εμπόδιο σε
         όλες τις προσομοιώσεις $\inf(d(\bm{\mathcal{P}},\bm{M}_W))$, μέση
         ελάχιστη απόσταση $\mu(d(\bm{\mathcal{P}},\bm{M}_W))$ από όλα τα
         εμπόδια για $N=10$ προσομοιώσεις στο χάρτη WILLOWGARAGE $\bm{M}_W$, και μέση τυπική απόκλιση
         $\sigma(d(\bm{\mathcal{P}},\bm{M}_W))$. Συνδυασμοί που
         ολοκλήρωσαν όλες τις αποστολές σημειώνονται με έντονη γραφή}
\label{tbl:info_ground_truth_map_willowgarage}
\end{table}

Όσον αφορά στην απόκλιση των διανυθέντων μονοπατιών από τα αντίστοιχα
σχεδιασθέντα μονοπάτια (πίνακας
\ref{tbl:info_deviation_from_global_plan_willowgarage}), δεν παρατηρείται
συνέπεια σε σύγκριση με τα αποτελέσματα των προσομοιώσεων στο περιβάλλον
CORRIDOR. Ενώ ο συνδυασμός του \texttt{teb\_local\_planner} με τον
\texttt{global\_planner}  στον χάρτη $\bm{M}_C$ του περιβάλλοντος CORRIDOR
παρουσίασε τη μεγαλύτερη μέση απόκλιση, τη μεγαλύτερη συνολική μέση απόκλιση,
και τη μεγαλύτερη μέση διακριτή απόσταση Frechet, στον χάρτη $\bm{M}_W$ του
περιβάλλοντος WILLOWGARAGE παρουσίασε τις μικρότερες---και ο συνδυασμός του με
τον \texttt{sbpl\_lattice\_planner} έδωσε τη μεγαλύτερη μέση διακριτή απόσταση
Frechet, όταν στον χάρτη $\bm{M}_C$ έδωσε τη μικρότερη.

\begin{table}[h]
\renewcommand{\arraystretch}{1.3}
\begin{tabular}{llccc}
  & & \multicolumn{3}{c}{Μετρικές σχετικές με την απόκλιση διαδρομών από σχεδιασθέντα μονοπάτια} \\
  \cline{3-5}
  GP & LP & $\mu_{\delta}(\bm{\mathcal{P}},\bm{\mathcal{G}})$ [m] & $\mu_{\delta}(\bm{\mathcal{P}},\bm{\mathcal{G}})$ [m] & $\mu_{\delta}^F(\bm{\mathcal{P}},\bm{\mathcal{G}})$ [m] \\ \toprule
  \texttt{navfn} & \texttt{dwa} & $0.14$ & $226.98$ & $34.12$ \\
  \texttt{global\_planner} & \texttt{dwa} & $0.12$ & $185.54$ & $34.06$ \\
  \texttt{sbpl} & \texttt{dwa} & $0.27$ & $457.42$ & $31.29$ \\
  \texttt{navfn} & \texttt{eband} & $0.12$ & $194.53$ & $11.98$ \\
  \texttt{global\_planner} & \texttt{eband} & $0.12$ & $190.02$ & $12.74$ \\
  \texttt{sbpl} & \texttt{eband} & $0.18$ & $271.59$ & $15.89$ \\
  \textbf{\texttt{navfn}} & \textbf{\texttt{teb}} & $\bm{0.11}$ & $\bm{174.18}$ & $\bm{0.43}$ \\
  \textbf{\texttt{global\_planner}} & \textbf{\texttt{teb}} & $\bm{0.10}$ & $\bm{152.02}$ & $\bm{0.43}$ \\
  \textbf{\texttt{sbpl}} & \textbf{\texttt{teb}} & $\bm{0.11}$ & $\bm{162.33}$ & $\bm{0.54}$ \\ \bottomrule
\end{tabular}
\caption{\small Μέση απόκλιση $\mu_{\delta}(\bm{\mathcal{P}},\bm{\mathcal{G}})$,
         μέση συνολική απόκλιση
         $\mu_{\Delta}(\bm{\mathcal{P}},\bm{\mathcal{G}})$, και μέση απόσταση
         Frechet $\mu_{\delta}^F(\bm{\mathcal{P}},\bm{\mathcal{G}})$ μεταξύ των
         πραγματικών διαδρομών $\bm{\mathcal{P}}$ που ακολούθησε το ρομπότ και
         των αντίστοιχων σχεδιασθέντων μονοπατιών $\bm{\mathcal{G}}$ για $N=10$
         προσομοιώσεις στο χάρτη WILLOWGARAGE $\bm{M}_W$. Συνδυασμοί που
         ολοκλήρωσαν τουλάχιστον μία αποστολή σημειώνονται με έντονη γραφή}
\label{tbl:info_deviation_from_global_plan_willowgarage}
\end{table}



%%%%%%%%%%%%%%%%%%%%%%%%%%%%%%%%%%%%%%%%%%%%%%%%%%%%%%%%%%%%%%%%%%%%%%%%%%%%%%%%
\subsection{Στοιχεία αξιολόγησης στο περιβάλλον CSAL}
\label{appendix:evaluation_csal}

\subsubsection{Σχετικά με τους αλγορίθμους κατασκευής μονοπατιών}

Οι πίνακες \ref{tbl:info_global_plan_csal} και
\ref{tbl:info_global_plan_map_csal} καταγράφουν τις τιμές των ποσοτικών
μετρικών που αφορούν στους αλγορίθμους κατασκευής μονοπατιών που ορίζονται στον
πίνακα \ref{tbl:metrics_and_proportionality_global_planners} και που προέκυψαν
κατά τις $N = 10$ προσομοιώσεις στο περιβάλλον CSAL.

Όσον αφορά στα παραγόμενα μονοπάτια (πίνακας \ref{tbl:info_global_plan_csal}),
και όσον αφορά στις μετρικές αξιολόγησης των αλγορίθμων χάραξης
μονοπατιών, ο \texttt{navfn} παρήγαγε τα μονοπάτια με το μικρότερο μήκος, και ο
\texttt{sbpl\_lattice\_planner} εκείνα με το μεγαλύτερο, με τη
μεγαλύτερη διακύμανση, με τη χαμηλότερη πυκνότητας στάσεων, αλλά και
την υψηλότερης ομαλότητας. O \texttt{global\_planner} παρήγαγε τα λιγότερο
πυκνά σχέδια.

\begin{table}[h]
\renewcommand{\arraystretch}{1.3}
\begin{tabular}{llccccc}
& & \multicolumn{5}{c}{Μετρικές επίδοσης αλγορίθμων χάραξης μονοπατιών} \\
\cline{3-7}
GP & LP & $\mu_{l}(\bm{\mathcal{G}})$ [m] & $\sigma_{l}(\bm{\mathcal{G}})$ [m] & $\mu_r(\bm{\mathcal{G}})$ [στάσεις/m] & $\mu_{s}(\bm{\mathcal{G}})$ [rad] & $\sigma_{s}(\bm{\mathcal{G}})$ [rad] \\ \toprule
\texttt{navfn} & \texttt{dwa} & $21.87$ & $0.14$ & $199.84$ & $2.33$ & $0.00$ \\
\texttt{navfn} & \texttt{eband} & $21.78$ & $0.13$ & $199.96$ & $2.33$ & $0.00$ \\
\texttt{navfn} & \texttt{teb} & $21.87$ & $0.16$ & $199.95$ & $2.33$ & $0.00$ \\
\texttt{global\_planner} & \texttt{dwa} & $21.90$ & $0.06$ & $200.06$ & $2.33$ & $0.00$ \\
\texttt{global\_planner} & \texttt{eband} & $21.89$ & $0.10$ & $200.06$ & $2.33$ & $0.00$ \\
\texttt{global\_planner} & \texttt{teb} & $21.84$ & $0.13$ & $200.07$ & $2.33$ & $0.00$ \\
\texttt{sbpl} & \texttt{dwa} & $22.07$ & $0.04$ & $131.61$ & $2.31$ & $0.01$ \\
\texttt{sbpl} & \texttt{eband} & $22.09$ & $0.10$ & $131.66$ & $2.31$ & $0.01$ \\
\texttt{sbpl} & \texttt{teb} & $22.12$ & $0.39$ & $133.03$ & $2.30$ & $0.04$ \\ \bottomrule
\end{tabular}
\caption{\small Μέσο συνολικό μήκος μονοπατιών $\mu_{l}(\bm{\mathcal{G}})$ και
         τυπική απόκλιση $\sigma_{l}(\bm{\mathcal{G}})$, μέση ανάλυση
         μονοπατιών $\mu_r(\bm{\mathcal{G}})$, μέση τιμή ομαλότητας
         $\mu_{s}(\bm{\mathcal{G}})$, και τυπική απόκλιση
         $\sigma_{s}(\bm{\mathcal{G}})$, για $N=10$ προσομοιώσεις στο
         περιβάλλον CSAL}
\label{tbl:info_global_plan_csal}
\end{table}

Ο αλγόριθμος χάραξης μονοπατιών \texttt{sbpl\_lattice\_planner} παρουσίασε την
ίδια συμπεριφορά με εκείνη στον προσομοιωμένο κόσμο WILLOGARAGE: η ολικά
ελάχιστη απόσταση των σχεδίων του από τα εμπόδια (πίνακας
\ref{tbl:info_global_plan_map_csal}) ήταν μηδέν, όπως και εκείνη του
\texttt{global\_planner}, ο οποίος και στις τρεις περιπτώσεις σχεδίασε σταθερά
μέσα από εμπόδια. Εκτός από αυτό, ο \texttt{sbpl\_lattice\_planner} παρουσίασε
και πάλι τη χαμηλότερη μέση ελάχιστη απόσταση από τα εμπόδια και τη μεγαλύτερη
συνέπεια απόστασης γύρω από αυτά. Από την άλλη πλευρά, o \texttt{navfn}
παρήγαγε τα καλύτερα σχέδια όσον αφορά στη μέση απόσταση από εμπόδια, αλλά με τη
μεγαλύτερη ασυνέπεια μεταξύ των τριών αλγορίθμων.

\begin{table}[h]
\renewcommand{\arraystretch}{1.3}
\begin{tabular}{llccc}
  & & \multicolumn{3}{c}{Μετρικές επίδοσης αλγορίθμων χάραξης μονοπατιών σχετικές με εμπόδια} \\
  \cline{3-5}
  GP & LP & $\inf(d(\bm{\mathcal{G}},\bm{M}_L))$ [m] & $\mu(d(\bm{\mathcal{G}}, \bm{M}_L))$ [m] & $\sigma(d(\bm{\mathcal{G}},\bm{M}_L))$ [m] \\ \toprule
  \texttt{navfn} & \texttt{dwa} & $0.01$ & $0.47$ & $0.42$ \\
  \texttt{navfn} & \texttt{eband} & $0.01$ & $0.47$ & $0.42$ \\
  \texttt{navfn} & \texttt{teb} & $0.01$ & $0.47$ & $0.42$ \\
  \texttt{global\_planner} & \texttt{dwa} & \hspace{1.1cm} $0.00$ (-$0.02$) & $0.45$ & $0.40$ \\
  \texttt{global\_planner} & \texttt{eband} & \hspace{1.1cm} $0.00$ (-$0.02$) & $0.45$ & $0.40$ \\
  \texttt{global\_planner} & \texttt{teb} & \hspace{1.1cm} $0.00$ (-$0.02$) & $0.45$ & $0.41$ \\
  \texttt{sbpl} & \texttt{dwa} & \hspace{1.1cm} $0.00$ (-$0.02$) & $0.41$ & $0.37$ \\
  \texttt{sbpl} & \texttt{eband} & \hspace{1.1cm} $0.00$ (-$0.02$) & $0.41$ & $0.37$ \\
  \texttt{sbpl} & \texttt{teb} & \hspace{1.1cm} $0.00$ (-$0.02$) & $0.41$ & $0.37$ \\ \bottomrule
\end{tabular}
\caption{\small Ολικά ελάχιστη απόσταση μονοπατιών $\bm{\mathcal{G}}$ από
         οποιοδήποτε εμπόδιο $\inf(d(\bm{\mathcal{G}},\bm{M}_L))$, μέση
         ελάχιστη απόσταση $\mu(d(\bm{\mathcal{G}},\bm{M}_L))$ και τυπική
         απόκλιση $\sigma(d(\bm{\mathcal{G}},\bm{M}_L))$ από όλα τα εμπόδια,
         για $N=10$ πειράματα στο περιβάλλον CSAL}
\label{tbl:info_global_plan_map_csal}
\end{table}


\subsubsection{Σχετικά με τους ελεγκτές κίνησης}

O πίνακας \ref{tbl:info_failures_csal} καταγράφει τις τιμές των ποσοτικών
μετρικών που αφορούν στους ελεγκτές κίνησης που ορίζονται στον πίνακα
\ref{tbl:metrics_and_proportionality_local_planners} και που προέκυψαν κατά τα
$N = 10$ πειράματα στο περιβάλλον CSAL.

Αυτό που είναι εντυπωσιακό εδώ είναι ότι όλοι οι συνδυασμοί των αλγορίθμων
χάραξης μονοπατιών με τον ελεγκτή κίνησης \texttt{dwa\_local\_planner} απέτυχαν
και πάλι να μεταφέρουν το ρομπότ από την αρχική στην τελική στάση σε όλα τα
πειράματα που διεξήχθησαν. Όσον αφορά στην επίδοση του
\texttt{teb\_local\_planner} στο περιβάλλον CSAL, δεν ματαίωσε ποτέ αποστολή
και δεν έκανε καμία προσπάθεια ανάκαμψης, παρόλο που παρουσίασε έναν μικρό
αριθμό αποτυχιών διαδρομής. Όσον αφορά στον \texttt{eband\_local\_planner}
αυτός παρουσίασε ελάχιστες προσπάθειες ανάκαμψης, αλλά σημαντικές αποτυχίες
διαδρομής όταν συνδυάστηκε με αλγορίθμους χάραξης μονοπατιών που δεν λαμβάνουν
υπόψη τους το κινηματικό μοντέλο του ρομπότ.

\begin{table}[h]
\renewcommand{\arraystretch}{1.3}
\begin{tabular}{llcccccccc}
& & \multicolumn{8}{c}{Μετρικές επίδοσης ελεγκτών κίνησης} \\
\cline{3-10}
  GP & LP & $\mu_{A} / N_s$ & $\mu_{RR}$ & $\sigma_{RR}$ & $\mu_{CC}$ & $\sigma_{CC}$ & $\mu_{PF}$ & $\sigma_{PF}$ & $\mu_{PF} / \mu_{LPC}$ \\ \toprule
  \texttt{navfn} & \texttt{dwa} & $1.00$ & $2.20$ & $0.42$ & $3.00$ & $0.00$ & $37.40$ & $17.85$ & $0.08$ \\
  \texttt{global\_planner} & \texttt{dwa} & $1.00$ & $2.60$ & $0.70$ & $3.20$ & $0.63$ & $30.20$ & $23.66$ & $0.06$ \\
  \texttt{sbpl} & \texttt{dwa} & $1.00$ & $2.40$ & $0.70$ & $3.30$ & $0.95$ & $4.10$ & $3.14$ & $0.01$ \\
  \texttt{navfn} & \texttt{eband} & $0.00$ & $0.60$ & $0.97$ & $0.90$ & $1.45$ & $57.00$ & $26.72$ & N/A \\
  \texttt{global\_planner} & \texttt{eband} & $0.00$ & $1.00$ & $0.67$ & $0.40$ & $0.97$ & $65.00$ & $29.84$ & N/A \\
  \texttt{sbpl} & \texttt{eband} & $0.00$ & $1.40$ & $0.52$ & $1.10$ & $1.37$ & $5.80$ & $4.13$ & N/A \\
  \texttt{navfn} & \texttt{teb} & $0.00$ & $0.00$ & $0.00$ & $0.00$ & $0.00$ & $1.10$ & $0.88$ & $0.00$ \\
  \texttt{global\_planner} & \texttt{teb} & $0.00$ & $0.00$ & $0.00$ & $0.00$ & $0.00$ & $1.40$ & $1.17$ & $0.00$ \\
  \texttt{sbpl} & \texttt{teb} & $0.00$ & $0.00$ & $0.00$ & $0.00$ & $0.00$ & $2.70$ & $3.27$ & $0.00$ \\ \bottomrule
\end{tabular}
\caption{\small Μέσος αριθμός ματαιωμένων αποστολών επί του αριθμού των
      προσομοιώσεων που πραγματοποιήθηκαν $\mu_A / N$, μέσος αριθμός ανακτήσεων
      με περιστροφή $\mu_{RR}$ και η τυπική τους απόκλιση $\sigma_{RR}$, μέσος
      αριθμός εκκαθαρίσεων χαρτών κόστους $\mu_{CC}$ και η τυπική τους απόκλιση
      $\sigma_{CC}$, μέσος αριθμός αποτυχιών διαδρομής $\mu_{PF}$ και η τυπική
      τους απόκλιση $\sigma_{PF}$, και μέσος αριθμός αποτυχιών διαδρομής επί
      του μέσου αριθμού των κλήσεων του ελεγκτή κίνησης $\mu_{PF} / \mu_{LPC}$,
      για όλους τους συνδυασμούς αλγορίθμων χάραξης μονοπατιών και ελεγκτών
      κίνησης που παρουσιάζονται στον πίνακα \ref{tbl:planners_sifted_list},
      για $N=10$ πειράματα στο περιβάλλον CSAL}
\label{tbl:info_failures_csal}
\end{table}



\subsubsection{Σχετικά με το συνδυασμό τους}

Οι πίνακες \ref{tbl:info_pose_csal}, \ref{tbl:info_ground_truth_csal},
\ref{tbl:info_ground_truth_map_csal}, και
\ref{tbl:info_deviation_from_global_plan_csal} καταγράφουν τις τιμές των
ποσοτικών μετρικών που αφορούν στους συνδυασμούς αλγορίθμων χάραξης μονοπατιών
και ελεγκτών κίνησης που ορίζονται στον πίνακα
\ref{tbl:metrics_and_proportionality_combination_planners}, και που προέκυψαν
κατά τα $N = 10$ πειράματα στο περιβάλλον CSAL.

Όσον αφορά στο χρόνο που απαιτείται για την επίτευξη της πλοήγησης από τη στάση
$\bm{p}_0^L$ προς τη στάση-στόχο $\bm{p}_G^L$ (πίνακας
\ref{tbl:info_pose_csal}), όλοι οι συνδυασμοί που περιλαμβάνουν τον
\texttt{dwa\_local\_planner} αποκλείονται από αξιολόγηση λόγω της αδυναμίας
του να πλοηγήσει το ρομπότ με τρόπο τέτοιο ώστε να φτάσει στην επιθυμητή στάση.
Ο ελεγκτής κίνησης \texttt{teb\_local\_planner} διέσχισε τα σχεδιασθέντα
μονοπάτια σε λιγότερο μέσο χρόνο σε σύγκριση με τον
\texttt{eband\_local\_planner} για τον ίδιο αλγόριθμο χάραξης μονοπατιών.

\begin{table}[h]
\renewcommand{\arraystretch}{1.3}
\begin{tabular}{llcc}
& & \multicolumn{2}{c}{Μετρικές που αφορούν στον χρόνο πραγματικής διαδρομής} \\
\cline{3-4}
GP & LP & $\mu_{t}$ [sec] & $\sigma_{t}$ [sec] \\  \toprule
\texttt{navfn} & \texttt{dwa} & $47.47$ & $15.29$ \\
\texttt{global\_planner} & \texttt{dwa} & $56.24$ & $15.44$ \\
\texttt{sbpl} & \texttt{dwa} & $60.30$ & $22.58$ \\
  \textbf{\texttt{navfn}} & \textbf{\texttt{eband}} & $\bm{356.25}$ & $\bm{9.88}$ \\
  \textbf{\texttt{global\_planner}} & \textbf{\texttt{eband}} & $\bm{354.89}$ & $\bm{10.05}$ \\
  \textbf{\texttt{sbpl}} & \textbf{\texttt{eband}} & $\bm{392.16}$ & $\bm{21.29}$ \\
  \textbf{\texttt{navfn}} & \textbf{\texttt{teb}} & $\bm{326.70}$ & $\bm{12.88}$ \\
  \textbf{\texttt{global\_planner}} & \textbf{\texttt{teb}} & $\bm{330.25}$ & $\bm{13.77}$ \\
  \textbf{\texttt{sbpl}} & \textbf{\texttt{teb}} & $\bm{363.16}$ & $\bm{42.35}$ \\ \bottomrule
\end{tabular}
\caption{\small Μέσος χρόνος διαδρομής $\mu_{t}$ από την αρχική στην τελική
         στάση και τυπική απόκλιση $\sigma_{t}$ για $N=10$ πειράματα στο
         περιβάλλον CSAL. Οι συνδυασμοί που ολοκλήρωσαν όλες τις αποστολές
         και οι τιμές των αντίστοιχων μετρικών σημειώνονται με έντονη γραφή}
\label{tbl:info_pose_csal}
\end{table}

Όσον αφορά στο μέσο μήκος των διανυόμενων διαδρομών
(πίνακας \ref{tbl:info_ground_truth_csal}), ο \texttt{teb\_local\_planner} δεν
παραμόρφωσε τα σχεδιασθέντα μονοπάτια στο βαθμό που το έκανε στις
προσομοιώσεις, και αυτό είναι παρατηρήσιμο καθώς παρήγαγε διαδρομές με το
μικρότερο μέσο μήκος. Από την άλλη πλευρά, οι διαδρομές που παρήγαγε ο
\texttt{eband\_local\_planner} ήταν οι μακρύτερες, αλλά οι πιο συνεπείς σε μήκος.
Ο τελευταίος παρήγαγε μονοπάτια με χαμηλότερη ομαλότητα σε σύγκριση με τον
\texttt{teb\_local\_planner}, και μονοπάτια με τη μεγαλύτερη συνέπεια όσον αφορά
την ομαλότητα.

\begin{table}[h]
\renewcommand{\arraystretch}{1.3}
\begin{tabular}{llcccc}
& & \multicolumn{4}{c}{Μετρικές σχετικές με τις διανυθείσες διαδρομές} \\
\cline{3-6}
  GP & LP & $\mu_{l}(\bm{\mathcal{P}})$ [m] & $\sigma_{l}(\bm{\mathcal{P}})$ [m] & $\mu_{s}(\bm{\mathcal{P}})$ [rad] & $\sigma_{s}(\bm{\mathcal{P}})$ [rad] \\ \toprule
  \texttt{navfn} & \texttt{dwa} & $2.97$ & $1.00$ & $0.58$ & $0.42$ \\
  \texttt{global\_planner} & \texttt{dwa} & $2.65$ & $1.50$ & $1.16$ & $0.51$ \\
  \texttt{sbpl} & \texttt{dwa} & $2.99$ & $1.36$ & $0.79$ & $0.54$ \\
  \textbf{\texttt{navfn}} & \textbf{\texttt{eband}} & $\bm{22.81}$ & $\bm{0.12}$ & $\bm{2.32}$ & $\bm{0.01}$ \\
  \textbf{\texttt{global\_planner}} & \textbf{\texttt{eband}} & $\bm{22.79}$ & $\bm{0.13}$ & $\bm{2.33}$ & $\bm{0.01}$ \\
  \textbf{\texttt{sbpl}} & \textbf{\texttt{eband}} & $\bm{22.78}$ & $\bm{0.13}$ & $\bm{2.32}$ & $\bm{0.01}$ \\
  \textbf{\texttt{navfn}} & \textbf{\texttt{teb}} & $\bm{22.71}$ & $\bm{0.18}$ & $\bm{2.35}$ & $\bm{0.02}$ \\
  \textbf{\texttt{global\_planner}} & \textbf{\texttt{teb}} & $\bm{22.73}$ & $\bm{0.28}$ & $\bm{2.34}$ & $\bm{0.02}$ \\
  \textbf{\texttt{sbpl}} & \textbf{\texttt{teb}} & $\bm{23.47}$ & $\bm{0.87}$ & $\bm{2.30}$ & $\bm{0.04}$ \\ \bottomrule
\end{tabular}
\caption{\small Μέσο μήκος διαδρομής $\mu_{l}(\bm{\mathcal{P}})$ και τυπική
         απόκλιση $\sigma_{l}(\bm{\mathcal{P}})$, και μέση τιμή ομαλότητας
         διαδρομής $\mu_{s}(\bm{\mathcal{P}})$ και τυπική απόκλιση
         $\sigma_{s}(\bm{\mathcal{P}})$ για $N=10$ πειράματα στο περιβάλλον
         CSAL. Συνδυασμοί που ολοκλήρωσαν όλες τις αποστολές  σημειώνονται
         με έντονη γραφή}
\label{tbl:info_ground_truth_csal}
\end{table}

Όσον αφορά στην απόσταση από τα εμπόδια στο χάρτη $\bm{M}_L$ (πίνακας
\ref{tbl:info_ground_truth_map_csal}), ο \texttt{teb\_local\_planner} δεν
κατάφερε να επιτύχει την ελάχιστη τεθειμένη απόσταση ρομπότ-εμποδίων (που είχε
οριστεί όπως προηγουμένως σε $0.10$ m), ωστόσο, υπό τον έλεγχό του, η μέση
ελάχιστη απόσταση του ρομπότ από τα εμπόδια και η τυπική απόκλισή της ήταν
μικρότερες από εκείνες του \texttt{eband\_local\_planner}. Ο τελευταίος ήταν ο
μόνος ελεγκτής που κατάφερε να παρουσιάσει απόσταση ρομπότ-εμποδίων μεγαλύτερη
από το κατώφλι που τέθηκε, και μάλιστα με συνέπεια.

\begin{table}[h]
\renewcommand{\arraystretch}{1.3}
\begin{tabular}{llccc}
  & & \multicolumn{3}{c}{Μετρικές που αφορούν στα εμπόδια και τις πραγματικές διαδρομές} \\
  \cline{3-5}
  GP & LP & $\inf(d(\bm{\mathcal{P}},\bm{M}_L))$ [m] & $\mu(d(\bm{\mathcal{P}},\bm{M}_L))$ [m] & $\sigma(d(\bm{\mathcal{P}},\bm{M}_L))$ [m] \\ \toprule
  \texttt{navfn} & \texttt{dwa} & $0.02$ & $0.24$ & $0.09$ \\
  \texttt{global\_planner} & \texttt{dwa} & $0.02$ & $0.25$ & $0.07$ \\
  \texttt{sbpl} & \texttt{dwa} & $0.03$ & $0.26$ & $0.04$ \\
  \textbf{\texttt{navfn}} & \textbf{\texttt{eband}} & $\bm{0.11}$ & $\bm{0.52}$ & $\bm{0.20}$ \\
  \textbf{\texttt{global\_planner}} & \textbf{\texttt{eband}} & $\bm{0.11}$ & $\bm{0.54}$ & $\bm{0.20}$ \\
  \textbf{\texttt{sbpl}} & \textbf{\texttt{eband}} & $\bm{0.13}$ & $\bm{0.57}$ & $\bm{0.19}$ \\
  \textbf{\texttt{navfn}} & \textbf{\texttt{teb}} & $\bm{0.08}$ & $\bm{0.51}$ & $\bm{0.18}$ \\
  \textbf{\texttt{global\_planner}} & \textbf{\texttt{teb}} & $\bm{0.08}$ & $\bm{0.52}$ & $\bm{0.19}$ \\
  \textbf{\texttt{sbpl}} & \textbf{\texttt{teb}} & $\bm{0.08}$ & $\bm{0.56}$ & $\bm{0.17}$ \\ \bottomrule
\end{tabular}
\caption{\small Ολικά ελάχιστη απόσταση των πραγματικών διαδρομών
         $\bm{\mathcal{P}}$ που διήνυσε το ρομπότ από οποιοδήποτε εμπόδιο σε
         όλες τις προσομοιώσεις $\inf(d(\bm{\mathcal{P}},\bm{M}_L))$, μέση
         ελάχιστη απόσταση $\mu(d(\bm{\mathcal{P}},\bm{M}_L))$ από όλα τα
         εμπόδια για $N=10$ πειράματα στο χάρτη CSAL $\bm{M}_L$, και μέση τυπική απόκλιση
         $\sigma(d(\bm{\mathcal{P}},\bm{M}_L))$. Συνδυασμοί που
         ολοκλήρωσαν όλες τις αποστολές σημειώνονται με έντονη γραφή}
\label{tbl:info_ground_truth_map_csal}
\end{table}

Όσον αφορά στην απόκλιση των διανυθέντων διαδρομών από τα αντίστοιχα
σχεδιασθέντα μονοπάτια (πίνακας
\ref{tbl:info_deviation_from_global_plan_csal}), o \texttt{teb\_local\_planner}
παρήγαγε μονοπάτια με τη χαμηλότερη μέση και συνολική απόκλιση, η οποία είναι
σύμφωνη με τη μέση ελάχιστη απόσταση ρομπότ-εμποδίων που παρουσιάστηκε.
Αντίθετα, ο \texttt{eband\_local\_planner} παρήγαγε διαδρομές με τη μεγαλύτερη
μέση και συνολική απόκλιση από σχεδιασθέντα μονοπάτια, και, συνολικά, διαδρομές
με τη μεγαλύτερη απόσταση Frechet.

\begin{table}[h]
\renewcommand{\arraystretch}{1.3}
\begin{tabular}{llccc}
& & \multicolumn{3}{c}{Μετρικές σχετικές με την απόκλιση διαδρομών από σχεδιασθέντα μονοπάτια} \\
\cline{3-5}
  GP & LP & $\mu_{\delta}(\bm{\mathcal{P}},\bm{\mathcal{G}})$ [m] & $\mu_{\delta}(\bm{\mathcal{P}},\bm{\mathcal{G}})$ [m] & $\mu_{\delta}^F(\bm{\mathcal{P}},\bm{\mathcal{G}})$ [m] \\ \toprule
  \texttt{navfn} & \texttt{dwa} & $0.03$ & $1.74$ & $12.69$ \\
  \texttt{global\_planner} & \texttt{dwa} & $0.05$ & $3.41$ & $12.43$ \\
  \texttt{sbpl} & \texttt{dwa} & $0.04$ & $2.32$ & $12.58$ \\
  \textbf{\texttt{navfn}} & \textbf{\texttt{eband}} & $\bm{0.12}$ & $\bm{48.34}$ & $\bm{0.35}$ \\
  \textbf{\texttt{global\_planner}} & \textbf{\texttt{eband}} & $\bm{0.13}$ & $\bm{51.63}$ & $\bm{0.35}$ \\
  \textbf{\texttt{sbpl}} & \textbf{\texttt{eband}} & $\bm{0.15}$ & $\bm{62.71}$ & $\bm{0.43}$ \\
  \textbf{\texttt{navfn}} & \textbf{\texttt{teb}} & $\bm{0.10}$ & $\bm{40.19}$ & $\bm{0.31}$ \\
  \textbf{\texttt{global\_planner}} & \textbf{\texttt{teb}} & $\bm{0.11}$ & $\bm{42.84}$ & $\bm{0.33}$ \\
  \textbf{\texttt{sbpl}} & \textbf{\texttt{teb}} & $\bm{0.12}$ & $\bm{50.56}$ & $\bm{0.35}$ \\ \bottomrule
\end{tabular}
\caption{\small Μέση απόκλιση $\mu_{\delta}(\bm{\mathcal{P}},\bm{\mathcal{G}})$,
         μέση συνολική απόκλιση
         $\mu_{\Delta}(\bm{\mathcal{P}},\bm{\mathcal{G}})$, και μέση απόσταση
         Frechet $\mu_{\delta}^F(\bm{\mathcal{P}},\bm{\mathcal{G}})$ μεταξύ των
         πραγματικών διαδρομών $\bm{\mathcal{P}}$ που ακολούθησε το ρομπότ και
         των αντίστοιχων σχεδιασθέντων μονοπατιών $\bm{\mathcal{G}}$ για $N=10$
         πειράματα στο χάρτη CSAL $\bm{M}_L$. Συνδυασμοί που
         ολοκλήρωσαν τουλάχιστον μία αποστολή σημειώνονται με έντονη γραφή}
\label{tbl:info_deviation_from_global_plan_csal}
\end{table}
