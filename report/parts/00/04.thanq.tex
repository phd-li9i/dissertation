\newgeometry{top=70mm, bottom=30mm, inner=30mm, outer=30mm, footskip = 15mm}
%\afterpage{\blankpage}
%\cleardoublepage

Η εκπόνηση αυτής της διατριβής δεν θα ήταν δυνατή χωρίς την παρότρυνση και τη
βοήθεια του Δρ. Αντώνη Δημητρίου, στον οποίο είμαι βαθύτατα ευγνώμων καθώς με
ευεργέτησε. Κατά τον ίδιο τρόπο ευχαριστώ τον καθηγητή Γεώργιο Σεργιάδη καθώς
δέχθηκε να με υποστηρίξει, και διότι με βοήθησε κατά τη συγγραφή των
δημοσιεύσεών μας. Ο καθηγητής Ανδρέας Συμεωνίδης και ο Δρ. Μάνος Τσαρδούλιας
είχαν την υπομονή και κατέβαλαν τον κόπο να διαβάσουν με επιμέλεια και να
βελτιώσουν τα κείμενα των προ-δημοσιεύσεών μας. Μέσα από αυτούς και την
εμπειρία τους κατάλαβα πώς (δεν) δομούνται τα επιστημονικά κείμενα και πώς
ξεχωρίζεται η ήρα από το στάρι. Για αυτά και για τη συνεργασία μας τους
ευχαριστώ---όπως και ευχαριστώ τους ανώνυμους διαιτητές των δημοσιεύσεών μας,
καθώς από όλους τους ανθρώπους που ανέφερα ακονίστηκα. Είμαι επίσης ευγνώμων
για την εγκάρδια συνεργασία των συνυποψηφίων διδακτόρων Αριστείδη
Ραπτόπουλο-Χατζηστεφάνου, Σπύρου Μεγάλου, Τάσου Τζιτζή και εμού.

Θέλω επίσης να εκφράσω δημοσίως την ευγνωμοσύνη μου στον Σ.Σ., τον αφανή ήρωα
αυτής της διατριβής---αν δεν με απέλυε από την επιχείρησή του τότε ποτέ δεν θα
είχα γίνει μέλος της ομάδας PANDORA, ποτέ δεν θα είχα γνωρίσει τη Ρομποτική,
ποτέ δεν θα είχα επιδιώξει να γίνω μαθητής της, συνεπώς ποτέ δεν θα είχα κάνει
τη θητεία μου στο εξωτερικό, ποτέ δεν θα είχα αγαπήσει τον Έλεγχο κατά
λάθος, και ποτέ δεν θα είχα φθάσει σε αυτό το σημείο εδώ.  \\ \\

Αφιερώνω όλα αυτά τα χρόνια στους ανθρώπους που έμμεσα και άμεσα με διέπλασαν:
στους γονείς του Ανδρέα Συμεωνίδη, Ηλεκτρολόγου Μηχανικού και Μηχανικού
Υπολογιστών Α.Π.Θ. (6191), Μαρία και Αλέξανδρο, και στους γονείς μου, Μαρίνα και
Χρήστο. Ευγνωμωνώ την Ευγενία Σακελλαρίου, η οποία ανέχθηκε την απουσία μου τα
χρόνια της ερευνωρυχείας.

\afterpage{\blankpage}
\restoregeometry
\cleardoublepage
