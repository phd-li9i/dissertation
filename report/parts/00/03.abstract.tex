\newgeometry{top=52mm, bottom=52mm, inner=35mm, outer=35mm}
\singlespace

\vfill
\begin{quotation}
\begin{center}
\textbf{Περίληψη}
\end{center}
\noindent
\end{quotation}

Την τελευταία δεκαετία η τροχιά της ρομποτικής κινητής βάσης βαίνει ανοδικώς ως
προς το εύρος των τομέων χρήσης της και του αριθμού των εφαρμογών της. Αυτόνομα
επίγεια και ιπτάμενα ρομπότ χρησιμοποιούνται σε αποθήκες για την απογραφή και
εύρεση της θέσης προϊόντων, σε νοσοκομεία για την απολύμανση χώρων, στη γεωργία
για παρακολούθηση και στοχευμένη προσαρμογή στόχων όσο αφορά σε καλλιέργειες,
και για τη μεταφορά εμπορευμάτων ή ανθρώπων. Σε όλες τις περιπτώσεις η γνώση
της στάσης ενός οχήματος στο χώρο, δηλαδή της θέσης και του προσανατολισμού
του, είναι απαραίτητη για την αυτόνομη κίνησή του και την ικανοποίηση των
στόχων της αποστολής του. Σε εσωτερικούς χώρους ή συμφραζόμενα όπου εξωτερικές
υποδομές αδυνατούν να προσδώσουν τη στάση του η μόνη εναλλακτική είναι η
εκτίμηση της, και η διαδικασία εκτίμησης αναλαμβάνεται από το ίδιο το όχημα, με
βάση τους αισθητήρες που αυτό φέρει.

Το αντικείμενο της παρούσας διατριβής είναι η ρομποτική αυτόνομης κινητής βάσης
σε συμφραζόμενα απουσίας μέτρησης της στάσης της. Το κυρίως περιεχόμενο της
αφορά στην εκτίμηση της στάσης μίας κινητής βάσης βάσει μετρήσεων που
προέρχονται από έναν αισθητήρα αποστάσεων τύπου lidar και στην ελάττωση του
σφάλματος εκτίμησης, η οποία εκτίμηση προέρχεται από παραδοσιακές μεθόδους της
βιβλιογραφίας. Πιο συγκεκριμένα το περιεχόμενο της διατριβής εκτυλίσσεται σε
δύο επίπεδα: σε επίπεδο εφαρμογών και σε επίπεδο εργαλείων.  Κατά βάθος
ερευνούμε τρόπους με τους οποίους η ευθυγράμμιση μετρήσεων από τον φυσικό
αισθητήρα με εικονικές σαρώσεις---υπολογιζόμενες με βάση είτε τον χάρτη του
περιβάλλοντος είτε μία δεύτερη μέτρηση---μπορεί να χρησιμοποιηθεί για την
ικανοποίηση των δύο παραπάνω στόχων, και με επίδοση καλύτερη από μεθόδους της
καθιερωμένης και τρέχουσας βιβλιογραφίας.

Αρχικά σχεδιάζουμε μία μεθοδολογία αξιολόγησης αλγορίθμων αυτόνομης πλοήγησης,
και μέσω της εφαρμογής της παρατηρούμε την ύπαρξη σφαλμάτων εκτίμησης στάσης.
Στη συνέχεια ερευνούμε τρόπους μείωσης αυτών των σφαλμάτων εντός και εκτός του
μηχανισμού ενός φίλτρου σωματιδίων. Ένας από αυτούς τους εξωτερικούς τρόπους
είναι η προσθετική ευθυγράμμιση μετρήσεων με εικονικές σαρώσεις.
Ανακαλύπτοντας πως η ευθυγράμμιση σαρώσεων εκτελείται στη βιβλιογραφία με τρόπο
μη εύρωστο ως προς το θόρυβο μέτρησης, το μέτρο των σφαλμάτων εκτίμησης θέσης,
και ρυθμιστικές παραμέτρους, αντιπροτείνουμε μεθόδους ευθυγράμμισης πραγματικών
με εικονικές σαρώσεις που αντικαθιστούν τους σχετικούς παθογόνους μηχανισμούς
της βιβλιογραφίας με μηχανισμούς που βασίζονται στο μετασχηματισμό Fourier και
τις ιδιότητές του. Σε επίπεδο εφαρμογών σχεδιάζουμε συστήματα που δεδομένου
του χάρτη του περιβάλλοντος της κινητής βάσης στοχεύουν στην εκ του μηδενός
εκτίμηση της στάσης του, και στην ελάττωση του σφάλματος παρατήρησής της. Στο
τέλος γενικεύουμε την προσέγγισή μας απουσία χάρτη, προκειμένου για την
παραγωγή οδομετρίας με βάση μετρήσεις αισθητήρα αποστάσεων τύπου lidar.


\newpage

\begin{quotation}
\begin{center}
\textbf{Abstract}
\end{center}
\noindent
\end{quotation}

\restoregeometry
\doublespace
