\newgeometry{top=52mm, bottom=52mm, inner=35mm, outer=35mm}
\singlespace

\vfill
\begin{center}
\textbf{Περίληψη}
\end{center}
\noindent

Η ρομποτική κινητής βάσης είναι ένας ολοένα ανερχόμενος κλάδος της ρομποτικής,
τόσο ως προς το εύρος των τομέων χρήσης της όσο και ως προς το περιεχόμενο των
εφαρμογών της. Αυτόνομα επίγεια και ιπτάμενα ρομπότ χρησιμοποιούνται σε
αποθήκες για την απογραφή προϊόντων και την εύρεση της θέσης τους, σε
νοσοκομεία για την απολύμανση χώρων, στη γεωργία για παρακολούθηση καλλιεργιών
και αυτοματοποίηση διαδικασιών, και για τη μεταφορά εμπορευμάτων ή ανθρώπων. Σε
όλες τις περιπτώσεις είναι απαραίτητη η γνώση της στάσης ενός οχήματος στο χώρο
από το ίδιο το όχημα, δηλαδή της θέσης και του προσανατολισμού του, για την
αυτόνομη κίνησή του στο χώρο και για την ικανοποίηση των στόχων της αποστολής
του. Σε εσωτερικούς χώρους, όπου είναι αδύνατη η πρόσβαση σε δορυφορικώς
μεταδιδόμενες πληροφορίες στάσης, ή σε συμφραζόμενα όπου υποδομές μέτρησης
εσωτερικού χώρου απουσιάζουν, η μόνη εναλλακτική είναι η εκτίμηση του
διανύσματος στάσεως, και η διαδικασία εκτίμησης αναλαμβάνεται από το ίδιο το
όχημα, με βάση τους αισθητήρες που αυτό φέρει.

Το αντικείμενο της παρούσας διατριβής είναι η ρομποτική αυτόνομης επίγειας
κινητής βάσης σε συμφραζόμενα απουσίας μέτρησης της στάσης της. Το κυρίως
περιεχόμενο της αφορά στην εκτίμηση της στάσης μίας κινητής βάσης βάσει
δισδιάστατων μετρήσεων που προέρχονται από έναν αισθητήρα αποστάσεων τύπου
lidar, και στην ελάττωση του σφάλματος εκτίμησης, η οποία εκτίμηση προέρχεται
από παραδοσιακές μεθόδους της βιβλιογραφίας. Πιο συγκεκριμένα ερευνούμε τρόπους
με τους οποίους η ευθυγράμμιση μετρήσεων από τον φυσικό αισθητήρα με εικονικές
σαρώσεις---υπολογιζόμενες με βάση είτε τον χάρτη του περιβάλλοντος είτε μία
δεύτερη μέτρηση---μπορεί να χρησιμοποιηθεί για την ικανοποίηση των δύο παραπάνω
στόχων, και με επίδοση καλύτερη από μεθόδους της καθιερωμένης και τρέχουσας
βιβλιογραφίας.

Αρχικά σχεδιάζουμε μία μεθοδολογία αξιολόγησης αλγορίθμων αυτόνομης πλοήγησης,
και μέσω της εφαρμογής της παρατηρούμε την ύπαρξη σφαλμάτων εκτίμησης στάσης.
Στη συνέχεια ερευνούμε τρόπους μείωσης αυτών των σφαλμάτων εντός και εκτός του
μηχανισμού του φίλτρου σωματιδίων. Ένας από αυτούς τους εξωτερικούς τρόπους
είναι η προσθετική ευθυγράμμιση μετρήσεων με εικονικές σαρώσεις.
Ανακαλύπτοντας πως η ευθυγράμμιση σαρώσεων εκτελείται στη βιβλιογραφία με τρόπο
μη εύρωστο ως προς το θόρυβο μέτρησης, το μέτρο των σφαλμάτων εκτίμησης θέσης,
και ρυθμιστικές παραμέτρους, αντιπροτείνουμε μεθόδους ευθυγράμμισης πραγματικών
με εικονικές σαρώσεις που αντικαθιστούν τους σχετικούς παθογόνους μηχανισμούς
της βιβλιογραφίας με μηχανισμούς που βασίζονται στο μετασχηματισμό Fourier και
τις ιδιότητές του. Σε επίπεδο εφαρμογών σχεδιάζουμε συστήματα που δεδομένου
του χάρτη του περιβάλλοντος της κινητής βάσης στοχεύουν στην εκ του μηδενός
εκτίμηση της στάσης του, και στην ελάττωση του σφάλματος παρατήρησής της. Στο
τέλος γενικεύουμε την προσέγγισή μας απουσία χάρτη, προκειμένου για την
παραγωγή οδομετρίας με βάση μετρήσεις πανοραμικού αισθητήρα αποστάσεων τύπου
lidar.


\newpage

\begin{quotation}
\begin{center}
\textbf{Abstract}
\end{center}
\noindent
\end{quotation}
\selectlanguage{english}

During the last decade mobile robotics has been on the rise in terms of its
range of use and content of applications. Autonomous ground and aerial robots
are used in warehouses for inventorying and localising products, in hospitals
for decontamination purposes, in agriculture for monitoring crops and process
automation, and for transporting goods or people. In all cases the
self-knowledge of a vehicle's pose in space, i.e. its position and orientation,
is essential for its autonomous motion and the satisfaction of objectives of
its mission. In indoor areas, where pose information transmitted via satellite
is denied, or contexts where indoor measuring infrastructure is absent, the
only alternative is the estimation of the pose vector, and the estimation
process is undertaken by the vehicle itself, based on the sensors it is equiped
with.

The subject of this dissertation is autonomous ground mobile robotics in the
context of the absence of pose measurements. Its main content is conserned with
the estimation of the pose of a mobile base on the basis of two dimensional
measurements derived from a lidar range sensor and the reduction of the
estimation error, with the pose estimate being derived from traditional methods
of the literature. Specifically we explore ways in which matching measurements
from the physical sensor against virtual scans---computed either from the map
of the environment or from a second measurement---can be used to satisfy the
two above-stated objectives, with performance superior to that of established
and state-of-the-art methods of the literature.

We first design a methodology for evaluating autonomous navigation algorithms,
and through its implementation we observe the existence of pose estimate
errors. We then investigate ways of reducing these errors with methods intenal
and external to the mechanisms of a particle filter. One of these external ways
is prosthetic scan--to--map-scan matching. Discovering that scan-matching is
performed in the literature in a manner that is not robust to measurement
noise, the magnitude of position estimate errors, and tuning parameters, we
propose scan--to--map-scan-matching methods that replace the relevant
pathogenic state-of-the-art mechanisms with mechanisms based on the Fourier
transform and its properties. At the application level we design systems that
given the map of the mobile base environment aim at solving the global
localisation problem, and reducing the pose error during pose tracking.
Finally, we generalize our approach in the absence of a map in order to
generate odometry based on a panoramic lidar's range measurements.


\selectlanguage{greek}
\restoregeometry
\doublespace
