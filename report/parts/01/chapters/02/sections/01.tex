Αυτό το κεφάλαιο έχει ως σκοπό την παροχή μίας συνοπτικής κάτοψης των
προβλημάτων στων οποίων τη λύση συμβάλλει η παρούσα διατριβή, των συμβολών της
ως προς τις λύσεις τους, και της δομής του παρόντος εγγράφου. Το σχήμα
\ref{fig:roadmap} λειτουργεί τροχιοδεικτικά ως προς τα προβλήματα-σταθμούς, των
συνδετικών βημάτων ανάμεσά τους, και τις ιδιότητές των λύσεών που παρουσιάζονται
στην παρούσα διατριβή.\\

%1%%%%%%%%%%%%%%%%%%%%%%%%%%%%%%%%%%%%%%%%%%%%%%%%%%%%%%%%%%%%%%%%%%%%%%%%%%%%%%
Η απαρχή του πρωτότυπου ερευνητικού περιεχομένου της διατριβής βρίσκεται στην
ανάγκη διαλεύκανσης ενός προβλήματος του οποίου η λύση είναι κρίσιμη σε
πρακτικές εφαρμογές ρομποτικής κινητής βάσης: της αποτελεσματικότητας και της
επίδοσης των διαφορετικών υλοποιήσεων που αφορούν στην αυτόνομη πλοήγηση με το
λειτουργικό σύστημα \texttt{ROS} (ενότητα \ref{subsec:01_01_02_9}). Καθώς η
δημοφιλία τού τελευταίου έχει εξαπλωθεί στην έρευνα, και ως το de facto μέσο
υλοποίησης εφαρμογών ρομποτικής, έχει ενσωματωθεί σε αυτό ένας ικανός αριθμός
αλγορίθμων αυτόνομους πλοήγησης (χάραξης μονοπατιών σε δισδιάστατο χάρτη, και
ελεγκτών κίνησης: ενότητα \ref{subsec:01_01_01_1}), των οποίων η συνδυαστική
χρήση αποτελεί αντικείμενο χρονοβόρας έρευνας και πειραματισμού για ερευνητές
και επαγγελματίες του πεδίου εφαρμογής \ref{scope}. Σκοπός της μελέτης η οποία
παρουσιάζεται στο κεφάλαιο \ref{part:02:chapter:01} είναι η παροχή μεθόδου
αξιολόγησης της επίδοσης αλγορίθμων αυτόνομους πλοήγησης, καθώς και η
πειραματική αξιοποίησή της σε ό,τι αφορά τρέχοντες διαθέσιμους αλγορίθμους.

\begin{bw_box}
\begin{customcontribution}{Σ1}
  \label{contribution:01}
  Ως προς τη λύση του προβλήματος της αξιολόγησης της αποτελεσματικότητας και
  της επίδοσης μεθόδων χάραξης μονοπατιών, ελεγκτών κίνησης, και των συνδυασμών
  τους, οι συμβολές της παρούσας διατριβής συνίστανται στον σχεδιασμό μίας
  περιεκτικής και επεκτάσιμης μεθοδολογίας αξιολόγησής τους, και στην εφαρμογή
  της για την αξιολόγηση και ιεράρχηση της επίδοσης τρεχουσών υλοποιήσεών τους
  στο- και μέσω του μεσολογισμικού \texttt{ROS}. \cite{Filotheou2020b}
\end{customcontribution}
\end{bw_box}

%-------------------------------------------------------------------------------
Κατά τη διενέργεια της πειραματικής αξιολόγησης των μεθόδων πλοήγησης
παρατηρούμε το φαινόμενο της αστάθειας της εκτίμησης της στάσης ενός ρομπότ από
τη μέθοδο παρατήρησης αυτής, ήτοι από το φίλτρο σωματιδίων, και κατά συνέπεια
το γενικευμένο φαινόμενο του σφάλματός της ως προς την πραγματική του στάση,
ανεξαρτήτως των υποκειμένων μεθόδων πλοήγησης (σχήμα \ref{fig:roadmap}-Α). Η
μικρή αυτή παρατήρηση αποδεικνύεται ότι είναι καίριας σημασίας, καθώς μάς
εισάγει στον δρόμο της έρευνας επί της ελάττωσης του σφάλματος εκτίμησης της
στάσης ενός ρομπότ.\\

%2%%%%%%%%%%%%%%%%%%%%%%%%%%%%%%%%%%%%%%%%%%%%%%%%%%%%%%%%%%%%%%%%%%%%%%%%%%%%%%
Προς αυτόν το στόχο αρχικά επικεντρωνόμαστε στην πηγή του προβλήματος: την
εκτίμηση της στάσης ενός ρομπότ βάσει περιορισμένης αβεβαιότητος (pose
tracking) με φίλτρο σωματιδίων (ενότητα \ref{subsec:01_01_02_3}). Σκοπός αυτής
της μελέτης, η οποία παρουσιάζεται στο κεφάλαιο \ref{part:02:chapter:02}, είναι
η μείωση του σφάλματος εκτίμησης στάσης του φίλτρου. Προς αυτή την κατεύθυνση
θέτουμε έναν αριθμό από υποθέσεις και εξακριβώνουμε πειραματικά την ευστάθειά
τους. Τα συμπεράσματα που εξάγουμε αφορούν στη βελτίωση της ακρίβειας εκτίμησης
του φίλτρου σωματιδίων (α) επιλέγοντας ως πηγές της τελικής εκτίμησης του
υποσύνολα των πιο βαρέων σωματιδίων, (β) με τον προσθετικό τρόπο χρήσης της
μεθόδου ευθυγράμμισης μετρήσεων lidar δύο διαστάσεων με σαρώσεις χάρτη (ενότητα
\ref{subsec:01_01_02_6}), και (γ) με την ανατροφοδότηση της εκτίμησης της
τελευταίας στον πληθυσμό σωματιδίων του φίλτρου (σχήμα \ref{fig:roadmap}-Β). Ως
προς τη λύση του προβλήματος της ελάττωσης του σφάλματος εκτίμησης ενός φίλτρου
σωματιδίων του οποίου η λειτουργία βασίζεται σε δισδιάστατες μετρήσεις
αισθητήρα απόστασης, οι συμβολές της διατριβής είναι οι εξής:

\begin{bw_box}
\begin{customcontribution}{Σ2}
  \label{contribution:02}
  Εισάγουμε την ορθολογική υπόθεση πως η ελάττωση του σφάλματος εκτίμησης του
  φίλτρου είναι εφικτή μέσω της απαγόρευσης ψηφοδοσίας ως προς την τελική
  εκτίμηση του φίλτρου συνόλων υποθέσεων-σωματιδίων των οποίων το
  βάρος---δηλαδή η πιθανότητα παρατήρησης της πραγματικής μέτρησης από την
  υπόθεση κατάστασης που κωδικοποιεί το κάθε σωματίδιο---είναι χαμηλότερο από
  άλλα σύνολα υποθέσεων του πληθυσμού του φίλτρου. \cite{Filotheou2020c}
\end{customcontribution}
\end{bw_box}

\begin{bw_box}
\begin{customcontribution}{Σ3}
  \label{contribution:03}
  Ελέγχουμε πειραματικά την ως άνω υπόθεση και ανακαλύπτουμε πως η ισχύς της
  επιβεβαιώνεται, όμως έως ενός σημείου: ενώ η υπόθεση-σωματίδιο της οποίας το
  βάρος είναι το μέγιστο εντός του πληθυσμού του φίλτρου οφείλει στη θεωρία να
  φέρει το ελάχιστο σφάλμα, τα αποδεικτικά στοιχεία δείχνουν ότι αυτός ο
  ισχυρισμός δεν είναι έγκυρος. Με βάση τα τελευταία καταλήγουμε στο θεωρητικό
  συμπέρασμα πως ένα φίλτρο σωματιδίων δεν αποτελεί άθροισμα υποθέσεων.
\end{customcontribution}
\end{bw_box}

\begin{bw_box}
\begin{customcontribution}{Σ4}
  \label{contribution:04}
  Εισάγουμε την ορθολογική υπόθεση πως η ανάδραση του αποτελέσματος της
  ευθυγράμμισης της πραγματικής σάρωσης με την εικονική σάρωση που συλλαμβάνεται
  από την εκτίμηση του φίλτρου πίσω στον πληθυσμό του με τη μορφή πλειάδας
  υποθέσεων-σωματιδίων προκαλεί χαμηλότερα σφάλματα εκτίμησης κατάστασης σε
  σχέση με τη συνθήκη ανατροφοδότησης μόνο ενός σωματιδίου.
\end{customcontribution}
\end{bw_box}

\begin{bw_box}
\begin{customcontribution}{Σ5}
  \label{contribution:05}
  Εισάγουμε την ορθολογική υπόθεση πως η ανάδραση του αποτελέσματος της
  ευθυγράμμισης της πραγματικής σάρωσης με την εικονική σάρωση που συλλαμβάνεται
  από την εκτίμηση του φίλτρου πίσω στον πληθυσμό του με τη μορφή πλειάδας
  υποθέσεων-σωματιδίων προκαλεί την αποφυγή απόκλισης του φίλτρου σε σχέση
  με τη συνθήκη επαναρχικοποίησης του φίλτρου γύρω από το αποτέλεσμα της
  ευθυγράμμισης.
\end{customcontribution}
\end{bw_box}

\begin{bw_box}
\begin{customcontribution}{Σ6}
  \label{contribution:06}
  Ελέγχουμε πειραματικά τις ως άνω δύο υποθέσεις και ανακαλύπτουμε πως η ισχύς
  τους επιβεβαιώνεται.
\end{customcontribution}
\end{bw_box}

%-------------------------------------------------------------------------------
Για την υλοποίηση της ευθυγράμμισης πραγματικών μετρήσεων lidar με σαρώσεις
χάρτη χρησιμοποιούμε τη μέθοδο ευθυγράμμισης μετρήσεων lidar (ενότητα
\ref{subsec:01_01_02_5}) με την καλύτερη επίδοση στη βιβλιογραφία. Κατά την
υλοποίηση της μεθόδου ευθυγράμμισης πραγματικών με εικονικές σαρώσεις
παρατηρούμε ότι οι λύσεις της εν λόγω μεθόδου παρουσιάζουν σημαντικές
διακυμάνσεις στην ακρίβειά τους (α) με μικρές αλλαγές στις παραμέτρους της
μεθόδου, μερικές από τις οποίες αφορούν στη διαδικασία υπολογισμού
αντιστοιχίσεων ανάμεσα στις ακτίνες των δύο σαρώσεων, και (β) με αυτήν να
μειώνεται όσο ο θόρυβος μέτρησης της πραγματικής σάρωσης αυξάνεται. Αυτές οι
δύο παρατηρήσεις είναι κρίσιμης σημασίας διότι στην πλειοψηφία τους όλες οι
μέθοδοι ευθυγράμμισης χρησιμοποιούν τον υπολογισμό αντιστοιχίσεων και,
θεωρητικά, όσο αυξάνει ο θόρυβος μέτρησης ή/και το επίπεδο διαφθοράς του χάρτη
ως προς το περιβάλλον που αναπαριστά τόσο δυσχερέστερη γίνεται η διάκριση
αληθών απο ψευδείς αντιστοιχίσεις ανάμεσα στα διανύσματα εισόδου. Ταυτόχρονα,
οι αισθητήρες απόστασης εμφανίζουν μη αμελητέα επίπεδα διαταραχών, των οποίων
το μέγιστο μέτρο είναι αντιστρόφως ανάλογο του εμπορικού κόστους τους. Κατά
συνέπεια σκεφτόμαστε ορθολογικά και συμπεραίνουμε πως ο σχεδιασμός μεθόδων
ευθυγράμμισης πραγματικών με εικονικές σαρώσεις οι οποίες δεν βασίζονται στον
υπολογισμό αντιστοιχίσεων ανάμεσα στις σαρώσεις-εισόδους τους είναι άξιος
έρευνας διότι εάν ισχύει η υπόθεση πως ``όσο αυξάνει ο θόρυβος μέτρησης ή/και
το επίπεδο διαφθοράς του χάρτη ως προς το περιβάλλον που αναπαριστά τόσο
δυσχερέστερη γίνεται η διάκριση αληθών απο ψευδείς αντιστοιχίσεις ανάμεσα στα
διανύσματα εισόδου" τότε η παροίχηση\footnote{Το αποτέλεσμα της πράξης που
δηλώνεται από το ρήμα \textit{παροίχομαι}, του οποίου μετοχή αποτελεί η λέξη
\textit{παρωχημένος} \cite{liddell_scott}, σελ. 491} του μηχανισμού υπολογισμού
αντιστοιχίσεων και η αντικατάστασή του από μεθόδους κλειστού τύπου ενδέχεται να
αποφέρει μεγαλύτερη ευρωστία σε διαταραχές, ενώ η μη προβλεψιμότητα της
απόκρισης μίας τέτοιας μεθόδου ενδέχεται να μειώνεται, καθώς ο καθορισμός
παραμέτρων που αφορά στο μηχανισμό υπολογισμού αντιστοιχίσεων ταυτόχρονα
καθίσταται αχρείαστος ως αδόκιμος.\\


%3%%%%%%%%%%%%%%%%%%%%%%%%%%%%%%%%%%%%%%%%%%%%%%%%%%%%%%%%%%%%%%%%%%%%%%%%%%%%%%
Για αυτούς τους λόγους ξεκινούμε να ερευνούμε τη βιβλιογραφία για μεθόδους
ευθυγράμμισης μετρήσεων lidar με σαρώσεις χάρτη που να μην χρησιμοποιούν
αντιστοιχίσεις και που να είναι εύρωστες ως προς τον θόρυβο εισόδου. Το
ενδιαφέρον εδώ είναι ότι τόσο οι μέθοδοι ευθυγράμμισης μετρήσεων lidar με
σαρώσεις χάρτη όσο και οι μέθοδοι ευθυγράμμισης μετρήσεων lidar (οι οποίες
είναι δυνατόν και αυτές να χρησιμοποιηθούν για την ευθυγράμμιση μετρήσεων με
σαρώσεις χάρτη) χρησιμοποιούν στο σύνολό τους αντιστοιχίσεις ανάμεσα σε δύο
εισόδους για να φέρουν εις πέρας την ευθυγράμμιση. Για να πετύχουμε τους
στόχους μας στρεφόμαστε στο πεδίο της μηχανικής όρασης, από όπου αξιοποιούμε
μία μέθοδο που εκπληρώνει και τα δύο παραπάνω κριτήρια. Για την πειραματική
εξακρίβωση του οφέλους χρήσης της τήν στρέφουμε στο πρόβλημα της εύρεσης της
στάσης ενός ρομπότ βάσει καθολικής αβεβαιότητος (σχήμα \ref{fig:roadmap}-Γ). Η
πειραματική διαδικασία της μεθόδου εστιάζει στην εξακρίβωση των ποσοστών των
αληθώς θετικών εκτιμήσεων στάσεων και των σφαλμάτων τους σε σχέση με την
καλύτερη μέθοδο ευθυγράμμισης μετρήσεων lidar με σαρώσεις χάρτη της
βιβλιογραφίας. Ως προς τη λύση του προβλήματος της εκτίμησης της στάσης ενός
πανοραμικού αισθητήρα αποστάσεων (ενότητα \ref{subsec:01_01_02_4}) βάσει
καθολικής αβεβαιότητος (ενότητα \ref{subsec:01_01_01_1}) με τη χρήση
ευθυγράμμισης πραγματικών με εικονικές σαρώσεις, η έρευνα επί της οποίας
παρουσιάζεται στο κεφάλαιο \ref{part:02:chapter:03}, οι συμβολές της διατριβής
είναι οι εξής:

\begin{bw_box}
\begin{customcontribution}{Σ7}
  \label{contribution:07}
  Εισάγουμε την πρώτη μέθοδο που είναι ικανή εκτίμησης του διανύσματος
  κατάστασης ενός αισθητήρα που βασίζεται εξ ολοκλήρου στην ευθυγράμμιση
  πραγματικών με εικονικές σαρώσεις χωρίς τον υπολογισμό αντιστοιχίσεων ανάμεσα
  στα διανύσματα εισόδου, της οποίας το σφάλμα εκτίμησης να είναι ισοδύναμο με
  την καλύτερη μέθοδο ευθυγράμμισης της πρότερης βιβλιογραφίας, και που να
  διαθέτει μικρότερο μέγεθος συνόλου παραμέτρων αναγκαίων καθορισμού.
  \cite{Filotheou2022e}
\end{customcontribution}
\end{bw_box}

\begin{bw_box}
\begin{customcontribution}{Σ8}
  \label{contribution:08}
  Εισάγουμε μία μέθοδο που είναι ικανή εκτίμησης της στάσης ενός αισθητήρα
  βάσει καθολικής αβεβαιότητος που βασίζεται εξ ολοκλήρου στην ευθυγράμμιση
  πραγματικών με εικονικές σαρώσεις χωρίς τον υπολογισμό αντιστοιχίσεων ανάμεσα
  στα διανύσματα εισόδου, η οποία να εμφανίζει ανώτερα ποσοστά επιτυχούς
  ανεύρεσης της στάσης του αισθητήρα σε σχέση με την καλύτερη μέθοδο
  ευθυγράμμισης σαρώσεων της βιβλιογραφίας που δύναται να χρησιμοποιηθεί για
  την ευθυγράμμιση πραγματικών με εικονικές σαρώσεις.
\end{customcontribution}
\end{bw_box}


%-------------------------------------------------------------------------------
Σε αυτό το σημείο μας έχουν γίνει κατανοητά τέσσερα σημεία: (α) η ευθυγράμμιση
πραγματικών μετρήσεων με σαρώσεις χάρτη είναι ικανή να επιλύσει με επιτυχία τα
προβλήματα εύρεσης και παρακολούθησης της στάσης ενός ρομπότ (δηλαδή βάσει
καθολικής και πεπερασμένης αβεβαιότητος), (β) η ευθυγράμμιση μετρήσεων με
σαρώσεις χάρτη με βάση τον υπολογισμό αντιστοιχίσεων ανάμεσα στις εισόδους---ο
de factο και καθ' ολοκληρίαν τρόπος επίλυσης του προβλήματος--- είναι υπό
συνθήκες επιβλαβής ως προς την ποιότητα της ευθυγράμμισης, (γ) η ανάπτυξη
μεθόδων ευθυγράμμισης μετρήσεων με σαρώσεις χάρτη χωρίς τη χρήση αντιστοιχίσεων
που εκτελείται σε πραγματικό χρόνο αποτελεί ως εκ τούτου σημαντική συμβολή στη
λύση του προβλήματος (η μέθοδος που παρουσιάζεται στο κεφάλαιο
\ref{part:02:chapter:03} δεν εκτελείται σε χρόνο τέτοιο που να μπορεί να
χρησιμοποιηθεί προς αντικατάσταση της μεθόδου που παρουσιάζεται στο κεφάλαιο
\ref{part:02:chapter:02}), και (δ) οποιαδήποτε προσπάθεια για τη δημιουργία
μεθόδου ευθυγράμμισης μετρήσεων με σαρώσεις χάρτη χωρίς τη χρήση αντιστοιχιών
και με εκτέλεση σε πραγματικό χρόνο θα έπρεπε να προέλθει από έρευνα έξω από τη
σχετική βιβλιογραφία.\\



%4%%%%%%%%%%%%%%%%%%%%%%%%%%%%%%%%%%%%%%%%%%%%%%%%%%%%%%%%%%%%%%%%%%%%%%%%%%%%%%
Ως εκ τούτων η έρευνά μας επικεντρώνεται στην επίλυση του προβλήματος της
ευθυγράμμισης μετρήσεων πανοραμικού αισθητήρα αποστάσεων με σαρώσεις χάρτη
χωρίς τη χρήση αντιστοιχίσεων και σε πραγματικό χρόνο (σχήμα
\ref{fig:roadmap}-Δ). Το αποτέλεσμα αυτής της έρευνας είναι η ανάπτυξη μίας
τριλογίας μεθόδων, αντλούσα την αποτελεσματικότητά και την ευρωστία της από
πρώτες αρχές, το πεδίο της κρυσταλλογραφίας, και το πεδίο της μηχανικής όρασης.
Ως προς τη λύση του προβλήματος της εκτίμησης της στάσης ενός πανοραμικού
αισθητήρα αποστάσεων βάσει περιορισμένης αβεβαιότητος (ενότητα
\ref{subsec:01_01_01_1}) με τη χρήση ευθυγράμμισης πραγματικών με εικονικές
σαρώσεις, η έρευνα επί της οποίας παρουσιάζεται στο κεφάλαιο
\ref{part:02:chapter:04}, οι κυριότερες συμβολές της διατριβής είναι οι εξής:

\begin{bw_box}
\begin{customcontribution}{Σ9}
  \label{contribution:09}
  Εισάγουμε τρεις μεθόδους εκτίμησης του προσανατολισμού ενός πανοραμικού
  αισθητήρα δισδιάστατων μετρήσεων που βασίζονται εξ ολοκλήρου στην
  ευθυγράμμιση πραγματικών με εικονικές σαρώσεις χωρίς τον υπολογισμό
  αντιστοιχίσεων ανάμεσα στα διανύσματα εισόδου. Οι δύο μέθοδοι αποτελούν
  προσαρμογές μεθόδων της βιβλιογραφίας της κρυσταλλογραφίας και της
  μηχανικής όρασης, ενώ η τρίτη είναι καινοφανής.
\end{customcontribution}
\end{bw_box}

\begin{bw_box}
\begin{customcontribution}{Σ10}
  \label{contribution:10}
  Δίνουμε μία λύση στο πρόβλημα αδυναμίας εκτίμησης με αυθαίρετο σφάλμα του
  πραγματικού προσανατολισμού του αισθητήρα για μεθόδους εκτίμησης
  προσανατολισμού που λειτουργούν στον διακριτό χώρο, χωρίς την ανάγκη
  παρεμβολών και συνεπώς χωρίς ακούσια βλάβη του σφάλματος εκτίμησης.
\end{customcontribution}
\end{bw_box}

\begin{bw_box}
\begin{customcontribution}{Σ11}
  \label{contribution:11}
  Εφευρίσκουμε μία μετρική που επιτρέπει την παρατήρηση της ιεραρχίας των
  σφαλμάτων εκτίμησης στάσεων μέσω των σαρώσεων που συλλαμβάνονται από αυτές.
\end{customcontribution}
\end{bw_box}

\begin{bw_box}
\begin{customcontribution}{Σ12}
  \label{contribution:12}
  Εσωκλείουμε τις τρεις ως άνω μεθόδους εκτίμησης προσανατολισμού, τη λύση της
  Συμβολής \ref{contribution:10}, και τη μετρική της Συμβολής
  \ref{contribution:11}, με μία μέθοδο εκτίμησης θέσης σε τρία συστήματα
  ευθυγράμμισης πραγματικών με εικονικές σαρώσεις χωρίς τον υπολογισμό
  αντιστοιχίσεων τα οποία είναι ικανά εκτέλεσης σε πραγματικό χρόνο.
\end{customcontribution}
\end{bw_box}

\begin{bw_box}
\begin{customcontribution}{Σ13}
  \label{contribution:13}
  Δοκιμάζουμε διεξοδικά την επίδοση και την ευρωστία των τριών ως άνω
  συστημάτων εκτίμησης στάσης έναντι αυτών μεθόδων της βιβλιογραφίας σε
  συνθήκες αρχικών σφαλμάτων και διαταραχών σαρώσεων εισόδου που συναντώνται
  στην πράξη, και καταλήγουμε στο συμπέρασμα πως τα σχεδιασθέντα συστήματα
  είναι (α) κατ' ελάχιστον ισοδύναμα των μεθόδων της βιβλιογραφίας οι οποίες
  χρησιμοποιούν αντιστοιχίσεις για να φέρουν εις πέρας το έργο της
  ευθυγράμμισης ως προς το μέσο σφάλμα εκτίμησης, και (β) εμφανίζουν πιο
  εύρωστα και πιο υψηλά ποσοστά περιπτώσεων ελάττωσης του αρχικού σφάλματος
  εκτίμησης.
\end{customcontribution}
\end{bw_box}
\vspace{1cm}



%5%%%%%%%%%%%%%%%%%%%%%%%%%%%%%%%%%%%%%%%%%%%%%%%%%%%%%%%%%%%%%%%%%%%%%%%%%%%%%%
Το επόμενο και τελευταίο βήμα είναι το πιο σημαντικό, το λιγότερο τεχνικό, και
με συμβολή που έχει την πιο ευρεία χρήση: εάν στην παραπάνω μέθοδο
ευθυγράμμισης πραγματικών με εικονικές σαρώσεις ο χάρτης αντικατασταθεί με μία
δεύτερη φυσική μέτρηση τότε η μέθοδος ευθυγράμμισης μετρήσεων με σαρώσεις χάρτη
μετατρέπεται στη γενικότερη μέθοδο ευθυγράμμισης μετρήσεων lidar, η οποία
δύναταται να χρησιμοποιηθεί ως μέσο οδομετρίας (απαραίτητη στα φίλτρα
σωματιδίων και Kalman), και βρίσκεται στην καρδιά της επίλυσης του προβλήματος
της ταυτόχρονης χαρτογράφησης και παρακολούθησης της στάσης ενός ρομπότ
(Παρατήρηση \ref{rem:sm_applications} και σχήμα \ref{fig:roadmap}-Ε). Το
τελευταίο λοιπόν βήμα είναι ο μετασχηματισμός της ήδη σχεδιασθείσας μεθόδου για
την ευθυγράμμιση δισδιάστατων μετρήσεων αισθητήρα αποστάσεων τύπου lidar που
δεν χρησιμοποιεί αντιστοιχίσεις, που εκτελείται σε πραγματικό χρόνο, και που
εμφανίζει μεγαλύτερη ευρωστία στο θόρυβο μέτρησης και μικρότερα σφάλματα
ευθυγράμμισης σε σχέση με αντίστοιχες μεθόδους της βιβλιογραφίας. Ως προς τη
λύση του προβλήματος της εκτίμησης της στάσης ενός πανοραμικού αισθητήρα
αποστάσεων άνευ εσωτερικής αναπαράστασης του περιβάλλοντός του με τη χρήση
ευθυγράμμισης πραγματικών σαρώσεων, η έρευνα επί της οποίας παρουσιάζεται στο
κεφάλαιο \ref{part:02:chapter:05}, οι κυριότερες συμβολές της διατριβής είναι
οι εξής:

\begin{bw_box}
\begin{customcontribution}{Σ14}
  \label{contribution:14}
  Εισάγουμε την πρώτη μέθοδο ευθυγράμμισης σαρώσεων πανοραμικού αισθητήρα
  lidar μετρήσεων δύο διαστάσεων η οποία, σε αντίθεση με τις μεθόδους της
  τρέχουσας βιβλιογραφίας δεν λειτουργεί με τον υπολογισμό αντιστοιχίσεων
  ανάμεσα στα διανύσματα εισόδου, και η οποία εκτελείται σε πραγματικό χρόνο.
\end{customcontribution}
\end{bw_box}

\begin{bw_box}
\begin{customcontribution}{Σ15}
  \label{contribution:15}
  Δοκιμάζουμε διεξοδικά την επίδοση και την ευρωστία της μεθόδου ευθυγράμμισης
  που εισαγάγαμε έναντι αυτών μεθόδων της βιβλιογραφίας σε
  συνθήκες αρχικών αποστάσεων και διαταραχών σαρώσεων εισόδου που συναντώνται
  στην πραγματικότητα, και καταλήγουμε στο συμπέρασμα πως το σχεδιασθέν
  σύστημα είναι (α) κατ' ελάχιστον ισοδύναμο των μεθόδων της βιβλιογραφίας οι
  οποίες χρησιμοποιούν αντιστοιχίσεις για να φέρουν εις πέρας το έργο της
  ευθυγράμμισης ως προς το μέσο και διάμεσο σφάλμα εκτίμησης, και (β)
  εμφανίζουν μεγαλύτερη ευρωστία απο αυτές ως προς το θόρυβο μέτρησης.
\end{customcontribution}
\end{bw_box}

\begin{figure}\hspace{-2cm}
  \input{./figures/parts/01/chapters/02/sections/01/roadmap.pdf_tex}
  \caption{\small Ο οδικός χάρτης της διατριβής}
  \label{fig:roadmap}
\end{figure}
