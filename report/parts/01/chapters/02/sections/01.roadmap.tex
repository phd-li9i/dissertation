Αυτό το κεφάλαιο έχει ως σκοπό την παροχή μίας συνοπτικής κάτοψης των
προβλημάτων στων οποίων τη λύση συμβάλλει η διατριβή. Το σχήμα
\ref{fig:roadmap} λειτουργεί τροχιοδεικτικά ως προς τα προβλήματα-σταθμούς,
των συνδετικών βημάτων ανάμεσά τους, και τις ιδιότητές των λύσεών τους.


Όλα ξεκινούν από την ανάγκη διαλεύκανσης ενός προβλήματος του οποίου η λύση
είναι κρίσιμη σε πρακτικές εφαρμογές ρομποτικής κινητής βάσης: της επίδοσης και
ποιότητας των διαφορετικών πακέτων λογισμικού που αφορούν στην αυτόνομη
πλοήγηση με το λειτουργικό σύστημα ROS (υποενότητα \ref{subsec:01_01_02_7}).
Καθώς η δημοφιλία τού τελευταίου έχει εξαπλωθεί στην έρευνα, έχει ενσωματωθεί
ένας ικανός αριθμός αλγορίθμων αυτόνομους πλοήγησης (χάραξης μονοπατιών σε
δισδιάστατο χάρτη και ελεγκτών κίνησης: υποενότητα \ref{subsec:01_01_01_1}),
των οποίων η συνδυαστική χρήση αποτελεί αντικείμενο χρονοβόρας έρευνας και
πειραματισμού για ερευνητές και επαγγελματίες του πεδίου εφαρμογής \ref{scope}.
Σκοπός αυτής της μελέτης είναι η παροχή μίας μεθόδου αξιολόγησης της επίδοσης
αλγορίθμων αυτόνομους πλοήγησης, καθώς και η πειραματική αξιοποίησή της σε ό,τι
αφορά τρέχοντες διαθέσιμους αλγορίθμους.

Κατά τη διενέργεια της πειραματικής αξιολόγησης των μεθόδων πλοήγησης
παρατηρήσαμε το φαινόμενο της αστάθειας της εκτίμησης της στάσης και το
γενικευμένο φαινόμενο του σφάλματός της ως προς την πραγματική στάση ενός
ρομπότ, ανεξαρτήτως μεθόδου πλοήγησης (σχήμα \ref{fig:roadmap}-Α). Η μικρή αυτή
παρατήρηση αποδεικνύεται ότι είναι καίριας σημασίας καθώς μας εισάγει στον
δρόμο της έρευνας επί της βελτίωσης της εκτίμησης της στάσης ενός ρομπότ.

%% <-- ?

Προς αυτόν το στόχο επικεντρωθήκαμε στην πηγή του προβλήματος: την εκτίμηση της
στάσης ενός ρομπότ βάσει περιορισμένης αβεβαιότητας (pose tracking) με φίλτρο
σωματιδίων (υποενότητα \ref{subsec:01_01_02_3}). Με σκοπό τη μείωση του
σφάλματος εκτίμησης θέσαμε έναν αριθμό από υποθέσεις και εξακριβώσαμε
πειραματικά την ευστάθειά τους. Τα συμπεράσματα που εξήγαμε αφορούν στη
βελτίωση της ακρίβειας εκτίμησης του φιλτρου σωματιδίων (α) επιλέγοντας ως
πηγές της τελικής εκτίμησης του υποσύνολα των πιό βαρέων σωματιδίων, (β) με τον
προσθετικό τρόπο χρήσης της μεθόδου ευθυγράμμισης μετρήσεων lidar με σαρώσεις
χάρτη (υποενότητα \ref{subsec:01_01_02_6}), και (γ) με την ανατροφοδότηση της
εκτίμησης της τελευταίας στον πληθυσμό σωματιδίων του φίλτρου (σχήμα
\ref{fig:roadmap}-Β).

Για την υλοποίηση της ευθυγράμμισης μετρήσεων lidar με σαρώσεις χάρτη
χρησιμοποιήσαμε τον αλγόριθμο ευθυγράμμισης μετρήσεων lidar με την καλύτερη
επίδοση στη βιβλιογραφία. Κατά την υλοποίηση της μεθόδου β$^\prime$ παρατηρήσαμε
ότι η λύσεις του εν λόγω αλγορίθμου παρουσίαζαν σημαντικές διακυμάνσεις στην
ακρίβειά τους (α) με μικρές αλλαγές στις παραμέτρους που αφορούν στη διαδικασία
υπολογισμού αντιστοιχιών ανάμεσα στις ακτίνες των δύο σαρώσεων, και (β) με την
ακρίβεια να μειώνεται όσο ο θόρυβος στις δύο σαρώσεις αυξάνεται.

Για αυτούς τους λόγους ξεκινήσαμε να ερευνούμε τη βιβλιογραφία για μεθόδους
ευθυγράμμισης μετρήσεων lidar με σαρώσεις χάρτη που να μην χρησιμοποιούν
αντιστοιχίες και που να είναι εύρωστες ως προς τον θόρυβο εισόδου. Το
ενδιαφέρον εδώ είναι ότι τόσο οι μέθοδοι ευθυγράμμισης μετρήσεων lidar με
σαρώσεις χάρτη όσο και οι μέθοδοι ευθυγράμμισης μετρήσεων lidar (οι οποίες
είναι δυνατόν και αυτές να χρησιμοποιηθούν για την ευθυγράμμιση μετρήσεων με
σαρώσεις χάρτη) χρησιμοποιούν στο σύνολό τους αντιστοιχίσεις ανάμεσα σε δύο
εισόδους για να φέρουν εις πέρας την ευθυγράμμιση. Για να πετύχουμε τους στόχους
στραφήκαμε εν τέλει στο πεδίο της μηχανικής όρασης, από όπου χρησιμοποιήσαμε
μία μέθοδο που εκπληρώνει και τα δύο κριτήρια. Για την πειραματική εξακρίβωση
του οφέλους χρήσης της τήν στρέψαμε στο πρόβλημα της εύρεσης της στάσης ενός
ρομπότ βάσει καθολικής αβεβαιότητος (σχήμα \ref{fig:roadmap}-Γ). Ο λόγος που η
μέθοδος δεν χρησιμοποιήθηκε απευθείας για την εκτίμηση της στάσης ενός οχήματος
βάσει περιορισμένης αβεβαιότητος είναι ότι ο χρόνος εκτέλεσής της είναι πιό
αργός από ότι απαιτείται σε αυτό το πρόβλημα, όπου, λόγω κίνησης του οχήματος,
απαιτούνται υψηλότεροι ρυθμοί ανανέωσης. Αντιθέτως, στο πρόβλημα της εκτίμησης
βάσει καθολικής αβεβαιότητος ο χαμηλός χρόνος εκτέλεσης είναι επιθυμητός αλλά
όχι αυστηρά απαιτητέος ή αναγκαίος. Τα πειραματικά αποτελέσματα έδειξαν ότι η
προτεινόμενη μέθοδος επιφέρει υψηλότερα ποσοστά εύρεσης στάσης σε σχέση με την
καλύτερη μέθοδο ευθυγράμμισης μετρήσεων lidar με σαρώσεις χάρτη, και παρόμοια
σφάλματα στάσης.






\begin{figure}\hspace{-2cm}
  \input{./figures/parts/01/chapters/02/sections/01/roadmap.pdf_tex}
  \caption{Ο οδικός χάρτης της διατριβής}
  \label{fig:roadmap}
\end{figure}
