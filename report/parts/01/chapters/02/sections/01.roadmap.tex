Αυτό το κεφάλαιο έχει ως σκοπό την παροχή μίας συνοπτικής κάτοψης των
προβλημάτων στων οποίων τη λύση συμβάλλει η διατριβή. Το σχήμα
\ref{fig:roadmap} λειτουργεί τροχιοδεικτικά ως προς τα προβλήματα-σταθμούς,
των συνδετικών βημάτων ανάμεσά τους, και τις ιδιότητές των λύσεών τους.


Όλα ξεκινούν από την ανάγκη διαλεύκανσης ενός προβλήματος του οποίου η λύση
είναι κρίσιμη σε πρακτικές εφαρμογές ρομποτικής κινητής βάσης: της επίδοσης και
ποιότητας των διαφορετικών πακέτων λογισμικού που αφορούν στην αυτόνομη
πλοήγηση με το λειτουργικό σύστημα \texttt{ROS} (ενότητα
\ref{subsec:01_01_02_9}).  Καθώς η δημοφιλία τού τελευταίου έχει εξαπλωθεί στην
έρευνα, έχει ενσωματωθεί σε αυτό ένας ικανός αριθμός αλγορίθμων αυτόνομους
πλοήγησης (χάραξης μονοπατιών σε δισδιάστατο χάρτη και ελεγκτών κίνησης:
ενότητα \ref{subsec:01_01_01_1}), των οποίων η συνδυαστική χρήση αποτελεί
αντικείμενο χρονοβόρας έρευνας και πειραματισμού για ερευνητές και
επαγγελματίες του πεδίου εφαρμογής \ref{scope}.  Σκοπός αυτής της μελέτης είναι
η παροχή μίας μεθόδου αξιολόγησης της επίδοσης αλγορίθμων αυτόνομους πλοήγησης,
καθώς και η πειραματική αξιοποίησή της σε ό,τι αφορά τρέχοντες διαθέσιμους
αλγορίθμους.

Κατά τη διενέργεια της πειραματικής αξιολόγησης των μεθόδων πλοήγησης
παρατηρήσαμε το φαινόμενο της αστάθειας της εκτίμησης της στάσης από το φίλτρο
σωματιδίων, και το γενικευμένο φαινόμενο του σφάλματός της ως προς την
πραγματική στάση ενός ρομπότ, ανεξαρτήτως μεθόδου πλοήγησης (σχήμα
\ref{fig:roadmap}-Α). Η μικρή αυτή παρατήρηση αποδεικνύεται ότι είναι καίριας
σημασίας καθώς μάς εισάγει στον δρόμο της έρευνας επί της βελτίωσης της
εκτίμησης της στάσης ενός ρομπότ.

%% <-- ?

Προς αυτόν το στόχο επικεντρωθήκαμε στην πηγή του προβλήματος: την εκτίμηση της
στάσης ενός ρομπότ βάσει περιορισμένης αβεβαιότητας (pose tracking) με φίλτρο
σωματιδίων (ενότητα \ref{subsec:01_01_02_3}). Με σκοπό τη μείωση του
σφάλματος εκτίμησης θέσαμε έναν αριθμό από υποθέσεις και εξακριβώσαμε
πειραματικά την ευστάθειά τους. Τα συμπεράσματα που εξήγαμε αφορούν στη
βελτίωση της ακρίβειας εκτίμησης του φιλτρου σωματιδίων (α) επιλέγοντας ως
πηγές της τελικής εκτίμησης του υποσύνολα των πιό βαρέων σωματιδίων, (β) με τον
προσθετικό τρόπο χρήσης της μεθόδου ευθυγράμμισης μετρήσεων lidar με σαρώσεις
χάρτη (ενότητα \ref{subsec:01_01_02_6}), και (γ) με την ανατροφοδότηση της
εκτίμησης της τελευταίας στον πληθυσμό σωματιδίων του φίλτρου (σχήμα
\ref{fig:roadmap}-Β).

Για την υλοποίηση της ευθυγράμμισης μετρήσεων lidar με σαρώσεις χάρτη
χρησιμοποιήσαμε τον αλγόριθμο ευθυγράμμισης μετρήσεων lidar με την καλύτερη
επίδοση στη βιβλιογραφία. Κατά την υλοποίηση της μεθόδου β$^\prime$ παρατηρήσαμε
ότι η λύσεις του εν λόγω αλγορίθμου παρουσίαζαν σημαντικές διακυμάνσεις στην
ακρίβειά τους (α) με μικρές αλλαγές στις παραμέτρους που αφορούν στη διαδικασία
υπολογισμού αντιστοιχιών ανάμεσα στις ακτίνες των δύο σαρώσεων, και (β) με την
ακρίβεια να μειώνεται όσο ο θόρυβος στις δύο σαρώσεις αυξάνεται.

Για αυτούς τους λόγους ξεκινήσαμε να ερευνούμε τη βιβλιογραφία για μεθόδους
ευθυγράμμισης μετρήσεων lidar με σαρώσεις χάρτη που να μην χρησιμοποιούν
αντιστοιχίες και που να είναι εύρωστες ως προς τον θόρυβο εισόδου. Το
ενδιαφέρον εδώ είναι ότι τόσο οι μέθοδοι ευθυγράμμισης μετρήσεων lidar με
σαρώσεις χάρτη όσο και οι μέθοδοι ευθυγράμμισης μετρήσεων lidar (οι οποίες
είναι δυνατόν και αυτές να χρησιμοποιηθούν για την ευθυγράμμιση μετρήσεων με
σαρώσεις χάρτη) χρησιμοποιούν στο σύνολό τους αντιστοιχίσεις ανάμεσα σε δύο
εισόδους για να φέρουν εις πέρας την ευθυγράμμιση. Για να πετύχουμε τους
στόχους στραφήκαμε εν τέλει στο πεδίο της μηχανικής όρασης, από όπου
χρησιμοποιήσαμε μία μέθοδο που εκπληρώνει και τα δύο κριτήρια. Για την
πειραματική εξακρίβωση του οφέλους χρήσης της τήν στρέψαμε στο πρόβλημα της
εύρεσης της στάσης ενός ρομπότ βάσει καθολικής αβεβαιότητος (σχήμα
\ref{fig:roadmap}-Γ). Η πειραματική διαδικασία της μεθόδου εστιάζει στην
εξακρίβωση των ποστοστών των αληθώς θετικών εκτιμήσεων στάσεων και των
σφαλμάτων τους σε σχέση με την καλύτερη μέθοδο ευθυγράμμισης μετρήσεων lidar με
σαρώσεις χάρτη της βιβλιογραφίας.

Ο λόγος που η μέθοδος δεν χρησιμοποιήθηκε απευθείας για την εκτίμηση
της στάσης ενός οχήματος βάσει περιορισμένης αβεβαιότητος είναι ότι ο χρόνος
εκτέλεσής της είναι τέτοιος που δεν μπορεί να συμβαδίσει με το ρυθμό ανανέωσης
των μετρήσεων που προέρχονται από έναν συμβατικό αισθητήρα lidar. Αντιθέτως,
στο πρόβλημα της εκτίμησης βάσει καθολικής αβεβαιότητος, ο χαμηλός χρόνος
εκτέλεσης είναι επιθυμητός αλλά όχι αυστηρά απαιτητέος ή αναγκαίος.

Σε αυτό το σημείο είχαν γίνει κατανοητά τέσσερα σημεία: (α) η ευθυγράμμιση
μετρήσεων με σαρώσεις χάρτη είναι ικανή να επιλύσει με επιτυχία τα προβλήματα
εύρεσης και παρακολούθησης της στάσης ενός ρομπότ (δηλαδή βάσει καθολικής και
πεπερασμένης αβεβαιότητος), (β) η ευθυγράμμιση μετρήσεων με σαρώσεις χάρτη με
βάση τον υπολογισμό αντιστοιχιών ανάμεσα στις εισόδους---ο de factο και καθ'
ολοκληρίαν τρόπος επίλυσης του προβλήματος--- είναι υπό συνθήκες επιβλαβής ως
προς την ποιότητα της ευθυγράμμισης, (γ) η ανάπτυξη μεθόδων ευθυγράμμισης
μετρήσεων με σαρώσεις χάρτη χωρίς τη χρήση αντιστοιχιών που εκτελείται σε
πραγματικό χρόνο αποτελεί ως εκ τούτου σημαντική συμβολή στη λύση του
προβλήματος, και (δ) οποιαδήποτε προσπάθεια για τη δημιουργία μεθόδου
ευθυγράμμισης μετρήσεων με σαρώσεις χάρτη χωρίς τη χρήση αντιστοιχιών θα έπρεπε
να προέλθει από έρευνα έξω από τη σχετική βιβλιογραφία.

Ως εκ τούτων η έρευνα μας επικεντρώθηκε στην επίλυση του προβλήματος της
ευθυγράμμισης μετρήσεων με σαρώσεις χάρτη χωρίς τη χρήση αντιστοιχιών και σε
πραγματικό χρόνο. Προς αυτόν το σκοπό εστιάσαμε σε μία κλάση αισθητήρων lidar
των οποίων η χρήση έχει αυξηθεί τα τελευταία χρόνια για σκοπούς εύρεσης της
στάσης, οι οποίοι έχουν να επωφεληθούν τα μέγιστα από τέτοιες μεθόδους λόγω
του αυξημένου θορύβου μέτρησης που φέρουν. Επιπρόσθετα, αυτή η κλάση αισθητήρων
εμφανίζει πανοραμικό γωνιακό πεδίο όρασης: κατά συνέπεια η περιοδικότητα του
σήματος μετρήσεων αποτελεί γόνιμο έδαφος για την απαλλαγή από τον υπολογισμό
αντιστοιχιών (σχήμα \ref{fig:roadmap}-Δ). Το αποτέλεσμα αυτής της έρευνας ήταν
η ανάπτυξη μίας τριλογίας μεθόδων, αντλούσα την αποτελεσματικότητά της από
πρώτες αρχές, το πεδίο της κρυσταλλογραφίας, και το πεδίο της μηχανικής όρασης.
Κάθε μία από τις τρεις μεθόδους εκτελείται σε πραγματικό χρόνο και εμφανίζει
μεγαλύτερη ευρωστία και μικρότερα σφάλματα στάσης από τις μεθόδους της
βιβλιογραφίας.

Το επόμενο και τελευταίο βήμα ήταν το πιό σημαντικό, το λιγότερο τεχνικό, και
με τη γενικότερη συμβολή: εάν ο χάρτης αντικατασταθεί με μία δεύτερη φυσική
μέτρηση τότε η ευθυγράμμιση μετρήσεων με σαρώσεις χάρτη μετατρέπεται στη
γενικότερη μέθοδο ευθυγράμμισης μετρήσεων lidar, η οποία χρησιμοποιείται
ως μέσο οδομετρίας (απαραίτητη στα φίλτρα σωματιδίων και Kalman), και βρίσκεται
στην καρδιά της επίλυσης του προβλήματος της ταυτόχρονης χαρτογράφησης και
παρακολούθησης της στάσης ενός ρομπότ (παρατήρηση \ref{rem:sm_applications}).
Το τελευταίο λοιπόν βήμα είναι η ανάπτυξη μίας μεθόδου για την ευθυγράμμιση
μετρήσεων 2D lidar που δεν χρησιμοποιεί αντιστοιχίες, που εκτελείται σε
πραγματικό χρόνο, και που εμφανίζει μεγαλύτερη ευρωστία στο θόρυβο μέτρησης
και μικρότερα σφάλματα ευθυγράμμισης σε σχέση με αντίστοιχες μεθόδους της
βιβλιογραφίας.


\begin{figure}\hspace{-2cm}
  \input{./figures/parts/01/chapters/02/sections/01/roadmap.pdf_tex}
  \caption{Ο οδικός χάρτης της διατριβής}
  \label{fig:roadmap}
\end{figure}
