%%%%%%%%%%%%%%%%%%%%%%%%%%%%%%%%%%%%%%%%%%%%%%%%%%%%%%%%%%%%%%%%%%%%%%%%%%%%%%%%
\subsection{Εκτιμητέο διάνυσμα κατάστασης}

Κεντρικής σημασίας στη διατριβή είναι το εκτιμητέο διάνυσμα κατάστασης ενός
επίγειου οχήματος. Μέχρι σε αυτό το σημείο χρησιμοποιείτο αντί αυτής η λέξη
``θέση" για εισαγωγικούς λόγους.

\begin{bw_box}
\begin{definition}
  \textit{Διάνυσμα κατάστασης ή στάση}

Ως διάνυσμα κατάστασης θεωρούμε τη στάση ενός οχήματος στο δισδιάστατο επίπεδο:
τον ειρμό της θέσης του με τον προσανατολισμό του, ως προς το σύστημα αναφοράς
του χάρτη του περιβάλλοντος στο οποίο βρίσκεται το όχημα (σχήμα
\ref{fig:pose_figure}):
  \begin{align}
    \bm{p} = [x \ y \ \theta]^\top
\end{align}

\end{definition}
\end{bw_box}

\begin{figure}[H]\centering
  \input{./figures/01.01.concepts/pose.eps_tex}
  \caption{\small Το διάνυσμα κατάστασης (στάση) $\bm{p} = [x,y,\theta]^\top$
    ενός επίγειου οχήματος στο οριζόντιο επίπεδο}
  \label{fig:pose_figure}
\end{figure}

  Η στάση του οχήματος είναι το αντικείμενο της
  Λόγω παραδοχής \ref{ass:01_01} η στάση είναι

%%%%%%%%%%%%%%%%%%%%%%%%%%%%%%%%%%%%%%%%%%%%%%%%%%%%%%%%%%%%%%%%%%%%%%%%%%%%%%%%
\subsection{Ο αισθητήρας lidar δισδιάστατων μετρήσεων}

\begin{bw_box}
\begin{definition}
  \label{def:lidar}
  \textit{Ορισμός μέτρησης αισθητήρα 2D lidar}

  Μία μέτρηση συμβατικού αισθητήρα 2D lidar αποτελείται από έναν πεπερασμένο
  αριθμό αποστάσεων σε αντικείμενα σε οπτική επαφή εντός της μέγιστης
  εμβέλειάς του. Οι μετρήσεις λαμβάνονται εγκαρσίως προς το σώμα του, σε
  κανονικά γωνιακά και χρονικά διαστήματα, σε ένα καθορισμένο γωνιακό εύρος
  \cite{Cooper2018a}.

  Μία μέτρηση-σάρωση $\mathcal{S}$ που απαρτίζεται από $N_s$ ακτίνες σε γωνιακό εύρος
  $\lambda$ είναι μία διατεταγμένη ακολουθία $\mathcal{S} : \Theta \rightarrow
  \mathbb{R}_{\geq 0}$, όπου
  \begin{align}
  \Theta = \{\theta_n \in [-\frac{\lambda}{2}, +\frac{\lambda}{2}) :
    \theta_n = -\frac{\lambda}{2} + \lambda \frac{n}{N_s},
  n = 0,1,\dots, N_s-1\}
  \end{align}

  Οι γωνίες $\theta_n$ εκφράζονται σε σχέση με τον προσανατολισμό του αισθητήρα
  στο τοπικό του σύστημα συντεταγμένων.
\end{definition}
\end{bw_box}

Το σχήμα \ref{fig:laser} απεικονίζει τη γεωμετρία του ενός τυπικού αισθητήρα
2D lidar, όπου $d_n = \mathcal{S}[-\frac{\lambda}{2} + \frac{\lambda n}{N_s}]$
είναι η απόσταση που αφορά στην ακτίνα με αναγνωριστικό $n$.

\begin{figure}[]\centering
  \usetikzlibrary{intersections}
\usetikzlibrary{angles, quotes}
\begin{tikzpicture}


  \coordinate (O) at (0,0);
  \node (O_n) at (0.2,-0.2) {$O$};
  \node (x_plus) at (3.5,0) {$x$};
  \node (y_plus) at (0,3) {$y$};
  \coordinate (x_minus) at (-2,0);
  \coordinate (y_minus) at (0,-2.5);
  \coordinate (first_ray) at (-2*0.70711, -2*0.70711);
  \coordinate (first_ray_far) at (-2.5*0.70711, -2.5*0.70711);
  \node (ray_0) at (-3.0*0.70711, -3.0*0.70711){ακτίνα $0$};
  \coordinate (last_ray) at (-2*0.70711, 2*0.70711);
  \coordinate (last_ray_far) at (-2.5*0.70711, 2.5*0.70711);
  \node (ray_N) at (-3.0*0.70711, 3.0*0.70711){ακτίνα $N_s$$-$$1$};
  \node (l) at (-1.0,0.2){$\scriptstyle{2\pi-\lambda}$};
  \coordinate(n_c) at (3.0,1.117);
  \node[right] (n_n) at (3.0,1.117){ακτίνα $n$};
  \draw [fill] (n_c) circle [radius=0.05];
  \draw [fill] (O) circle [radius=0.05];
  \node[above] (dn) at (1.0,0.35){$d_n$};

  % draw axes
  \draw [->] (x_minus) -- (x_plus);
  \draw [->] (y_minus) -- (y_plus);
  \draw [dashed] (O) -- (last_ray_far);
  \draw [dashed] (O) -- (first_ray_far);
  \draw [->] (O) -- (n_c);

  % draw laser arc
  \draw [black, thick, dotted] (first_ray) arc[start angle=-135, end angle=135,radius=2];

  % draw 2π - λ arc
  \pic [draw,  angle radius=5mm, angle eccentricity=1.4] {angle = last_ray--O--first_ray};

  % draw n angle arc
  \pic [draw, ->, angle radius=17mm, angle eccentricity=1.4] {angle = x_plus--O--n_c};
  \node (angle_n) at (2.6,0.44){${\scriptstyle-\dfrac{\scriptstyle\lambda}{\scriptstyle 2} + \dfrac{\scriptstyle \lambda n}{\scriptstyle N_s}}$};

\end{tikzpicture}

  \caption{\small Κάτοψη του τοπικού συστήματος αναφοράς ενός τυπικού αισθητήρα
           αποστάσεων τύπου lidar. Ο αισθητήρας είναι τοποθετημένος στο $O(0,0)$
           και ο προσανατολισμός του είναι αυτός του θετικού $x$ άξονα. Το
           γωνιακό πεδίο οράσεώς του είναι $\lambda$}
  \label{fig:laser}
\end{figure}

\begin{bw_box}
\begin{definition}
  \textit{Πανοραμικός αισθητήρας 2D lidar}

  Το γωνιακό εύρος ενός 2D lidar είναι συμμετρικά κατανεμημένο ως προς τον
  τοπικό του $x$ άξονα. Κάθε ακτίνα έχει την ίδια γωνιακή απόσταση από τις
  γειτονικές της, εξαιρέσει των δύο ακραίων ακτίνων όταν $\lambda < 2\pi$.
  Όταν $\lambda = 2\pi$ ο αισθητήρας ονομάζεται πανοραμικός.
\end{definition}
\end{bw_box}

%%%%%%%%%%%%%%%%%%%%%%%%%%%%%%%%%%%%%%%%%%%%%%%%%%%%%%%%%%%%%%%%%%%%%%%%%%%%%%%%
\subsection{Το φίλτρο σωματιδίων}
%%%%%%%%%%%%%%%%%%%%%%%%%%%%%%%%%%%%%%%%%%%%%%%%%%%%%%%%%%%%%%%%%%%%%%%%%%%%%%%%
\subsection{Ευθυγράμμιση σαρώσεων lidar}
%%%%%%%%%%%%%%%%%%%%%%%%%%%%%%%%%%%%%%%%%%%%%%%%%%%%%%%%%%%%%%%%%%%%%%%%%%%%%%%%
\subsection{Τα προβλήματα εύρεσης στάσης}
%%%%%%%%%%%%%%%%%%%%%%%%%%%%%%%%%%%%%%%%%%%%%%%%%%%%%%%%%%%%%%%%%%%%%%%%%%%%%%%%
\subsection{Το λειτουργικό σύστημα ρομπότ ROS}
