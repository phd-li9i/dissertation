Στην παρούσα διατριβή ξεκινήσαμε από την ανάγκη αξιολόγησης συνδυασμών μεθόδων
αυτόνομης πλοήγησης στον δισδιάστατο χώρο και καταλήξαμε στο σχεδιασμό της
πρώτης μεθόδου ευθυγράμμισης δισδιάστατων πανοραμικών μετρήσεων αισθητήρα lidar
η οποία δεν υπολογίζει αντιστοιχίσεις μεταξύ των ακτίνων των μετρήσεων για να
φέρει εις πέρας το έργο της ευθυγράμμισης. Η αφετηρία (κεφάλαιο
\ref{part:02:chapter:01}) συνδέεται με το τέρμα (κεφάλαιο
\ref{part:02:chapter:05}) με ένα νήμα που έχει ως κοινό παρονομαστή την
ευθυγράμμιση πραγματικών με εικονικές σαρώσεις αισθητήρα lidar μετρήσεων δύο
διαστάσεων ως ένα αποτελεσματικό μέσο εκτίμησης της στάσης του αισθητήρα σε
συμφραζόμενα και συνθήκες που απαντώνται στην ερευνητική θεωρία και ερευνητική
πράξη.

Στο κεφάλαιο \ref{part:02:chapter:01} σχεδιάσαμε μία ολοκληρωμένη και
επεκτάσιμη μεθοδολογία αξιολόγησης αλγορίθμων χάραξης μονοπατιών, ελεγκτών
κίνησης, και των συνδυασμών τους, για το σκοπό της αποσαφήνισης της ιεραρχίας
αποτελεσματικότητας και επίδοσης τρεχουσών υλοποιήσεών τους μέσω του- και στο
λογισμικο \texttt{ROS}, το de facto μέσο υλοποίησης ρομποτικών εφαρμογών κατά
το χρόνο της συγγραφής αυτής της διατριβής. Στην κεφαλή της ιεραρχίας
αναδείχθηκε ένας συνδυασμός ο οποίος σε προσομοιώσεις και πειράματα ανταπεξήλθε
σε όλες τις δοκιμαστικές συνθήκες που θέσαμε, και που, ανεκδοτολογικά,
χρησιμοποιήθηκε με επιτυχία για μεγάλες διάρκειες πλοήγησης στα πλαίσια των
έργων RELIEF\footnote{\textit{Ευφυείς Επαναλήπτες και Ρομπότ για Γρήγορη,
Αξιόπιστη, Χαμηλού Κόστους Απογραφή και Εύρεση της Θέσης Αντικειμένων, μέσω
Τεχνολογίας RFID}. Κωδικός έργου Τ1ΕΔΚ-03032.
\url{https://relief.web.auth.gr/}} και CULTUREID\footnote{\textit{Το Διαδίκτυο
του Πολιτισμού: Ενσωματώνοντας Τεχνολογία RFID στο Μουσείο}. Κωδικός έργου
Τ2ΕΔΚ-02000. \url{https://cultureid.web.auth.gr/}}. Τα ειδικά συμπεράσματα που
εξαγάγαμε κατά τη μελέτη μας παρατίθενται στην ενότητα
\ref{subsection:02_01_05:01}.

Κατά τη διάρκεια διεξαγωγής της εφαρμογής της σχεδιασθείσας μεθοδολογίας μάς
έγινε φανερή η ύπαρξη σφαλμάτων εκτίμησης στάσης---δηλαδή της διαφοράς της
εκτίμησης για τη στάση του ρομπότ από την πραγματική του στάση, η οποία διαφορά
έγκειται στις συνθήκες του πεδίου εφαρμογής \ref{scope} και πιο συγκεκριμένα
λόγω της Παραδοχής \ref{ass:01_01} και της Παρατήρησης \ref{remark:observable}.

Από αυτό το σημείο και έπειτα αποφασίσαμε να στραφούμε αποκλειστικά στο πεδίο
της Ρομποτικής που έχει ως αντικείμενο την εκτίμηση της στάσης ενός ρομπότ με
βάση μετρήσεις δύο διαστάσεων που προέρχονται από έναν αισθητήρα---τον
αισθητήρα αποστάσεων τύπου lidar. Αυτή η εστίαση αποφασίσθηκε με βάση τα εξής
κριτήρια: (α) οι αισθητήρες αποστάσεων τύπου lidar φέρουν ακριβέστερες και
με μεγαλύτερο εύρος όρασης μετρήσεις σε σχέση με άλλους αισθητήρες που
χρησιμοποιούνται στο έργο της εκτίμησης της στάσης ενός ρομπότ, (β) οι
αισθητήρες αποστάσεων τύπου lidar που παρέχουν μετρήσεις δύο διαστάσεων έχουν
μικρότερο κόστος σε σχέση με αισθητήρες μετρήσεων τριών διαστάσεων, (γ) το έργο
της εκτίμησης της στάσης ενός ρομπότ με βάση έναν αισθητήρα αποστάσεων τύπου
lidar είναι περισσότερο δύσκολο από αυτό της εκτίμησης με βάση αισθητήρες
διαφόρων και πολλαπλών τύπων, (δ) οι αισθητήρες αποστάσεων τύπου lidar επέχουν
θέση αισθητήρων εκ των ων ουκ άνευ στη σημερινή εποχή, ενώ αναμένεται ότι η
αγορά αυτών θα αποκτά μεγαλύτερο όγκο όσο περνάνε τα χρόνια, και (ε) στο έργο
της παρατήρησης της τροχιάς ενός ρομπότ που χρησιμοποιεί αισθητήρα lidar για
την εκτίμηση της στάσης του στο χώρο καθώς αυτό κινείται κατά τη διάρκεια του
χρόνου, το πρόβλημα της εκτίμησης θεωρείται λυμένο επί της αρχής, και όχι με
γνώμονα την επίτευξη ``βέλτιστων" σφάλμάτων εκτίμησης.

Με βάση τα παραπάνω κριτήρια ξεκινήσαμε την έρευνα με γνώμονα την ανάγκη
επίτευξης σφαλμάτων στάσης μικρότερα από αυτά που παράγονται από τα καθιερωμένα
φίλτρα παρατήρησης της στάσης ενός ρομπότ. Στο κεφάλαιο
\ref{part:02:chapter:02} επιβεβαιώσαμε ότι υπάρχουν συνθήκες τέτοιες ώστε
επιπρόσθετες μέθοδοι δύνανται να μειώσουν το σφάλμα εκτίμησης στάσης ενός
φίλτρου σωματιδίων. Συγκεκριμένα δοκιμάσαμε τρεις υποθέσεις. Η πρώτη αφορά στην
ορθολογική σκέψη πως η εκτίμηση του φίλτρου οφείλει να εμφανίζει χαμηλότερα
σφάλματα εκτίμησης όταν μόνο τα βαρύτερα σωματίδια επιτρέπεται να ψηφίσουν ως
προς αυτήν. Η σκέψη αυτή θεμελιώνεται στην ίδια θεωρία του φίλτρου σωματιδίων,
καθώς το βάρος ενός σωματιδίου ποσοτικοποιεί την πιθανότητα παρατήρησης της
υφιστάμενης μέσω του αισθητήρα αποστάσεων μέτρησης από την εκτίμηση στάσης που
αυτό κωδικοποιεί. Τα πειραματικά στοιχεία επιβεβαιώνουν αυτήν την υπόθεση, όμως
μέχρι ενός σημείου: στο όριο, το σωματίδιο που εμφανίζει το μέγιστο βάρος
ανάμεσα στα υπόλοιπα του πληθυσμού εμφανίζει μεγαλύτερο σφάλμα από αυτό της
συλλογικής εκτίμησης του φίλτρου. Με βάση αυτό το αποτέλεσμα καταλήξαμε στον
ισχυρισμό πως το φίλτρο σωματιδίων δεν αποτελεί \textit{άθροισμα} υποθέσεων
εκτίμησης στάσης. Η δεύτερη υπόθεση αφορά στην ικανότητα της μεθόδου
ευθυγράμμισης πραγματικών με εικονικές σαρώσεις να παράγει εκτιμήσεις στάσης με
σφάλμα μικρότερο από την ίδια την εκτίμηση του φίλτρου, την οποία επίσης
επιβεβαιώσαμε. Η τρίτη υπόθεση αφορά στην ορθολογική σκέψη πως η εκτίμηση
στάσης που παράγεται από το σύστημα ευθυγράμμισης είναι μία εκτίμηση για την
οποία το ίδιο το φίλτρο δεν έχει γνώση, και πως κατα συνέπεια εάν εισαχθεί στον
πληθυσμό του θα έχει ευεργετικά αποτελέσματα ως προς το σφάλμα στάσης του.
Συγκεκριμένα υποθέσαμε πως η εισαγωγή της εκτίμησης του συστήματος
ευθυγράμμισης στον πληθυσμό του φίλτρου με τη μορφή πλειάδας σωματιδίων οφείλει
(α) να εμφανίζει χαμηλότερα σφάλματα εκτίμησης σε σχέση με την εισαγωγή μόνο
ενός σωματιδίου, και (β) να διατηρεί την ευρωστία του φίλτρου σε αποτυχίες της
μεθόδου ευθυγράμμισης σε σχέση με την επαναρχικοποίησή του γύρω από αυτήν. Τα
πειραματικά αποτελέσματα έδειξαν ότι για τις πειραματικές συνθήκες που
δοκιμασθηκαν οι δύο υποθέσεις επιβεβαιώνονται. Τα ειδικά συμπεράσματα που
εξαγάγαμε κατά τη μελέτη μας παρατίθενται στην ενότητα
\ref{subsection:02_02_05:01}.


Σε αυτό το σημείο είχε επιβεβαιωθεί ότι υπάρχουν συνθήκες στις οποίες η
ευθυγράμμιση πραγματικών μετρήσεων αισθητήρα lidar με εικονικές σαρώσεις που
έχουν εξαχθεί από το χάρτη του περιβάλλοντος στον οποίο λειτουργεί ο αισθητήρας
είναι ικανή να εκτιμήσει με ακρίβεια τη στάση του. Το πρώτο πρόβλημα που
συναντήσαμε εδώ, και το οποίο τροχιοδεικνύει την έρευνα που έπεται, είναι ότι
οι μέθοδοι ευθυγράμμισης πραγματικών μετρήσεων, οι οποίες εμφανίζουν
αποτελεσματικές επιδόσεις στο έργο της ευθυγράμμισης πραγματικών με εικονικές
σαρώσεις, απαιτούν τον υπολογισμό αντιστοιχίσεων ανάμεσα στις σαρώσεις εισόδου,
και τον χειροκίνητο καθορισμό παραμέτρων που αφορούν και σε αυτή τη διαδικασία.
Ο καθορισμός αυτών των παραμέτρων δεν έχει καθολική εγκυρότητα και δεν είναι
διαισθητικός. Παράλληλα, το αποτέλεσμα της ευθυγράμμισης αυτών των μεθόδων
εμφανίζει σημαντικές διαφορές ανάλογα με το επίπεδο διαταραχών των σαρώσεων
εισόδου. Συνεπώς από αυτό το σημείο και πέρα επιζητήσαμε την αντικατάσταση
αυτών των μεθόδων στο έργο της ευθυγράμμισης πραγματικών με εικονικές σαρώσεις
από μεθόδους που δεν υπολογίζουν αντιστοιχίσεις ανάμεσα στα διανύσματα εισόδου
τους, και θέσαμε τον ισχυρισμό ότι μία αναγκαία συνθήκη είναι η πανοραμικότητα
του εύρους όρασης του αισθητήρα μετρήσεων lidar.

Στο κεφάλαιο \ref{part:02:chapter:03} επιζητήσαμε να δοκιμάσουμε επί της αρχής
την υπόθεση ότι είναι δυνατή η εφεύρεση μεθόδου ευθυγράμμισης πραγματικών
πανοραμικών μετρήσεων με εικονικές πανοραμικές σαρώσεις που δεν χρησιμοποιεί
τον υπολογισμό αντιστοιχίσεων ανάμεσα στα διανύσματα εισόδου για να φέρει εις
πέρας το έργο της ευθυγράμμισης. Προς αυτόν τον στόχο σχεδιάσαμε μία τέτοια
μέθοδο μεταγγίζοντας μία μέθοδο από το πεδίο της μηχανικής όρασης, η οποία
διαθέτει μικρότερο σύνολο παραμέτρων από τις αντίστοιχες μεθόδους ευθυγράμμισης
που υπολογίζουν αντιστοιχίσεις, και των οποίων ο καθορισμός είναι διαισθητικός.
Καθώς επιζητήσαμε τη δοκιμή της παραπάνω υπόθεσης επί της αρχής, δοκιμάσαμε το
σχεδιασθέν σύστημα εκτίμησης της στάσης πανοραμικού αισθητήρα lidar στα
συμφραζόμενα εκτίμησης της στάσης του βάσει καθολικής αβεβαιότητος, και
ταυτόχρονα δοκιμάσαμε στο ίδιο έργο το ίδιο σύστημα, το οποίο όμως χρησιμοποιεί
για το έργο της ευθυγράμμισης μία μέθοδο που χρησιμοποιεί τον υπολογισμό
αντιστοιχίσεων. Τα πειραματικά αποτελέσματα έδειξαν ότι τα δύο συστήματα δεν
έχουν ευδιάκριτες διαφορές ως προς το τελικό σφάλμα εκτίμησης, όμως η ικανότητα
διάκρισης αληθώς ορθών στάσεων από ψευδώς ορθών του πρώτου είναι μεγαλύτερη από
αυτή του δεύτερου, κάτι το οποίο οφείλεται στην αν-ικανότητα του δεύτερου να
διορθώνει ``μεγάλα" αρχικά σφάλματα εκτίμησης θέσης. Τα ειδικά συμπεράσματα που
εξαγάγαμε κατά τη μελέτη μας παρατίθενται στην ενότητα
\ref{subsection:02_03_05:01}.


Σε αυτό το σημείο είχαμε επιβεβαιώσει την αρχική μας υπόθεση, ότι δηλαδή είναι
επί της αρχής δυνατόν να αντικατασταθεί ο μηχανισμός υπολογισμού αντιστοιχίσεων
σε μεθόδους ευθυγράμμισης πραγματικών με εικονικές πανοραμικές σαρώσεις από
μηχανισμούς κλειστού τύπου, χωρίς βλάβη της ακρίβειας της εκτίμησης στάσης.
Καθώς το σχεδιασθέν σύστημα δεν εκτελείτο σε πραγματικό χρόνο ως προς το ρυθμό
ανανέωσης εκτιμήσεων από μία βασική μέθοδο παρατήρησης της στάσης ενός ρομπότ
καθώς αυτό κινείται μέσα στο χώρο και κατά τη διάρκεια του χρόνου, επιζητήσαμε
να προωθήσουμε την προσπάθειά μας σε αυτά τα συμφραζόμενα. Με αυτόν τον τρόπο
θα ήταν δυνατή η ελάττωση του σφάλματος εκτίμησης μεθόδων παρατήρησης της
στάσης ενός ρομπότ με τους τρόπους που εισαγάγαμε στο κεφάλαιο
\ref{part:02:chapter:02}, με σχετικά μεγαλύτερη ευρωστία όμως στον αυξημένο
θόρυβο μέτρησης που φέρουν διαθέσιμοι εμπορικά πανοραμικοί αισθητήρες, και στα
κατά συνέπεια αυξημένα σφάλματα εκτίμησης της βασικής μεθόδου παρατήρησης της
στάσης.

Για αυτούς τους λόγους στο κεφάλαιο \ref{part:02:chapter:04} αναπτύξαμε μία
τριλογία μεθόδων ευθυγράμμισης πραγματικών με εικονικές πανοραμικές σαρώσεις
που εκτελούνται σε πραγματικό χρόνο και που δεν χρησιμοποιούν τον υπολογισμό
αντιστοιχίσεων για να φέρουν εις πέρας το έργο της ευθυγράμμισης. Οι μέθοδοι
αυτές διακρίνονται με βάση την υποκείμενη μέθοδο εκτίμησης του προσανατολισμού
του αισθητήρα. Κατά τη διάρκεια του σχεδιασμού των τριών μεθοδων και του
συνολικού συστήματος ευθυγράμμισης σκοντάψαμε σε προβλήματα εγγενή των
χαρακτηριστικών του αισθητήρα και της γεωμετρικής φύσης του προβλήματος
ευθυγράμμισης και ανακάμψαμε εφευρίσκοντας μεθόδους και δίνοντας λύσεις που
έχουν ως στόχο την μη βλάβη της ευθυγράμμισης και την εκτέλεση της σε
πραγματικό χρόνο. Δοκιμάσαμε την επίδοση των τριών μεθόδων σε πειραματικές
συνθήκες που εμφανίζονται στην πράξη με διαθέσιμους εμπορικά πανοραμικούς
αισθητήρες και τα αποτελέσματα έδειξαν ότι οι τρεις μέθοδοι εμφανίζουν
μεγαλύτερη ευρωστία στο θόρυβο μέτρησης και τη διαφθορά του χάρτη του
περιβάλλοντος από αντίστοιχες μεθόδους της βιβλιογραφίας ως προς το σφάλμα
εκτίμησης και το ποσοστό των περιπτώσεων όπου το σφάλμα εκτίμησης μειώνεται
μετά την εφαρμογή τους. Τα ειδικά συμπεράσματα που εξαγάγαμε κατά τη μελέτη
μας παρατίθενται στην ενότητα \ref{subsection:02_04_07:01}.

Σε αυτό το σημείο η εργασία μας θα βρισκόταν στο τέλος της εάν δεν
παρατηρούσαμε πως με έναν απλό μετασχηματισμό της μεθόδου που υλοποιήσαμε και
των εισόδων της η ίδια θα ήταν σε θέση να επιλύσει το γενικότερο και
απαιτητικότερο πρόβλημα της ευθυγράμμισης πραγματικών δισδιάτατων πανοραμικών
μετρήσεων. Στο κεφάλαιο \ref{part:02:chapter:05} εισαγάγαμε την πρώτη μέθοδο
που το επιλύει σε πραγματικό χρόνο ως προς το ρυθμό ανανέωσης μετρήσεων τυπικών
αισθητήρων, χωρίς τον υπολογισμό αντιστοιχίσεων, με επίδοση ακρίβειας και
ευρωστίας ως προς θόρυβο μέτρησης και απόστασης στάσεων ανώτερη από αυτές των
μεθόδων της βιβλιογραφίας. Τα ειδικά συμπεράσματα που εξαγάγαμε κατά τη μελέτη
μας παρατίθενται στην ενότητα \ref{section:02_05_05}.
