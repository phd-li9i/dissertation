Στόχος του παρόντος κεφαλαίου είναι η επίλυση του προβλήματος \ref{prob:02_05}:

\begin{bw_box}
\begin{customproblem}{Π4}
  \label{prob:02_05}
  Έστω ένα ρομπότ κινητής βάσης του πεδίου εφαρμογής \ref{scope}, ικανό να
  κινείται στο επίπεδο $x-y$, εξοπλισμένο με έναν οριζόντια τοποθετημένο
  αισθητήρα lidar μετρήσεων δύο διαστάσεων που εκπέμπει $N_s$ ακτίνες. Έστω
  επίσης ότι τα ακόλουθα είναι διαθέσιμα:
  \begin{itemize}
    \item Μια δισδιάστατη μέτρηση $\mathcal{S}_0$
    \item Μια δισδιάστατη μέτρηση $\mathcal{S}_1$
  \end{itemize}
\end{customproblem}
Τότε ο στόχος είναι η εκτίμηση της στάσης από την οποία συνελήφθη η μέτρηση
$\mathcal{S}_1$ σε σχέση με τη στάση από την οποία συνελήφθη η μέτρηση
$\mathcal{S}_1$ δεδομένων των κάτωθι παραδοχών και περιορισμών:
\begin{itemize}
  \item Το γωνιακό εύρος του αισθητήρα lidar ισούται με $\lambda = 2\pi$
  \item Η επίλυση του προβλήματος γίνεται χωρίς τον υπολογισμό αντιστοιχίσεων
        ανάμεσα στις εισόδους της μεθόδου επίλυσης
  \item Η εκτέλεση της επίλυσης του προβλήματος πρέπει γίνεται σε χρόνο που να
        συμβαδίζει με το ρυθμό ανανέωσης των μετρήσεων που παρέχει ο φυσικός
        αισθητήρας lidar
\end{itemize}

\end{bw_box}

Στην ακόλουθη ενότητα παρουσιάζεται το σκεπτικό της δυνατότητας επίλυσης του
προβλήματος \ref{prob:02_05} μέσω της επίλυσης του προβλήματος ευθυγράμμισης
πραγματικών με εικονικές σαρωσεις, δηλαδή μέσω της επίλυσης του προβλήματος
\ref{prob:02_04}. Στην ίδια ενότητα εκτίθεται η δομή του προτεινόμενου
αλγορίθμου, ο οποίος κατά κύριο λόγο βασίζεται στον Αλγόριθμο
\ref{alg:algorithm_fsmsm}, και συνεπώς στο σκεπτικό και τις αναλύσεις του
προηγούμενου κεφαλαίου. Η προτεινόμενη μέθοδος δοκιμάζεται ως προς την
ικανότητα επίλυσης του προβλήματος στο οποίο στοχεύει στην ενότητα
\ref{section:02_05_03}, όπου η επίδοσή της συγκρίνεται με καθιερωμένες και
νεότερες μεθόδους επίλυσης του ίδιου προβλήματος. Στην επόμενη ενότητα
(\ref{section:02_05_04}) παρέχονται κύρια χαρακτηριστικά γνωρίσματα και
βασικοί περιορισμοί της κατασκευαστείσας μεθόδου, ενώ η τελευταία ενότητα
ανακεφαλαιώνει τον παρόν κεφάλαιο.
