Στο παρόν κεφάλαιο επιζητήσαμε την επίλυση του προβλήματος ευθυγράμμισης
δισδιάστατων πανοραμικών σαρώσεων, χωρίς τον υπολογισμό αντιστοιχίσεων ανάμεσα
στις ακτίνες των εισόδων, μέσω του μετασχηματισμού των μεθόδων ευθυγράμμισης
πραγματικών με εικονικές σαρώσεις που εισαγάγαμε στο προηγούμενο κεφάλαιο.
Προς αυτόν τον σκοπό μελετήσαμε τη διαφορά ανάμεσα στα δύο προβλήματα
ευθυγράμμισης και, σκεπτόμενοι ορθολογικα, καταλήξαμε στον ισχυρισμό πως το
πρώτο πρόβλημα μπορεί να λυθεί μέσω της λύσης του δεύτερου. Αυτός ο ισχυρισμός
στηρίζεται στην παρατήρηση-υπόθεση ότι η απόσταση μεταξύ των θέσεων από τις
οποίες συλλαμβάνονται δύο διαδοχικές πραγματικές πανοραμικές σαρώσεις είναι
τέτοια για ρομπότ του πεδίου εφαρμογής \ref{scope} ώστε η κάθεμία να αποτελεί
προσέγγιση του μοντέλου του τοπικού περιβάλλοντος στο οποίο κινείται ο
αισθητήρας, δηλαδή του χάρτη του, την ύπαρξη του οποίου προϋποθέτει η επίλυση
του προβλήματος του προηγούμενου κεφαλαίου.

Καθώς το πρόβλημα του παρόντος κεφαλαίου έχει αυστηρότερες απαιτήσεις ως προς
το χρόνο επίλυσής του, υιοθετήσαμε την πιο γρήγορη μέθοδο του προηγούμενου
κεφαλαίου, και δοκιμάσαμε τη επίδοσή της έναντι της επίδοσης καθιερωμένων και
νεότερων μεθόδων ευθυγράμμισης, μαζί με την ισχύ του ισχυρισμού-υπόθεσης της
παραπάνω παραγράφου σε ό,τι αφορά την προτεινόμενη μέθοδο. Με βάση τα
αποτελέσματα της πειραματικής διαδικασίας συνάγουμε τα εξής συμπεράσματα:

\begin{itemize}
  \item Τα σφάλματα εκτίμησης θέσης και προσανατολισμού της σχεδιασθείσας
        μεθόδου (FSM) εμφανίζουν το χαμηλότερο μέσο όρο και τη χαμηλότερη
        διάμεση τιμή ανάμεσα στις μεθόδους που είναι ικανές για εκτέλεση σε
        πραγματικό χρόνο, για κάθε δοκιμασθείσα διαμόρφωση απόστασης θέσεων και
        προσανατολισμών των στάσεων από τις οποίες συλλαμβάνονται
        σαρώσεις-είσοδοι, και κάθε επίπεδο διαταραχών μέτρησης (σχήματα
        \ref{fig:02_05_03:02:01}, \ref{fig:02_05_03:02:02}, και
        \ref{fig:02_05_03:02:03})
  \item Η διάμεση τιμή των σφαλμάτων εκτίμησης θέσης και προσανατολισμού
        της FSM είναι αναλλοίωτη της απόστασης των στάσεων από τις οποίες
        συλλαμβάνονται σαρώσεις για ένα δεδομένο επίπεδο διαταραχών μέτρησης
  \item Η FSM εκτελείται σε πραγματικό χρόνο για αριθμό ακτίνων $N_s = 360$
        (σχήμα \ref{fig:02_05_03:02:03})
  \item Τα σφάλματα εκτίμησης θέσης και προσανατολισμού της FSM είναι
        ανεξάρτητα της διαφοράς προσανατολισμού των στάσεων από τις οποίες
        συλλαμβάνονται οι είσοδοί της (σχήματα
        \ref{fig:appendix_05_01:01}-\ref{fig:appendix_05_01:12})
  \item Το σφάλμα εκτίμησης προσανατολισμού της FSM είναι ανεξάρτητο της
        διαφοράς θέσης των στάσεων από τις οποίες συλλαμβάνονται οι είσοδοί της,
        τουλάχιστον έως αποστάσεις θέσεων με μέτρο μικρότερο ή ίσο από
        $\sqrt{2}\cdot 0.20$ m
  \item Το σφάλμα εκτίμησης θέσης της FSM εξαρτάται από τη διαφορά θέσης των
        στάσεων από τις οποίες συλλαμβάνονται οι είσοδοί της
  \item Σε συνθήκες όπου μέρη του περιβάλλοντος βρίσκονται εκτός του μεγίστου
        εύρους του αισθητήρα, η μέθοδος FSM, η οποία δεν υπολογίζει
        αντιστοιχίσεις, και συνεπώς δεν είναι ικανή να εξετάσει τμηματικά τις
        εισόδους της όπως οι μέθοδοι που επιλύουν την ευθυγράμμιση μέσω
        αντιστοιχίσεων, εμφανίζει τουλάχιστον ισοδύναμη επίδοση ως προς τα
        σφάλματα εκτίμησης με την καλύτερη τέτοια μέθοδο ευθυγράμμισης, έως
        ότου και δύο στις δέκα μετρήσεις ακτίνων να είναι απούσες νοήματος
        (σχήμα \ref{fig:02_05_04:04})
\end{itemize}
