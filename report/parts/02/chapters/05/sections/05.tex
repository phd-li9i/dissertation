Στο παρόν κεφάλαιο επιζητήσαμε την επίλυση του προβλήματος ευθυγράμμισης
δισδιάστατων πανοραμικών σαρώσεων, χωρίς τον υπολογισμό αντιστοιχίσεων ανάμεσα
στις ακτίνες των εισόδων, μέσω του μετασχηματισμού των μεθόδων ευθυγράμμισης
πραγματικών με εικονικές σαρώσεις που εισαγάγαμε στο προηγούμενο κεφάλαιο.
Προς αυτόν τον σκοπό μελετήσαμε τη διαφορά ανάμεσα στα δύο προβλήματα
ευθυγράμμισης και, σκεπτόμενοι ορθολογικα, καταλήξαμε στον ισχυρισμό πως το
πρώτο πρόβλημα μπορεί να λυθεί μέσω της λύσης του δεύτερου. Αυτός ο ισχυρισμός
στηρίζεται στην παρατήρηση-υπόθεση ότι η απόσταση μεταξύ των θέσεων από τις
οποίες συλλαμβάνονται δύο διαδοχικές πραγματικές πανοραμικές σαρώσεις είναι
τέτοια για ρομπότ του πεδίου εφαρμογής \ref{scope} ώστε η κάθεμία να αποτελεί
προσέγγιση του μοντέλου του τοπικού περιβάλλοντος στο οποίο κινείται ο
αισθητήρας, δηλαδή του χάρτη του, την ύπαρξη του οποίου προϋποθέτει η επίλυση
του προβλήματος του προηγούμενου κεφαλαίου, και την απουσία του αυτό του
παρόντος.

Καθώς το πρόβλημα του παρόντος κεφαλαίου έχει αυστηρότερες απαιτήσεις ως προς
το χρόνο επίλυσής του, υιοθετήσαμε την πιο γρήγορη μέθοδο του προηγούμενου
κεφαλαίου, και δοκιμάσαμε τη επίδοσή της έναντι της επίδοσης καθιερωμένων και
νεότερων μεθόδων ευθυγράμμισης, μαζί με την ισχύ του ισχυρισμού-υπόθεσης της
παραπάνω παραγράφου σε ό,τι αφορά την προτεινόμενη μέθοδο. Με βάση τα
αποτελέσματα της πειραματικής διαδικασίας συνάγουμε τα εξής συμπεράσματα:

\begin{itemize}
  \item Τα σφάλματα εκτίμησης θέσης και προσανατολισμού της σχεδιασθείσας
        μεθόδου FSM εμφανίζουν το χαμηλότερο μέσο όρο και τη χαμηλότερη
        διάμεση τιμή ανάμεσα στις μεθόδους που είναι ικανές για εκτέλεση σε
        πραγματικό χρόνο, για κάθε δοκιμασθείσα διαμόρφωση απόστασης θέσεων και
        προσανατολισμών των στάσεων από τις οποίες συλλαμβάνονται
        σαρώσεις-είσοδοι, και κάθε επίπεδο διαταραχών μέτρησης που αναφέρεται
        για εμπορικά διαθέσιμους αισθητήρες απόστασης (σχήματα
        \ref{fig:02_05_03:02:01}, \ref{fig:02_05_03:02:02}, και
        \ref{fig:02_05_03:02:03})
  \item Η διάμεση τιμή των σφαλμάτων εκτίμησης θέσης και προσανατολισμού
        της FSM είναι αναλλοίωτη της απόστασης των στάσεων από τις οποίες
        συλλαμβάνονται σαρώσεις για ένα δεδομένο επίπεδο διαταραχών μέτρησης
  \item Η FSM εκτελείται σε πραγματικό χρόνο για αριθμό ακτίνων $N_s = 360$
        (σχήμα \ref{fig:02_05_03:02:03})
  \item Τα σφάλματα εκτίμησης θέσης και προσανατολισμού της FSM είναι
        ανεξάρτητα της διαφοράς προσανατολισμού των στάσεων από τις οποίες
        συλλαμβάνονται οι είσοδοί της (σχήματα
        \ref{fig:02_05_03:02:04}-\ref{fig:02_05_03:02:07})
  \item Το σφάλμα εκτίμησης προσανατολισμού της FSM είναι ανεξάρτητο της
        διαφοράς θέσης των στάσεων από τις οποίες συλλαμβάνονται οι είσοδοί της,
        τουλάχιστον έως αποστάσεις θέσεων με μέτρο μικρότερο ή ίσο από
        $\sqrt{2}\cdot 0.20$ m
  \item Το σφάλμα εκτίμησης θέσης της FSM εξαρτάται από τη διαφορά θέσης των
        στάσεων από τις οποίες συλλαμβάνονται οι είσοδοί της, και είναι
        συγκλίνον τουλάχιστον για αποστάσεις θέσεων με μέτρο μικρότερο ή ίσο από
        $\sqrt{2}\cdot 0.20$ m
  \item Η μέθοδος FSM είναι περισσότερο έυρωστη ως προς τα σφάλματα εκτίμησης
        θέσης και προσανατολισμού από τις μεθόδους PLICP, NDT, FastGICP,
        FastVGICP, και NDT-PSO όσο αυξάνει ο θόρυβος μέτρησης
  \item Τα σφάλματα εκτίμησης θέσης της μεθόδου PLICP είναι τα βέλτιστα ανάμεσα
        στις εκδόσεις των μεθόδων ICP και NDT όταν το μέτρο της αρχικής
        διαφοράς θέσης δεν ξεπερνά τα $\sqrt{2}\cdot 0.20$ m (σχήμα
        \ref{fig:02_05_03:02:01}). Όσο αυξάνει ο θόρυβος μέτρησης τόσο
        χειρότερη γίνεται η επίδοσή της όσο αυξάνει η αρχική απόσταση μεταξύ
        των δύο στάσεων κατά προσανατολισμό (σχήματα \ref{fig:02_05_03:02:04}
        και \ref{fig:02_05_03:02:06})
  \item Τα σφάλματα εκτίμησης θέσης της μεθόδου NDT-PSO εξαρτώνται σε μεγαλύτερο
        βαθμό από την αρχική διαφορά στάσεων κατά προσανατολισμό σε σχέση με τις
        εκδόσεις των μεθόδων ICP και NDT, και επιδεικνύουν τις χαμηλότερές
        τιμές τους όταν ο αισθητήρας κινείται σε σχετικά ευθείες τροχιές
  \item Το ίδιο πρότυπο παρατηρείται για τα σφάλματα εκτίμησης του
        προσανατολισμού της ιδίας μεθόδου και της PLICP (σχήματα
        \ref{fig:02_05_03:02:05} και \ref{fig:02_05_03:02:07})
  \item Τα σφάλματα εκτίμησης προσανατολισμού των μεθόδων NDT, FastGICP και
        FastVGICP εμφανίζουν απότομες αυξήσεις όταν η αρχική απόσταση των
        στάσεων σύλληψης μετρήσεων απέχουν κατά προσανατολισμό $\simeq \pi/6$
        rad
  \item Σε συνθήκες όπου μέρη του περιβάλλοντος βρίσκονται εκτός του μεγίστου
        εύρους του αισθητήρα, η μέθοδος FSM, η οποία δεν υπολογίζει
        αντιστοιχίσεις, και συνεπώς δεν είναι ικανή να εξετάσει τμηματικά τις
        εισόδους της όπως οι μέθοδοι που επιλύουν την ευθυγράμμιση μέσω
        αντιστοιχίσεων, εμφανίζει ως προς το σφάλμα εκτίμησης
        προσανατολισμού επίδοση συγκρίσιμη με την ονομαστική έως ότου οκτώ
        στις δέκα ακτίνες του αισθητήρα παύουν να φέρουν ουσιαστική πληροφορία.
        Ως προς το σφάλμα εκτίμησης θέσης, αυτό είναι αντιστρόφως ανάλογο του
        ποσοστού των τελευταίων (σχήμα \ref{fig:02_05_04:04})
\end{itemize}
