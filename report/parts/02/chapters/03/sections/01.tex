Σε αυτό το κεφάλαιο επιζητούμε τη λύση στο πρόβλημα του εντοπισμού της στάσης
ενός ρομπότ του πεδίου εφαρμογής \ref{scope} βάσει καθολικής αβεβαιότητας για
αυτήν. Πιο συγκεκριμένα επιζητούμε τη λύση της παθητικής έκδοσης του προβλήματος
\ref{prob:02_03:the_problem} (Ορισμός \ref{definition:01_01_02_02:01}), βάσει
των Παραδοχών \ref{assumption:02_03_01:01}, \ref{assumption:02_03_01:02}, και
\ref{assumption:02_03_01:03}, δεδομένων των Παρατηρήσεων
\ref{remark:01_01_02_02:01} και \ref{remark:01_01_02_02:02}.

\begin{bw_box}
\begin{customproblem}{Π2}
  \label{prob:02_03:the_problem}
  Έστω ένα ρομπότ κινητής βάσης του πεδίου εφαρμογής \ref{scope}, εξοπλισμένο με
  έναν οριζόντια τοποθετημένο αισθητήρα lidar μετρήσεων δύο διαστάσεων που
  εκπέμπει $N_s$ ακτίνες. Έστω επίσης ότι τα ακόλουθα είναι διαθέσιμα ή
  ευσταθούν:
  \begin{itemize}
    \item Ο χάρτης $\bm{M}$ του περιβάλλοντος στο οποίο βρίσκεται το ρομπότ
    \item Μια δισδιάστατη μέτρηση $\mathcal{S}_R$, που λαμβάνεται από
          την---άγνωστη και αναζητούμενη---στάση $\bm{p}(\bm{l},\theta)$,
          $\bm{l} = (x,y)$
  \end{itemize}
\end{customproblem}
Τότε, δεδομένων των Παρατηρήσεων \ref{remark:01_01_02_02:01} και
\ref{remark:01_01_02_02:02} ο στόχος είναι η εκτίμηση της στάσης $\bm{p}$ του
ρομπότ στο σύστημα αναφοράς του χάρτη $\bm{M}$.
\end{bw_box}

\begin{bw_box}
  \begin{assumption}
    \label{assumption:02_03_01:01}
    Συνέπεια του Ισχυρισμού \ref{claim:02_03:01} είναι ότι το γωνιακό εύρος του
    αισθητήρα lidar ισούται με $\lambda = 2\pi$ rad.
  \end{assumption}
\end{bw_box}

\begin{bw_box}
  \begin{assumption}
    \label{assumption:02_03_01:02}
    Η λύση του προβλήματος \ref{prob:02_03:the_problem} δίνεται μέσω
    ευθυγράμμισης μετρήσεων με σαρώσεις χάρτη.
  \end{assumption}
\end{bw_box}

\begin{bw_box}
  \begin{assumption}
    \label{assumption:02_03_01:03}
    Η επίλυση του προβλήματος στο οποίο αφορά η Παραδοχή
    \ref{assumption:02_03_01:02} δίνεται χωρίς τον υπολογισμό αντιστοιχίσεων
    ανάμεσα στις εισόδους της μεθόδου επίλυσης.
  \end{assumption}
\end{bw_box}

Το κεφάλαιο αυτό συνεχίζει με την επισκόπηση των μεθόδων επίλυσης του
προβλήματος \ref{prob:02_03:the_problem} που απαντώνται στη βιβλιογραφία, και
τις χρήσεις που έχει βρει στον κλάδο της ρομποτικής η μέθοδος που προέρχεται
από τον κλάδο της υπολογιστικής όρασης (ενότητα \ref{subsec:01_01_02_7}) την
οποία χρησιμοποιούμε για την ευθυγράμμιση πραγματικών με εικονικές σαρώσεις δύο
διαστάσεων.

Στην ενότητα \ref{section:02_03_03} αναλύεται ο τρόπος με τον οποίο η τελευταία
μπορεί να προσαρμοσθεί στα συμφραζόμενα επίλυσης του προβλήματος
\ref{prob:02_03:the_problem}. Η προτεινόμενη μέθοδος επίλυσης του προβλήματος
της εύρεσης της στάσης ενός ρομπότ βάσει καθολικής αβεβαιότητας χρησιμοποιεί
την ως άνω μέθοδο για την εκτίμηση του προσανατολισμού του (ενότητα
\ref{subsection:02_03_03:02}), και για την εκτίμηση της θέσης του μία μέθοδο
ανάδρασης που βασίζεται στη διαφορά της γεωμετρίας μίας πραγματικής και μίας
εικονικής σάρωσης (ενότητα \ref{subsection:02_03_03:03}). Οι δύο μέθοδοι
εκτίμησης του προσανατολισμού και της θέσης του ρομπότ εκτελούνται για ένα
σύνολο υποθέσεων στάσης που διασπείρονται στον χάρτη, και η διάκριση της αληθώς
ορθής υπόθεσης στάσης ανάμεσα στο σύνολο των υποθέσεων πραγματοποιείται μέσω
αξιόπιστων μετρικών ομοιότητας που εξάγονται από τη μέθοδο εκτίμησης του
προσανατολισμού (ενότητα \ref{subsection:02_03_03:04}).

Η ενότητα \ref{section:02_03_04} ελέγχει την ευστάθεια της υπόθεσης
\ref{hypothesis:02_03:01} στα συμφραζόμενα του προβλήματος
\ref{prob:02_03:the_problem} μέσω πειραματικής διαδικασίας που εκτελείται σε
προσομοιωμένα και πραγματικά περιβάλλοντα. Στην ενότητα \ref{section:02_03_05}
παρουσιάζουμε τα συμπεράσματα της μελέτης μας, και τις αιτίες για περαιτέρω
έρευνα στο πεδίο της ευθυγράμμισης δισδιάστατων μετρήσεων αισθητήρα lidar με
σαρώσεις χάρτη.
