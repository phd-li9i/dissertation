\begin{bw_box}
\begin{customproblem}{Π2}
  \label{prob:02_03:the_problem}
  Έστω ένα ρομπότ κινητής βάσης του πεδίου εφαρμογής \ref{scope}, εξοπλισμένο με
  έναν οριζόντια τοποθετημένο αισθητήρα lidar μετρήσεων δύο διαστάσεων που
  εκπέμπει $N_s$ ακτίνες. Έστω επίσης ότι τα ακόλουθα είναι διαθέσιμα ή
  ευσταθούν:
  \begin{itemize}
    \item Ο χάρτης $\bm{M}$ του περιβάλλοντος στο οποίο κινείται το ρομπότ
    \item Μια δισδιάστατη μέτρηση $\mathcal{S}_R$, που λαμβάνεται από
          την---άγνωστη και αναζητούμενη---στάση $\bm{p}(\bm{l},\theta)$,
          $\bm{l} = (x,y)$
  \end{itemize}
\end{customproblem}
Τότε, δεδομένων των παρατηρήσεων \ref{remark:01_01_02_02:01} και
\ref{remark:01_01_02_02:02} ο στόχος είναι η εκτίμηση της στάσης $\bm{p}$ του
ρομπότ στο σύστημα αναφοράς του χάρτη $\bm{M}$.
\end{bw_box}

\begin{bw_box}
  \begin{assumption}
    \label{assumption:02_03_01:01}
    Το γωνιακό εύρος του αισθητήρα lidar είναι $360^\circ$.
  \end{assumption}
\end{bw_box}

\begin{bw_box}
  \begin{assumption}
    \label{assumption:02_03_01:02}
    Η λύση του προβλήματος \ref{prob:02_03:the_problem} πρέπει δοθεί μέσω
    ευθυγράμμισης μετρήσεων με σαρώσεις χάρτη.
  \end{assumption}
\end{bw_box}

\begin{bw_box}
  \begin{assumption}
    \label{assumption:02_03_01:03}
    Η επίλυση του προβλήματος της παραδοχής \ref{assumption:02_03_01:02}
    πρέπει να δοθεί χωρίς τον υπολογισμό αντιστοιχίσεων ανάμεσα στις εισόδους
    της μεθόδου επίλυσης.
  \end{assumption}
\end{bw_box}
