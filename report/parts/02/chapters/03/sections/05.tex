%%%%%%%%%%%%%%%%%%%%%%%%%%%%%%%%%%%%%%%%%%%%%%%%%%%%%%%%%%%%%%%%%%%%%%%%%%%%%%%%
\subsection{Συμπεράσματα κεφαλαίου}
\label{subsection:02_03_05:01}

Σε αυτό το κεφάλαιο μάς απασχόλησε η δημιουργία μεθόδου ευθυγράμμισης
πραγματικών με εικονικές δισδιάστατες σαρώσεις, η οποία δεν υπολογίζει
αντιστοιχίσεις μεταξύ των ακτίνων των εισόδων, με αιτία τα ευρήματά μας στο
προηγούμενο κεφάλαιο (ενότητα \ref{subsection:02_02_05:02}). Δοκιμάσαμε την
επίδοσή της όσο αφορά στην επίλυση του προβλήματος της εύρεσης της στάσης ενός
οχήματος βάσει καθολικής αβεβαιότητας, και, αντιπαραβάλλοντας την με αυτήν του
αλγορίθμου που χρησιμοποιήσαμε στο προηγούμενο κεφάλαιο---τον οποίο επιζητούμε
να αντικαταστήσουμε λόγω του μηχανισμού υπολογισμού αντιστοιχίσεων που
χρησιμοποιεί---συμπεραίνουμε ότι οι δύο μέθοδοι δεν εμφανίζουν διακριτή διαφορά
όσο αφορά στα σφάλματα εκτιμήσεων που εξάγουν.

Πιο συγκεκριμένα, το προτεινόμενο σύστημα εμφανίζει χαμηλότερους ρυθμούς
αναφοράς ψευδώς θετικών στάσεων σε σχέση με το σύστημα που προτάθηκε για λύση
του προβλήματος εκτίμησης στάσης βάσει καθολικής αβεβαιότητος στην ενότητα
\ref{subsection:02_02_05:02} και δοκιμάστηκε στην πειραματική διαδικασία της
προηγούμενης ενότητας. Δεδομένου ότι στα δύο συστήματα δόθηκε ως όρισμα ο ίδιος
αριθμός υποθέσεων, σε συνέπεια με τα ευρήματα της βιβλιογραφίας (π.χ.
\cite{Olson2009a,bernreiter2021phaser}), καταλήγουμε ότι τα σφάλματα στάσης
εξόδου (α) του (PL)ICP εξαρτώνται από την απόσταση των θέσεων από τις οποίες
συλλήφθησαν οι είσοδοί του, και (β) του FMIC, εάν εξαρτώνται από την απόσταση
των θέσεων σύλληψής τους, τότε εξαρτώνται σε μικρότερο βαθμό σε σχέση με τον
PLICP.

Ως προς το χρόνο εκτέλεσης, αυτός του τελευταίου κλιμακώνεται με μεγαλύτερο
ρυθμό ως προς τον αριθμό των ακτίνων που εκπέμπει ο αισθητήρας lidar σε
σύγκριση με αυτόν του προτεινόμενου συστήματος. Σε χαμηλές πληθικότητες ακτίνων
η προτεινόμενη μέθοδος εμφανίζει μεγαλύτερους χρόνους εκτέλεσης από τον PLICP,
αλλά, δεδομένης της παρατήρησης της παραπάνω παραγράφου, μόνο ένας μεγαλύτερος
αριθμός υποθέσεων θα καθιστούσε τον PLICP συγκρίσιμο ως προς τα ποσοστά ψευδώς
θετικών αποκρίσεων, ο οποίος κατά συνέπεια θα αύξανε τον χρόνο εκτέλεσής του.

%%%%%%%%%%%%%%%%%%%%%%%%%%%%%%%%%%%%%%%%%%%%%%%%%%%%%%%%%%%%%%%%%%%%%%%%%%%%%%%%
\subsection{Αιτίες περαιτέρω έρευνας}
\label{subsection:02_03_05:02}
