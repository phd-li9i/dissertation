%%%%%%%%%%%%%%%%%%%%%%%%%%%%%%%%%%%%%%%%%%%%%%%%%%%%%%%%%%%%%%%%%%%%%%%%%%%%%%%%
\subsection{Συμπεράσματα κεφαλαίου}
\label{subsection:02_03_05:01}

Σε αυτό το κεφάλαιο μάς απασχόλησε η δημιουργία μεθόδου ευθυγράμμισης
πραγματικών με εικονικές δισδιάστατες σαρώσεις, η οποία δεν υπολογίζει
αντιστοιχίσεις μεταξύ των ακτίνων των εισόδων, με αιτία τα ευρήματά μας στο
προηγούμενο κεφάλαιο (ενότητα \ref{subsection:02_02_05:02}). Δοκιμάσαμε την
επίδοσή της όσο αφορά στην επίλυση του προβλήματος της εύρεσης της στάσης ενός
οχήματος βάσει καθολικής αβεβαιότητας, και, αντιπαραβάλλοντας την με αυτήν του
αλγορίθμου που χρησιμοποιήσαμε στο προηγούμενο κεφάλαιο---τον οποίο επιζητούμε
να αντικαταστήσουμε λόγω του μηχανισμού υπολογισμού αντιστοιχίσεων που
χρησιμοποιεί---συμπεραίνουμε ότι οι δύο μέθοδοι δεν εμφανίζουν διακριτή διαφορά
όσο αφορά στα σφάλματα εκτιμήσεων που εξάγουν.

Πιο συγκεκριμένα, το προτεινόμενο σύστημα εμφανίζει χαμηλότερους ρυθμούς
αναφοράς ψευδώς θετικών στάσεων σε σχέση με το σύστημα που προτάθηκε για λύση
του προβλήματος εκτίμησης στάσης βάσει καθολικής αβεβαιότητος στην ενότητα
\ref{subsection:02_02_05:02} και δοκιμάστηκε στην πειραματική διαδικασία της
προηγούμενης ενότητας. Δεδομένου ότι στα δύο συστήματα δόθηκε ως όρισμα ο ίδιος
αριθμός υποθέσεων, σε συνέπεια με τα ευρήματα της βιβλιογραφίας (π.χ.
\cite{Olson2009a,bernreiter2021phaser}), καταλήγουμε ότι τα σφάλματα στάσης
εξόδου του (PL)ICP εξαρτώνται από την απόσταση των θέσεων από τις οποίες
συνελήφθησαν οι είσοδοί του. Για τον δε FMIC, εάν εξαρτώνται από αυτήν, τότε
εξαρτώνται σε μικρότερο βαθμό σε σχέση με τον PLICP. Αυτές οι παρατηρήσεις είναι
σημαντικές, καθώς σημαίνουν ότι

\begin{itemize}
  \item Στο έργο της ευθυγράμμισης μετρήσεων με σκοπό την παραγωγή οδομετρίας
        (Παρατήρηση \ref{rem:sm_applications}), ο (PL)ICP μπορεί να λειτουργήσει
        με αξιοπιστία μόνο σε χαμηλές ταχύτητες κίνησης, ενώ ο FMIC, εάν
        εκτελείτο σε πραγματικό χρόνο ($< 50$ ms), θα λειτουργούσε με αξιοπιστία
        σε μεγαλύτερες ταχύτητες κίνησης, και
  \item Στο έργο της ευθυγράμμισης πραγματικών με εικονικές σαρώσεις με σκοπό
        την ελάττωση του σφάλματος εκτίμησης στάσης κατά την παρακολούθηση
        της τροχιάς ενός ρομπότ (Παρατήρηση \ref{remark:smsm_benefit}), ο
        FMIC---εάν εκτελείτο σε πραγματικό χρόνο ως προς το ρυθμό εξαγωγής
        εκτιμήσεων της βασική μεθόδου παρακολούθησης ($< 200$ ms)---θα
        λειτουργούσε με αξιοπιστία σε υψηλότερα επίπεδα αβεβαιότητας στάσης,
        θορύβου μέτρησης ή/και διαφθοράς του χάρτη σε σχέση με το περιβάλλον
        που αναπαριστά σε σχέση με τον (PL)ICP
\end{itemize}

Εν κατακλείδι, οι στόχοι που τέθηκαν στην αρχή του κεφαλαίου επετεύχθησαν:
σε αυτό το κεφάλαιο αναπτύξαμε μία μέθοδο

\begin{quote}
``ευθυγράμμισης πραγματικών με εικονικές σαρώσεις η οποία λειτουργεί (α) χωρίς
τον υπολογισμό αντιστοιχίσεων ανάμεσα στις ακτίνες των εισόδων της, (β) με
`μικρό' σύνολο ρυθμιζόμενων παραμέτρων, και (γ) με επίδοση ως προς το σφάλμα
στάσης τουλάχιστον ισάξια με την καλύτερη μέθοδο ευθυγράμμισης μέσω
αντιστοιχιών." (Αρχή κεφαλαίου)
\end{quote}


%%%%%%%%%%%%%%%%%%%%%%%%%%%%%%%%%%%%%%%%%%%%%%%%%%%%%%%%%%%%%%%%%%%%%%%%%%%%%%%%
\subsection{Αιτίες περαιτέρω έρευνας}
\label{subsection:02_03_05:02}

Η κυριότερη αιτία περαιτέρω έρευνας αφορά στο χρόνο εκτέλεσης της μεθόδου που
εισαγάγαμε σε αυτό το κεφάλαιο: καθώς αυτή γίνεται σε χρόνο μεγαλύτερο από
αυτόν με τον οποίο παράγονται εκτιμήσεις από μία μέθοδο παρακολούθησης στάσης,
ο κυριότερος περιορισμός της είναι η αδυναμία της να εφαρμοσθεί στα
συμφραζόμενα της παρακολούθησης της τροχιάς ενός ρομπότ καθώς αυτό κινείται,
δηλαδή στα συμφραζόμενα του προηγούμενου κεφαλαίου.

Επιπρόσθετα, η μεταφορά του προβλήματος ευθυγράμμισης πραγματικών με εικονικές
μετρήσεις στο πεδίο της συχνότητας ίσως αποτελεί την αιτία της ευρωστίας της
προτεινόμενης σε αυτό το κεφάλαιο μεθόδου (α) στην απόσταση στάσεων από τις
οποίες συλλαμβάνονται οι προς ευθυγράμμιση σαρώσεις, (β) στο θόρυβο μέτρησης
και (γ) στη διαφθορά του χάρτη ως προς το περιβάλλον που αναπαριστά.  Αυτό το
ενδεχόμενο τροχιοδεικνύει την πορεία περαιτέρω έρευνας: σωρευτικά, με βάση τις
συνθήκες στις οποίες αναπτύχθηκε η προτεινόμενη μέθοδος του παρόντος κεφαλαίου
και τα συμπεράσματα που εξαγάγαμε, οι νέοι μας στόχοι είναι η ανάπτυξη μεθόδων
ευθυγράμμισης πραγματικών με εικονικές σαρώσεις, οι οποίες:

\begin{itemize}
  \item λειτουργούν χωρίς τον υπολογισμό αντιστοιχίσεων μεταξύ των ακτίνων των
        εισόδων
  \item απαιτούν τη ρύθμιση ``μικρού" συνόλου παραμέτρων
  \item εμφανίζουν επίδοση ως προς το σφάλμα στάσης τουλάχιστον ισάξια με την
        καλύτερη μέθοδο ευθυγράμμισης μέσω αντιστοιχιών
  \item δέχονται ως εισόδους σαρώσεις γωνιακού εύρους $360$ μοιρών
  \item εκτελούνται σε χρόνο τέτοιο που να συμβαδίζει με το ρυθμό
        εξαγωγής εκτιμήσεων από μία τυπική βασική μέθοδο εκτίμησης στάσης
  \item βασίζονται στον μετασχηματισμό Fourier και τις ιδιότητές του
\end{itemize}
