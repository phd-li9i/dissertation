%%%%%%%%%%%%%%%%%%%%%%%%%%%%%%%%%%%%%%%%%%%%%%%%%%%%%%%%%%%%%%%%%%%%%%%%%%%%%%%%
\subsection{Εκτίμηση στάσης βάσει καθολικής αβεβαιότητος}
\label{subsection:02_03_02:01}


Το πρόβλημα της εκτίμησης της στάσης ενός ρομπότ βάσει καθολικής αβεβαιότητας,
όπως ορίζεται στην ενότητα \ref{} ως διακριτή συνιστώσα του προβλήματος της
εκτίμησης της στάσης ενός ρομπότ κινητής βάσης, έχει τύχει εκτεταμένης προσοχής
με την πάροδο των ετών. Οι προσεγγίσεις που υιοθετήθηκαν για την επίλυσή του,
πέρα από την κατηγοριοποίηση σε παθητικές και ενεργητικές (ibid??), μπορούν
επίσης να χωριστούν σε δύο ακόμη κατηγορίες: προσεγγίσεις που λειτουργούν στο
χώρο των χαρακτηριστικών (features), και προσεγγίσεις που εκμεταλλεύονται άμεσα
μόνο τις ακατέργαστες μετρήσεις των αισθητήρων που φέρει το ρομπότ. Επιπλέον,
οι προσεγγίσεις μπορεί να είναι πιθανοτικές, βασισμένες σε φίλτρα, ή
ντετερμινιστικές. Ορισμένες προσεγγίσεις λαμβάνουν την μινιμαλιστική οδό,
δηλαδή βασίζονται αποκλειστικά σε μετρήσεις που εξάγονται από έναν αισθητήρα,
ενώ άλλες συγχωνεύουν πληροφορίες από διάφορες πηγές για να επιτύχουν ένα
βελτιωμένο αποτέλεσμα.

Οι προσεγγίσεις ενεργής εκτίμησης της στάσης ενός ρομπότ κινητής βάσης απαιτούν
την παραδοχή ότι το ρομπότ είναι ικανό να κινείται ελεύθερα στο περιβάλλον του
ακόμη και πριν καθοριστεί οριστικά η στάση του---καθιστώντας το δυνητικά
ευάλωτο σε συγκρούσεις με το περιβάλλον του. Η κίνηση του ρομπότ στο χώρο
είναι, κατ' αρχήν, επωφελής για τη λύση του εντοπισμού του, καθώς αυτή αυξάνει
τις μετρήσεις των αισθητήρων, και επομένως την πιθανότητα δειγματοληψίας
ποικίλων και μοναδικών τμημάτων του περιβάλλοντος, η οποία στη συνέχεια αυξάνει
την πιθανότητα επιτυχούς εκτίμησης της στάσης του, και τη μείωση της
αβεβαιότητάς της. Στο \cite{Jensfelt2001a} προτείνεται μια υβριδική προσέγγιση
με βάση την θεωρία του Bayes \cite{thrun2005probabilistic} και την
παρακολούθηση πολλαπλών υποθέσεων στάσης με χρήση του φίλτρου Kalman
\cite{Maybeck1979}. Η τελευταία και ένας τοπολογικός χάρτης του κόσμου
χρησιμοποιούνται για τη δημιουργία εντολών κίνησης ώστε να αποκομίζονται
περισσότερες πληροφορίες από το περιβάλλον και να αντισταθμίζεται η παρέκκλιση
της οδομετρίας, προκειμένου να φιλτράρονται οι τρέχουσες υποθέσεις στάσης και
να επιλύεται η ασάφεια που αφορά στη στάση του ρομπότ. Οι εντολές κίνησης
παράγονται ευρηστικά με την κίνηση στις άκρες του τοπολογικού χάρτη (έτσι ώστε
το ρομπότ είναι απίθανο να συγκρουστεί με εμπόδια), αποφεύγοντας να επισκεφθεί
το ίδιο μέρος δύο φορές (αφού είναι απίθανο να αποκτηθούν νέες πληροφορίες από
την ίδια θέση), και επιλέγοντας να επισκεφθεί τον γειτονικό κόμβο με τον
μέγιστο αριθμό χαρακτηριστικών στην περιοχή του. Τα χαρακτηριστικά γνωρίσματα
σε αυτή την περίπτωση εξάγονται από ένα αισθητήρα δισδιάστατων μετρήσεων
απόστασης, και περιλαμβάνουν χαρακτηριστικά πορτών, γραμμών, και ζευγών
σημείων, και χρησιμοποιούνται είτε για τη δημιουργία νέων υποθέσεων είτε για
την υποστήριξη των ήδη υπαρχουσών.  Κάθε ανιχνευόμενο χαρακτηριστικό δημιουργεί
ένα σύνολο πιθανών στάσεων του ρομπότ, οι οποίες αντιμετωπίζονται ως μετρήσεις
της πραγματικής στάσης του.  Οι εσωτερικές λειτουργίες κάθε φίλτρου Kalman που
συνδέεται με κάθε υποψήφια στάση στη συνέχεια εξασφαλίζουν ότι η πιθανότητα των
έγκυρων υποθέσεων αυξάνεται, ενώ οι λανθασμένες απορρίπτονται φυσικά μέσω
πρόσθετης εξωτερικής κατωφλίωσης (thresholding).

Μια άλλη προσέγγιση ενεργής εκτίμησης βάσει καθολικής αβεβαιότητος \cite{OKane}
εκκινεί από μια ακόμη πιο μινιμαλιστική στάση όσον αφορά στον αριθμό και τον
τύπο των αισθητήρων που χρησιμοποιούνται, χρησιμοποιώντας μόνο αισθητήρες
οδομετρίας και προφυλακτήρα (bumper). Η προσέγγιση φτάνει στο εκπληκτικό
(θεωρητικό) συμπέρασμα ότι, δεδομένου ενός ακριβούς χάρτη, η λύση του
προβλήματος μπορεί να επιτευχθεί με τη χρήση μόνο αυτών των δύο τύπων
αισθητήρων, παρόλο που το κινηματικό μοντέλο του ρομπότ περιορίζεται στο να
είναι χωρίς σφάλματα και ότι το ρομπότ επιτρέπεται να συγκρούεται με το
περιβάλλον του. Η υποκείμενη μέθοδος θέτει το πρόβλημα στον χώρο πληροφορίας
του ρομπότ και επιλύει ένα πρόβλημα σχεδιασμού διακριτού χρόνου, δείχνοντας
ότι, με σχετικά συνήθεις περιορισμούς του κόσμου/χάρτη, η εκτίμηση της στάσης
του ρομπότ είναι πράγματι δυνατή, αλλά με ένα βαθμό ασάφειας ανάλογο του βαθμού
συμμετριών που υπάρχουν στον περιβάλλον.

Στο \cite{Gasparri2007a} για την επίλυση του προβλήματος χρησιμοποιείται ένα
φίλτρο σωματιδίων, λόγω της εγγενούς του ικανότητας του να αναπαριστά αυθαίρετα
πολυτροπικές (multi-modal) κατανομές πυκνότητας πιθανότητας της στάσης ενός
ρομπότ. Σε αντίθεση με το \cite{Jensfelt2001a}, χρησιμοποιείται ένας αισθητήρας
σόναρ ως μέσο αντίληψης του περιβάλλοντος, και δεν χρησιμοποιούνται
χαρακτηριστικά των μετρήσεων ή του χάρτη.  Στη φάση αρχικοποίησης τα σωματίδια
διασκορπίζονται ομοιόμορφα εντός του ελεύθερου χώρου του χάρτη, και
υπολογίζεται ένα βάρος για το καθένα, σύμφωνα με το σφάλμα ελάχιστης
τετραγωνικής απόστασης μεταξύ της πραγματικής μέτρησης του αισθητήρα και της
εξόδου του μοντέλου παρατήρησης του αισθητήρα για τη συγκεκριμένη υπόθεση
στάσης. Ο αλγόριθμος εκτελείται επαναληπτικά, χωρίς το ρομπότ να κινείται,
υλοποιώντας έτσι μια αλγόριθμο βελτιστοποίησης. Προκειμένου να αποφευχθεί η
εξάντληση των σωματιδίων σε αυτό το στάδιο, νέα δείγματα δημιουργούνται στη
βάση της εξελικτικής θεωρίας, όπου τα επιζώντα σωματίδια---αυτά που δεν έχουν
αμελητέα βάρη---χρησιμεύουν ως ο τόπος γύρω από το τον οποίο εισάγονται νέα
σωματίδια, προκειμένου να ενισχυθεί η παρουσία των σωματιδίων όπου η πιθανότητα
εύρεσης του ρομπότ είναι υψηλότερη. Προκειμένου να μειωθεί η πιθανότητα
παγίδευσης του φίλτρου σε τοπικά ελάχιστα, νέα σωματίδια που εγγυώνται την
ελάχιστη κάλυψη του χώρου εισάγονται τυχαία στο χώρο στάσεων. Ο αλγόριθμος
σταματά όταν εντοπιστούν ευσταθείς λύσεις. Αυτές χρησιμοποιούνται στη συνέχεια
ως αρχικές θέσεις από τις οποίες εντοπίζεται το πλησιέστερο εμπόδιο σε μια
υπόθεση μέσω του αισθητήρα σόναρ, και στην περιοχή του οποίου δίνεται εντολή
στο ρομπότ να κινηθεί. Κατά τη διάρκεια της πλοήγησης κάθε υποψήφια υπόθεση
παρακολουθείται από ένα Εκτεταμένο φίλτρο Kalman, και ένα τετραγωνικό τεστ
$\chi$ με χρήση της απόστασης Mahalanobis χρησιμοποιείται για την επικύρωση των
υποψηφίων στάσεων κατά τη διάρκεια της συσχέτισης δεδομένων μεταξύ
παρατηρούμενων και θεωρητικά αναμενόμενων μετρήσεων.

Η έρευνα στις ενεργές μεθόδους παγκόσμιου εντοπισμού φαίνεται να έχει μειωθεί
τα τελευταία χρόνια, με παλαιότερες προσεγγίσεις, συμπεριλαμβανομένων των
\cite{Manasse1988,Kleinberg,Romanik1996,Dudek1998,OKanea,Rao2007}, να δίνουν
ώθηση στην έρευνα προς την κατεύθυνση του πιο απαιτητικού προβλήματος, δηλαδή
του παθητικής εκτίμησης της στάσης του. Σε αυτόν τον τρόπο λύσης, το ρομπότ δεν
εκτελεί κινήσεις πλοήγησης προτού εκτιμηθεί η στάση του και στερείται πρόσθετες
εισόδους αισθητήρων εκτός από εκείνες που λαμβάνονται από την άγνωστη αρχική
του στάση.

Στο \cite{Se} χρησιμοποιείται μια τριοπτρική (trinocular) κάμερα για τη
δημιουργία ενός τρισδιάστατου χάρτη των χαρακτηριστικών SIFT που υπάρχουν στο
περιβάλλον του ρομπότ, ο οποίος ενημερώνεται με την πάροδο του χρόνου, ενώ
προσαρμόζεται σε δυναμικά περιβάλλοντα, δημιουργώντας, διατηρώντας, και
ενημερώνοντας ένα φίλτρο Kalman για κάθε ξεχωριστό ορόσημο-χαρακτηριστικό που
ανιχνεύεται στο οπτικό πεδίο της κάμερας ανά καρέ εισόδου. Αυτός ο
τρισδιάστατος χάρτης ορόσημων κατασκευάζεται πριν από το χρόνο εκτίμησης της
στάσης του ρομπότ. Χάρει στην ιδιαιτερότητα των χαρακτηριστικών SIFT το ρομπότ
δεν χρειάζεται να μετακινηθεί στο περιβάλλον του για να εντοπιστεί. Για την
επίλυση του προβλήματος εκτίμησης της στάσης του, τα χαρακτηριστικά SIFT που
εξάγονται μέσω της κάμερας από την πραγματική στάση του ρομπότ αντιστοιχίζονται
με εκείνα που είναι ήδη αποθηκευμένα στον τρισδιάστατο χάρτη μέσω
κατακερματισμού μετασχηματισμού Hough (Hough Transform
Hashing---\cite{Hough1960}), προκειμένου να αποκτηθεί μια πρόχειρη εκτίμηση της
στάσης του ρομπότ (τα bins του μετασχηματισμού με τις περισσότερες ψήφους
αντιστοιχούν σε εκτιμήσεις στάσεων που είναι πιο πιθανό να επιτύχουν μεγαλύτερο
αριθμό αντιστοιχίσεων). Στη συνέχεια, εκτελείται επαναληπτική ελαχιστοποίηση
ελαχίστων τετραγώνων προκειμένου να επιτευχθεί προοδευτικά καλύτερη εκτιμήση
της στάσης. Η τελική στάση είναι εκείνη που έχει τον μέγιστο αριθμό ταυτίσεων
αντιστοιχίσεων και το μικρότερο σφάλμα ελαχίστων τετραγώνων.

Αντιθέτως, στο \cite{Hernandez-alamilla2006}, ο αισθητήρας αντίληψης του
περιβάλλοντος είναι ένας αισθητήρας δισδιάστατων μετρήσεων τύπου lidar. Σε
πρώτη, προεπεξεργαστική φάση, εξάγονται χαρακτηριστικά από το χάρτη του
περιβάλλοντος του ρομπότ, και δημιουργείται μία βάση δεδομένων ορατών
χαρακτηριστικών από κάθε κελί πλέγματος του χάρτη. Τα χαρακτηριστικά αυτά είναι
φυσικά ορόσημα που βρίσκονται τόσο στο χάρτη όσο και στις μετρήσεις του
αισθητήρα του ρομπότ, και περιλαμβάνουν τοίχους, που εξάγονται ως ευθύγραμμα
τμήματα μέσω της χρήσης του μετασχηματισμού Hough, κοίλες γωνίες, και
ασυνέχειες μεταξύ διαδοχικών ακτίνων σάρωσης. Στη συνέχεια τα ορόσημα και από
τις δύο πηγές αντιστοιχίζονται μεταξύ τους με μια διαδικασία δύο βημάτων: ένα
αρχικό φίλτρο αφαιρεί το μεγαλύτερο μέρος των λανθασμένων υποθέσεων στάσης
μετρώντας τον αριθμό των χαρακτηριστικών που έχουν εξαχθεί από τις μετρήσεις
του αισθητήρα τα οποία ταιριάζουν με την απόσταση, τον προσανατολισμό, και τον
τύπο του ορόσημου που είναι αποθηκευμένα στη βάση δεδομένων του χάρτη. Στη
συνέχεια χρησιμοποιείται ένας τροποποιημένος αλγόριθμος διακριτής χαλάρωσης,
χρησιμοποιώντας τις πληροφορίες των χαρακτηριστικών που σχετίζονται με κάθε
αποθηκευμένο ορόσημο. Το κελί πλέγματος που αντιστοιχεί στη θέση του ρομπότ
προσδιορίζεται μεταξύ των υποθέσεων με τη χρήση ενός κριτηρίου ελαχίστων
τετραγώνων της διαφοράς απόστασης μεταξύ (α) αυτού του κελιού από κάθε ορόσημο
του χάρτη, και (β) της απόστασης από ένα ορόσημο που προκύπτει από τη μέτρηση
του αισθητήρα, για όλα τα υποψήφια κελιά.  Η στάση του ρομπότ υπολογίζεται στη
συνέχεια ως η μέση γωνιακή απόκλιση μεταξύ (α) της θέσης του ρομπότ και ενός
ορόσημου που έχει εξαχθεί από το διάνυσμα της μέτρησης, και (β) της εκτιμώμενης
θέσης του ρομπότ και κάθε ορατού ορόσημου στο χάρτη από το συγκεκριμένο κελί.

Μια γενική μεθοδολογία για την ανάλυση του σχεδιασμού σημείων-κλειδιών
(keypoints) για την εκτίμηση στάσης παρουσιάζεται στο \cite{Bosse2009}. Η
μεθοδολογία αυτή μπορεί να χρησιμοποιηθεί για την επίλυση του προβλήματος ??
στα πλαίσια της πλοήγησης ρομπότ κινητής βάσης και της χαρτογράφησης, και είναι
ιδιαίτερα χρήσιμη στη διαδικασία καθορισμού των παραμέτρων για την επιλογή των
συγκεκριμένων τύπων σημείων-κλειδιών, καθώς δεν είναι όλα αξιόπιστα παρουσία
θορύβου ή τυφλών περιοχών (occlusions). Οι συγγραφείς θέτουν το πρόβλημα της
εκτίμησης της στάσης ως ένα πρόβλημα αναζήτησης πλησιέστερου γείτονα,
επιλέγοντας πρώτα ένα σύνολο σημείων-κλειδιών που εξάγονται από οπτικό πεδίο
ενός αισθητήρα δισδιάστατων μετρήσεων, τα οποία κωδικοποιούν την τοπική περιοχή
γύρω από το ρομπότ. Στη συνέχεια αναζητούν σε μια βάση δεδομένων τα σημεία
κλειδιά που έχουν προηγουμένως δημιουργηθεί από το χάρτη, για τον εντοπισμό
σημείων με κοινά χαρακτηριστικά. Ο χάρτης γύρω από το ρομπότ και τα τμήματα του
χάρτη του περιβάλλοντος που εντοπίζονται να έχουν κοινά χαρακτηριστικά
τροφοδοτούνται στη συνέχεια σε μία αλυσιδωτή διαδικασία τεσσάρων μεθόδων
επαλήθευσης (μία εκ των οποίων είναι η παραδοσιακή μέθοδος ευθυγράμμισης
σαρώσεων ICP), οι οποίες φιλτράρουν τα ψευδή θετικά αποτελέσματα, έως ότου
βρεθεί η βέλτιστη αντιστοίχιση ανάμεσά τους.

Αν και το \cite{Brenner2010} ασχολείται με το πρόβλημα της εκτίμησης ενός
οχήματος κινητής βάσης σε εξωτερικά περιβάλλοντα, η αρχή λειτουργίας της
μεθόδου του είναι ανάλογη με αυτές που χρησιμοποιούνται σε εσωτερικούς χώρους.
Ο συγγραφέας υποστηρίζει ότι η ευθυγράμμιση τρισδιάστατων νεφών σημείων που
εξάγονται μέσω ενός αισθητήρα lidar με νέφη σημείων που εξάγονται από το χάρτη
του περιβάλλοντός του δεν είναι αποδοτική στο πλήρες τρισδιάστατο σενάριο και,
ως εκ τούτου, είναι επιθυμητό να κατασκευαστούν αφαιρέσεις (abstractions) με τη
μορφή ορόσημων. Το ορόσημα που χρησιμοποιούνται εδώ για την ευθυγράμμιση τους
είναι στύλοι, όπως αυτές των πινακίδων κυκλοφορίας, των φαναριών, και τα
δέντρων. Τα ευρήματα του συγγραφέα δείχνουν ότι αυτός ο τύπος ορόσημου δεν
είναι αξιόπιστος για την εκτίμηση της στάσης ενός οχήματος βάσει καθολικής ή
πεπερασμένης αβεβαιότητος, αφού περίπου το $40\%$ όλων των λαμβανόμενων
μετρήσεων δεν περιλαμβάνουν στύλους, είτε λόγω πραγματικών απουσίας τους είτε
λόγω δημιουργίας ψευδώς θετικών αποτελεσμάτων από τον προτεινόμενο μηχανισμό
ανίχνευσης στύλων.

Στο \cite{Zhu2011a} οι συγγραφείς υποστηρίζουν ότι σε εσωτερικά περιβάλλοντα,
οι σημαντικότερες δομές είναι οι τοίχοι, οι πόρτες, και τα ντουλάπια, δηλαδή
δομές που μπορούν να προσεγγιστούν με ευθείες γραμμές στον δισδιάστατο χώρo.
Χρησιμοποιώντας αυτή την υπόθεση κατασκευάζουν μία μέθοδο που αρχικά εκτιμά τον
προσανατολισμό του ρομπότ εξάγοντας χαρακτηριστικά γραμμών από το χάρτη του
περιβάλλοντος και από έναν αισθητήρα δισδιάστατων μετρήσεων απόστασης, προτού
στη συνέχεια ευθυγραμμιστούν με τη χρήση του αλγορίθμου HSM. Η εξαγώμενη
εκτίμηση εξαρτάται από την απουσία συμμετριών στο περιβάλλον. Δεδομένης της
εκτίμησης του προσανατολισμού του ρομπότ, η θέση του εκτιμάται μέσω του
μοντέλου ακραίου σημείου δέσμης \cite{thrun2005probabilistic}, όπου
υπολογίζεται η πιθανότητα ότι ένα κελί του χάρτη πλέγματος ήταν η θέση από την
οποία έγινε η μέτρηση σάρωσης, για όλα τα κελιά του χάρτη. Μετά από αυτήν την
εξαντλητική αναζήτηση, το κελί με την υψηλότερη πιθανότητα επιλέγεται ως η
εκτίμηση της θέσης του ρομπότ. Στη συνέχεια, χρησιμοποιείται ένας αλγόριθμος
κατάβασης (gradient descent) για την περαιτέρω εξάλειψη του σφάλματος
διακριτοποίησης που προκαλείται από την αναπαράσταση του χάρτη μέσω πλέγματος.

Εκτός από τον χάρτη, η μέθοδος που παρουσιάζεται στο \cite{Xie2010}
χρησιμοποιεί προαποθηκευμένες σαρώσεις αναφοράς, οι οποίες εξάγονται μέσω ενός
αισθητήρα δισδιάστατων μετρήσεων lidar, και τις στάσεις από τις οποίες
αποτυπώθηκαν αυτές στο σύστημα αναφοράς του χάρτη, σε ένα βήμα πριν από την
εκτέλεση του αλγορίθμου εκτίμησης της στάσης του ρομπότ βάσει καθολικής
αβεβαιότητος. Κατά την έναρξη της εκτέλεσής του κατασκευάζεται ένας τοπικός
χάρτης πλέγματος με βάση την πρώτη μέτρηση από τον πραγματικό αισθητήρα. Στη
συνέχεια κατασκευάζεται ένας αριθμός αντιγράφων του μέσω περιστροφής κατά
ακέραια πολλαπλάσια τεσσάρων μοιρών, έως ότου σχηματίστεί ένας πλήρης κύκλος.
Στη συνέχεια δημιουργείται ένας τοπικός χάρτης πλέγματος από τις
προαποθηκευμένες σαρώσεις αναφοράς για κάθε αντίστοιχη στάση αναφοράς, όπου
κάθε μία θεωρείται ως μια πιθανή υποψήφια στάση του ρομπότ. Στη συνέχεια τα δύο
σύνολα τοπικών χαρτών ευθυγραμμίζονται μεταξύ τους μέσω μιας διαδικασία
βελτιστοποίησης που έχει ως στόχο τη μεγιστοποίηση της επικάλυψης μεταξύ των
δύο συνόλων υπο-χαρτών. Η διαδικασία αυτή ξεκινά με χάρτες χαμηλής ανάλυσης,
και η ανάλυσή τους σταδιακά αυξάνεται.

Μια εναλλακτική λύση για τη σύγκριση των περιγραφών τόπων ανά ζεύγη προτείνεται
στο \cite{Bosse2013}, η οποία μειώνει τον γραμμικό χρόνο αναζήτησης σε
υπο-γραμμικά επίπεδα. Η προτεινόμενη μέθοδος αποσκοπεί στην αντικατάσταση των
συγκρίσεων σε επίπεδο περιγραφέα τόπου με περιγραφείς σημείων-κλειδιών,
δεδομένου ότι οι τελευταίοι βρίσκονται σε χαμηλότερο επίπεδο από τους πρώτους,
και ότι τα αποτελέσματα σε δύο διαστάσεις έχουν δείξει ότι η χρήση τους οδηγεί
σε υψηλά ποσοστά αναγνώρισης τόπων. Από μία βάση δεδομένων προ-αποθηκευμένων
τοπικών περιγραφέων του χάρτη, ένας σταθερός αριθμός πλησιέστερων γειτόνων
ψηφίζει για κάθε σημείο-κλειδί που εξάγεται από το τρισδιάστατες μετρήσεις
lidar, και η άθροισή τους καθορίζει τις πιθανές αντιστοιχίες τόπων. Οι
συγγραφείς διαπιστώνουν ότι ένα τέτοιο σύστημα οδηγεί σε αποτελέσματα
ψηφοφορίας των οποίων η κατανομή ακολουθεί μια λογαριθμοκανονική κατανομή, και
έτσι είναι σε θέση να προσαρμόσουν ένα παραμετρικό μοντέλο υπερπαραμέτρων
προκειμένου να καθοριστεί ένα ουσιαστικό κατώφλι ψηφοφορίας, το οποίο μπορεί να
διακρίνει αξιόπιστα μεταξύ αληθινών και ψευδών θετικών αποτελεσμάτων,
παρέχοντας έναν αυτόματο τρόπο ρύθμισης κρίσιμων αλγοριθμικών παραμέτρων.

Η πρώτη χρήση της τεχνικής ευθυγράμμισης πραγματικών μετρήσεων με σαρώσεις
χάρτη σε δύο διαστάσεις (ενότητα \ref{}) στα πλαίσια της εκτίμησης της στάσης
ρομπότ κινητής βάσης βάσει καθολικής αβεβαιότητας εμφανίζεται στο
\cite{Park2014a}.  Η προτεινόμενη μέθοδος παράγει πρώτα το γενικευμένη
διάγραμμα Voronoi του δισδιάστατου χάρτη πλέγματος, του οποίου οι κόμβοι
λαμβάνονται ως αρχικές υποθέσεις για τη θέση του ρομπότ. Από αυτές τις θέσεις
λαμβάνονται εικονικές σαρώσεις σε ένα γωνιακό εύρος $360$ μοιρών με τη χρήση
δεσμοβολής εντός του χάρτη. Οι αντιστοιχίες μεταξύ κάθε εικονικής σάρωσης και
της σάρωσης που λαμβάνεται από τον αισθητήρα εκτελούνται με τη χρήση της
φασματικής τεχνικής \cite{Leordeanu2005a}, η οποία βρίσκει γεωμετρικές σχέσεις
ανά ζεύγη σαρώσεων. Αυτές οι αντιστοιχίες χρησιμοποιούνται στη συνέχεια για τη
δημιουργία δισδιάστατων γεωμετρικών ιστογραμμάτων που κωδικοποιούν ένα μέτρο
της ομοιότητας μεταξύ της πραγματικής σάρωσης και όλων των εικονικών σαρώσεων.
Οι κόμβοι από τους οποίους αποτυπώθηκαν οι τελευταίες κατατάσσονται στη
συνέχεια σύμφωνα με αυτό το μέτρο ομοιότητας, και ένα κατώφλι που βασίζεται
στον συντελεστή συσχέτισης όλων των συνδυασμών σαρώσεων χρησιμοποιείται για την
εξαγωγή ενός υποσυνόλου υποψήφιων στάσεων. Η τελική εκτίμηση της στάσης είναι
εκείνη που επιτυγχάνει τον μέγιστο αριθμό αντιστοιχούντων ζευγών μετά από την
ίδια φασματική μέθοδο ευθυγράμμισης που χρησιμοποιήθηκε στο προηγούμενο βήμα.

Εμπνευσμένη από έρευνα του κλάδου υπολογιστικής όρασης, η μέθοδος που
παρουσιάζεται στο \cite{Himstedt2014} χρησιμοποιεί υπογραφές ορόσημων που
εξάγονται από δισδιάστατες μετρήσεις αισθητήρα lidar για την εύρεση της στάσης
του ρομπότ.  Τα ορόσημα που χρησιμοποιούνται είναι σημεία υψηλής καμπυλότητας,
που αποδεικνύεται ότι είναι επαρκώς περιγραφικά στον τομέα των δεδομένων
απόστασης \cite{Tipaldi2010}. Για κάθε σύνολο ορόσημων που εξάγονται και
αποθηκεύονται εκ των προτέρων κατά τη διάρκεια της ταυτόχρονης χαρτογράφησης
και παρακολούθησης της στάσης το ρομπότ (SLAM), καταγράφεται και αποθηκεύται η
κατανομή των χωρικών σχέσεων μεταξύ τους. Στη συνέχεια δημιουργείται ένα
δισδιάστατο ιστόγραμμα από ένα πλέγμα δυαδικών ψηφίδων, στο οποίο οι σχέσεις
αυτές κωδικοποιούνται κεντράροντας μια κανονική κατανομή σε κάθε κάδο (bin) του
ιστογράμματος. Η υπογραφή κάθε ορόσημου υπολογίζεται ως το άθροισμα όλων των
κατανομών πάνω από το εν λόγω ορόσημο, και η υπογραφή της σάρωσης υπολογίζεται
συνεπώς ως το άθροισμα όλων των υπογραφών των ορόσημων της εν λόγω σάρωσης.
Αφού αποθηκευτούν οι υπογραφές σε μια βάση δεδομένων, η εκτίμηση της στάσης
πραγματοποιείται με βάση τον κατά προσέγγιση πλησιέστερο γείτονα των υπογραφών
που είναι αποθηκευμένες στη βάση δεδομένων που αυτών εξάγονται από τον
αισθητήρα πραγματικής σάρωσης κατά τη στιγμή της εκτέλεσης του αλγορίθμου
εκτίμησης. Η στάση-έξοδος του συστήματος είναι εκείνη της οποίας η υπογραφή
έχει τη μικρότερη απόσταση από εκείνη της μέτρησης εισόδου.

Μακριά από τις καθιερωμένες τεχνικές της έρευνας γύρω από την επίλυση του
προβλήματος \ref{}, η μέθοδος που παρουσιάζεται στο \cite{Lyrio2014}
χρησιμοποιεί νευρωνικά δίκτυα ως μέσο εκτίμησης της στάσης ηρεμίας ενός ρομπότ,
και χωρίς τη χρήση ενός χάρτη. Αντ' αυτού, στο αρχικό βήμα, το ρομπότ διασχίζει
το περιβάλλον του, και ένα ζεύγος μιας εικόνας από την εμπρόσθια κάμερα RGB του
ρομπότ και μιας στάσης από την οποία αποτυπώθηκε αποθηκεύονται σε μια βάση
δεδομένων. Μετά τη συλλογή όλων των ζευγών, ένα νευρωνικό δίκτυο εκπαιδεύεται
στις εικόνες που έχουν ληφθεί, έτσι ώστε το σύστημα να μάθει να εξάγει τον
μοναδικό δείκτη κάθε εικόνας. Κατά τη διαδικασία αυτή, κάθε νευρώνας
δειγματοληπτεί την εικόνα εισόδου στο σύνολό της, και τη φιλτραρισμένη με
φίλτρο κανονικής κατανομής εκδοχή της ίδιας εικόνας, έτσι ώστε να εξαλειφθεί ο
θόρυβος υψηλής συχνότητας, αλλά να διατηρούνται οι λεπτομέρειες κάθε σκηνής.
Όταν κατά τη διάρκεια της εκτίμησης της στάσης του ρομπότ το σύστημα συλλαβει
μια εικόνα, την εισάγει στο νευρωνικό δίκτυο, όπου όλοι οι νευρώνες εξάγουν
έναν δείκτη για αυτήν. Η ψηφοφορία για κάθε δείκτη αποφασίζει την τελική
εκτίμηση της στάσης του ρομπότ.

Οι συγγραφείς του \cite{Kallasi2016} υποστηρίζουν ότι στο \cite{Himstedt2014} η
γεωμετρικές υπογραφές σχέσεων ορόσημων, αν και συμπερασματικά ισχυρές, δεν
είναι αναλλοίωτες σε περιστροφές. Η μέθοδος τους χρησιμοποιεί βελτιωμένα
σημεία-κλειδιά τύπου falko και γρήγορη σημείο-προς-σημείο αντιστοίχιση των
σαρώσεων για την αναγνώριση τόπων. Αν και αυτή η τεχνική περιορίζεται στην
αναγνώριση τόπων στο πλαίσιο του κλεισίματος βρόχου κατά τη διάρκεια του SLAM,
διαπιστώνουμε ότι θα μπορούσε να επεκταθεί και στην ευθυγράμμιση πραγματικών με
εικονικές σαρώσεις για τον εντοπισμό της στάσης του ρομπότ.

Η μέθοδος που παρουσιάζεται στα \cite{Su2017} και \cite{Chen2019b} εντοπίζει
ένα ρομπότ χρησιμοποιώντας έναν οπτικό αισθητήρα RGBD μαζί με έναν αισθητήρα
δισδιάστατων μετρήσεων τύπου lidar, χρησιμοποιώντας οπτικά χαρακτηριστικά που
εξάγονται από τις μετρήσεις του πρώτου σε βοήθεια των μετρήσεων του τελευταίου.
Για το σκοπό αυτό, κατά τη διάρκεια κατασκευής του χάρτη του περιβάλλοντος,
ταυτόχρονα με τον χάρτη που οικοδομείται από τον αισθητήρα απόστασης, ένας
χάρτης οπτικών πληροφοριών κατασκευάζεται σε ένα βήμα προεπεξεργασίας, και
αποθηκεύονται οι στάσεις από τις οποίες καταγράφονται οι οπτικές πληροφορίες.
Οι οπτικές πληροφορίες που αποθηκεύονται είναι μια σειρά από καρέ-κλειδιά, και
οι αντίστοιχοι περιγραφείς GIST τους \cite{Azzi2015,Singh2010}. Όταν εκτελείται
ο αλγόριθμος εντοπισμού, το υπολογίζεται το διάνυσμα GIST από την εικόνα RGB
του πραγματικού αισθητήρα της κάμερας, και η απόσταση Minkowski από όλα τα
προ-αποθηκευμένα διανύσματα GIST χρησιμοποιείται για την κατάταξη των $n$
κορυφαίων αντιστοιχίσεων ανάμεσά τους.  Αυτές στη συνέχεια ομαδοποιούνται, και
ο μετασχηματισμός μεταξύ της στάσης του ρομπότ και του συστήματος αναφοράς του
χάρτη προκύπτει από τον υπολογισμό του μετασχηματισμού μεταξύ της εικόνας
εισόδου και πλησιέστερης εικόνας στο κεντροειδές της συστάδας, λαμβάνοντας
υπόψη τη στάση από την οποία αποτυπώθηκε η πλησιέστερη σε αυτό εικόνα.

Οι συγγραφείς του \cite{Cop2018} παρουσιάζουν έναν νέο περιγραφέα για
τρισδιάστατες σαρώσεις αισθητήρων lidar, ο οποίος βασίζεται όχι στη συνιστώσα
της απόστασης, αλλά στη συνιστώσα της έντασής (intensity). Αυτός ο περιγραφέας
χρησιμοποιείται για κάθε ένα από τα τρισδιάστατα νέφη σημείων που έχουν
προ-αποθηκευθεί κατά τη στιγμή της δημιουργίας του χάρτη του περιβάλλοντος, και
επίσης για το νέφος σημείων που καταγράφηκε κατά την εκτέλεση της εκτίμησης της
στάσης του ρομπότ.  Το τελευταίο διαιρείται αρχικά σε κάδους, για τους οποίους
υπολογίζεται ένα ιστόγραμμα έντασης. Τα ιστογράμματα αυτά συνδυάζονται στη
συνέχεια σε έναν καθολικό περιγραφέα με βάση την ένταση, του οποίου η ομοιότητα
συγκρίνεται με εκείνους που εξάγονται από τα προ-αποθηκευμένα νέφη. Όταν
ταυτοποιηθεί το περισσότερο όμοιο νέφος σημείων, χρησιμοποιούνται τοπικοί
γεωμετρικοί περιγραφείς για την εύρεση αντιστοιχίσεων
σημείου-κλειδιού-προς-σημείο-κλειδί, και αυτές χρησιμοποιούνται για την παροχή
του πλήρους μετασχηματισμού 6DOF μεταξύ των δύο νεφών σημείων εισόδου.

Η μέθοδος που προτείνεται στο \cite{Wang2019a}, ονομαζόμενη GLFP, δεν βασίζεται
σε έναν εκ των προτέρων κατασκευασμένο χάρτη, αλλά σε ένα σχέδιο κάτοψης του
περιβάλλοντος στο οποίο το ρομπότ καλείται να εντοπιστεί. Προκειμένου να
ξεπεραστεί το χάσμα μεταξύ της έλλειψης προηγούμενης επίσκεψης στο περιβάλλον
και της διαθεσιμότητας μόνο μίας κάτοψής του, η GLFP εντοπίζει χαρακτηριστικά
που συνυπάρχουν τόσο σε έναν χάρτη χαμηλής ποιότητας της κάτοψης, όσο και στις
τρισδιάστατες μετρήσεις του αισθητήρα lidar του ρομπότ.  Στη συνέχεια εξάγει
τις κάθετες ακμές από το νέφος σημείων εισόδου και τις γωνίες από την κάτοψη
του χάρτη. Η συσχέτιση των δεδομένων πραγματοποιείται στη συνέχεια με τη χρήση
max-mixtures \cite{Olson2013} και της μεθόδου αναζήτησης του πλησιέστερου
γείτονα.  Η στάση του ρομπότ εκτιμάται στη συνέχεια μέσω ενός αλγορίθμου
βασισμένου στον γράφο παραγόντων (factor-based graph), ένα πρόβλημα
βελτιστοποίησης όπου και τα δύο ορόσημα και η στάση του ρομπότ αντιμετωπίζονται
ως μεταβλητές.

Η μέθοδος που παρουσιάζεται στο \cite{Yilmaz2019} επεκτείνει τη μέθοδο SA-MCL
\cite{Zhang2009} για χρήση σε συνθήκες που χρησιμοποιούνται πολλαπλοί
δισδιάστατοι ή τρισδιάστατοι αισθητήρες lidar για την εκτίμηση της στάσης ενός
ρομπότ κινητής βάσης. Και οι δύο μέθοδοι χρησιμοποιούν τον υποκείμενο μηχανισμό
MCL \cite{Gustafsson2002} για την εκτέλεση της εκτίμησης της στάσης του:
σωματίδια διασκορπίζονται στο χάρτη και το πεδίο πιθανοφάνειας (likelihood
field \cite{thrun2005probabilistic}) χρησιμοποιείται ως μοντέλο μέτρησης
προκειμένου να εντοπιστεί το σωματίδιο που εξηγεί καλύτερα τις μετρήσεις μεταξύ
όλων των σωματιδίων. Ωστόσο, στον SA-MCL τα σωματίδια δεν διασκορπίζονται
ομοιόμορφα στο σύνολο του χάρτη, αλλά μόνο σε παρόμοιες ενεργειακά περιοχές. Σε
ένα βήμα προεπεξεργασίας, κάθε κελί του χάρτη πλέγματος 2D συσχετίζεται με μια
τιμή ενέργειας η οποία κωδικοποιεί την εγγύτητα του ρομπότ σε εμπόδια. Στη
συνέχεια, κατά την εκτέλεση της εκτίμησης της στάσης του ρομπότ, υπολογίζεται η
ενέργεια της μέτρησης εισόδου και συγκρίνεται με εκείνη όλων των κελιών του
πλέγματος.  Εκείνα που βρίσκονται κάτω από ένα ορισμένο όριο που έχει οριστεί
χειροκίνητα είναι εκείνα στα οποία θα κατενεμηθούν σωματίδια-υποθέσεις για τη
στάση του ρομπότ.



%%%%%%%%%%%%%%%%%%%%%%%%%%%%%%%%%%%%%%%%%%%%%%%%%%%%%%%%%%%%%%%%%%%%%%%%%%%%%%%%
\subsection{Ο μετασχηματισμός Fourier-Mellin στη ρομποτική}
\label{subsection:02_03_02:02}

Ο μετασχηματισμός Fourier-Mellin (Fourier-Mellin Transform---FMT) έχει λάβει
περιορισμένη προσοχή στα πλαίσια του πεδιου εφαρμογής \ref{scope}, κυρίως λόγω
του περιορισμού εφαρμογής του σε δισδιάστατα πλέγματα-εικόνες. Οι περισσότερες
από τις εφαρµογές περιορίζονται συνεπώς σε ροµπότ που φέρουν αισθητήρες των
οποίων οι μετρήσεις είναι ή μπορουν να χρησιμοποιηθούν για να παράγουν
δισδιάστατες εικόνες, π.χ. αισθητήρες κάμερας, σόναρ, ή ραντάρ.  Επιπλέον, έχει
χρησιμοποιηθεί στη συγχώνευση/ευθυγράμμιση ψηφιακών χαρτών, ενώ οι περισσότερες
εφαρμογές εκμεταλλεύονται τον FMT για το έργο της χαρτογράφησης ή της εκτίμησης
της οδομετρίας ενός οχήματος.

Μία μέθοδος για την εκτίμηση της κίνησης ενός ρομπότ εξοπλισμένο με πανοραμικό
αισθητήρα ραντάρ παρουσιάζεται στα \cite{Checchin2010} και \cite{Vivet2013},
βασισμένη σε τεχνολογία Συνεχούς Κύματος με Διαμόρφωση Συχνότητας
\cite{Monod1995}, στα πλαίσια εκτέλεσης αλγορίθμου SLAM. Οι συγγραφείς
υποστηρίζουν ότι η ευαισθησία των αισθητήρων δισδιάστατων μετρήσεων lidar στις
ατμοσφαιρικές συνθήκες εξωτερικών χώρων έχει δώσει το έναυσμα για τη διεξαγωγή
SLAM με ραντάρ και σόναρ. Η έρευνά τους επικεντρώνεται στη χρήση ραντάρ μεγάλης
εμβέλειας με χαμηλή ισχύ εκπομπής, και εκμεταλλεύεται την εγγενή ικανότητά του
να εκτιμά πιο εύκολα τις απότομες μεταβολές των χρονικών μεταβλητών στη
συχνότητα παρά στο πεδίο του χρόνου. Η εκτίμηση της κίνησης του ρομπότ με βάση
το ραντάρ πραγματοποιείται μέσω του FMT, στον οποίον εισάγονται διαδοχικές
εικόνες ραντάρ και από τον οποίον εξάγεται η σχετική τους μετατόπιση και
περιστροφή, οι οποίες είναι ακριβώς εκείνες που αφορούν στη στάση του ρομπότ
από την οποία λήφθηκε η δεύτερη εικόνα σε σχέση με εκείνη από την οποία λήφθηκε
η πρώτη.

Στα πλαίσια του ελέγχου αυτόνομων υποβρύχιων οχημάτων και χαρτογράφησης σε
υποβρύχιες συνθήκες \cite{Bulow2010}, η βασική αρχή του FMT συνδυάζεται με την
τεχνική Phase-Only Matched Filter (POMF, σε αντίθεση με την Symmetric
Phase-Only Matched Filtering---SPOMF, ενότητα \ref{}), λόγω της έλλειψης
ανάγκης εξαγωγής της κλίμακας ανάμεσα σε διαδοχικές σαρώσεις απόστασης ενός
σόναρ. Οι συγγραφείς παρατηρούν ότι ένας βασικός αλγόριθμος ευθυγράμμισης
μετρήσεων από σόναρ με τη χρήση αντιστοιχίσεων ανάμεσά τους, όπως ο ICP ή η
ιδιαίτερα αποδοτικότερη παραλλαγή του, ο PL-ICP, δεν μπορούν να χρησιμοποιηθούν
με μετρήσεις απόστασης τύπου σόναρ ως είσοδο, καθώς αυτοί αναφέρουν όχι μόνο
μία μέτρηση απόστασης, αλλά ένα πλήθος τιμών, οι οποίες αντιστοιχούν σε πλάτη
ηχών σε διαφορετικές αποστάσεις, παραβιάζοντας έτσι την τη θεμελιώδη παραδοχή
της μοναδικότητας της απόστασης που υποθέτει ο ICP και οι παραλλαγές του.
Επιπλέον, οι αισθητήρες σόναρ παρουσιάζουν τέτοια επίπεδα θορύβου που προκαλούν
τους αλγορίθμους τύπου ICP να εμφανίζουν μη βέλτιστα αποτελέσματα, και
συνηθέστερα όταν μια δέσμη σόναρ προσκρούει σε μια επιφάνεια υπό γωνία. Στο
\cite{Bulow2011} αυτή η προσέγγιση επεκτείνεται σε τρεις διαστάσεις, και στο
\cite{Pfingsthorn2010} προσαρμόζεται σε πιθανοτικά πλαίσια, όπου οι πίνακες
συνδιακύμανσης προσαρμόζονται γύρω από τις τρεις μετατοπίσεις, περιστροφές, και
κλίμακώσεις, ανάλογα με την ένταση σε κάθε παράμετρικό χώρο, και
αντιμετωπίζονται ως συναρτήσεις πυκνότητας πιθανότητας. Η διαδικασία αυτή
ενσωματώνεται σε ένα πλαίσιο χαρτογράφησης μέγιστης πιθανοφάνειας που
χρησιμοποιείται για τη δημιουργία χαρτών των υποβρύχιων δομών από ακολουθίες
μονοπτρικών εικόνων (αντί για αισθητήρες σόναρ) μέσω βελτιστοποίησης γράφων με
βάση τη στάση (pose-based graph optimisation).

Η ίδια αρχή εφαρμόζεται στο \cite{Bulow2009} αλλά με μια βελτιωμένη μέθοδο FMT,
στο πλαίσιο μη επανδρωμένων εναέριων οχημάτων (Unmanned Aerial Vehicles---UAV),
με σκοπό την καταγραφή χαρτών κάτοψης από έναν αισθητήρα κάμερας, μέσω της
συρραφής διαδοχικών εικόνων. Η μέθοδος αυτή μπορεί ταυτόχρονα να χρησιμοποιηθεί
ως μια μορφή (οπτικής) οδομετρίας του οχήματος. Η ίδια μέθοδος χρησιμοποιείται
στο \cite{Birk2010} για την επίλυση του προβλήματος του εντοπισμού δομικών
σφαλμάτων στους χάρτες πλέγματος κατάληψης που παράγονται από αλγορίθμους SLAM
που εκτελούνται σε μη επανδρωμένα επίγεια οχήματα (Unmanned Ground
Vehicles---UGV), τα οποία προκύπτουν όταν περιοχές του συνολικού χάρτη είναι
τοπικά συνεπείς σε σχέση με έναν πιστό χάρτη, αλλά ασυνεπείς μεταξύ τους,
εισάγοντας την έννοια του σπασίματος (brokenness) ενός χάρτη. Η επίλυση αυτού
του προβλήματος είναι ιδιαίτερα χρήσιμη στο έργο χαρτογράφησης πολλαπλών ρομπότ
ή κατά τη διάρκεια εκτελέσεων SLAM όπου συμμετρίες στο περιβάλλον οδηγούν σε
λανθασμένο κλείσιμο βρόχου. Οι κατατμήσεις της ευθυγράμμισης μεταξύ ενός χάρτη
αναφοράς και ενός δυνητικά εσφαλμένα ευθυγραμμισμένου χάρτη ανιχνεύονται με τη
χρήση ενός μέτρου ομοιότητας που παρέχεται από αυτή τη βελτιωμένη έκδοση του
FMT.

Στο \cite{Kazik2011} ο FMT χρησιμοποιείται για να παρέχει τη μετατόπιση και την
περιστροφή ενός UGV σε σχέση με μια παρελθούσα στάση, δηλαδή για την εξαγωγή
οδομετρικών πληροφοριών. Αυτές οι παράμετροι εξάγονται με την τροφοδοσία δύο
εικόνων που έχουν ληφθεί σε διαδοχικές χρονικές στιγμές από μια κάμερα RGB
τοποθετημένη στην κάτω πλευρά ενός οχήματος και στραμμένη προς την έδαφος σε
έναν αλγόριθμο POMF, από τον οποίο προκύπτει η μετατόπιση και η περιστροφή της
πιο πρόσφατης εικόνας σε σχέση με την προηγούμενηι. Στη συνέχεια οι παράμετροι
ανά άξονα της εικόνας μετατρέπονται σε κινήσεις του ρομπότ μέσω των εγγενών
παραμέτρων βαθμονόμησης του αισθητήρα της κάμερας, και από από εκεί εκφράζεται
η στάση από την οποία λήφθηκε η δεύτερη εικόνασε σχέση με εκείνη από την οποία
λήφθηκε η πρώτη.

Ομοίως με το \cite{Bulow2010}, ο FMT χρησιμοποιείται σε συνθήκες υποβρύχιας
χαρτογράφησης στο \cite{Hurtos2012}. Οι συγγραφείς σημειώνουν ότι η χρήση της
οπτικών καμερών είναι απαγορευτική σε υποβρύχιες καταστάσεις λόγω του
περιορισμένου εύρους ορατότητάς τους, αλλά οι αισθητήρες σόναρ δεν επηρεάζονται
από αυτή την άποψη. Ωστόσο, η εχθρικότητα του περιβάλλοντος, σε συνδυασμό με
την ακουστική φύση της αρχής λειτουργίας των αισθητήρων του σόναρ θέτει σοβαρές
προκλήσεις, καθώς οι εικόνες του έχουν χαμηλή ανάλυση, χαμηλό λόγο σήματος προς
θόρυβο, ενώ είναι ιδιαίτερα ευαίσθητες σε ανομοιογενή ηχοβολισμό
(insonification), και σε αλλοιώσεις της έντασης λόγω αλλαγών της οπτικής γωνίας
του αισθητήρα. Οι δυσκολίες αυτές εμποδίζουν την επιτυχή λειτουργία των μεθόδων
που βασίζονται σε χαρακτηριστικά \cite{Kim2005,Lowe2004}, ειδικά όταν οι πρέπει
να δημιουργηθούν ακριβή κλεισίματα βρόχων. Θέτοντας το πρόβλημα χαρτογράφησης
με ένα αισθητήρα σόναρ ως πρόβλημα βελτιστοποίησης γράφου με βάση τη στάση, οι
συγγραφείς δείχνουν ότι η ευθυγράμμιση εικόνων σόναρ με βάση τον FMT είναι
εύρωστη στις προαναφερθείσες πηγές εμποδίων και στην έλλειψη χαρακτηριστικών,
με τη μέθοδό τους να παράγει συνολικά συνεπή αποτελέσματα.

Ο FMT χρησιμοποιείται στο \cite{Oberlander2013} στα πλαίσια των δημιουργούμενων
από SLAM χαρτών πλέγματος μέσω της χρήσης τυπικών αισθητήρων δισδιάστατων
σαρώσεων απόστασης. Ο FMT χρησιμοποιείται για την ευθυγράμμισης υπο-χαρτών
μεταξύ τους: κατά τη διάρκεια της χαρτογράφησης, κάθε φορά που ένας σταθερός
αριθμός νέων μετρήσεων έχει υποστεί επεξεργασία, δημιουργείται ένας τοπικός
υπο-χάρτης ο οποίος αποθηκεύεται σε μια βάση δεδομένων, μαζί με τη στάση του σε
σχέση με το σύστημα αναφοράς του συνολικού χάρτη, και τη στάση του ρομπότ σε
σχέση με τον υποχάρτη. Όταν χρειάζεται να γίνει κλείσιμο βρόχου ή όταν δύο
χάρτες από δύο διαφορετικές συνεδρίες χαρτογράφησης του ίδιου περιβάλλοντος
πρέπει να συγχωνευθούν, ο FMT χρησιμοποιείται για την εκτίμηση του σχετικού
μετασχηματισμού μεταξύ του τελευταίου κατασκευασμένου υποχάρτη και ενός
υποχάρτη που δημιουργήθηκε από την προηγούμενη φορά που το ρομπότ επισκέφθηκε
το ίδιο μέρος, ο οποίος είναι αποθηκευμένος στη μνήμη, ή μεταξύ των υποχαρτών
που δημιουργήθηκαν και αποθηκεύτηκαν κατά τη διάρκεια των δύο διαφορετικών
συνεδριών, αυξάνοντας έτσι την επιχειρησιακή λειτουργικότητα, την αξιοπιστία,
και το χρόνο λειτουργίας ενός τυπικού αλγορίθμου SLAM.

Στo \cite{Rohde2016} o FMT χρησιμοποιείται στa πλαίσιο του εντοπισμού ενός
οχήματος σε εξωτερικούς χώρους με την χρήση ενός αισθητήρα lidar μετρήσεων
τρίων διαστάσεων, οδομετρίας, και ενός αισθητήρα GPS, ως μέσο υπολογισμού μιας
πρόσθετης πηγής οδομετρίας και της παρακολούθησης της στάσης του οχήματος. Όσον
αφορά στην οδομετρία, διαδοχικές τρισδιάστατες σαρώσεις προβάλλονται στο
οριζόντιο επίπεδο, μετατρέπονται σε πλέγματα, και ευθυγραμμίζονται μεταξύ τους
με τη χρήση FMT-SPOMF. Οι εξαγόμενες παράμετροι μετασχηματισμού του τελευταίου
παρέχουν τη μετατόπιση και την περιστροφή μιας σάρωσης σε σχέση με αυτήν που
προηγήθηκε, και επομένως εκείνες μεταξύ των δύο διαδοχικών στάσεων από τις
οποίες αποτυπώθηκαν οι μετρήσεις. Όσον αφορά στην εκτίμηση στάσης, μια
τρισδιάστατη σάρωση μετατρέπεται σε εικόνα πλέγματος μετά την προβολή της στο
οριζόντιο επίπεδο, με κέντρο τη στάση που μετράει ο αισθητήρας GPS. Στη
συνέχεια αυτή η εικόνα ευθυγραμμίζεται με έναν χάρτη του οποίου οι διαστάσεις
εξαρτώνται από την αβεβαιότητα της στάσης του οχήματος. Η χονδροειδής γνώση της
θέσης του αισθητήρα lidar καθιστά δυνατή την εξαγωγή μίας διόρθωσης της
εκτίμησης της στάσης που παρέχεται από τον αισθητήρα GPS.
