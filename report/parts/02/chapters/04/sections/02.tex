Έστω οι παραδοχές του προβλήματος ??. Έστω επιπλέον ότι $\hat{\bm{l}} =
\bm{l}$, δηλαδή μόνο ο προσανατολισμός του αισθητήρα πρέπει να εκτιμηθεί. Τότε
ας υπολογιστεί η εικονική σάρωση $\mathcal{S}_V$ μέσω δεσμοβολής (raycasting)
από την εκτίμηση $\hat{\bm{p}}$ στον χάρτη $\bm{M}$. Η εκτίμηση της περιστροφής
της εικονικής σάρωσης $\mathcal{S}_V$ σε σχέση με την πραγματική σάρωση
$\mathcal{S}_R$ μπορεί να βρεθεί μέσω των μεθόδων που παρουσιάζονται στις
ενότητες \ref{subsection:02_04_02:01}, \ref{subsection:02_04_02:02}, και
\ref{subsection:02_04_02:03}. Το σφάλμα της εκτίμησης μπορεί να μειωθεί
περαιτέρω μέσω της μεθόδου που παρουσιάζεται στην ενότητα
\ref{subsection:02_04_02:05}.

Στα συμφραζόμενα του παρόντος κεφαλαίου, έστω $\mathcal{F}\{\mathcal{S}\}$ ο
διακριτός μετασχηματισμός Fourier του σήματος $\mathcal{S}$,
$\mathcal{F}^{-1}\{\mathcal{S}\}$ ο αντίστροφός του, $\bm{c}^\ast$ ο συζυγής
του μιγαδικού αριθμού $\bm{c}$, και $|\bm{c}|$ το μέτρο του.

%%%%%%%%%%%%%%%%%%%%%%%%%%%%%%%%%%%%%%%%%%%%%%%%%%%%%%%%%%%%%%%%%%%%%%%%%%%%%%%%
\subsection{Η μέθοδος Fourier-Mellin σε μία διάσταση}
\label{subsection:02_04_02:01}

Έστω ότι ο χώρος δειγματοληπτείται αρκετά πυκνά γωνιακά, τότε για
$k,\xi \in \mathbb{Z}$: $k,\xi \in [0, N_s-1]$:
\begin{align}
  \mathcal{S}_V[k] &\simeq \mathcal{S}_R[(k - \xi) \mod N_s] \Rightarrow \nonumber \\
  \mathcal{F}\{\mathcal{S}_V\}(u) &\simeq e^{-j 2\pi \xi u / N_s} \cdot \mathcal{F}\{\mathcal{S}_R\}(u) \nonumber
\end{align}
και, επομένως, αφού $2\pi \dfrac{\xi}{N_s} = \xi \dfrac{2\pi}{N_s} = \xi \gamma$,
όπου $\gamma$ είναι η διακριτική γωνία του αισθητήρα:
\begin{align}
  Q_{\mathcal{S}_V, \mathcal{S}_R}(u) & \triangleq \dfrac{\mathcal{F}\{\mathcal{S}_V\}^{\ast} \cdot \mathcal{F}\{\mathcal{S}_R\}}{|\mathcal{F}\{\mathcal{S}_V\}| \cdot |\mathcal{F}\{\mathcal{S}_R\}|} \nonumber \\
  &\simeq \dfrac{e^{-j \xi \gamma u} \cdot \mathcal{F}\{\mathcal{S}_R\}^\ast \cdot \mathcal{F}\{\mathcal{S}_R\}}{|e^{- j \xi \gamma u} \cdot \mathcal{F}\{\mathcal{S}_R\}^\ast | \cdot | \mathcal{F}\{\mathcal{S}_R\}|} \nonumber \\
  &= e^{-j \xi \gamma u} \cdot \dfrac{\mathcal{F}\{\mathcal{S}_R\}^\ast \cdot \mathcal{F}\{\mathcal{S}_R\}}{|\mathcal{F}\{\mathcal{S}_R\} | \cdot | \mathcal{F}\{\mathcal{S}_R\}|} \nonumber \\
  &= e^{-j \xi \gamma u}
  \label{eq:Q0}
\end{align}

Συνεπώς ο αντίστροφος διακριτός μετασχηματιμός Fourier του
$Q_{\mathcal{S}_V, \mathcal{S}_R}$ είναι μία Kronecker $\delta$-συνάρτηση
$q_{\mathcal{S}_V, \mathcal{S}_R} = \mathcal{F}^{-1}\{Q_{\mathcal{S}_V, \mathcal{S}_R}\}$
με κέντρο $\xi$:
\begin{align}
  \xi = \operatorname*{arg\,max}\limits_u \ q_{\mathcal{S}_V, \mathcal{S}_R}(u)
\end{align}

Εάν η διαφορά του προσανατολισμού μεταξύ των στάσεων από τις οποίες λήφθησαν οι
σαρώσεις $\mathcal{S}_R$ και $\mathcal{S}_V$ είναι $\Delta\theta$, τότε
$\Delta\theta = \xi\gamma + \delta\theta$, όπου
$\mod(|\delta\theta|, \gamma) = \phi \in [0,\gamma/2]$. Επομένως, για δεδομένο
αριθμό εκπεμπόμενων ακτίνων $N_s$ (ισοδύναμα, για δεδομένη διακριτική γωνία
$\gamma$), ενημερώνοντας την εκτίμηση προσανατολισμού σε $\hat{\theta}^\prime$:
\begin{align}
  \hat{\theta}^\prime = \hat{\theta} + \xi \gamma \label{eq:update_t1}
\end{align}
οδηγεί σε ένα σφάλμα προσανατολισμού $\phi$:
\begin{align}
  \phi \leq \dfrac{\gamma}{2}  \label{eq:phi_1}
\end{align}

Η συμβολή αυτού του σφάλματος στην
σάρωσης-αντιστοίχισης είναι διπλή, καθώς η ύπαρξή του διαδίδεται επίσης στο
μέθοδο εκτίμησης θέσης. ??


%%%%%%%%%%%%%%%%%%%%%%%%%%%%%%%%%%%%%%%%%%%%%%%%%%%%%%%%%%%%%%%%%%%%%%%%%%%%%%%%
\subsection{Η μέθοδος πρώτων αρχών}
\label{subsection:02_04_02:02}

Έστω μία δισδιάστατη σάρωση $\mathcal{S}$ που έχει ληφθεί από τη στάση
$(x,y,\theta)$ σε κάποιο σύστημα συντεταγμένων (ορισμός \ref{def:lidar}). Έστω
ότι το γωνιακό εύρος της $\mathcal{S}$ είναι $\lambda = 2\pi$. Οι συντεταγμένες
του τελικού σημείου της $n$-οστής ακτίνας της $\mathcal{S}$,
$n=0,1,\dots,N_s-1$, στο σύστημα συντεταγμένων είναι $(x_n,y_n)$:
\begin{align}
  x_n &= x + d_n \cos(\theta + \frac{2 \pi n}{N_s} - \pi) = -d_n \cos(\theta + \frac{2 \pi n}{N_s}) \label{eq:x_n}\\
  y_n &= y + d_n \sin(\theta + \frac{2 \pi n}{N_s} - \pi) = -d_n \sin(\theta + \frac{2 \pi n}{N_s}) \label{eq:y_n}
\end{align}
Σε αυτό το σημείο παρατηρούμε ότι $-(x_n-x)$ και $(y_n-y)$ είναι αντίστοιχα
το πραγματικό και το φανταστικό μέρος της μιγαδικής ποσότητας
\begin{align}
  d_n e^{-i(\theta + \frac{2 \pi n}{N_s})} &= d_n \cos(\theta + \frac{2 \pi n}{N_s}) - i \cdot d_n \sin(\theta + \frac{2 \pi n}{N}) \nonumber \\
                                         &\leftstackrel{(\ref{eq:x_n}),(\ref{eq:y_n})}{=} -(x_n-x) + i \cdot (y_n-y) \label{eq:dne_complex_x_y}
\end{align}
και, επομένως
\begin{align}
  d_n e^{-i 2 \pi n/N_s} &= e^{i\theta}(-(x_n-x) + i \cdot (y_n-y)) \label{eq:dne_complex}
\end{align}
Αθροίζοντας την εξίσωση (\ref{eq:dne_complex}) επί του συνόλου των $N_s$
ακτίνων λαμβάνουμε τον πρώτο όρο του διακριτού μετασχηματισμού Fourier
του σήματος $\{d_n\}$, $n=0,1,\dots,N_s-1$:
\begin{align}
  \mathcal{F}\{\mathcal{S}\} = \sum\limits_{n=0}^{N_s-1}d_n \cdot e^{-i 2 \pi n/N_s} \ &\leftstackrel{(\ref{eq:dne_complex})}{=} \ \sum\limits_{n=0}^{N_s-1} e^{i\theta}(-(x_n-x) + i \cdot (y_n-y)) \nonumber  \\
           &= e^{i\theta} \sum\limits_{n=0}^{N_s-1} [ (x- i \cdot y) + (-x_n + i \cdot y_n) ]\nonumber \\
           &= e^{i\theta} N_s (x-i\cdot y) - e^{i\theta}\Delta \label{eq:F1}
\end{align}
όπου $\Delta \triangleq \sum\limits_{n=0}^{N_s-1} (x_n -i \cdot y_n)$.


Συμβολίζοντας με το γράμμα $R$ τις ποσότητες που αντιστοιχούν στην πραγματική
σάρωση $\mathcal{S}_R$, η οποία έχει ληφθεί από τη στάση του φυσικού αισθητήρα
$\bm{p}(x,y,\theta)$, και με $V$ εκείνες που αντιστοιχούν στην εικονική σάρωση
$\mathcal{S}_V$, η οποία έχει ληφθεί από τη στάση
$\hat{\bm{p}}(x,y,\hat{\theta})$:
\begin{align}
  \mathcal{F}\{\mathcal{S}_R\} &= \sum\limits_{n=0}^{N_s-1}d_n^R \cdot e^{-i 2 \pi n/N_s} \stackrel{(\ref{eq:F1})}{=} N_s e^{i\theta}(x-i\cdot y) - e^{i\theta} \Delta_R \label{eq:dne_complex_r} \\
  \mathcal{F}\{\mathcal{S}_V\} &= \sum\limits_{n=0}^{N_s-1}d_n^V \cdot e^{-i 2 \pi n/N_s} \stackrel{(\ref{eq:F1})}{=} N_s e^{i\hat{\theta}}(x-i\cdot y) - e^{i\hat{\theta}} \Delta_V \label{eq:dne_complex_v}
\end{align}

Έστω τώρα ότι
\begin{align}
  \Delta_R - \Delta_V &= \sum\limits_{n=0}^{N_s-1}(x_n^R-x_n^V) - i \cdot \sum\limits_{n=0}^{N_s-1}(y_n^R-y_n^V) \nonumber \\
                      &= N_s (\delta_x - i \cdot \delta_y)
\end{align}
όπου
\begin{align}
  \delta_x &\triangleq \frac{1}{N_s}\sum\limits_{n=0}^{N_s-1}(x_n^R-x_n^V) \label{eq:delta_x}\\
  \delta_y &\triangleq \frac{1}{N_s}\sum\limits_{n=0}^{N_s-1}(y_n^R-y_n^V) \label{eq:delta_y}
\end{align}
τότε
\begin{align}
  \Delta_V &= \Delta_R -N_s(\delta_x - i \cdot \delta_y) \label{eq:Delta_V_eq_Delta_R}
\end{align}

Ο πρώτος όρος του διακριτού μετασχηματισμού Fourier του σήματος που
αποτελείται από τη διαφορά των δύο σημάτων (\ref{eq:dne_complex_r}) και
(\ref{eq:dne_complex_v}) είναι:
\begin{align}
  \mathcal{F}\{\mathcal{S}_R\} - \mathcal{F}\{\mathcal{S}_V\} &= \sum\limits_{n=0}^{N_s-1}(d_n^R - d_n^V) \cdot e^{-i 2 \pi n/N_s} \nonumber \\
           &\leftstackrel{(\ref{eq:dne_complex_r}),(\ref{eq:dne_complex_v})}{=} N_s (x-i \cdot y)(e^{i\theta}-e^{i\hat{\theta}}) -e^{i\theta}\Delta_R+e^{i\hat{\theta}}\Delta_V \nonumber \\
          &\leftstackrel{(\ref{eq:Delta_V_eq_Delta_R})}{=} N_s (x-i \cdot y)(e^{i\theta}-e^{i\hat{\theta}}) -e^{i\theta}\Delta_R +e^{i\hat{\theta}}(\Delta_R -N_s(\delta_x - i \cdot \delta_y)) \nonumber \\
          &= N_s (x-i \cdot y)(e^{i\theta}-e^{i\hat{\theta}}) - \Delta_R (e^{i\theta}-e^{i\hat{\theta}})- N_s e^{i\hat{\theta}}(\delta_x - i\cdot \delta_y) \nonumber \\
          &= (e^{i\theta}-e^{i\hat{\theta}}) [N_s(x-i\cdot y) - \Delta_R]- N_s e^{i\hat{\theta}}(\delta_x - i\cdot \delta_y) \nonumber   \\
          &\leftstackrel{(\ref{eq:dne_complex_r})}{=} (e^{i\theta}-e^{i\hat{\theta}}) \dfrac{\mathcal{F}\{\mathcal{S}_R\}}{e^{i\theta}}- N_s e^{i\hat{\theta}}(\delta_x - i\cdot \delta_y) \nonumber \\
          &= (1-e^{-i(\theta-\hat{\theta})}) \mathcal{F}\{\mathcal{S}_R\}- N_s e^{i\hat{\theta}}(\delta_x - i\cdot \delta_y) \nonumber
\end{align}
άρα
\begin{align}
  -\mathcal{F}\{\mathcal{S}_V\} &= -e^{-i(\theta-\hat{\theta})} \mathcal{F}\{\mathcal{S}_R\}- N_s e^{i\hat{\theta}}(\delta_x - i\cdot \delta_y) \nonumber \\
  e^{-i(\theta-\hat{\theta})} &= \dfrac{\mathcal{F}\{\mathcal{S}_V\}}{\mathcal{F}\{\mathcal{S}_R\}} - \dfrac{N_s e^{i\hat{\theta}}}{\mathcal{F}\{\mathcal{S}_R\}}(\delta_x - i\cdot \delta_y) \nonumber
\end{align}
Χρησιμοποιώντας την πολική αναπαράσταση $\bm{A} = |\bm{A}| e^{i\angle \bm{A}}$:
\begin{align}
  e^{-i(\theta-\hat{\theta})} &= \dfrac{|\mathcal{F}\{\mathcal{S}_V\}|}{|\mathcal{F}\{\mathcal{S}_R\}|} e^{i(\angle \mathcal{F}\{\mathcal{S}_V\} - \angle \mathcal{F}\{\mathcal{S}_R\})} - \dfrac{e^{i(\hat{\theta}-\angle \mathcal{F}\{\mathcal{S}_R\})}}{|\mathcal{F}\{\mathcal{S}_R\}|} (N_s \delta_x - i\cdot N_s \delta_y) \label{eq:x1_final_big_eq}
\end{align}

Λόγω του γεγονότος ότι ο προσανατολισμός $\theta$ του αισθητήρα είναι άγνωστος,
τα τελικά σημεία $\{(x_n^R,y_n^R)\}$ καθίστανται ομοίως άγνωστα, και συνεπώς
και οι ποσότητες $\delta_x, \delta_y$. Προκειμένου να αποκτήσουμε μια αρχική
διαίσθηση ως προς τα μέτρα των τελευταίων κάνουμε την παρατήρηση ότι, εξ
ορισμού, οι ποσότητες $N_s \delta_x$ και $N_s \delta_y$ ποσοτικοποιούν τη
διαφορά της προσέγγισης των επικαμπύλιων ολοκληρωμάτων επί των καμπύλων
που ορίζουν τα τελικά σημεία των δύο σαρώσεων στους δύο κύριους άξονες $x$ και
$y$. Η προσέγγιση αυτή οφείλεται στο πεπερασμένο μέγεθος των εκπεμπόμενων ακτίνων
$N_s$. Επομένως υπό τις υποθέσεις ότι (α) ο χάρτης του περιβάλλοντος είναι
τέλεια αναπαράστασή του και (β) οι μετρήσεις του φυσικού αισθητήρα δεν
επηρεάζονται από διαταραχές: καθώς $N_s \rightarrow \infty$, $N_s \delta_x$,
$N_s \delta_y$ $\rightarrow 0$, τα οποία με τη σειρά τους σημαίνουν λόγω της
εξίσωσης (\ref{eq:x1_final_big_eq}) ότι
$\dfrac{|\mathcal{F}\{\mathcal{S}_V\}|}{|\mathcal{F}\{\mathcal{S}_R\}|} \rightarrow 1$
και $\theta-\hat{\theta} \rightarrow \angle \mathcal{F}\{\mathcal{S}_R\} - \angle \mathcal{F}\{\mathcal{S}_V\}$.

Ενημερώνοντας την εκτίμηση προσανατολισμού σε $\hat{\theta}^\prime$:
\begin{align}
\hat{\theta}^\prime = \hat{\theta} + \angle \mathcal{F}\{\mathcal{S}_R\} - \angle\mathcal{F}\{\mathcal{S}_V\}
  \label{eq:update_t2}
\end{align}
οδηγεί σε ένα σφάλμα προσανατολισμού $\phi$:
\begin{align}
  \phi &= \tan^{-1}\dfrac{N_s \delta_x \tan(\theta - \angle \mathcal{F}\{\mathcal{S}_R\}) - N_s \delta_y}{|\mathcal{F}\{\mathcal{S}_R\}| + N_s \delta_x + N_s \delta_y \tan(\theta - \angle \mathcal{F}\{\mathcal{S}_R\})} \label{eq:phi2}
\end{align}
του οποίου το μέτρο είναι αντιστρόφως ανάλογο του αριθμού των ακτίνων $N_s$ που
εκπέμπει ο αισθητήρας στην περίπτωση που τόσο η πραγματική μέτρηση
$\mathcal{S}_R$ όσο και η εικονική σάρωση $\mathcal{S}_V$ δεν διαταράσσονται
από θόρυβο. Το πεπερασμένο των εκπεμπόμενων ακτίνων του φυσικού αισθητήρα, σε
συνδυασμό με το αυθαίρετο του ρυθμού των αλλαγών του περιβάλλοντος (σχήμα
\ref{fig:02_02:the_problem}), μπορούν να οδηγήσουν σε υποδειγματοληψία τμημάτων
του περιβάλλοντος και του χάρτη του. Επιπλέον, ο αριθμός των εκπεμπόμενων
ακτίνων από τον φυσικό αισθητήρα είναι αμετάβλητος. ??

Η συμβολή αυτού του σφάλματος στην
σάρωσης-αντιστοίχισης είναι διπλή, καθώς η ύπαρξή του διαδίδεται επίσης στο
μέθοδο εκτίμησης θέσης. ?? (και στα τρια subsections γραφεται αυτη η παραγραφος)



%%%%%%%%%%%%%%%%%%%%%%%%%%%%%%%%%%%%%%%%%%%%%%%%%%%%%%%%%%%%%%%%%%%%%%%%%%%%%%%%
\subsection{Η μέθοδος του Προκρούστη}
\label{subsection:02_04_02:03}

Έστω ότι η προβολή των τελικών σημείων των ακτίνων της σάρωσης $\mathcal{S}_V$
γύρω από τη στάση $\hat{\bm{p}}(x,y,\hat{\theta})$ παράγει το σύνολο σημείων
$\bm{P}_V$ στο οριζόντιο επίπεδο. Έστω ότι η ίδια προβολή για τη σάρωση
$\mathcal{S}_R$ ως προς τη στάση $\bm{p}(x,y,\theta)$ παράγει το σύνολο
$\bm{P}_R$. Η περιστροφή της στάσης $\hat{\bm{p}}$ που ευθυγραμμίζει βέλτιστα
το σύνολο σημείων $\bm{P}_V$ σε σχέση με το $\bm{P}_R$ μπορεί να βρεθεί από τη
λύση του Ορθογώνιου Προσκρούστειου πρόβληματος \cite{Schonemann1966a} για
πίνακες εισόδου $\bm{P}_V$ και $\bm{P}_R$. Στην περίπτωση που ο πίνακας
μετασχηματισμού περιορίζεται στο να έχει τη δομή πίνακα περιστροφής $\bm{R}$:
$\det{(\bm{R})} = 1$, το πρόβλημα ευθυγράμμισης ονομάζεται Περιορισμένο
Ορθογώνιο Προσκρούστειο πρόβλημα. Η λύση του δίνεται στο \cite{Umeyama1991} και
περιγράφεται παρακάτω.

Δεδομένου ότι στα συμφραζόμενα του προβλήματός ?? μας η θέση $\bm{l}$ είναι
άγνωστη, τα τελικά σημεία κάθε σάρωσης λαμβάνονται με την προβολή κάθε σάρωσης
στο επίπεδο $x-y$ σύμφωνα με το τοπικό σύστημα αναφοράς της κάθεμίας, δηλαδή
σαν να είχε ληφθεί η κάθε μιά από το $O(0,0,0)$. Ο πίνακας περιστροφής $\bm{R}$
που ευθυγραμμίζει βέλτιστα το σύνολο $\bm{P}_V$ με το $\bm{P}_R$ είναι ο
πίνακας που ελαχιστοποιεί την απόκλιση των περιεστραμμένων σημείων
$\bm{R}\bm{P}_V$ από το $\bm{P}_R$:
\begin{align}
  \operatorname*{arg\,min}\limits_{\bm{R}} \|\bm{P}_R - \bm{R} \cdot \bm{P}_V\|_F^2 \nonumber
\end{align}
όπου $\|\bm{A}\|_F = (\bm{A}^\top\bm{A})^{1/2}$ δηλώνει το μέτρο Frobenius του
πίνακα πραγματικών τιμών $\bm{A}$. Έστω ο τελεστής $\text{tr}(\bm{A})$ να
δηλώνει το ίχνος του πίνακα $\bm{A}$. Τότε
\begin{align}
  \|\bm{P}_R - \bm{R} \bm{P}_V\|_F^2 = \text{tr}(\bm{P}_R^\top \bm{P}_R + \bm{P}_V^\top \bm{P}_V) - 2 \text{tr}(\bm{R} \bm{P}_R \bm{P}_V^\top)
  \label{eq:expanded_frob_norm}
\end{align}

Δεδομένου ότι μόνο ο δεύτερος όρος της δεξιάς πλευράς εξαρτάται από τον πίνακα
$\bm{R}$, αρκεί να βρεθεί ο πίνακας περιστροφής $\bm{R}$ που μεγιστοποιεί
το ίχνος $\text{tr}(\bm{R} \bm{P}_V \bm{P}_R^\top)$. Ο βέλτιστος πίνακας
$\bm{R}$ δίνεται από το λήμμα \ref{lm:umeyama}:

\begin{lemma}
  \label{lm:umeyama}
  Έστω $\bm{P}_R$ και $\bm{P}_V$ πίνακες διαστάσεων $2 \times N_s$, $\bm{R}$
  πίνακας διαστάσεων $2 \times 2$, και $\bm{U} \bm{D} \bm{V}^\top$ η αποσύνθεση
  του $\bm{P}_R \bm{P}_V^\top$ σε ιδιάζουσες τιμές (Singular Value
  Decomposition---SVD). Τότε ο πίνακας $\bm{R}$ που ελαχιστοποιεί το μέτρο
  $\|\bm{P}_R - \bm{R} \cdot \bm{P}_V\|_F^2$ δίνεται από τη σχέση
  $\bm{R} = \bm{U} \bm{S} \bm{V}^\top$, όπου
  $\bm{S} = \text{diag}(1,\det{(\bm{U}\bm{V})})$.
\end{lemma}

\begin{corollary}
  \label{col:umeyama}
  Η τιμή του μέγιστου ίχνους
  $T(\bm{P}_R, \bm{P}_V) \triangleq \max\text{tr}(\bm{R} \bm{P}_R \bm{P}_V^\top)$
  είναι
  \begin{align}
  \max\text{tr}(\bm{R} \bm{P}_R \bm{P}_V^\top) = \text{tr}(\bm{D}\bm{S})
  \end{align}
\end{corollary}

Το παραπάνω λήμμα παρέχει τον βέλτιστο πίνακα περιστροφής $\bm{R}$ υπό την
προϋπόθεση ότι τόσο το σύνολο $\bm{P}_R$ όσο και το $\bm{P}_V$ είναι γνωστά.
Ωστόσο, στα συμφραζόμενα του προβλήματος ?? τα τελικά σημεία $\bm{P}_R$
υπολογίζονται από έναν αυθαίρετο προσανατολισμό επειδή ο επιθυμητός
προσανατολισμός είναι θεμελιωδώς άγνωστος. Επομένως, ο υπολογισμός του πίνακα
$\bm{R}$ και η εξαγωγή  του σχετικού προσανατολισμού του $\bm{P}_V$ σε σχέση με
το $\bm{P}_R$ από τον πίνακα $\bm{R}$ \textit{σε ένα βήμα} είναι αδύνατη. Αυτό
που μπορεί να γίνει σε αυτή την περίπτωση είναι το εξής. Υπολογίζεται το
γινόμενο $\bm{P}_R \bm{P}_V^\top$ σε $O(N_s^2)$, η αποσύνθεσή του σε ιδιάζουσες
τιμές σε $O(1)$, καταγράφεται η τιμή του ίχνους $\text{tr}(\bm{D}\bm{S})$ σε
$O(1)$, μετατοπίζεται ο πίνακας $\bm{P}_V$ κατά στήλες προς τα αριστερά μία
φορά,  και επαναλαμβάνεται αυτή τη διαδικασία $N_s-1$ φορές. Έστω ότι η
επανάληψη $i$ να καταγράψει το μέγιστο ίχνος:---τότε η περιστροφή της στάσης
$\hat{\bm{p}}$ κατά $\hat{\theta}^\prime = 2 \pi i / N_s$ μεγιστοποιεί το ίχνος
$\text{tr}(\bm{R} \bm{P}_R \bm{P}_V^\top)$ και ελαχιστοποιεί το μέτρο του
σφάλματος ευθυγράμμισης σε (\ref{eq:expanded_frob_norm}) για μία δεδομένη
διακριτική γωνία $\gamma$. Η παραπάνω διαδικασία αποδίδει τη βέλτιστη
περιστροφή επειδή το ίχνος $\text{tr}(\bm{D}\bm{S})$ ουσιαστικά αναλαμβάνει το
ρόλο ενός μέτρου ευθυγράμμισης μεταξύ των συνόλων σημείων $\bm{P}_V$ και
$\bm{P}_R$.

Η παραπάνω διαδικασία καταγραφής $N_s$ ιχνών μπορεί να υπολογιστεί είτε με ευθύ
τρόπο πολυπλοκότητας $O(N_s^3)$, είτε μέσω με της μεθόδου που παρουσιάζεται στο
\cite{Dogan2015} σε $O(N_s \log N_s)$, η οποία θα αναφέρεται στο εξής ως
μέθοδος DBH. Η μέθοδος αυτή περιγράφεται παρακάτω.

% ΤΟDO απο εδω και κατω

Έστω $\mathcal{F}\{\cdot\}$ που συμβολίζει τον εμπρόσθιο διακριτό
μετασχηματισμό Fourier και $\mathcal{F}^{-1}\{\cdot\}$ το αντίστροφό του. Έστω
επίσης $\widetilde{\bm{A}}$ που συμβολίζει τον πίνακα $\bm{A}$ με αντίστροφη
σειρά στηλών, $\bm{P}_R = [\bm{p}_R^x; \bm{p}_R^y]$, $\widetilde{\bm{P}}_V =
[\bm{p}_V^x; \bm{p}_V^y]$, και ο τελεστής $\odot$ υποδηλώνουν τον
πολλαπλασιασμό κατά στοιχείο. Τότε τέσσερα διανύσματα μεγέθους $N_s$ είναι
κατασκευάζονται:
\begin{align}
  \bm{m}_{11} = \mathcal{F}^{-1}\{ \mathcal{F}\{\bm{p}_R^x\} \odot \mathcal{F}\{\bm{p}_V^x\} \} \} \nonumber  \\
  \bm{m}_{12} = \mathcal{F}^{-1}\{ \mathcal{F}\{\bm{p}_R^y\} \odot \mathcal{F}\{\bm{p}_V^x\} \} \} \nonumber \\
  \bm{m}_{21} = \mathcal{F}^{-1}\{ \mathcal{F}\{\bm{p}_R^x\} \odot \mathcal{F}\{\bm{p}_V^y\} \} \} \nonumber \\
  \bm{m}_{22} = \mathcal{F}^{-1}\{ \mathcal{F}\{\bm{p}_R^y\} \odot \mathcal{F}\{\bm{p}_V^y\} \} \} \nonumber
\end{align}
Μετά τον υπολογισμό των πινάκων $\bm{m}_{kj}$, $k,j = 1,2$, $N_s$ μεγέθους
$2\times2$ κατασκευάζονται σύμφωνα με:
\begin{align}
  \bm{M}_i =
  \begin{bmatrix}
    \bm{m}_{11}^i & \bm{m}_{12}^i \\
    \bm{m}_{21}^i & \bm{m}_{22}^i
  \end{bmatrix}
\end{align}
όπου $i = 0,\dots,N-1$, και $\bm{m}_{kj}^i$ είναι το $i$-οστό στοιχείο του
διανύσματος $\bm{m}_{kj}$. Ο πίνακας $\bm{M}_i$ είναι ίσος με τον πίνακα
$\bm{P}_R (\bm{P}_V^{N_s-1-i})^\top$, όπου χρησιμοποιείται ο συμβολισμός
$\bm{A}^k$.  για να δηλώσει τον πίνακα $\bm{A}$ του οποίου οι στήλες
μετατοπίζονται $k$ φορές προς τα αριστερά.  Η απόδειξη επικαλείται το θεώρημα
κυκλικής συνέλιξης του DFT και παραλείπεται.

Αφού υπολογιστούν και σχηματιστούν όλοι οι $N_s$ $\bm{M}_i$ πίνακες, κάθε
πίνακας είναι αποσυντίθεται με τη χρήση της αποσύνθεσης μοναδιαίων τιμών. Το
ίχνος κάθε $\bm{R}_i \bm{M}_i$ καταγράφεται με την εφαρμογή του λήμματος
\ref{lm:umeyama} και την επίκληση του Πόρισμα \ref{col:umeyama}. Ας καταγραφεί
το μέγιστο ίχνος κάτω από τον δείκτη $i$, τότε περιστρέφοντας το $\hat{\bm{p}}$
κατά $\hat{\theta}^\prime = 2 \pi (N_s-1-i)/ N_s$ επιτυγχάνει το ίδιο
αποτέλεσμα με την αφελή μέθοδο υψηλότερης πολυπλοκότητας για μια δεδομένη
προσαύξηση γωνίας του αισθητήρα εμβέλειας $\gamma$. Αλγόριθμος
\ref{alg:algorithm_ufrc} παρουσιάζει τη διαδικασία διόρθωσης προσανατολισμού
που χρησιμοποιεί τη μέθοδο DBH σε ψευδοκώδικα.

\begin{algorithm}[]
  \caption{\texttt{uf\_rc}}
  \begin{spacing}{1.2}
  \begin{algorithmic}[1]
    \REQUIRE $\bm{P}_R$, $\bm{P}_V$
    \ENSURE $i$, $T(\bm{P}_R, \bm{P}_V)$
    \STATE \texttt{reverse}$(\bm{P}_V)$
    \STATE $\bm{p}_R^x \leftarrow \text{first row of } \bm{P}_R$
    \STATE $\bm{p}_R^y \leftarrow \text{second row of } \bm{P}_R$
    \STATE $\bm{p}_V^x \leftarrow \text{first row of } \bm{P}_V$
    \STATE $\bm{p}_V^y \leftarrow \text{second row of } \bm{P}_V$
    \STATE $\bm{m}_{11} \leftarrow \texttt{IDFT}(\texttt{DFT}(\bm{p}_R^x) \odot \texttt{DFT}(\bm{p}_V^x))$
    \STATE $\bm{m}_{12} \leftarrow \texttt{IDFT}(\texttt{DFT}(\bm{p}_R^y) \odot \texttt{DFT}(\bm{p}_V^x))$
    \STATE $\bm{m}_{21} \leftarrow \texttt{IDFT}(\texttt{DFT}(\bm{p}_R^x) \odot \texttt{DFT}(\bm{p}_V^y))$
    \STATE $\bm{m}_{22} \leftarrow \texttt{IDFT}(\texttt{DFT}(\bm{p}_R^y) \odot \texttt{DFT}(\bm{p}_V^y))$
    \STATE $\bm{T} \leftarrow \{\varnothing\}$
    \FOR{\texttt{$i = 0:N_s-1$}}
      \vspace{0.1cm}
      \STATE $\bm{M}_i \leftarrow
        \begin{bmatrix}
          \bm{m}_{11}(i) & \bm{m}_{12}(i) \\
          \bm{m}_{21}(i) & \bm{m}_{22}(i)
        \end{bmatrix}$
        \vspace{0.1cm}
        \STATE $(\bm{U}, \bm{D}, \bm{V}) \leftarrow \texttt{SVD}(\bm{M}_i)$
        \STATE Append $\texttt{trace}(\bm{D} \cdot \texttt{diag}(1,\det(\bm{U}\bm{V})))$ to $\bm{T}$
    \ENDFOR
    \STATE \texttt{reverse}$(\bm{T})$
    \STATE $T_{\max} \leftarrow \max \{\bm{T}\}$
    \STATE $i_{\max} \leftarrow \texttt{index}(\bm{T}, T_{\max})$
    \RETURN $(i_{\max},T_{\max})$
  \end{algorithmic}
  \end{spacing}
  \label{alg:algorithm_ufrc}
\end{algorithm}

%%%%%%%%%%%%%%%%%%%%%%%%%%%%%%%%%%%%%%%%%%%%%%%%%%%%%%%%%%%%%%%%%%%%%%%%%%%%%%%%
\subsection{Ο ύφαλος της διακριτικής γωνίας του αισθητήρα}
\label{subsection:02_04_02:04}

%%%%%%%%%%%%%%%%%%%%%%%%%%%%%%%%%%%%%%%%%%%%%%%%%%%%%%%%%%%%%%%%%%%%%%%%%%%%%%%%
\subsection{Σίνις ο Πιτυοκάμπτης}
\label{subsection:02_04_02:05}

