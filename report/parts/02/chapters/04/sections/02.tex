Έστω οι παραδοχές του προβλήματος ??. Έστω επιπλέον ότι $\hat{\bm{l}} =
\bm{l}$, δηλαδή μόνο ο προσανατολισμός του αισθητήρα πρέπει να εκτιμηθεί. Τότε
ας υπολογιστεί η εικονική σάρωση $\mathcal{S}_V$ μέσω δεσμοβολής (raycasting)
από την εκτίμηση $\hat{\bm{p}}$ στον χάρτη $\bm{M}$. Η εκτίμηση της περιστροφής
της εικονικής σάρωσης $\mathcal{S}_V$ σε σχέση με την πραγματική σάρωση
$\mathcal{S}_R$ μπορεί να βρεθεί μέσω των μεθόδων που παρουσιάζονται στις
ενότητες \ref{subsection:02_04_02:01}, \ref{subsection:02_04_02:02}, και
\ref{subsection:02_04_02:03}. Το σφάλμα της εκτίμησης μπορεί να μειωθεί
περαιτέρω μέσω της μεθόδου που παρουσιάζεται στην ενότητα
\ref{subsection:02_04_02:05}.

%%%%%%%%%%%%%%%%%%%%%%%%%%%%%%%%%%%%%%%%%%%%%%%%%%%%%%%%%%%%%%%%%%%%%%%%%%%%%%%%
\subsection{Μέσω πρώτων αρχών}
\label{subsection:02_04_02:01}


Έστω μια πανοραμική σάρωση εμβέλειας $\mathcal{S}$ που έχει ληφθεί από τη θέση
$(x,y,\theta)$.  σε κάποιο πλαίσιο συντεταγμένων (ορισμός
\ref{def:definition_1}). Οι συντεταγμένες του τελικού σημείου της $n$-οστής
ακτίνας της σάρωσης $n=0,1,\dots,N_s-1$ εντός του εν λόγω πλαισίου αναφοράς
είναι $(x_n,y_n)$:
\begin{align}
  x_n &= x + d_n \cos(\theta + \frac{2 \pi n}{N_s} - \pi) = -d_n \cos(\theta + \frac{2 \pi n}{N_s}) \label{eq:x_n}\\
  y_n &= y + d_n \sin(\theta + \frac{2 \pi n}{N_s} - \pi) = -d_n \sin(\theta + \frac{2 \pi n}{N_s}) \label{eq:y_n}
\end{align}
Εδώ κάνουμε την παρατήρηση ότι $-(x_n-x)$ και $(y_n-y)$ είναι αντίστοιχα,
το πραγματικό και το φανταστικό μέρος της μιγαδικής ποσότητας
\begin{align}
  d_n e^{-i(\theta + \frac{2 \pi n}{N_s})} &= d_n \cos(\theta + \frac{2 \pi n}{N_s}) - i \cdot d_n \sin(\theta + \frac{2 \pi n}{N}) \nonumber \\
                                         &\leftstackrel{(\ref{eq:x_n}),(\ref{eq:y_n})}{=} -(x_n-x) + i \cdot (y_n-y) \label{eq:dne_complex_x_y}
\end{align}
και, επομένως, ότι
\begin{align}
  d_n e^{-i \frac{2 \pi n}{N_s}} &= e^{i\theta}(-(x_n-x) + i \cdot (y_n-y)) \label{eq:dne_complex}
\end{align}
Το τελικό γινόμενο του αθροίσματος (\ref{eq:dne_complex}) επί $N_s$ ακτίνων είναι ίσο με
με τον πρώτο όρο του διακριτού μετασχηματισμού Fourier του σήματος $\{d_n\}$,
$n=0,1,\dots,N_s-1$, $\bm{F}_1$:
\begin{align}
  \bm{F}_1 &= \sum\limits_{n=0}^{N_s-1}d_n \cdot e^{-i \frac{2 \pi n}{N_s}} \nonumber \\
           &\leftstackrel{(\ref{eq:dne_complex})}{=} \sum\limits_{n=0}^{N_s-1} e^{i\theta}(-(x_n-x) + i \cdot (y_n-y)) \nonumber  \\
           &= e^{i\theta} \sum\limits_{n=0}^{N_s-1} [ (x- i \cdot y) + (-x_n + i \cdot y_n) ]\nonumber \\
           &= e^{i\theta} N_s (x-i\cdot y) - e^{i\theta}\Delta \label{eq:F1}
\end{align}
όπου $\Delta \triangleq \sum\limits_{n=0}^{N_s-1} (x_n -i \cdot y_n)$.


Συμβολίζοντας με το γράμμα $R$ τις ποσότητες που αντιστοιχούν στις
πραγματική σάρωση $\mathcal{S}_R$, η οποία έχει ληφθεί από τη θέση του αισθητήρα
$\bm{p}(x,y,\theta)$, και με $V$ εκείνες που αντιστοιχούν στην εικονική
σάρωση $\mathcal{S}_V$, η οποία έχει ληφθεί από τη θέση
$\hat{\bm{p}}(x,y,\hat{\theta})$:
\begin{align}
  \bm{R}_1 &= \sum\limits_{n=0}^{N_s-1}d_n^R \cdot e^{-i \frac{2 \pi n}{N_s}} \leftstackrel{(\ref{eq:F1})}{=} N_s e^{i\theta}(x-i\cdot y) - e^{i\theta} \Delta_R \label{eq:dne_complex_r} \\
  \bm{V}_1 &= \sum\limits_{n=0}^{N_s-1}d_n^V \cdot e^{-i \frac{2 \pi n}{N_s}} \leftstackrel{(\ref{eq:F1})}{=} N_s e^{i\hat{\theta}}(x-i\cdot y) - e^{i\hat{\theta}} \Delta_V \label{eq:dne_complex_v}
\end{align}

Έστω τώρα
\begin{align}
  \Delta_R - \Delta_V &= \sum\limits_{n=0}^{N_s-1}(x_n^R-x_n^V) - i \cdot \sum\limits_{n=0}^{N_s-1}(y_n^R-y_n^V)
                      &= N_s (\delta_x - i \cdot \delta_y)
\end{align}
όπου
\begin{align}
  \delta_x &\triangleq \frac{1}{N_s}\sum\limits_{n=0}^{N_s-1}(x_n^R-x_n^V) \label{eq:delta_x}\\
  \delta_y &\triangleq \frac{1}{N_s}\sum\limits_{n=0}^{N_s-1}(y_n^R-y_n^V) \label{eq:delta_y}
\end{align}
Τότε
\begin{align}
  \Delta_V &= \Delta_R -N_s(\delta_x - i \cdot \delta_y) \label{eq:Delta_V_eq_Delta_R}
\end{align}

Ο πρώτος όρος του διακριτού μετασχηματισμού Fourier του σήματος που
αποτελείται από τη διαφορά των δύο σημάτων (\ref{eq:dne_complex_r}) και
(\ref{eq:dne_complex_v}) είναι $\bm{X}_1$:
\begin{align}
  \bm{X}_1 &= \bm{R}_1 - \bm{V}_1 \nonumber \\
           &= \sum\limits_{n=0}^{N_s-1}(d_n^R - d_n^V) \cdot e^{-i \frac{2 \pi n}{N_s}} \nonumber \\
           &\leftstackrel{(\ref{eq:dne_complex_r}),(\ref{eq:dne_complex_v})}{=} N_s (x-i \cdot y)(e^{i\theta}-e^{i\hat{\theta}}) -e^{i\theta}\Delta_R+e^{i\hat{\theta}}\Delta_V \nonumber \\
          &\leftstackrel{(\ref{eq:Delta_V_eq_Delta_R})}{=} N_s (x-i \cdot y)(e^{i\theta}-e^{i\hat{\theta}}) -e^{i\theta}\Delta_R +e^{i\hat{\theta}}(\Delta_R -N_s(\delta_x - i \cdot \delta_y)) \nonumber \\
          &= N_s (x-i \cdot y)(e^{i\theta}-e^{i\hat{\theta}}) - \Delta_R (e^{i\theta}-e^{i\hat{\theta}})- N_s e^{i\hat{\theta}}(\delta_x - i\cdot \delta_y) \nonumber \\
          &= (e^{i\theta}-e^{i\hat{\theta}}) [N_s(x-i\cdot y) - \Delta_R]- N_s e^{i\hat{\theta}}(\delta_x - i\cdot \delta_y) \nonumber   \\
          &\leftstackrel{(\ref{eq:dne_complex_r})}{=} (e^{i\theta}-e^{i\hat{\theta}}) \dfrac{\bm{R}_1}{e^{i\theta}}- N_s e^{i\hat{\theta}}(\delta_x - i\cdot \delta_y) \nonumber \\
          &= (1-e^{-i(\theta-\hat{\theta})}) \bm{R}_1- N_s e^{i\hat{\theta}}(\delta_x - i\cdot \delta_y) \nonumber
\end{align}

Επομένως, αφού $\bm{X}_1 = \bm{R}_1 -\bm{V}_1$:

\begin{align}
  -\bm{V}_1 &= -e^{-i(\theta-\hat{\theta})} \bm{R}_1- N_s e^{i\hat{\theta}}(\delta_x - i\cdot \delta_y) \nonumber \\
  e^{-i(\theta-\hat{\theta})} &= \dfrac{\bm{V_1}}{\bm{R_1}} - \dfrac{N_s e^{i\hat{\theta}}}{\bm{R_1}}(\delta_x - i\cdot \delta_y) \nonumber \\
  e^{-i(\theta-\hat{\theta})} &= \dfrac{|\bm{V_1}|}{|\bm{R_1}|} e^{i(\angle \bm{V_1} - \angle \bm{R_1})} - \dfrac{e^{i(\hat{\theta}-\angle \bm{R_1})}}{|\bm{R_1}|} (N_s \delta_x - i\cdot N_s \delta_y) \nonumber
\end{align}
όπου η πολική αναπαράσταση του μιγαδικού $\bm{A}$ είναι
$\bm{A} = |\bm{A}| e^{i\angle \bm{A}}$.

Λόγω του γεγονότος ότι ο προσανατολισμός του αισθητήρα $\theta$ είναι άγνωστος, έτσι είναι και οι
τελικά σημεία $\{(x_n^R,y_n^R)\}$, και επομένως οι ποσότητες $\delta_x, \delta_y$.
Προκειμένου να αποκτήσουμε μια αρχική διαίσθηση ως προς τα μεγέθη των τελευταίων θα
κάνουμε την παρατήρηση ότι, εξ ορισμού, $N_s \delta_x$ και $N_s \delta_y$
ποσοτικοποιούν τη διαφορά της προσέγγισης των γραμμικών ολοκληρωμάτων πάνω από το κλειστό
διαδρομές που παρέχουν τα τελικά σημεία των δύο σαρώσεων πάνω στους δύο κύριους άξονες $x$ και
$y$. Η προσέγγιση αυτή οφείλεται στο πεπερασμένο του $N_s$. Επομένως, υπό την
υποθέσεων ότι (α) ο χάρτης του περιβάλλοντος είναι η τέλεια αναπαράστασή του
και (β) η φυσική σάρωση εύρους δεν επηρεάζεται από διαταραχές, καθώς $N_s
\rightarrow \infty$, $N_s \delta_x$, $N_s \delta_y$ $\rightarrow 0$, τα οποία σε
με τη σειρά του σημαίνει ότι $|\bm{V}_1| \rightarrow |\bm{R}_1|$ και $\theta-\hat{\theta}
\rightarrow \angle \bm{R_1} - \angle \bm{V_1}$.

Ενημέρωση της εκτίμησης προσανατολισμού με
\begin{align}
\hat{\theta}^\prime = \hat{\theta} + \angle \bm{R}_1 - \angle \bm{V}_1
  \label{eq:update_t}
\end{align}
οδηγεί σε ένα υπολειπόμενο σφάλμα προσανατολισμού $\phi$:
\begin{align}
  \phi &= \tan^{-1}\dfrac{N_s \delta_x \tan(\theta - \angle \bm{R}_1) - N_s \delta_y}{|\bm{R}_1| + N_s \delta_x + N_s \delta_y \tan(\theta - \angle \bm{R}_1)} \label{eq:phi}
\end{align}
του οποίου το μέγεθος είναι αντιστρόφως ανάλογο του αριθμού των ακτίνων της φυσικής
αισθητήρα εμβέλειας $N_s$ στην περίπτωση όπου τόσο ο $\mathcal{S}_R$ όσο και ο $\mathcal{S}_V$
δεν διαταράσσονται από θόρυβο. Το πεπερασμένο των εκπεμπόμενων ακτίνων του φυσικού αισθητήρα, σε συνδυασμό με το
αυθαιρεσία του ρυθμού των αλλαγών στο περιβάλλον (σχ.
\ref{fig:the_problem}), μπορεί να οδηγήσει σε υποδειγματοληψία τμημάτων του χάρτη.
Επιπλέον, το γεγονός ότι ο αριθμός των εκπεμπόμενων ακτίνων από τον φυσικό αισθητήρα είναι
αμετάβλητος.


%%%%%%%%%%%%%%%%%%%%%%%%%%%%%%%%%%%%%%%%%%%%%%%%%%%%%%%%%%%%%%%%%%%%%%%%%%%%%%%%
\subsection{Μέσω Fourier-Mellin σε μία διάσταση}
\label{subsection:02_04_02:02}

%%%%%%%%%%%%%%%%%%%%%%%%%%%%%%%%%%%%%%%%%%%%%%%%%%%%%%%%%%%%%%%%%%%%%%%%%%%%%%%%
\subsection{Μέσω Προκρούστη}
\label{subsection:02_04_02:03}

%%%%%%%%%%%%%%%%%%%%%%%%%%%%%%%%%%%%%%%%%%%%%%%%%%%%%%%%%%%%%%%%%%%%%%%%%%%%%%%%
\subsection{Ο ύφαλος της διακριτικής γωνίας του αισθητήρα}
\label{subsection:02_04_02:04}

%%%%%%%%%%%%%%%%%%%%%%%%%%%%%%%%%%%%%%%%%%%%%%%%%%%%%%%%%%%%%%%%%%%%%%%%%%%%%%%%
\subsection{Σίνις ο Πιτυοκάμπτης}
\label{subsection:02_04_02:05}

