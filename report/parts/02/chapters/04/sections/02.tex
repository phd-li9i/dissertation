Έστω οι παραδοχές του προβλήματος ??. Έστω επιπλέον ότι $\hat{\bm{l}} =
\bm{l}$, δηλαδή μόνο ο προσανατολισμός του αισθητήρα πρέπει να εκτιμηθεί. Τότε
ας υπολογιστεί η εικονική σάρωση $\mathcal{S}_V$ μέσω δεσμοβολής (raycasting)
από την εκτίμηση $\hat{\bm{p}}$ στον χάρτη $\bm{M}$. Η εκτίμηση της περιστροφής
της εικονικής σάρωσης $\mathcal{S}_V$ σε σχέση με την πραγματική σάρωση
$\mathcal{S}_R$ μπορεί να βρεθεί μέσω των μεθόδων που παρουσιάζονται στις
ενότητες \ref{subsection:02_04_02:01}, \ref{subsection:02_04_02:02}, και
\ref{subsection:02_04_02:03}. Το σφάλμα της εκτίμησης μπορεί να μειωθεί
περαιτέρω μέσω της μεθόδου που παρουσιάζεται στην ενότητα
\ref{subsection:02_04_02:05}.


%%%%%%%%%%%%%%%%%%%%%%%%%%%%%%%%%%%%%%%%%%%%%%%%%%%%%%%%%%%%%%%%%%%%%%%%%%%%%%%%
\subsection{Η μέθοδος Fourier-Mellin σε μία διάσταση}
\label{subsection:02_04_02:01}

Έστω $\mathcal{F}\{\mathcal{S}\}$ ο διακριτός μετασχηματισμός Fourier
του σήματος $\mathcal{S}$, $\mathcal{F}^{-1}\{\mathcal{S}\}$
ο αντίστροφός του, $\bm{c}^\ast$ ο συζυγής του μιγαδικού $\bm{c}$, και
$|\bm{c}|$ το μέτρο του. Έστω επίσης $Q_{\mathcal{S}_R, \mathcal{S}_V}$:
\begin{align}
  Q_{\mathcal{S}_R, \mathcal{S}_V} \triangleq \dfrac{\mathcal{F}\{\mathcal{S}_R\}^{\ast} \cdot \mathcal{F}\{\mathcal{S}_V\}}{|\mathcal{F}\{\mathcal{S}_R\}| \cdot |\mathcal{F}\{\mathcal{S}_V\}|}
\end{align}
Αν ο χώρος δειγματοληπτείται αρκετά πυκνά γωνιακά, για
$k,\xi \in \mathbb{Z}$: $k,\xi \in [0, N_s-1]$:
\begin{align}
  \mathcal{S}_V[k] &\simeq \mathcal{S}_R[(k - \xi) \mod N_s] \Rightarrow \nonumber \\
  \mathcal{F}\{\mathcal{S}_V\}(u) &\simeq e^{-j 2\pi \xi u / N_s} \cdot \mathcal{F}\{\mathcal{S}_R\}(u) \nonumber
\end{align}
και, επομένως, αφού $2\pi \dfrac{\xi}{N_s} = \xi \dfrac{2\pi}{N_s} = \xi \gamma$,
όπου $\gamma$ είναι η διακριτική γωνία του αισθητήρα:
\begin{align}
  Q_{\mathcal{S}_V, \mathcal{S}_R}(u) &= \dfrac{\mathcal{F}\{\mathcal{S}_V\}^{\ast} \cdot \mathcal{F}\{\mathcal{S}_R\}}{|\mathcal{F}\{\mathcal{S}_V\}| \cdot |\mathcal{F}\{\mathcal{S}_R\}|} \nonumber \\
  &\simeq \dfrac{e^{-j \xi \gamma u} \cdot \mathcal{F}\{\mathcal{S}_R\}^\ast \cdot \mathcal{F}\{\mathcal{S}_R\}}{|e^{- j \xi \gamma u} \cdot \mathcal{F}\{\mathcal{S}_R\}^\ast | \cdot | \mathcal{F}\{\mathcal{S}_R\}|} \nonumber \\
  &= e^{-j \xi \gamma u} \cdot \dfrac{\mathcal{F}\{\mathcal{S}_R\}^\ast \cdot \mathcal{F}\{\mathcal{S}_R\}}{|\mathcal{F}\{\mathcal{S}_R\} | \cdot | \mathcal{F}\{\mathcal{S}_R\}|} \nonumber \\
  &= e^{-j \xi \gamma u}
  \label{eq:Q0}
\end{align}

Ο αντίστροφος διακριτό μετασχηματιμός Fourier του
$Q_{\mathcal{S}_V, \mathcal{S}_R}$ είναι μία Kronecker $\delta$-συνάρτηση
$q_{\mathcal{S}_V, \mathcal{S}_R} = \mathcal{F}^{-1}\{Q_{\mathcal{S}_V, \mathcal{S}_R}\}$
με κέντρο $\xi$:
\begin{align}
  \xi = \operatorname*{arg\,max}\limits_u \ q_{\mathcal{S}_V, \mathcal{S}_R}(u)
\end{align}

Εάν η διαφορά του προσανατολισμού μεταξύ των στάσεων από τις οποίες λήφθησαν οι
σαρώσεις $\mathcal{S}_R$ και $\mathcal{S}_V$ είναι $\Delta\theta$, τότε
$\Delta\theta = \xi\gamma + \delta\theta$, όπου
$\mod(|\delta\theta|, \gamma) = \phi \in [0,\gamma/2]$. Επομένως, για δεδομένο
αριθμό εκπεμπόμενων ακτίνων $N_s$ (ισοδύναμα, για δεδομένη διακριτική γωνία
$\gamma$) παραμένει ένα ανεπίλυτο σφάλμα προσανατολισμού
$|\delta\theta| = \phi \leq \gamma/2$.

Η συμβολή αυτού του σφάλματος στην
σάρωσης-αντιστοίχισης είναι διπλή, καθώς η ύπαρξή του διαδίδεται επίσης στο
μέθοδο εκτίμησης θέσης. ??


%%%%%%%%%%%%%%%%%%%%%%%%%%%%%%%%%%%%%%%%%%%%%%%%%%%%%%%%%%%%%%%%%%%%%%%%%%%%%%%%
\subsection{Η μέθοδος πρώτων αρχών}
\label{subsection:02_04_02:02}

Έστω μία δισδιάστατη σάρωση $\mathcal{S}$ που έχει ληφθεί από τη στάση
$(x,y,\theta)$ σε κάποιο σύστημα συντεταγμένων (ορισμός \ref{def:lidar}). Έστω
ότι το γωνιακό εύρος της $\mathcal{S}$ είναι $\lambda = 2\pi$. Οι συντεταγμένες
του τελικού σημείου της $n$-οστής ακτίνας της $\mathcal{S}$,
$n=0,1,\dots,N_s-1$, στο σύστημα συντεταγμένων είναι $(x_n,y_n)$:
\begin{align}
  x_n &= x + d_n \cos(\theta + \frac{2 \pi n}{N_s} - \pi) = -d_n \cos(\theta + \frac{2 \pi n}{N_s}) \label{eq:x_n}\\
  y_n &= y + d_n \sin(\theta + \frac{2 \pi n}{N_s} - \pi) = -d_n \sin(\theta + \frac{2 \pi n}{N_s}) \label{eq:y_n}
\end{align}
Σε αυτό το σημείο παρατηρούμε ότι $-(x_n-x)$ και $(y_n-y)$ είναι αντίστοιχα
το πραγματικό και το φανταστικό μέρος της μιγαδικής ποσότητας
\begin{align}
  d_n e^{-i(\theta + \frac{2 \pi n}{N_s})} &= d_n \cos(\theta + \frac{2 \pi n}{N_s}) - i \cdot d_n \sin(\theta + \frac{2 \pi n}{N}) \nonumber \\
                                         &\leftstackrel{(\ref{eq:x_n}),(\ref{eq:y_n})}{=} -(x_n-x) + i \cdot (y_n-y) \label{eq:dne_complex_x_y}
\end{align}
και, επομένως
\begin{align}
  d_n e^{-i 2 \pi n/N_s} &= e^{i\theta}(-(x_n-x) + i \cdot (y_n-y)) \label{eq:dne_complex}
\end{align}
Αθροίζοντας την εξίσωση (\ref{eq:dne_complex}) επί του συνόλου των $N_s$
ακτίνων λαμβάνουμε τον πρώτο όρο του διακριτού μετασχηματισμού Fourier
του σήματος $\{d_n\}$, $n=0,1,\dots,N_s-1$:
\begin{align}
  \mathcal{F}\{\mathcal{S}\} = \sum\limits_{n=0}^{N_s-1}d_n \cdot e^{-i 2 \pi n/N_s} \ &\leftstackrel{(\ref{eq:dne_complex})}{=} \ \sum\limits_{n=0}^{N_s-1} e^{i\theta}(-(x_n-x) + i \cdot (y_n-y)) \nonumber  \\
           &= e^{i\theta} \sum\limits_{n=0}^{N_s-1} [ (x- i \cdot y) + (-x_n + i \cdot y_n) ]\nonumber \\
           &= e^{i\theta} N_s (x-i\cdot y) - e^{i\theta}\Delta \label{eq:F1}
\end{align}
όπου $\Delta \triangleq \sum\limits_{n=0}^{N_s-1} (x_n -i \cdot y_n)$.


Συμβολίζοντας με το γράμμα $R$ τις ποσότητες που αντιστοιχούν στην πραγματική
σάρωση $\mathcal{S}_R$, η οποία έχει ληφθεί από τη στάση του φυσικού αισθητήρα
$\bm{p}(x,y,\theta)$, και με $V$ εκείνες που αντιστοιχούν στην εικονική σάρωση
$\mathcal{S}_V$, η οποία έχει ληφθεί από τη στάση
$\hat{\bm{p}}(x,y,\hat{\theta})$:
\begin{align}
  \mathcal{F}\{\mathcal{S}_R\} &= \sum\limits_{n=0}^{N_s-1}d_n^R \cdot e^{-i 2 \pi n/N_s} \stackrel{(\ref{eq:F1})}{=} N_s e^{i\theta}(x-i\cdot y) - e^{i\theta} \Delta_R \label{eq:dne_complex_r} \\
  \mathcal{F}\{\mathcal{S}_V\} &= \sum\limits_{n=0}^{N_s-1}d_n^V \cdot e^{-i 2 \pi n/N_s} \stackrel{(\ref{eq:F1})}{=} N_s e^{i\hat{\theta}}(x-i\cdot y) - e^{i\hat{\theta}} \Delta_V \label{eq:dne_complex_v}
\end{align}

Έστω τώρα ότι
\begin{align}
  \Delta_R - \Delta_V &= \sum\limits_{n=0}^{N_s-1}(x_n^R-x_n^V) - i \cdot \sum\limits_{n=0}^{N_s-1}(y_n^R-y_n^V) \nonumber \\
                      &= N_s (\delta_x - i \cdot \delta_y)
\end{align}
όπου
\begin{align}
  \delta_x &\triangleq \frac{1}{N_s}\sum\limits_{n=0}^{N_s-1}(x_n^R-x_n^V) \label{eq:delta_x}\\
  \delta_y &\triangleq \frac{1}{N_s}\sum\limits_{n=0}^{N_s-1}(y_n^R-y_n^V) \label{eq:delta_y}
\end{align}
τότε
\begin{align}
  \Delta_V &= \Delta_R -N_s(\delta_x - i \cdot \delta_y) \label{eq:Delta_V_eq_Delta_R}
\end{align}

Ο πρώτος όρος του διακριτού μετασχηματισμού Fourier του σήματος που
αποτελείται από τη διαφορά των δύο σημάτων (\ref{eq:dne_complex_r}) και
(\ref{eq:dne_complex_v}) είναι $\bm{X}_1$:
\begin{align}
  \bm{X}_1 &= \mathcal{F}\{\mathcal{S}_R\} - \mathcal{F}\{\mathcal{S}_V\} \nonumber \\
           &= \sum\limits_{n=0}^{N_s-1}(d_n^R - d_n^V) \cdot e^{-i 2 \pi n/N_s} \nonumber \\
           &\leftstackrel{(\ref{eq:dne_complex_r}),(\ref{eq:dne_complex_v})}{=} N_s (x-i \cdot y)(e^{i\theta}-e^{i\hat{\theta}}) -e^{i\theta}\Delta_R+e^{i\hat{\theta}}\Delta_V \nonumber \\
          &\leftstackrel{(\ref{eq:Delta_V_eq_Delta_R})}{=} N_s (x-i \cdot y)(e^{i\theta}-e^{i\hat{\theta}}) -e^{i\theta}\Delta_R +e^{i\hat{\theta}}(\Delta_R -N_s(\delta_x - i \cdot \delta_y)) \nonumber \\
          &= N_s (x-i \cdot y)(e^{i\theta}-e^{i\hat{\theta}}) - \Delta_R (e^{i\theta}-e^{i\hat{\theta}})- N_s e^{i\hat{\theta}}(\delta_x - i\cdot \delta_y) \nonumber \\
          &= (e^{i\theta}-e^{i\hat{\theta}}) [N_s(x-i\cdot y) - \Delta_R]- N_s e^{i\hat{\theta}}(\delta_x - i\cdot \delta_y) \nonumber   \\
          &\leftstackrel{(\ref{eq:dne_complex_r})}{=} (e^{i\theta}-e^{i\hat{\theta}}) \dfrac{\mathcal{F}\{\mathcal{S}_R\}}{e^{i\theta}}- N_s e^{i\hat{\theta}}(\delta_x - i\cdot \delta_y) \nonumber \\
          &= (1-e^{-i(\theta-\hat{\theta})}) \mathcal{F}\{\mathcal{S}_R\}- N_s e^{i\hat{\theta}}(\delta_x - i\cdot \delta_y) \nonumber
\end{align}
Επομένως, χρησιμοποιώντας την πολική αναπαράσταση του $\bm{A}$ ως
$\bm{A} = |\bm{A}| e^{i\angle \bm{A}}$, και αφού $\bm{X}_1 = \mathcal{F}\{\mathcal{S}_R\} -\mathcal{F}\{\mathcal{S}_V\}$:
\begin{align}
  -\mathcal{F}\{\mathcal{S}_V\} &= -e^{-i(\theta-\hat{\theta})} \mathcal{F}\{\mathcal{S}_R\}- N_s e^{i\hat{\theta}}(\delta_x - i\cdot \delta_y) \nonumber \\
  e^{-i(\theta-\hat{\theta})} &= \dfrac{\mathcal{F}\{\mathcal{S}_V\}}{\mathcal{F}\{\mathcal{S}_R\}} - \dfrac{N_s e^{i\hat{\theta}}}{\mathcal{F}\{\mathcal{S}_R\}}(\delta_x - i\cdot \delta_y) \nonumber \\
  e^{-i(\theta-\hat{\theta})} &= \dfrac{|\mathcal{F}\{\mathcal{S}_V\}|}{|\mathcal{F}\{\mathcal{S}_R\}|} e^{i(\angle \mathcal{F}\{\mathcal{S}_V\} - \angle \mathcal{F}\{\mathcal{S}_R\})} - \dfrac{e^{i(\hat{\theta}-\angle \mathcal{F}\{\mathcal{S}_R\})}}{|\mathcal{F}\{\mathcal{S}_R\}|} (N_s \delta_x - i\cdot N_s \delta_y) \nonumber
\end{align}

Λόγω του γεγονότος ότι ο προσανατολισμός $\theta$ του αισθητήρα είναι άγνωστος,
τα τελικά σημεία $\{(x_n^R,y_n^R)\}$ καθίστανται ομοίως άγνωστα, και συνεπώς
και οι ποσότητες $\delta_x, \delta_y$. Προκειμένου να αποκτήσουμε μια αρχική
διαίσθηση ως προς τα μέτρα των τελευταίων κάνουμε την παρατήρηση ότι, εξ
ορισμού, οι ποσότητες $N_s \delta_x$ και $N_s \delta_y$ ποσοτικοποιούν τη
διαφορά της προσέγγισης των επικαμπύλιων ολοκληρωμάτων επί των καμπύλων
που ορίζουν τα τελικά σημεία των δύο σαρώσεων στους δύο κύριους άξονες $x$ και
$y$. Η προσέγγιση αυτή οφείλεται στο πεπερασμένο μέγεθος των εκπεμπόμενων ακτίνων
$N_s$. Επομένως υπό τις υποθέσεις ότι (α) ο χάρτης του περιβάλλοντος είναι
τέλεια αναπαράστασή του και (β) οι μετρήσεις του φυσικού αισθητήρα δεν
επηρεάζονται από διαταραχές: καθώς $N_s \rightarrow \infty$, $N_s \delta_x$,
$N_s \delta_y$ $\rightarrow 0$, τα οποία με τη σειρά τους σημαίνουν ότι
$|\mathcal{F}\{\mathcal{S}_V\}| \rightarrow |\mathcal{F}\{\mathcal{S}_R\}|$ και $\theta-\hat{\theta} \rightarrow \angle
\mathcal{F}\{\mathcal{S}_R\} - \angle \mathcal{F}\{\mathcal{S}_V\}$.

Ενημερώνοντας την εκτίμηση προσανατολισμού σε $\hat{\theta}^\prime$
\begin{align}
\hat{\theta}^\prime = \hat{\theta} + \angle \mathcal{F}\{\mathcal{S}_R\} - \angle\mathcal{F}\{\mathcal{S}_V\}
  \label{eq:update_t}
\end{align}
οδηγεί σε ένα σφάλμα προσανατολισμού $\phi$
\begin{align}
  \phi &= \tan^{-1}\dfrac{N_s \delta_x \tan(\theta - \angle \mathcal{F}\{\mathcal{S}_R\}) - N_s \delta_y}{|\mathcal{F}\{\mathcal{S}_R\}| + N_s \delta_x + N_s \delta_y \tan(\theta - \angle \mathcal{F}\{\mathcal{S}_R\})} \label{eq:phi}
\end{align}
του οποίου το μέτρο είναι αντιστρόφως ανάλογο του αριθμού των ακτίνων $N_s$ που
εκπέμπει ο αισθητήρας στην περίπτωση που τόσο η πραγματική μέτρηση
$\mathcal{S}_R$ όσο και η εικονική σάρωση $\mathcal{S}_V$ δεν διαταράσσονται
από θόρυβο. Το πεπερασμένο των εκπεμπόμενων ακτίνων του φυσικού αισθητήρα, σε
συνδυασμό με το αυθαίρετο του ρυθμού των αλλαγών του περιβάλλοντος (σχήμα
\ref{fig:02_02:the_problem}), μπορεί να οδηγήσει σε υποδειγματοληψία τμημάτων
του περιβάλλοντος και του χάρτη του. Επιπλέον, ο αριθμός των εκπεμπόμενων
ακτίνων από τον φυσικό αισθητήρα είναι αμετάβλητος. ??

Η συμβολή αυτού του σφάλματος στην
σάρωσης-αντιστοίχισης είναι διπλή, καθώς η ύπαρξή του διαδίδεται επίσης στο
μέθοδο εκτίμησης θέσης. ?? (και στα τρια subsections γραφεται αυτη η παραγραφος)





%%%%%%%%%%%%%%%%%%%%%%%%%%%%%%%%%%%%%%%%%%%%%%%%%%%%%%%%%%%%%%%%%%%%%%%%%%%%%%%%
\subsection{Η μέθοδος του Προκρούστη}
\label{subsection:02_04_02:03}

%%%%%%%%%%%%%%%%%%%%%%%%%%%%%%%%%%%%%%%%%%%%%%%%%%%%%%%%%%%%%%%%%%%%%%%%%%%%%%%%
\subsection{Ο ύφαλος της διακριτικής γωνίας του αισθητήρα}
\label{subsection:02_04_02:04}

%%%%%%%%%%%%%%%%%%%%%%%%%%%%%%%%%%%%%%%%%%%%%%%%%%%%%%%%%%%%%%%%%%%%%%%%%%%%%%%%
\subsection{Σίνις ο Πιτυοκάμπτης}
\label{subsection:02_04_02:05}

