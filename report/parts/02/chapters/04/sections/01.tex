Στόχος του παρόντος κεφαλαίου είναι η επίλυση του προβλήματος \ref{prob:02_04}:

\begin{bw_box}
\begin{customproblem}{Π3}
  \label{prob:02_04}
  Έστω ένα ρομπότ κινητής βάσης του πεδίου εφαρμογής \ref{scope}, ικανό να
  κινείται στο επίπεδο $x-y$, εξοπλισμένο με έναν οριζόντια τοποθετημένο
  αισθητήρα lidar μετρήσεων δύο διαστάσεων που εκπέμπει $N_s$ ακτίνες. Έστω
  επίσης ότι τα ακόλουθα είναι διαθέσιμα ή ευσταθούν:
  \begin{itemize}
    \item Ο χάρτης $\bm{M}$ του περιβάλλοντος στο οποίο κινείται το ρομπότ
    \item Μια δισδιάστατη μέτρηση $\mathcal{S}_R$ μεγέθους $N_s$, που λαμβάνεται
          από την---άγνωστη και αναζητούμενη---στάση $\bm{p}(\bm{l},\theta)$,
          $\bm{l} = (x,y)$
    \item Μια εκτίμηση της θέσης του αισθητήρα
          $\hat{\bm{p}}(\hat{\bm{l}}, \hat{\theta})$ στο σύστημα αναφοράς του
          χάρτη, όπου $\hat{\bm{l}} = (\hat{x}, \hat{y})$ είναι σε μία γειτονιά
          του $\bm{l}$
  \end{itemize}
\end{customproblem}
Τότε ο στόχος είναι να μειωθεί το μέτρο του σφάλματος στάσης του αισθητήρα
$\bm{e}(\bm{p}, \hat{\bm{p}}) \triangleq \bm{p}- \hat{\bm{p}}$ από την αρχική
του τιμή $\|\bm{e}(\bm{p}, \hat{\bm{p}})\|_2$
βελτιώνοντας την εκτίμηση της στάσης του αισθητήρα σε
$\hat{\bm{p}}^\prime(\hat{x}^\prime, \hat{y}^\prime, \hat{\theta}^\prime)$
έτσι ώστε
\begin{align}
  \|\bm{e}(\bm{p}, \hat{\bm{p}}^\prime)\|_2 < \|\bm{e}(\bm{p}, \hat{\bm{p}})\|_2
  \tag{$\astt$}
  \label{objective:02_04}
\end{align}
δεδομένων των κάτωθι παραδοχών και περιορισμών:
\begin{itemize}
  \item Το γωνιακό εύρος του αισθητήρα lidar ισούται με $\lambda = 2\pi$
        (Παραδοχή \ref{assumption:02_03_01:01})
  \item Η λύση του προβλήματος δίνεται μέσω ευθυγράμμισης πραγματικών με εικονικές
        σαρώσεις (Παραδοχή \ref{assumption:02_03_01:02})
  \item Η επίλυση του προβλήματος γίνεται χωρίς τον υπολογισμό αντιστοιχίσεων
        ανάμεσα στις εισόδους της μεθόδου επίλυσης (Παραδοχή
        \ref{assumption:02_03_01:03})
  \item Η εκτέλεση της επίλυσης του προβλήματος πρέπει γίνεται σε χρόνο που να
        συμβαδίζει με το ρυθμό ανανέωσης εκτιμήσεων στάσης που παρέχει η βασική
        μέθοδος εκτίμησης στάσης (Επακόλουθο \ref{corollary:02_02_01:02})
\end{itemize}

\end{bw_box}


Σε αυτό το κεφάλαιο θα προσεγγίσουμε τη λύση του παραπάνω προβλήματος
αποσυνθέτοντάς το σε τρία διακριτά προβλήματα.

Το πρώτο είναι η εκτίμηση του προσανατολισμού της πραγματικής στάσης του
αισθητήρα δεδομένου ότι η πραγματική θέση και η εκτιμώμενη θέση συμπίπτουν.
Προς αυτήν την κατεύθυνση αναπτύσσουμε τρεις κύριες μεθόδους εκτίμησης, οι
οποίες παρουσιάζονται στην ενότητα \ref{section:02_04_02}. Οι δύο πρώτες
μέθοδοι προκύπτουν μέσω του διακριτού μετασχηματισμού Fourier: η πρώτη αποτελεί
την προσαρμογή της δισδιάστατης μεθόδου SPOMF σε μία διάσταση, ενώ η δεύτερη
εφορμάται από πρώτες αρχές και είναι καινοφανής. Η τρίτη μέθοδος έχει την
απαρχή της στο πεδίο της κρυσταλλογραφίας και, όπως ακριβώς και η πρώτη, και σε
αντίθεση με τη δεύτερη, λειτουργεί στον διακριτό γωνιακό χώρο. Λόγω αυτού του
τελευταίου γεγονότος το τελικό σφάλμα των δύο αυτών μεθόδων είναι, εν γένει, μη
μηδενικό, και εξαρτάται από την κατά γωνία διακριτική ικανότητα του φυσικού
αισθητήρα lidar. Για την ελάττωση αυτού του σφάλματος σχεδιάζουμε μία
διαδικασία η οποία εγγυάται την φραγή του εντός ανωφλίου το οποίο είναι
ορισμένο εκ των προτέρων.

Το δεύτερο πρόβλημα είναι η εκτίμηση της θέσης της πραγματικής στάσης του
αισθητήρα δεδομένου ότι ο πραγματικός προσανατολισμός και ο εκτιμώμενος
προσανατολισμός συμπίπτουν. Προς αυτή την κατεύθυνση υιοθετούμε μία μέθοδο
εκτίμησης θέσης η οποία βασίζεται στο διακριτό μετασχηματισμό Fourier,
λειτουργεί στον συνεχή δισδιάστατο χώρο, και εφορμάται από πρώτες αρχές. Η εν
λόγω μέθοδος παρουσιάζεται στην ενότητα \ref{section:02_04_03}.

Το τρίτο πρόβλημα προκύπτει από την ασυμβατότητα των δύο παραπάνω συνθηκών, των
οποίων το χάσμα\footref{quote:aristotle} γεφυρώνουμε προσεγγίζοντας το πρόβλημα
κατ' αναλογία με τη φύση του προβλήματος της ευθυγράμμισης πραγματικών με
εικονικές σαρώσεις (Παρατήρηση \ref{rem:iterative}), δηλαδή με τρόπο
επαναληπτικό: σε κάθε επανάληψη διενεργείται πρώτα εκτίμηση του
προσανατολισμού, και στη συνέχεια εκτίμηση της θέσης, με αποτέλεσμα τη σταδιακή
μείωση και των δύο. Η σχετική λύση αναλύεται και αιτιολογείται στην ενότητα
\ref{section:02_04_04}.

Ταυτόχρονα σε αυτό το κεφάλαιο κατασκευάζουμε μεθόδους που απευθύνονται σε
αισθητήρες που διαθέτουν ``μικρό" αριθμό ακτίνων, δηλαδή αισθητήρες που
δειγματοληπτούν αραιά το χώρο, κάτι που έχει ως αποτέλεσμα τη δυσχέρεια
εκτίμησης του προσανατολισμού και συνεπώς της θέσης του αισθητήρα και του
ρομπότ.

Στην ενότητα \ref{section:02_04_05} δοκιμάζουμε εκ νέου την υπόθεση
\ref{hypothesis:02_03:01} για τις τρεις εκδόσεις της κατασκευασθείσας μεθόδου,
και συγκρίνουμε τις επιδόσεις τους με καθιερωμένους και τρέχοντες αλγορίθμους
της βιβλιογραφίας. Στην ενότητα \ref{section:02_04_06} παρέχουμε κύρια
χαρακτηριστικά γνωρίσματα και βασικούς περιορισμούς της κατασκευασθείσας
μεθόδους. Η τελευταία ενότητα ανακεφαλαιώνει και παρέχει αιτίες για περαιτέρω
έρευνα.
