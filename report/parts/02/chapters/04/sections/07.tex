%%%%%%%%%%%%%%%%%%%%%%%%%%%%%%%%%%%%%%%%%%%%%%%%%%%%%%%%%%%%%%%%%%%%%%%%%%%%%%%%
\subsection{Συμπεράσματα κεφαλαίου}
\label{subsection:02_04_07:01}

Στο παρόν κεφάλαιο επιζητήσαμε την κατασκευή μίας μεθόδου ευθυγράμμισης
πραγματικών με εικονικές σαρώσεις, η οποία έχει ως σκοπό τη δράση προσθετικά σε
μία κύρια μέθοδο παρακολούθησης της στάσης ενός ρομπότ του πεδίου εφαρμογής
\ref{scope}, με σκοπό τη μείωση του σφάλματος εκτίμησής της (σκοπός
(\ref{objective:02_04}))--- όπως ακριβώς στο κεφάλαιο
\ref{part:02:chapter:02}---αλλά χωρίς τον υπολογισμό αντιστοιχίσεων ανάμεσα στα
διανύσματα εισόδου.

Με βάση τα πειραματικά αποτελέσματα που εκτίθενται στην ενότητα
\ref{subsection:02_04_05:02} δεδομένης της πειραματικής διάταξης της ενότητας
\ref{subsection:02_04_05:01}, τα χαρακτηριστικά και τους περιορισμούς της μεθόδου
FSMSM (ενότητα \ref{section:02_04_06}), συνάγουμε τα εξής συμπεράσματα:

\begin{itemize}
  \item Ο στόχος (\ref{objective:02_04}), ο οποίος τέθηκε στην αρχή του παρόντος
        κεφαλαίου ικανοποιείται για τη μέθοδο \texttt{x1} σε ποσοστό άνω του
        $97.5\%$ για $\sigma_R \in [0.01, 0.20]$ m και
        $\sigma_{\bm{M}} \in [0.0, 0.05] m$, και για τις \texttt{uf} και
        \texttt{fm} σε ποσοστό άνω του $99.2\%$
  \item Οι τρεις άνωθεν μέθοδοι εκτελούνται σε πραγματικό χρόνο για αριθμό
        ακτίνων $N_s = 360$, αλλά η \texttt{x1} υπολείπεται σε χρόνο εκτέλεσης
        σε χαμηλά επίπεδα διαταραχών των μετρήσεων του φυσικού αισθητήρα lidar
        (σχήμα \ref{fig:02_04_05:03})
  \item Τo μέσo σφάλμα εκτίμησης στάσης των \texttt{x1}, \texttt{uf}, και
        \texttt{fm} έχει για άνω όριο τo ελάχιστo μέσο σφάλμα της καλύτερης
        ανά διαμόρφωση μεθόδου (σχήμα \ref{fig:02_04_05:05})
  \item Οι μέθοδοι \texttt{uf} και \texttt{fm} είναι πρακτικά ισοδύναμες ως προς
        όλες τις μετρικές αξιολόγησης (ποσοστό επιτυχίας, σφάλματα εκτίμησης
        θέσης και προσανατολισμού, αριθμό επανεκκινήσεων, χρόνο εκτέλεσης)
\end{itemize}


%%%%%%%%%%%%%%%%%%%%%%%%%%%%%%%%%%%%%%%%%%%%%%%%%%%%%%%%%%%%%%%%%%%%%%%%%%%%%%%%
\subsection{Αιτίες περαιτέρω έρευνας}
\label{subsection:02_04_07:02}
