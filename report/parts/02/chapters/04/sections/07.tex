%%%%%%%%%%%%%%%%%%%%%%%%%%%%%%%%%%%%%%%%%%%%%%%%%%%%%%%%%%%%%%%%%%%%%%%%%%%%%%%%
\subsection{Συμπεράσματα κεφαλαίου}
\label{subsection:02_04_07:01}

Στο παρόν κεφάλαιο επιζητήσαμε την κατασκευή μεθόδων ευθυγράμμισης
πραγματικών με εικονικές σαρώσεις, των οποίων η δράση είναι προσθετική ως προς
μία κύρια μέθοδο παρακολούθησης της στάσης ενός ρομπότ του πεδίου εφαρμογής
\ref{scope}, με σκοπό τη μείωση του σφάλματος εκτίμησής της (στόχος
(\ref{objective:02_04}))---όπως ακριβώς στο κεφάλαιο
\ref{part:02:chapter:02}---αλλά χωρίς τον υπολογισμό αντιστοιχίσεων ανάμεσα στα
διανύσματα εισόδου. Προς αυτήν την κατεύθυνση σκεφθήκαμε ορθολογικά, μελετώντας
το πρόβλημα και λύσεις της ευρύτερης επιστημονικής βιβλιογραφίας, και
σχεδιάσαμε τη μέθοδο FSMSM, η οποία διακρίνεται σε τρεις εκδόσεις ανάλογα με
την υποκείμενη μέθοδο εκτίμησης περιστροφής. Αυτές ονομάστηκαν \texttt{x1},
\texttt{uf}, και \texttt{fm}.

Με βάση τα πειραματικά αποτελέσματα που εκτίθενται στην ενότητα
\ref{subsection:02_04_05:02} δεδομένης της πειραματικής διάταξης της ενότητας
\ref{subsection:02_04_05:01}, τα χαρακτηριστικά, και τους περιορισμούς της
μεθόδου FSMSM (ενότητα \ref{section:02_04_06}), συνάγουμε τα εξής συμπεράσματα:

\begin{itemize}
  \item Ο στόχος (\ref{objective:02_04}), ο οποίος τέθηκε στην αρχή του παρόντος
        κεφαλαίου, ικανοποιείται για τη μέθοδο \texttt{x1} σε ποσοστό άνω του
        $97.5\%$ για $\sigma_R \in [0.01, 0.20]$ m και
        $\sigma_{\bm{M}} \in [0.0, 0.05]$ m, και για τις \texttt{uf} και
        \texttt{fm} σε ποσοστό άνω του $99.2\%$ (σχήμα \ref{fig:02_04_05:01})
  \item Τα ποσοστά των στάσεων των οποίων το σφάλμα εκτίμησης μειώθηκε ως
        αποτέλεσμα της εφαρμογής των μεθόδων \texttt{x1}, \texttt{uf}, και
        \texttt{fm} είναι εύρωστα ως προς τις διαταραχές που επιδρούν στις
        μετρήσεις εμπορικά διαθέσιμων πανοραμικών αισθητήρων, και διαφθορές του
        χάρτη---σε αντίθεση με τις εκδόσεις των μεθόδων ICP και NDT
  \item Οι τρεις σχεδιασθείσες μέθοδοι εκτελούνται σε πραγματικό χρόνο για
        αριθμό ακτίνων $N_s = 360$, αλλά η \texttt{x1} υπολείπεται σε χρόνο
        εκτέλεσης σε χαμηλά επίπεδα διαταραχών των μετρήσεων του φυσικού
        αισθητήρα lidar (σχήμα \ref{fig:02_04_05:03})
  \item Τo μέσo σφάλμα εκτίμησης στάσης των \texttt{x1}, \texttt{uf}, και
        \texttt{fm} έχει για άνω όριο τo ελάχιστo μέσο σφάλμα της καλύτερης
        ανά διαμόρφωση μεθόδου της βιβλιογραφίας (σχήμα \ref{fig:02_04_05:05})
  \item Οι μέθοδοι \texttt{uf} και \texttt{fm} είναι πρακτικά ισοδύναμες ως προς
        όλες τις μετρικές αξιολόγησης (ποσοστό επιτυχίας, σφάλματα εκτίμησης
        θέσης και προσανατολισμού, αριθμό επανεκκινήσεων, χρόνο εκτέλεσης)
  \item Το τελικό σφάλμα προσανατολισμού των \texttt{uf} και \texttt{fm} είναι
        ανεξάρτητο από το αρχικό σφάλμα προσανατολισμού για ένα δεδομένο επίπεδο
        διαταραχών μετρήσεων και διαφθοράς χάρτη, της \texttt{x1} είναι
        αντιστρόφως ανάλογο, ενώ των ICP εκδόσεων ανάλογο πέραν ενός κατωφλίου
  \item Το τελικό σφάλμα θέσης των \texttt{uf} και \texttt{fm} είναι αναλλοίωτο
        του αρχικού σφάλματος προσανατολισμού, σε αντίθεση με όλες τις μεθόδους
        της βιβλιογραφίας
  \item Δεδομένης της απόκλισης της λύσης της FSMSM στο παράδειγμα του σχήματος
        \ref{fig:02_04_06:03} παρατηρούμε πως η υπόθεση\footnote{``[\dots] η
        μεταφορά του προβλήματος ευθυγράμμισης πραγματικών με εικονικές
        μετρήσεις στο πεδίο της συχνότητας ίσως αποτελεί την αιτία της
        ευρωστίας της προτεινόμενης σε αυτό το κεφάλαιο μεθόδου (α) στην
        απόσταση στάσεων από τις οποίες συλλαμβάνονται οι προς ευθυγράμμιση
        σαρώσεις [\dots]"} που διαμορφώσαμε στην ενότητα
        \ref{subsection:02_03_05:02} είναι εσφαλμένη: η ευρωστία της PGL-FMIC
        όσο αφορά στα ``μεγάλα" αρχικά σφάλματα εκτίμησης οφείλεται στη
        δισδιάστατη φύση του μετασχηματισμού Fourier-Mellin της μεθόδου της
        προηγούμενης ενότητος, και όχι στη μεταφορά του προβλήματος στο πεδίο
        της συχνότητος
\end{itemize}


%%%%%%%%%%%%%%%%%%%%%%%%%%%%%%%%%%%%%%%%%%%%%%%%%%%%%%%%%%%%%%%%%%%%%%%%%%%%%%%%
\subsection{Αιτίες περαιτέρω έρευνας}
\label{subsection:02_04_07:02}
