Στόχος αυτού του κεφαλαίου είναι (α) ο σχεδιασμός μίας ολοκληρωμένης,
περιεκτικής, και επεκτάσιμης μεθοδολογίας αξιολόγησης μεθόδων αυτόνομους
πλοήγησης κινητών βάσεων ρομπότ του πεδίου εφαρμογής \ref{scope}, και (β) η
εφαρμογή της για την αξιολόγηση της επίδοσης τρεχόντων υλοποιήσεών τους στο και
μέσω του μεσολογισμικού \texttt{ROS}.

Στην ενότητα \ref{section:02_01_02} γίνεται η επισκόπηση των χαρακτηριστικών
γνωρισμάτων των αλγορίθμων χάραξης μονοπατιών και ελεγκτών κίνησης που
απαντώνται στη ερευνητική βιβλιογραφία ως θεωρητικές μέθοδοι και ως υλοποιήσεις
πακέτων λογισμικού. Στην ενότητα \ref{section:02_01_03} εκτίθεται η μεθοδολογία
αξιολόγησης. Αρχικά παρουσιάζεται η διάταξη της πειραματικής διαδικασίας, και
στη συνέχεια η μεθοδολογία αξιολόγησης με βάση ποσοτικές μετρικές, οι οποίες
αποτελούν αντικειμενικά κριτήρια της επίδοσης ενός ρομπότ στο έργο της
αυτόνομους πλοήγησης. Όσο αφορά στις υλοποιήσεις των μεθόδων αυτόνομους
πλοήγησης, δεδομένου ότι η αξιολόγηση είναι στραμμένη στην πράξη, συστήνεται
μία μεθοδολογία προκαταρκτικής αξιολόγησής τους με βάσει ποιοτικά κριτήρια που
τίθενται από την εμπειρία ανάπτυξης και συντήρησης λογισμικού, προκειμένου να
διακριθούν τα εύρωστα και εύχρηστα πακέτα λογισμικού από τα μη.  Η πειραματική
αξιολόγηση διενεργείται επί μεθόδων αυτόνομους πλοήγησης των οποίων οι
υλοποιήσεις δεν απορρίπτονται με βάση αυτά τα ποιοτικά κριτήρια. Η πειραματική
διαδικασία και η εφαρμογή της μεθοδολογίας ποσοτικής αξιολόγησης διενεργείται
σε εννιά συνδυασμούς πακέτων λογισμικού στην ενότητα \ref{section:02_01_04}. Το
αποτέλεσμα είναι μία ιεράρχηση των συνδυασμών τους.  Τέλος, η ενότητα
\ref{section:02_01_05} παρέχει τα συμπεράσματα του κεφαλαίου και οδούς για
περαιτέρω έρευνα, οι οποίες οδηγούν στο κεφάλαιο \ref{part:02:chapter:02}.
