\subsection{Προκαταρκτική αξιολόγηση}
  \label{subsection:02_01_04:01}

Τα πακέτα \texttt{navfn} και \texttt{global\_planner} θεωρούνται ότι είναι οι
προεπιλεγμένες επιλογές για αλγορίθμους κατασκευής μονοπατιών στο ROS: είναι οι
παλαιότερες και (θεωρούνται ότι είναι οι---) ασφαλέστερες επιλογές για το έργο
της πλοήγησης. Επιπλέον, απαιτούν ελάχιστη παραμετροποίηση. Συντηρούνται συνεχώς
από την απαρχή του ROS (αυτό ισχύει ειδικά για το \texttt{navfn}) και,
ως οι de facto global planners του ROS, θα θεωρηθούν ως το βασικό μέτρο
σύγκρισης για όλους τους άλλους global planners που θα περάσουν από την αρχική
φάση διαλογής.

Θεωρούμε ότι το πακέτο \texttt{asr\_navfn} είναι περιττό καθώς (α) η
συμπεριφορά του είναι ακριβώς η ίδια με αυτή του \texttt{navfn} και (β) η
δυνατότητά χρήσης του βασίζεται στην πιθανή αποτυχία του επιλογέα στόχου του
ρομπότ. Επιπλέον, δεν συντηρείται επί του παρόντος. Ως εκ τούτου, αυτό το
πακέτο δεν θα αξιολογηθεί σύμφωνα με την δεύτερο κριτήριο της ενότητος
\ref{subsection:02_01_03:04}.

Παρόλο που το πακέτο \texttt{MoveIt!} είναι επαρκώς τεκμηριωμένο,
υποστηριζόμενο, και ενημερωμένο, δεν απευθύνεται σε πλοήγηση κινητών βάσεων
στον δισδιάστατο χώρο. Επομένως, το πακέτο αυτό θα δεν θα αξιολογηθεί, σύμφωνα
με το δεύτερο κριτήριο.

Το πακέτο \texttt{sbpl\_lattice\_planner} τεκμηριώνεται τόσο στη θεωρία όσο και
από άποψης παραμέτρων. Είναι επί του παρόντος ενημερωμένο στην τελευταία έκδοση
του ROS (έκδοση \texttt{melodic}), συντηρείται, και υποστηρίζεται από τους
συντηρητές του ROS (ένα σφάλμα λογισμικού που ανακαλύφθηκε κατά τη διάρκεια της
αξιολόγησής του εξαλείφθηκε μέσα σε $8$ ημέρες). Για την εγκατάστασή του
(εκτός από εκείνη της βασικής βιβλιοθήκης \texttt{SBPL}) δεν απαιτείται κάποια
ιδιαίτερη προσπάθεια.

Η δυναμική έκδοση του \texttt{sbpl\_lattice\_planner}, το πακέτο
\texttt{sbpl\_dynamic\_env\_global\_planner}, θεωρείται περιττό δεδομένου ότι το
παρόν άρθρο ασχολείται με την πλοήγηση σε στατικά περιβάλλοντα.  Παρ' όλα αυτά,
η σελίδα αναφοράς του προειδοποιεί τον αναγνώστη ότι ο ιχνηλάτης που
χρησιμοποιείται για την παρακολούθηση κινούμενων αντικειμένων δεν είναι
εύρωστος (ειδικά όταν το ρομπότ κινείται), συμβουλεύοντάς τον να κατευθυνθεί σε
κάποια καλύτερη εναλλακτική λύση. Επιπλέον, απαιτεί την αντικατάσταση ολόκληρου
του πακέτου \texttt{move\_base} με μια τροποποίηση αυτού, έτσι ώστε οι global
και local planners να εκτελούνται ταυτόχρονα. Τέλος, αξιολογείται ως μη
επικαιροποιημένο πακέτο, δεδομένου ότι η τελευταία υποστηριζόμενη διανομή ROS
είναι η \texttt{diamondback}, και η τελευταία ενημέρωσή της ήταν πάνω από οκτώ
χρόνια πριν κατά το χρόνο συγγραφής της διατριβής. Συνεπώς αυτό το πακέτο δεν
είναι αυτοτελές, ενημερωμένο, και, σύμφωνα με τo δεύτερο, το τέταρτο, και το
έβδομο κριτήριo, δεν θα ληφθεί υπόψη στην προσεχή αξιολόγηση.

Παρόλο που το πακέτο \texttt{lattice\_planner} είναι τεκμηριωμένο και
αυτοτελές, δεν συντηρείται ενεργά (η τελευταία του έκδοση στο \texttt{github}
είναι πέντε ετών) και, ως εκ τούτου, δεν θα αξιολογηθεί, σύμφωνα με την δεύτερο
κριτήριο αξιολόγησης ποιότητας.

Το ίδιο ισχύει και για το πακέτο \texttt{waypoint\_global\_planner}: είναι
ελάχιστα τεκμηριωμένο, δεν συντηρείται ενεργά, και δεν είναι αυτοτελές στο με
την έννοια ότι η παροχή της αρχικής και τελικής στάσης του ρομπότ δεν επαρκούν
για τη δημιουργία μιας διαδρομής που συνδέει τις συνδέει, δεδομένου ότι η
σχεδιαστής δεν είναι σε θέση να λάβει υπόψη του τα εμπόδια του χάρτη κόστους.
Συνεπώς, θα δεν θα ληφθεί υπόψη για αξιολόγηση, σύμφωνα με το πρώτο, το
δεύτερο και το τέταρτο κριτήριο.

Όσον αφορά το πακέτο \texttt{voronoi\_planner}, είναι επίσης ανεπαρκώς
τεκμηριωμένο, και δεν συντηρείται ενεργά (η τελευταία υποστηριζόμενη έκδοση ROS
είναι η \texttt{indigo} και η τελευταία έκδοσή του στο \texttt{github} είναι
έξι ετών). Ως εκ τούτου, δεν θα ληφθεί υπόψη για αξιολόγηση, σύμφωνα με τo
πρώτo και δεύτερο κριτήριο.

Όσον αφορά στους ελεγκτές κίνησης, η κατάσταση του \texttt{dwa\_local\_planner}
είναι ισοδύναμη με εκείνη των \texttt{navfn} και \texttt{global\_planner}:
είναι το βασικό πακέτο υλοποίησης ελεγκτή κίνησης στο ROS.

Ο ελεγκτής κίνησης \texttt{eband\_local\_planner} τεκμηριώνεται, εγκαθίσταται
μέσω της τυπικής διαδικασίας εγκατάστασης πακέτων, και είναι αυτοτελής. Ωστόσο,
δεν έχει ενημερωθεί ώστε να ταιριάζει με την τελευταία έκδοση του
ROS\footnote{\url{https://github.com/utexas-bwi/eband\_local\_planner/issues/28}},
και φαίνεται ότι δεν συντηρείται επί του παρόντος. Παρ' όλα αυτά θα το
συμπεριλάβουμε στην αξιολόγηση των τοπικών σχεδιαστών μας ως εξαίρεση λόγω της
κρίσιμης έλλειψης ελεγκτών κίνησης στο ROS. Στον πίνακα
\ref{tbl:qualitative_metrics}, κάτω από τη στήλη για τις υπολογιστικές ανάγκες,
ο \texttt{eband\_local\_planner} λαμβάνει δύο κύκλους λόγω της ανάγκης επίλυσης
ενός προβλήματος μη γραμμικής βελτιστοποίησης με περιορισμούς κατά τη διάρκεια
εκτέλεσης.

Τέλος, ο ελεγκτής κίνησης \texttt{teb\_local\_planner} είναι ο πιο διεξοδικά
τεκμηριωμένος αλγόριθμος μεταξύ όλων που έχουμε αναφέρει μέχρι στιγμής, τόσο σε
θεωρητικό επίπεδο, όσο και σε επίπεδο παραμέτρων. Είναι ενημερωμένος στην
τελευταία έκδοση του ROS, αυτοτελής, και είναι ο πιο παραμετροποιήσιμος
ελεγκτής κίνησης. Ακριβώς όπως και ο \texttt{eband\_local\_planner}, ο
\texttt{teb\_local\_planner} λαμβάνει δύο κύκλους στη στήλη για υπολογιστικών
αναγκών στον πίνακα \ref{tbl:qualitative_metrics} λόγω της αρχής λειτουργίας
του που περιλαμβάνει την επίλυση ενός προβλήματος μη γραμμικής βελτιστοποίησης
με χωροχρονικούς περιορισμούς κατά τη διάρκεια εκτέλεσής του.

Συνολικά, κανένας από τους σχεδιαστές που συζητήθηκαν παραπάνω δεν έχει
υπερβολικές απαιτήσεις σε πόρους, και επομένως η ταυτόχρονη λειτουργία τους
μαζί με άλλα πακέτα (παρακολούθησης στάσης ή χαρτογράφησης SLAM, για παράδειγμα)
δεν θέτει σε κίνδυνο τη λειτουργία των τελευταίων.

Ο πίνακας \ref{tbl:qualitative_metrics} απεικονίζει τον πλήρη κατάλογο
αξιολόγησης με βάση τα ποιοτικά κριτήρια της ενότητος
\ref{subsection:02_01_03:04} γιά όλα τα πακέτα λογισμικού αυτόνομους πλοήγησης
των ενοτήτων \ref{subsubsection:02_01_02:03_01} και
\ref{subsubsection:02_01_02:03_02}.

\begin{table*}\hspace{-1cm}
\renewcommand{\arraystretch}{1.3}
\begin{tabular}{lccccccccc|c}
  & \multicolumn{7}{c}{Ποιοτικές Μετρικές} \\
  \cline{2-8}
  Planner                            & DOC                       & UTD         & INST              & SC/C      & PARAM                   & CON              & COMP                    & Αποδοχή      \\ \toprule
  \texttt{navfn}                     & $\bullet$                 & $\bullet$   & $\bullet\bullet$  & $\bullet$ & $\bullet$               & $\bullet$        & $\bullet$               & $\bullet$    \\
  \texttt{global\_planner}           & $\bullet$                 & $\bullet$   & $\bullet\bullet$  & $\bullet$ & $\bullet$               & $\bullet$        & $\bullet$               & $\bullet$    \\
  \texttt{asr\_navfn}                & $\bullet$                 & $\circ$     & $\bullet$         & $\bullet$ & $\bullet$               & $\bullet$        & $\bullet$               & $\circ$      \\
  \texttt{MoveIt!}                   & $\bullet\bullet\bullet$   & $\bullet$   & $\bullet\bullet$  & $\bullet$ & $\bullet\bullet\bullet$ & ?                & $\bullet\bullet\bullet$ & $\circ$      \\
  \texttt{sbpl\_lattice\_planner}    & $\bullet\bullet$          & $\bullet$   & $\bullet\bullet$  & $\bullet$ & $\bullet$               & $\circ$          & $\bullet$               & $\bullet$    \\
  \texttt{sbpl\_dynamic\_}[$\dots$]  & $\bullet$                 & $\circ$     & $\bullet$         & $\circ$   & $\bullet$               & ?                & $\bullet$               & $\circ$      \\
  \texttt{lattice\_planner}          & $\bullet$                 & $\circ$     & $\bullet$         & $\bullet$ & $\bullet$               & $\bullet$        & $\bullet$               & $\circ$      \\
  \texttt{waypoint\_global\_planner} & $\bullet$                 & $\circ$     & $\bullet$         & $\circ$   & $\circ$                 & $\bullet$        & $\bullet$               & $\circ$      \\
  \texttt{voronoi\_planner}          & $\bullet$                 & $\circ$     & $\bullet$         & $\bullet$ & $\bullet$               & $\bullet$        & $\bullet$               & $\circ$      \\ \midrule
  \texttt{dwa\_local\_planner}       & $\bullet$                 & $\bullet$   & $\bullet\bullet$  & $\bullet$ & $\bullet$               & $\bullet$        & $\bullet$               & $\bullet$    \\
  \texttt{eband\_local\_planner}     & $\bullet$                 & $\circ$     & $\bullet\bullet$  & $\bullet$ & $\bullet\bullet$        & $\bullet$        & $\bullet\bullet$        & $\bullet$    \\
  \texttt{teb\_local\_planner}       & $\bullet\bullet\bullet$   & $\bullet$   & $\bullet\bullet$  & $\bullet$ & $\bullet\bullet\bullet$ & $\bullet\bullet$ & $\bullet\bullet$        & $\bullet$    \\ \bottomrule
\end{tabular}
\caption{\small Αξιολόγηση των πακέτων ROS που αποτελούν συνιστώσες αυτόνομους
         πλοήγησης με βάση τις μετρικές που ορίζονται στην ενότητα
         \ref{subsection:02_01_03:04}, και απόφαση αποδοχής για συμπερίληψη στην
         πειραματική αξιολόγηση. Οι συντομογραφίες εισάγονται για λόγους
         εξοικονόμησης χώρου. DOC: συντομογραφία ποιότητας τεκμηρίωσης, UTD
         περί του αν είναι ενημερωμένο, INST της ευκολίας εγκατάστασης του,
         SC/C για την αυτοτέλεια/πληρότητα του, PARAM για την
         παραµετροποιησιµότητα του, CON της συνέπειας στην εκτέλεσή του, και COMP
         για τις ανάγκες του σε υπολογιστικούς πόρους. Οι κενές κουκκίδες
         υποδηλώνουν ανεπάρκεια σε σχέση με κάθε μετρική. Τα ερωτηματικά
         υποδηλώνουν άγνωστη κατάσταση.}
\label{tbl:qualitative_metrics}
\end{table*}

Ο πίνακας \ref{tbl:planners_sifted_list} δείχνει την τελική λίστα των πακέτων
ROS που θα αξιολογηθούν πειραματικά. Οι συμβολισμοί GP και LP που
χρησιμοποιούνται στη συνέχεια στην επικεφαλίδα των πινάκων είναι συντομογραφία
για τις φράσεις ``Global Planner" και ``Local Planner`".
αντίστοιχα.

\begin{table}[h]\centering
\begin{tabular}{l|l}
  Global planners (GP) & Local planners (LP) \\ \toprule
  \texttt{navfn} & \texttt{dwa\_local\_planner} \\
  \texttt{global\_planner} & \texttt{eband\_local\_planner} \\
  \texttt{sbpl\_lattice\_planner} & \texttt{teb\_local\_planner} \\ \bottomrule
\end{tabular}
  \caption{\small Ο κατάλογος των πακέτων ROS των αλγορίθμων κατασκευής
           μονοπατιών (Global Planners) και ελεγκτών κίνησης (Local Planners)
           των οποίων η ξεχωριστή και συνδυαστική χρήση θα αξιολογηθεί
           πειραματικά}
\label{tbl:planners_sifted_list}
\end{table}
