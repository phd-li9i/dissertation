Αυτή η εργασία αξιολογεί τους τελευταίας τεχνολογίας παγκόσμιους και τοπικούς σχεδιαστές των οποίων η πηγή
κώδικας είναι διαθέσιμος για άμεση ανάπτυξη μέσω του Robot Operating System (ROS)
στο έργο της πλοήγησης μη επανδρωμένων επίγειων οχημάτων $-$ συγκεκριμένα σε
αυτή την εργασία ένα Turtlebot v2 $-$ σε χάρτες πλέγματος κατάληψης $2$D.
Πραγματοποιήθηκε ένα αρχικό κοσκίνισμα προκειμένου να διακριθούν εύρωστοι σχεδιαστές με βάση
με βάση ποιοτικές μετρήσεις προσανατολισμένες στο λογισμικό. Οι σχεδιαστές που πέρασαν από
αυτό το κόσκινο συνδυάστηκαν έτσι ώστε η απόδοση όλων των συνδυασμών των
παγκόσμιων και τοπικών σχεδιαστών θα αξιολογούνταν. Η απόδοσή τους ελέγχθηκε
σε δύο ετερογενή προσομοιωμένα περιβάλλοντα και σε ένα περιβάλλον του πραγματικού κόσμου,
έτσι ώστε και τα δύο στοιχεία να αντιμετωπίσουν τις κατάλληλες δοκιμασίες καταπόνησης.

Συνολικά, διακρίνουμε τον τοπικό σχεδιαστή \texttt{teb\_local\_planner} ως τον πιο
στιβαρός και οριστικά επιτυχημένος τοπικός σχεδιαστής μεταξύ αυτών που δοκιμάστηκαν: είναι
ποτέ δεν απέτυχε να πλοηγηθεί από μια αρχική θέση σε μια θέση-στόχο, και πάντα έκανε
το έκανε πάντα στον ελάχιστο χρόνο. Τοπικός σχεδιαστής \texttt{eband\_local\_planner}
ήρθε δεύτερος, αποτυγχάνοντας να πλοηγηθεί σε κατάλληλο χρονικό διάστημα σε μία
προσομοιωμένο χάρτη περιβάλλοντος, ενώ κατεύθυνε σαφώς την κίνηση του ρομπότ
με υπερβολικά ασφαλή τρόπο. Ωστόσο, σε πειράματα στον πραγματικό κόσμο, είχε επιδόσεις
καλύτερα από ό,τι στις προσομοιώσεις. Ο τρίτος τοπικός σχεδιαστής,
\texttt{dwa\_local\_planner} δεν ολοκλήρωσε επιτυχώς ούτε μία αποστολή σε
σε κανένα περιβάλλον, προσομοιωμένο ή πραγματικό.

Όσον αφορά τους παγκόσμιους σχεδιαστές, οι \texttt{navfn} και \texttt{global\_planner}
είναι σχεδόν ισοδύναμοι, με τον πρώτο να υπερέχει του δεύτερου συνολικά.
Από την άλλη πλευρά, η διαφορά απόδοσης του \texttt{sbpl\_lattice\_planner}
μεταξύ του \texttt{global\_planner} είναι μεγαλύτερη από εκείνη μεταξύ του τελευταίου
και του \texttt{navfn}.
