%%%%%%%%%%%%%%%%%%%%%%%%%%%%%%%%%%%%%%%%%%%%%%%%%%%%%%%%%%%%%%%%%%%%%%%%%%%%%%%%
\subsection{Ευθυγράμμιση δισδιάστατων μετρήσεων αισθητήρα lidar}
\label{subsection:02_02_02:1}

Η ευθυγράμμιση δισδιάστατων μετρήσεων αισθητήρα lidar (ενότητα
\ref{subsec:01_01_02_5}) ήταν το αποτέλεσμα της γενικότερης έρευνας επί της
ευθυγράμμισης νεφών σημείων (point clouds) στον τρισδιάστατο χώρο. Η
ευθυγράμμιση νεφών σημείων προέκυψε από τις ανάγκες της κοινότητας της
υπολογιστικής όρασης με το θεμελιώδες άρθρο των Besl και McKay
\cite{Besl1992c}, στο οποίο εισήγαγαν τον αλγόριθμο Iterative Closest Point
(ICP). Ο ICP προσδιορίζει τον πλήρη χωρικό μετασχηματισμό έξι βαθμών ελευθερίας
που ευθυγραμμίζει βέλτιστα δύο σύνολα τρισδιάστατων σημείων ελεύθερης
μορφής---βέλτιστα σύμφωνα με τη μετρική της μέσης τετραγωνικής ευκλείδειας
απόστασης. Στον ICP κάθε σημείο του δεύτερου συνόλου συσχετίζεται με ένα σημείο
αναφοράς του πρώτου μέσω του προσδιορισμού του πλησιέστερου σε αυτό σημείου.
Μόλις προσδιοριστεί αυτό το σύνολο αντιστοιχίσεων, ο ICP υπολογίζει το
μετασχηματισμό που ελαχιστοποιεί το μέσο τετραγωνικό σφάλμα μεταξύ των δύο
συνόλων ζευγαρωμένων σημείων. Σε κάθε του βήμα εφαρμόζει αυτόν τον
μετασχηματισμό στα σημεία του δεύτερου συνόλου σημείων και ενημερώνει το μέσο
τετραγωνικό σφάλμα. Αυτή η διαδικασία επαναλαμβάνεται μέχρι το μέσο τετραγωνικό
σφάλμα να πέσει κάτω από ένα προκαθορισμένο όριο.

Οι Lu και Milios ήταν οι πρώτοι που διατύπωσαν και χρησιμοποίησαν μεθόδους
ευθυγράμμισης μετρήσεων δισδιάστατου αισθητήρα lidar προκειμένου να εκτιμηθεί η
σχετική μετατόπιση και περιστροφή μεταξύ δύο στάσεων του. Παρόλο που μια
μέτρηση αντιπροσωπεύει ένα 2D σχήμα (ένα περίγραμμα του ορατού κόσμου από την
προοπτική του ρομπότ), το σχήμα αυτό αναπαρίσταται από θορυβώδη διακριτά σημεία
αντί για ένα μοντέλο υψηλής ποιότητας και πιστότητας, γεγονός που καθιστά τον
αξιόπιστο ορισμό ή την εξαγωγή χαρακτηριστικών (features) από αυτό ένα δύσκολο
και ενδεχομένως ανακριβές εγχείρημα \cite{Grimson}. Στο \cite{FengLu1994a}
διατυπώνονται δύο αλγόριθμοι που χειρίζονται τον θόρυβο του αισθητήρα και δεν
βασίζονται σε διακριτά χαρακτηριστικά του περιβάλλοντος λειτουργίας του
αισθητήρα (όπως γραμμές ή γωνίες), αποφεύγοντας έτσι τη διαδικασία εξαγωγής
χαρακτηριστικών και της αντιστοίχισής τους. Η πρώτη μέθοδος, που ονομάστηκε
Iterative Matching Range to Point (IMPR), εξετάζει ξεχωριστά τις περιστροφικές
και μεταφορικές συνιστώσες του προβλήματος ευθυγράμμισης, θεωρώντας σταθερή
τη μία και βελτιστοποιώντας την άλλη εναλλάξ. Η λύση του προβλήματος
διατυπώνεται ως αναζήτηση σε μια συνάρτηση απόστασης για την εκτίμηση της
σχετικής περιστροφής μεταξύ των μετρήσεων εισόδου και χρησιμοποιεί μια
διαδικασία ελαχίστων τετραγώνων για την επίλυση της σχετικής μετατόπισης. Η
δεύτερη µέθοδος, η οποία ονοµάζεται Iterative Dual Correspondence (IDC),
αποδίδει σημαντικά ακριβέστερες εκτιμήσεις της περιστροφής από την IMPR, και
βασίζεται σε επαναληπτικές λύσεις ελαχίστων τετραγώνων, χρησιμοποιώντας
αντιστοιχίσεις από σημείο σε σημείο, παρομοίως με τον αλγόριθμο ICP. Ουσιαστικά
ο IDC συνδυάζει τον ICP και τον IMPR, χρησιμοποιώντας τον ICP για τον υπολογισμό
της μετατόπισης μεταξύ των δύο σαρώσεων και τον IMPR για τον υπολογισμό της
σχετικής τους περιστροφής.

Οι συγγραφείς του \cite{Pfistera} ήταν οι πρώτοι που επεξέτειναν τη συμπερίληψη
πηγών αβεβαιότητας στη λύση του προβλήματος ευθυγράμμισης, αναπτύσσοντας
μοντέλα που λαμβάνουν υπόψη τους τις επιδράσεις του θορύβου των μετρήσεων, της
γωνίας πρόσπτωσης του αισθητήρα, και του σφάλματος ευθυγράμμισης μεταξύ των
μετρημένων οριακών σημείων του περιβάλλοντος. Παρόλο που δεν λαμβάνουν υπόψη
τους την αβεβαιότητα της στάσης του αισθητήρα, ενσωματώνουν την οδομετρία του
ρομπότ. Σε αυτό το άρθρο εισάγεται ένας σταθμισμένος αλγόριθμος ευθυγράμμισης
δισδιάστατων μετρήσεων lidar για την εκτίμηση του μετασχηματισμού
κίνησης του ρομπότ μεταξύ των στάσεων από όπου λαμβάνονται διαδοχικές
μετρήσεις, με καλύτερες επιδόσεις από τις μη σταθμισμένες μεθόδους, όπως οι
αλγόριθμοι του \cite{FengLu1994a}.  Επιπλέον, με τον υπολογισμό της πραγματικής
συνδιακύμανσης των μετασχηματισμών, ο σταθμισμένος αλγόριθμος ευθυγράμμισης
παρέχει την βάση για τη βέλτιστη συγχώνευση αυτών των εκτιμήσεων με οδομετρικές
ή/και αδρανειακές μετρήσεις, καθιστώντας τον έτσι υποψήφιο για την υποστήριξη
των έργων του εντοπισμού στάσης και της ταυτόχρονης χαρτογράφησης και
παρακολούθησης της στάσης ενός ρομπότ.

Στο \cite{Chetverikova} παρουσιάζεται μια εύρωστη επέκταση του ICP, η οποία
ονομάζεται Trimmed Iterative Closest Point (TrICP). Αυτός ο αλγόριθμος αίρει
την υπόθεση του ICP για την ισότητα των μεγεθών των δύο μετρήσεων εισόδου και,
υποθέτοντας τη γνώση (ή τη δυνατότητα απόκτησης) του ελάχιστου αριθμού
εγγυημένων αντιστοιχίσεων μεταξύ των δύο συνόλων, κάνει εκτεταμένη χρήση της
μεθόδου ελαχίστων τετραγώνων \cite{Rousseeuw1984} προκειμένου να ενισχύσει
τον βρόχο εκτέλεσης του ICP.  Ο ICP διαπιστώνεται ότι αποτελεί ειδική περίπτωση
του TrICP στην περίπτωση που τα δύο σύνολα έχουν το ίδιο μέγεθος και όλα τα
σημεία των δύο συνόλων μπορούν να αντιστοιχιστούν το ένα με το άλλο.

Οι Biber και Strasser εισήγαγαν τον μετασχηματισμό κανονικών κατανομών Normal
Distributions Transform (NDT) στο \cite{Bibera}, σε μια προσπάθεια να
υποστηρίξουν το έργο της κατασκευής χαρτών από ρομπότ. Υποθέτοντας 2D μετρήσεις
lidar ως είσοδο, ο NDT υποδιαιρεί το οριζόντιο επίπεδο σε ένα πλέγμα παρόμοιo
με ένα πλέγμα κατάληψης---αλλά σε αντίθεση με το πλέγμα κατάληψης που
αντιπροσωπεύει δυαδικά εάν ένα κελί είναι κατειλημμένο, ο NDT αναπαριστά
σε αυτό την πιθανότητα μέτρησης ενός σημείου-δείγματος μιας μέτρησης απόστασης
για κάθε θέση \textit{εντός} ενός κελιού. Το αποτέλεσµα του µετασχηµατισµού
είναι μία συνεχής και διαφορίσιμη πυκνότητα πιθανότητας που μπορεί να
χρησιμοποιηθεί για την αντιστοίχιση μιας δεύτερης μέτρησης, εδώ χρησιμοποιώντας
τον αλγόριθμο του Νεύτωνα \cite{Suli2003}. Τα πλεονεκτήματα αυτού του τρόπου
αναπαράστασης είναι ότι (α) όλες οι εμπλεκόμενες παράγωγοι μπορούν να
υπολογιστούν αναλυτικά, γρήγορα, και με εγκυρότητα, και, το σημαντικότερο, (β)
η αντιστοίχιση σημείων της δεύτερης μέτρησης σε κατανομές σημείων της πρώτης
σημαίνει ότι δεν υπάρχει καμία ρητή αντιστοίχιση μεταξύ δύο διακριτών σημείων
μεταξύ δύο μετρήσεων, μειώνοντας έτσι το βαθμό στον οποίο όλες οι
αντιστοιχίσεις σάρωσης είναι επιρρεπείς σε σφάλματα. Η σχετική μετατόπιση και
περιστροφή μεταξύ μιας μέτρησης εισόδου και μιας μέτρησης αναφοράς
υπολογίζονται επαναληπτικά με μεγιστοποίηση (μέσω της εκτέλεσης ενός βήματος
του αλγορίθμου του Νεύτωνα) μιας συνάρτησης βαθμολογίας, η οποία είναι το
άθροισμα της αξιολόγησης όλων των κατανομών της μέτρησης εισόδου που
αντιστοιχίζονται στο σύστημα αναφοράς της πρώτης μέτρησης, με βάση μια αρχική
εκτίμησή τους (π.χ. που λαμβάνεται από οδομετρικές μετρήσεις).

Στο \cite{Censia} οι συγγραφείς επεκτείνουν τη δυνατότητα εφαρμογής της
ευθυγράμμισης μετρήσεων στα προβλήματα εύρεσης της στάσης ενός ρομπότ βάσει
πεπερασμένης και καθολικής αβεβαιότητας. Μεταφέροντας το πρόβλημα της
ευθυγράμμισης μετρήσεων στο χώρο Hough εκμεταλλεύονται αρκετές από τις
ιδιότητές του: συγκεκριμένα ότι δεν υπάρχει απώλεια πληροφορίας κατά τη
διαδικασία μετασχηματισμού, και ότι η αναλλοιωσιμότητα (invariance) του
μετασχηματισμού επιτρέπει την απεμπλοκή του προβλήματος της εύρεσης της
διαφοράς προσανατολισμού ανάμεσα σε δύο μετρήσεις από το πρόβλημα εύρεσης της
διαφοράς της θέσης από τις οποίες ελήφθησαν. Ο αλγόριθμος Hough Scan Matching
(HSM) είναι ένας καθολικός, πολυτροπικός (multi-modal), μη επαναληπτικός
αλγόριθμος ευθυγράμμισης δισδιάστατων μετρήσεων απόστασης που μπορεί να
λειτουργήσει σε μη δομημένα περιβάλλοντα. Ο HSM δεν βασίζεται στην εξαγωγή
χαρακτηριστικών, αλλά αντιστοιχίζει πυκνά δεδομένα, δηλαδή σαρώσεις απόστασης
που μπορούν να ερμηνευθούν ως κατανομές χαρακτηριστικών σε ένα διαφορετικό χώρο
παραμέτρων, επιτρέποντας του την ευθυγράμμιση μη γραμμικών επιφανειών, με
ταυτόχρονη ευρωστία ως προς το θόρυβο μέτρησης.

Στο \cite{Montesano2005} οι συγγραφείς παρουσιάζουν μια πιθανοτική μέθοδο για
την ευθυγράμμιση μετρήσεων που λαμβάνονται σε μη δομημένα περιβάλλοντα. Ενώ ο
αλγόριθμός τους, probabilistic Iterative Correspondence (pIC), έχει σχεδιαστεί
ώστε να χειρίζεται τον θόρυβο μέτρησης του αισθητήρα, σε αντίθεση με τον
\cite{Pfistera}, είναι επίσης σε θέση να χειριστεί και την αβεβαιότητα της
στάσης του αισθητήρα. O pIC ακολουθεί μια διαδικασία δύο βημάτων, σύμφωνα με
την οποία προσδιορίζει πιθανοτικά αντιστοιχίσεις μεταξύ δύο μετρήσεων. Στη
συνέχεια αποτυπώνει ταυτόχρονα τη σχετική μετατόπιση και περιστροφή τους.
Πειράματα έναντι των ICP και IDC δείχνουν ταχύτερη σύγκλιση της ευθυγράμμισης
σε σχέση με τον ICP και, στο έργο της χαρτογράφησης, αποδεικνύεται ότι ο pIC
υπερτερεί και των δύο όσον αφορά στην ευρωστία, την ακρίβεια, και την ταχύτητα
σύγκλισης.

Σε αντίθεση με τις προαναφερθείσες μεθόδους, οι συγγραφείς του \cite{Diosi2005}
υποστηρίζουν ότι είναι επωφελές για έναν αλγόριθμο ευθυγράμμισης μετρήσεων να
λειτουργεί στο εγγενές πολικό σύστημα συντεταγμένων του αισθητήρα. Η μέθοδός
τους, Polar Scan Matching (PSM), ανήκει στην οικογένεια των προσεγγίσεων
ευθυγράμμισης αναζήτησης αντιστοιχίσεων από-σημείο-σε-σημείο. O PSM αποφεύγει
την αναζήτηση συσχετίσεων μέσω της αντιστοίχισης σημείων παρόμοιας κατεύθυνσης,
και εναλλάσσεται μεταξύ προβολής μέτρησης ακολουθούμενη από εκτίμηση
μετατόπισης, και προβολής μέτρησης ακολουθούμενη από εκτίμηση προσανατολισμού.
Σε πειράματα ο PSM αποδεικνύεται ότι είναι υπολογιστικά ταχύτερος από τον ICP,
τόσο όσον αφορά στον αριθμό επαναλήψεων όσο και στο χρόνο επεξεργασίας, καθώς
και από απόψεως ακρίβειας.

Μια νέα συνάρτηση απόστασης εισάγεται στο \cite{Minguezb}, κατάλληλη για
διανύσματα καταστάσεων στο χώρο $(x,y,\theta)$: η απόσταση μεταξύ δύο πλήρων
διανυσμάτων στάσης (σε αντίθεση με την περίληψη μόνο της θέσης στη μετρική της
απόστασης) είναι ο μικρότερος μετασχηματισμός στερεού σώματος που οδηγεί τη μία
στάση στην άλλη. Αυτό το μέτρο απόστασης χρησιμοποιείται και στα δύο βήματα του
αλγορίθμου που εισάγουν οι συγγραφείς του, ο οποίος ονομάζεται Metric-Based
Iterative Closest Point (MB-ICP): τα σημεία από μια δισδιάστατη μέτρηση
αποστάσεων αρχικά αντιστοιχίζονται με εκείνα μιας μέτρησης αναφοράς (η ακριβώς
προηγούμενη μέτρηση), και στη συνέχεια η σχετική μετατόπιση και περιστροφή
μεταξύ των δύο μετρήσεων υπολογίζεται ταυτόχρονα, χάρει στην ενσωμάτωση του
ολικού μετασχηματισμού στη μετρική απόστασης, με επαναληπτική ελαχιστοποίηση
του σφάλματος ελαχίστου τετραγώνου της νέας μετρικής σε σχέση με τα
αντιστοιχισμένα σημεία. Με αυτή τη διατύπωση οι συγγραφείς είναι σε θέση να
βελτιώσουν τον IDC όσον αφορά στην ευρωστία, την ταχύτητα σύγκλισης, και την
ακρίβεια.

O Censi, με κίνητρο το γεγονός ότι οι αντιστοιχίσεις μεταξύ δύο μετρήσεων
μπορεί να μην υπάρχουν---αφού ο αισθητήρας δειγματοληπτεί αραιά το περιβάλλον
και συνεπώς διαφορετικές μετρήσεις μπορεί να δειγματοληπτούν διαφορετικά μέρη
του---, αντί να χρησιμοποιήσει μια μετρική απόστασης σημείου-προς-σημείο,
χρησιμοποιεί μια μετρική σημείου-προς-γραμμή στο \cite{Censi2008a}. Σε αυτό το
άρθρο εισάγεται ο αλγόριθμος Point-to-Line Iterative Closest Point (PLICP). Το
πλεονέκτημα της χρήσης μιας μετρικής απόστασης από σημείο σε γραμμή είναι ότι
μπορεί να βρεθεί μία κλειστή μορφή για την ελαχιστοποίηση αυτής της μετρικής
απόστασης, αυξάνοντας έτσι την ακρίβεια και την ταχύτητα σύγκλισης της
ευθυγράμμισης. Πράγματι, ο αλγόριθμος που προκύπτει συγκλίνει τετραγωνικά (ενώ
ο ICP συγκλίνει γραμμικά) και σε πεπερασμένο αριθμό βημάτων. Στο ίδιο άρθρο ο
PLICP συγκρίνεται με τους ICP, IDC και MB-ICP, και διαπιστώνεται ότι είναι
ανώτερος ως προς την ακρίβεια, τον αριθμό των επαναλήψεων που απαιτούνται για
τη σύγκλιση (περισσότερες από τρεις φορές λιγότερο), και στο χρόνο εκτέλεσης
(περισσότερο από 40 φορές λιγότερο). Η διαισθητική εξήγηση πίσω από αυτή την
αύξηση της ακρίβειας είναι ότι η μετρική σημείου-προς-γραμμή προσεγγίζει την
πραγματική απόσταση επιφάνειας καλύτερα από τη μετρική σημείο-προς-σημείο.
Ωστόσο, ειδικά σε αντίθεση με την MB-ICP, ο PLICP είναι τόσο επιρρεπής σε
σφάλματα όσο αυξάνει η απόσταση θέσης και η περιστροφή μεταξύ των στάσεων από
όπου ελήφθησαν οι μετρήσεις. Παρ' όλα αυτά ο αλγόριθμος PLICP έχει υιοθετηθεί
ευρέως λόγω της αυξημένης του ακρίβειας ανάμεσα στις παραλλαγές του ICP, και
της διαθεσιμότητας του πηγαίου κώδικα της υλοποίησής του.

Γενικότερα, ο ICP και οι παραλλαγές του παρουσιάζουν διακυμάνσεις στις
επιδόσεις τους \cite{Donoso2017b}, οι οποίες ορίζονται από το επίπεδο θορύβου
που φέρουν οι μετρήσεις εισόδου, την ανάγκη για επιλογή της εκ-των-προτέρων
υπόθεσης για τον μετασχηματισμό εξόδου και της τιμής του, και τη διαμόρφωση των
τιμών των παραμέτρων που διέπουν την απόκρισή τους. Για τους λόγους αυτούς,
καθώς και για λόγους ευρωστίας, η έρευνα πάνω στις μεθόδους ευθυγράμμισης με τη
χρήση αντιστοιχίσεων μετατοπίστηκε από προσεγγίσεις βασισμένες στην εύρεση
αντιστοιχίσεων από-σημείο-προς-σημείο ή σημείο-προς-γραμμή σε
χαρακτηριστικά-προς-χαρακτηριστικά (feature-to-feature). Συνήθως
χρησιμοποιούμενα χαρακτηριστικά για αναγνώριση είναι τα ευθύγραμμα τμήματα
\cite{XuZezhong,Mohamed2017,Wen2018}, οι γωνίες \cite{Wang2018b},
χαρακτηριστικά SIFT \cite{Li2016}, ή, ιδίως τα τελευταία χρόνια, χαρακτηριστικά
που εξάγονται μέσω της χρήσης τεχνικών βαθειάς μάθησης (deep learning)
\cite{Li2017,Li2020}. Παράλληλα, και για λόγους ανεξαρτησίας από τυχαία
χαρακτηριστικά του περιβάλλοντος ή των αισθητήρων, ή προσαρμογής των μεθόδων σε
συγκεκριμένες συνθήκες τους, αναπτύχθηκε έρευνα γύρω από μεθόδους που εξάγουν ή
εκμεταλλεύονται μαθηματικές ιδιότητες από μετρήσεις αποστάσεων, ή που εξετάζουν
το πρόβλημα της ευθυγράμμισης σαρώσεων ως πρόβλημα βελτιστοποίησης.
Παραδείγματα είναι οι τεχνικές βασισμένες στην συσχέτιση (correlation)
\cite{Olson2009a,Olson2015,Konecny2016}, και στη θεωρία πιθανοτήτων. Μεταξύ των
τελευταίων, o NDT έχει κερδίσει δημοτικότητα λόγω της ρητής μοντελοποίησης των
αβεβαιοτήτων μέτρησης και στάσης, και την επεκτασιμότητά του σε τρεις
διαστάσεις
\cite{Magnusson2007,Zhou2017,Wen2018a,Choi2019,Qingshan2019b,Lee2020}.

Συγκεκριμένα, ο Olson \cite{Olson2009a} υποστηρίζει ότι τα σύγχρονα
υπολογιστικά μηχανήματα είναι αρκετά ικανά επεξεργαστικά ώστε οι μέθοδοι
ευθυγράμμισης μετρήσεων να προχωρήσουν από τις ευρηστικές μεθόδους σε
αναλυτικές μεθόδους μεγαλύτερης ακριβείας. Πράγματι, ο κύριος όγκος των μεθόδων
ευθυγράμμισης δισδιάστατων μετρήσεων χρησιμοποιεί ευρηστικές μεθόδους οι οποίες
είναι ατελείς και επιρρεπείς σε αστάθεια λόγω αδύναμων εκ-των-προτέρων
υποθέσεων για τη στάση του ρομπότ. Αυτές οι ευρηστικές εφαρμόζονται για να
προσδώσουν ταχύτητα εκτέλεσης, αντί να εστιάζουν πρώτα στην ποιότητα
ευθυγράμμισης και στη συνέχεια στην ταχύτητα. Υποστηρίζει ότι η ευθυγράμμιση
σαρώσεων είναι σπανίως κυρτό πρόβλημα βελτιστοποίησης, με την επιφάνεια της
συνάρτησης κόστους να έχει πολλά τοπικά ελάχιστα, καθιστώντας έτσι έναν τοπικό
βελτιστοποιητή ευάλωτο στο να παγιδευτεί σε αυτά. Ταυτόχρονα, καθίσταται
δυσκολότερος ο εντοπισμός του ολικού ελαχίστου της συνάρτησης. Στο
\cite{Olson2009a} διατυπώνεται ένας πιθανοτικός αλγόριθμος ευθυγράμμισης
δισδιάστατων μετρήσεων αποστάσεων ο οποίος παράγει αποτελέσματα υψηλότερης
ποιότητας σε σχέση με αυτούς της βιβλιογραφίας, με κόστος τον πρόσθετο
υπολογιστικό χρόνο εκτέλεσης, αν και η μέθοδος είναι σε θέση να εκτελεστεί σε
πραγματικό χρόνο. Αντί να εμπιστεύεται έναν τοπικό αλγόριθμο αναζήτησης για την
εύρεση του ολικού μεγίστου (ο μετασχηματισμός στερεού σώματος που μεγιστοποιεί
την πιθανότητα να έχει παρατηρηθεί η δεύτερη μέτρηση), ο προτεινόμενος
αλγόριθμος εκτελεί αναζήτηση σε ολόκληρο το χώρο των πιθανών μετασχηματισμών. Η
περιοχή αυτή προκύπτει από μια εκ των προτέρων πιθανότητα, η οποία με τη σειρά
της προκύπτει από οδομετρικές μετρήσεις. Η μέθοδος που παρουσιάζεται με βάση τη
συσχέτιση αποδεικνύεται ότι είναι πολύ ακριβής και εύρωστη στην αβεβαιότητα της
στάσης του αισθητήρα.

Η Πολική Ευθυγράμμιση μετρήσεων με βάση την Περίμετρο (Perimeter-based polar
scan matching---PB-PSM) \cite{Friedman2015} είναι μια τεχνική που βασίζεται
στην \cite{Diosi2005}, η οποία ευνοεί τις ευθυγραμμίσεις με τη μεγαλύτερη
περιμετρική επικάλυψη μεταξύ των δύο εισόδων, ενώ χρησιμοποιεί μια διαδικασία
ελαχιστοποίησης του κόστους, δηλαδή μια προσαρμοστική μέθοδο άμεσης αναζήτησης,
η οποία καθίσταται δυνατή χάρη σε μία τεχνική συσχέτισης δεδομένων γραμμικής
πολυπλοκότητας. Σε αντίθεση με άλλες μεθόδους δεν απαιτείται ως είσοδος μία
αρχική υπόθεση για τη στάση του ρομπότ, αν και, εάν αυτή παρέχεται, έχει ως
αποτέλεσμα την αύξηση της ευρωστίας και τη μείωση των υπολογιστικών απαιτήσεων.
Οι μετρήσεις εισόδου πρώτα φιλτράρονται, στη συνέχεια αναζητούνται συσχετίσεις
σημείων ανάμεσά τους και, τέλος, προσδιορίζεται μια συνάρτηση κόστους η οποία
κατασκευάζεται από τα συσχετιζόμενα ζεύγη σημείων. Το κόστος είναι ανάλογο της
επικάλυψης μεταξύ των μετρήσεων, και η βέλτιστη λύση βρίσκεται με την
ελαχιστοποίηση του, χρησιμοποιώντας μια μορφή εξαντλητικής αναζήτησης. Η
διαδικασία ελαχιστοποίησης εκτελείται πρώτα για την περιστροφή, ακολουθούμενη
από τη μετατόπιση, και η διαδικασία επαναλαμβάνεται έως ότου επιτευχθεί επαρκής
σύγκλιση.

Οι συγγραφείς του \cite{Konecny2016} ανέπτυξαν μια μέθοδο για την εξαγωγή του
μετασχηματισμού μεταξύ δύο μετρήσεων με την κατασκευή μιας συνάρτησης
συσχέτισης που αντικατοπτρίζει το βαθμό ευθυγράμμισής τους σύμφωνα με κάποιο
όρισμα περιστροφο-μετατόπισης. Αυτό εξάγεται φυσικά μεγιστοποιώντας την εν λόγω
συνάρτηση. Η μέθοδός τους έχει χαμηλές υπολογιστικές απαιτήσεις και απευθύνεται
σε ενσωματωμένες συσκευές χαμηλής κατανάλωσης ενέργειας, που συναντώνται συνήθως
σε πλατφόρμες ρομπότ κινητής βάσης.  Σε σύγκριση με τον ICP είναι περίπου δέκα
φορές ταχύτερη και έχει ευρύτερο εύρος λειτουργίας. Ωστόσο, τα διανύσματα
μετασχηματισμού πάσχουν από μια σχετικά αδρή ανάλυση και ελαφρώς υψηλότερα
σχετικά σφάλματα στάσης.

Στα \cite{Yu2018a} και \cite{Jiang2018a} εμφανίζονται δύο από τις ελάχιστες
μεθόδους που δεν χρησιμοποιούν αντιστοιχίσεις: χρησιμοποιούν Phase-Only Matched
Filtering (POMF) \cite{Qin-ShengChen1994a} για τη λύση της εκτίμησης της
περιστροφής και της μετατόπισης: η πρώτη σε μία διάσταση και η δεύτερη σε δύο
διαστάσεις. Στην τελευταία οι απαιτήσεις για λύση σε πραγματικό χρόνο και
επαρκή ακρίβεια δεν μπορούν να ικανοποιηθούν ταυτόχρονα λόγω της αδυναμίας
εξισορρόπησης της υψηλής ανάλυσης πλέγματος (και συνεπώς υψηλής ακρίβειας) με
τακτικές ενημερώσεις των μετρήσεων του αισθητήρα απόστασης. Η πρώτη αμβλύνει
αυτόν τον περιορισμό λειτουργώντας σε μία διάσταση, αλλά η ακρίβεια των λύσεών
της πάσχει από τις ίδιες αιτίες με την πρώτη, δηλαδή τα σφάλματα
διακριτοποίησης. Ενώ η δεύτερη εξαρτάται από την ανάλυση του πλέγματος, η πρώτη
εξαρτάται από την αμετάβλητη διακριτική γωνία του αισθητήρα (angle increment).
Και στις δύο περιπτώσεις επηρεάζεται τόσο η περιστροφική όσο και η μεταφορική
συνιστώσα, αλλά καμία τεχνική αποσόβησης ή μετριασμού δεν χρησιμοποιείται για
τη μείωση των σφαλμάτων των συνιστωσών τους.

Τα τελευταία χρόνια ένας αριθμός νέων μεθόδων ευθυγράμμισης μετρήσεων έχει
παρουσιαστεί, οι οποίες προσφέρουν βελτιώσεις σε καθιερωμένες μεθόδους ή
εισάγουν νέες καινοτομίες. Στα  \cite{Bouraine2020a,Bouraine2021} ο NDT
χρησιμοποιείται για τη μοντελοποίηση του περιβάλλοντος του αισθητήρα
προκειμένου να αντιμετωπιστούν οι αβεβαιότητες και οι περιορισμοί του. Ο
μετασχηματισμός μεταξύ διαδοχικών στάσεων---η λύση του προβλήματος
βελτιστοποίησης της εξίσωσης \ref{eq:sm_def}---δίνεται από μία τροποποιημένη
προσέγγιση βελτιστοποίησης στοχαστικού σμήνους σωματιδίων που ενσωματώνει στη
διατύπωσή της βάρη αδράνειας. Αυτά τα βάρη κωδικοποιούν την ορμή που εκφράζεται
από δυνάμεις που έλκουν το σωματίδιο στη διατήρηση της τρέχουσας ταχύτητάς του,
δυνάμεις που στρέφουν την κίνησή του προς την κατεύθυνση της ατομικά βέλτιστης
στάσης του σωματιδίου, και δυνάμεις που το κατευθύνουν προς την ολικά βέλτιστη
στάση του σμήνους.

Σε αντίθεση όμως με τον NDT, ο οποίος υπολογίζει αντιστοιχίσεις λαμβάνοντας
υπόψη την απόσταση των θέσεων των σημείων από κατανομές κελιών πλέγματος, η
VGICP \cite{Koide2021a} αθροίζει την κατανομή κάθε σημείου στο κελί και
υπολογίζει αντιστοιχίσεις μεταξύ αυτών των κατανομών και των κατανομών της
μέτρησης-στόχου, καθιστώντας έτσι τον VGICP μια προσέγγιση ευθυγράμμισης
αναζήτησης αντιστοιχίσεων κατανομών-με-κατανομές. Αυτή η προσέγγιση αποδίδει
έγκυρες κατανομές πλέγματος ακόμη και όταν υπάρχουν λίγα σημεία σε ένα κελί, με
αποτέλεσμα έναν αλγόριθμο που είναι εύρωστος σε αλλαγές στην ανάλυση του
πλέγματος. Ο VGICP επεκτείνει τον GICP \cite{Segal2009a} προκειμένου να
αποφεύγονται οι δαπανηροί υπολογισμοί αναζήτησης των πλησιέστερων γειτόνων,
μειώνοντας παράλληλα τον χρόνο εκτέλεσής του.

Στο \cite{Yang2021} εισάγεται ένας πιστοποιήσιμος αλγόριθμος ευθυγράμμισης
νεφών σημείων στις τρεις διαστάσεις. Ο μετασχηματισμός του δεύτερου νέφους προς
το πρώτο γίνεται αρχικά αναίσθητος σε μεγάλο αριθμό ή ψευδείς αντιστοιχίες με
την αναδιατύπωση του προβλήματος εύρεσης του μετασχηματισμού ευθυγράμμισης με
τρόπο που χρησιμοποιεί ένα αποκομμένο κόστος ελαχίστων τετραγώνων. Η
περιστροφή, η μετατόπιση, και η κλίμακα μεταξύ των δύο εισόδων απεμπλέκονται
μεταξύ τους με τη χρήση ενός γενικού θεωρητικού πλαισίου γράφων, το οποίο
επιτρέπει το κλάδεμα των ακραίων τιμών με την εύρεση της μέγιστης ``κλίκας" του
γράφου.  Η κλίμακα και η μετατόπιση αποδεικνύεται ότι είναι επιλύσιμες σε
πολυωνυμικό χρόνο μέσω ενός προσαρμοστικού συστήματος ψηφοφορίας, ενώ η
περιστροφή επιλύεται με χαλάρωση σε ένα ημιπεριορισμένο πρόγραμμα
(semi-definite programming).

%%%%%%%%%%%%%%%%%%%%%%%%%%%%%%%%%%%%%%%%%%%%%%%%%%%%%%%%%%%%%%%%%%%%%%%%%%%%%%%%
\subsection{Ευθυγράμμιση δισδιάστατων μετρήσεων αισθητήρα lidar με σαρώσεις
χάρτη}
\label{subsection:02_02_02:2}

Το έργο της ευθυγράμμισης πραγματικών μετρήσεων αισθητήρα lidar με εικονικές
σαρώσεις (ενότητα \ref{subsec:01_01_02_6}) είναι υποπρόβλημα αυτού της
ευθυγράμμισης μετρήσεων αισθητήρα lidar, καθώς το πρώτο κάνει την επιπρόσθετη
παραδοχή ότι ο χάρτης του περιβάλλοντος είναι διαθέσιμος. Ως εκ τούτου το πρώτο
πρόβλημα επιδέχεται λύσης από οποιονδήποτε αλγόριθμο που επιλύει το γενικό
πρόβλημα. Παρ' όλα αυτά, έχει αναπτυχθεί ένας αριθμός μεθόδων που προσπαθεί
στοχευμένα να επιλύσει το υπο-πρόβλημα, μοχλεύοντας επιπρόσθετες πληροφορίες
για την επίλυση του προβλήματος από τη γνώση του χάρτη.

Η ευθυγράμμιση μετρήσεων για τον εντοπισμό ρομπότ βάσει καθολικής αβεβαιότητος
έχει διερευνηθεί στο \cite{XuZezhong}. Υποθέτοντας ότι το περιβάλλον
λειτουργίας ενός ρομπότ είναι δομημένο και ότι ευθύγραμμα τμήματα είναι
διάσπαρτα σε αυτό, ο αλγόριθμος των συγγραφέων, που ονομάστηκε Complete Line
Segments (CLS), αντιστοιχίζει πλήρη ευθύγραμμα τμήματα που εξάγονται από τις
δισδιάστατες μετρήσεις εισόδου σε πλήρη ευθύγραμμα τμήματα που εξάγονται από το
χάρτη που αναπαριστά το περιβάλλον λειτουργίας του ρομπότ, παρέχοντας έτσι έναν
ακριβή τρόπο εξαγωγής της συνολικής στάσης του ρομπότ στο χάρτη.

Στο \cite{Lingemann2005a} παρουσιάζεται ένας αλγόριθμος ευθυγράμμισης
πραγματικών με εικονικές μετρήσεις που εξάγει χαρακτηριστικά από αυτές για την
επίλυση του προβλήματος ευθυγράμμισης.  Ο αλγόριθμος λειτουργεί ανιχνεύοντας
χαρακτηριστικά των πραγματικών και των εικονικών σαρώσεων που είναι αναλλοίωτα
κατά την περιστροφή και τη μετατόπιση, τα οποία είναι υπολογίσιμα μόνο σε
πραγματικό χρόνο (όπως ακραίες τιμές στην πολική αναπαράσταση των μετρήσεων).
Στη συνέχεια δημιουργούνται αντιστοιχίσεις μεταξύ των χαρακτηριστικών που
εξήχθησαν. Η μετατόπιση μεταξύ των δύο σαρώσεων υπολογίζεται ως ο βέλτιστος
μετασχηματισμός για την αντιστοίχιση των χαρακτηριστικών της δεύτερης στα
χαρακτηριστικά της πρώτης.

Στο \cite{Sandberg2009a} παρουσιάζεται ένας στοιχειώδης στοχαστικός αλγόριθμος
αναζήτησης που διορθώνει τα μεταφορικά και περιστροφικά σφάλματα ενός ρομπότ
λόγω της αποκλίνουσας οδομετρίας του. Αυτή η βοηθητική συμπεριφορά εντοπισμού
και διόρθωσης της στάσης του ρομπότ ενεργοποιείται κάθε φορά που μία μετρική
σφάλματος διαπιστώνεται ότι είναι πάνω από ένα προκαθορισμένο όριο. Η μετρική
αυτή βασίζεται στη σχετική απόκλιση των ανιχνευόμενων αποστάσεων μεταξύ των
ακτίνων μίας πραγματικής σάρωσης και μιας σάρωσης χάρτη. Για την αποφυγή της
διόρθωση της κίνησης του ρομπότ κατά τη διάρκεια της ευθυγράμμισης το ρομπότ
θεωρείται ότι παραμένει ακίνητο καθ' όλη τη διάρκεια της. Επομένως, όποτε η
μετρική σφάλματος βρίσκεται πάνω από το προκαθορισμένο όριο, ο αλγόριθμος
σταματά την κίνηση του ρομπότ και επιλέγει μια τυχαία στάση στη γειτονιά της
εκτιμώμενης στάσης του. Στη συνέχεια, πραγματοποιεί μία εικονική σάρωση από
αυτή τη στάση και υπολογίζει την τιμή της παραπάνω μετρικής. Εάν αυτή είναι
χαμηλότερη από αυτήν που βρέθηκε για την προηγούμενη εκτιμώμενη στάση, ξεκινά
μια νέα επανάληψη, αυτή τη φορά με επίκεντρο τη νέα στάση. Εάν όχι, ο
αλγόριθμος συνεχίζει να μαντεύει στάσεις με τυχαίο τρόπο, μέχρι να βρει μία της
οποίας το σφάλμα είναι μικρότερο από το προηγούμενο. Η τελική στάση
λαμβάνεται στη συνέχεια ως η πραγματική στάση του ρομπότ, επιτρέποντας τη
διόρθωση της οδομετρίας. Τα πειράματα που πραγματοποιήθηκαν με αυτή τη μέθοδο
έδειξαν ότι ήταν σε θέση να διορθώσει ένα ακτινικό σφάλμα της τάξεως των
$0.3$ m στα $0.07$ m, και ένα γωνιακό σφάλμα από $0.393$ rad σε $0.01$ rad.

Οι συγγραφείς του \cite{Zhu2011a} χρησιμοποιούν την ευθυγράμμιση πραγματικών με
εικονικές σαρώσεις προκειμένου να βελτιώσουν τη λύση του προβλήματος εντοπισμού
της στάσης ενός ρομπότ βάσει καθολικής αβεβαιότητος. Υποθέτοντας ότι το
περιβάλλον του είναι δομημένο και χωρίς κανενός είδους συμμετρίες, η μέθοδος
προσδιορίζει τον προσανατολισμό του ρομπότ χρησιμοποιώντας τη μέθοδο του HSM.
Έχοντας βρει τον προσανατολισμό του ρομπότ, η μέθοδος εκτιμά τη θέση του ρομπότ
υπολογίζοντας την πιθανότητα ότι κάθε θέση στο πλέγμα του χάρτη παρήγαγε την
μέτρηση που προέρχεται από τον φυσικό αισθητήρα lidar. Αυτή η πιθανότητα
εξάγεται με τη χρήση του μοντέλου τελικού σημείου δέσμης
\cite{thrun2005probabilistic}. Η θέση του ρομπότ είναι η θέση από την οποία
αποτυπώθηκε η εικονική σάρωση που σημειώνει τη μέγιστη πιθανότητα.

Ομοίως, στα πλαίσια της επίλυσης του προβλήματος εκτίμησης της στάσης ενός
ρομπότ βάσει καθολικής αβεβαιότητας, η μέθοδος του \cite{Park2014a} παράγει
πρώτα το γενικευμένο διάγραμμα Voronoi του δεδομένου δισδιάστατου χάρτη
πλέγματος. Οι κόμβοι του θεωρούνται ως αρχικές υποθέσεις για τη θέση του
ρομπότ. Από αυτούς τους κόμβους υπολογίζονται εικονικές σαρώσεις σε γωνιακό
εύρος $2\pi$ με τη χρήση δεσμοβολής (raycasting) στο χάρτη. Στη συνέχεια
υπολογίζονται αντιστοιχίσεις μεταξύ κάθε εικονικής σάρωσης και της σάρωσης που
καταγράφηκε από τον φυσικό αισθητήρα με τη χρήση μίας φασματικής τεχνικής
\cite{Leordeanu2005a}. Η τελευταία βρίσκει γεωμετρικές σχέσεις μεταξύ των δύο
σαρώσεων εισόδου κατά ζεύγη σημείων. Αυτές οι αντιστοιχίσεις χρησιμοποιούνται
στη συνέχεια για τη δημιουργία γεωμετρικών δισδιάστατων ιστογραμμάτων που
κωδικοποιούν μια αίσθηση ομοιότητας μεταξύ της πραγματικής σάρωσης και όλων των
εικονικών σαρώσεων. Στη συνέχεια οι κόμβοι από τους οποίους αποτυπώθηκαν οι
τελευταίες κατατάσσονται σύμφωνα με αυτό το μέτρο ομοιότητας, και ένα κατώφλι
που βασίζεται στο συντελεστή συσχέτισης όλων των συνδυασμών των σαρώσεων
χρησιμοποιείται για την εξαγωγή ενός υποσυνόλου υποψήφιων στάσεων. Αυτή η
διαδικασία χρησιμοποιείται για να φιλτράρει γρήγορα όλες τις υποψήφιες στάσεις.
Η τελική εκτίμηση στάσης είναι εκείνη που επιτυγχάνει τον μέγιστο αριθμό
αντιστοιχούντων ζευγών.

Στο \cite{Zhang2017a} ο χάρτης πλέγματος πληρότητας μετατρέπεται πρώτα σε μία
μορφή χάρτη προσημασμένης καταλληλότητας (fitness map). Αυτός κωδικοποιεί σε
κάθε κελί του την απόσταση του πλησιέστερου εμποδίου για μία δεδομένη εκτίμηση
της θέσης του αισθητήρα. Μέσω του χάρτη καταλληλότητας οι μετρήσεις του
αισθητήρα συσχετίζονται με το χάρτη του περιβάλλοντος χωρίς να εξάγονται
χαρακτηριστικά από κανένα από τα δύο δεδομένα. Το πρόβλημα της εύρεσης της
στάσης του αισθητήρα διατυπώνεται στη συνέχεια ως ένα πρόβλημα βελτιστοποίησης.
Εδώ χρησιμοποιείται βελτιστοποίηση σμήνους σωματιδίων για την εξερεύνηση του
χώρου στάσεων για την αναζήτηση της πιο πιθανής λύσης. Αυτό γίνεται με τη
μεγιστοποίηση της συνάρτησης καταλληλότητας. Για την περαιτέρω βελτίωση της
ακρίβειας αναζήτησης, γίνεται μία σειρά ευθυγραμμίσεων της πραγματικής μέτρησης
με εικονικές μετρήσεις από τις εκτιμήσεις της στάσης του αισθητήρα μέσω του
ICP: από τις στάσεις των σωματιδίων που κατέχουν την κορυφαίες καταλληλότητες
συλλαμβάνονται εικονικές σαρώσεις και αντιστοιχίζονται με την πιο πρόσφατη
σάρωση-μέτρηση. Η εκτίμηση της στάσης που εξάγεται από τον αλγόριθμο είναι
αυτή της οποίας η ενημερωμένη τιμή καταλληλότητας είναι η μέγιστη μεταξύ όλων
των υπο-επεξεργασία σωματιδίων.

Αντίθετα, για την επίλυση του προβλήματος \ref{prob:02_02:the_problem}, στα
πλαίσια της εκτίμησης της στάσης αυτόνομων περονοφόρων ανυψωτικών μηχανημάτων,
η μέθοδος που παρουσιάζεται στο \cite{Vasiljevic2016a} επιλύει την ευθυγράμμιση
πραγματικών με εικονικές σαρώσεις σε δύο βήματα: Δεδομένης της εκτίμησης της
στάσης ενός οχήματος, που γίνεται μέσω της χρήσης φίλτρου σωματιδίων με
δειγματοληψία KLD (ενότητα \ref{subsec:01_01_02_3}), επιχειρείται η μείωση του
σφάλματος της εκτίμησης του προσανατολισμού της στάσης μέσω ευθυγράμμισης της
πραγματικής και της εικονικής σάρωσης χρησιμοποιώντας μία παραλλαγή του ICP.
Συγκεκριμένα, η ευθυγράμμιση πραγματοποιείται μέσω της ολοκληρωμένης, υψηλής
ακρίβειας, αποδοτική, και με τις καλύτερες επιδόσεις μέθοδο του PLICP.  Τα
ευρήματα των συγγραφέων δείχνουν ότι η βελτίωση της εκτίμησης της θέσης μέσω
του ίδιου μηχανισμού και με τη χρήση του PLICP είναι ασταθής. Επομένως
καταλήγουν στο συμπέρασμα ότι η χρήση του PLICP προκειμένου να εξαχθεί η
σχετική μετατόπιση μεταξύ των δύο σαρώσεων σε βιομηχανικές συνθήκες, όπου
απαιτείται ακρίβεια χιλιοστών του μέτρου,  είναι επισφαλής και ακατάλληλη. Για
αυτόν τον λόγο, δεδομένου ότι το σφάλμα εκτίμησης του προσανατολισμού ενός
οχήματος έχει μειωθεί σε μόλις $0.13^\circ$ ($0.0011$ rad), το σφάλμα εκτίμησης
θέσης διορθώνεται με επαναληπτική εκτέλεση ευθυγράμμισης μετρήσεων με εικονικές
σαρώσεις μέσω μίας διαδικασίας που εκτιμά το σφάλμα μετατόπισης μεταξύ των δύο
με έναν επαναληπτικό τρόπο και μέσω μίας συνάρτησης του πρώτου όρου του
Διακριτού Μετασχηματισμού Fourier της διαφοράς των δύο σημάτων-σαρώσεων.

Ένας παρόμοιος τρόπος ευθυγράμμισης παρουσιάζεται στο \cite{Peng2018a}. Αντί
της χρήσης του PLICP οι συγγραφείς αναπτύσσουν έναν αλγόριθμο ευθυγράμμισης με
τη χρήση του αλγορίθμου Gauss-Newton.  Αυτή η ευθυγράμμιση πραγματοποιείται με
κλιμακωτά αυξανόμενη ανάλυση του χάρτη από τον οποίον εξάγονται οι εικονικές
μετρήσεις.  Τα πειράματα που πραγματοποιήθηκαν με ένα πραγματικό ρομπότ σε μη
δομημένα περιβάλλοντα δείχνουν ότι η μέθοδός τους επιτυγχάνει κατά μέσο όρο
ακρίβεια θέσης $0.017$ m και μέση ακρίβεια προσανατολισμού $0.5^\circ$
($0.0087$ rad). Στα \cite{Chen2019a} και \cite{Liu2019a} ο PLICP
χρησιμοποιείται επιπλέον ως μέσο οδομετρίας κάθε φορά που μία μετρική σφάλματος
οδομετρίας βρίσκεται να είναι μεγαλύτερη από ένα κατώτατο όριο. Ωστόσο, σε
αντίθεση με τη μέθοδο του \cite{Peng2018a}, η ευθυγράμμιση πραγματικών με
εικονικές μετρήσεις πραγματοποιείται με την αλυσιδωτή σύνδεση του PLICP με τον
GPM \cite{Censib} προκειμένου να μετριαστούν οι επιπτώσεις των μεγάλων γωνιακών
σφαλμάτων του PLICP.

Η μέθοδος που παρουσιάζεται στo \cite{Bresson2019a} εξετάζει από κοινού την
οδομετρία, την ευθυγράμμιση δισδιάστατων μετρήσεων αισθητήρα lidar, και την
ευθυγράμμιση πραγματικών με εικονικές σαρώσεις χάρτη, σε κτηματολογικούς
χάρτες, και συγκεκριμένα για τον εντοπισμό αυτόνομων οχημάτων σε εξωτερικούς
χώρους. Αυτές χρησιμοποιούνται ως περιορισμοί στην επίλυση ενός προβλήματος
βελτιστοποίησης γράφου που υπολογίζει την πιο πιθανή στάση του οχήματος
δεδομένων των μετρήσεων αισθητήρα απόστασης. Όσον αφορά στα κτηματολογικά
σχέδια, τα μη κτιριακά αντικείμενα φιλτράρονται από την πραγματική παρατήρηση
του αισθητήρα με τη χρήση μιας προσέγγισης διαχωρισμού και συγχώνευσης (split
and merge), η οποία συνδυάζεται με σταθμισμένη άρμωση γραμμών (line fitting). Η
μέτρηση εισόδου και αυτή που προκύπτει από την εκτιμώμενη στάση της στο χάρτη
ευθυγραμμίζονται στη συνέχεια μέσω του GICP, και ο προκύπτον μετασχηματισμός
στάσης προστίθεται στο γράφο εάν και μόνο εάν ο αλγόριθμος που τον παρήγαγε
έχει συγκλίνει. Παράλληλα, στο ίδιο άρθρο εισάγεται μια μέθοδος για τη λύση της
ασάφειας όσον αφορά στη διαμήκη θέση του οχήματος που προκύπτει σε περιβάλλοντα
που προσομοιάζουν σε διαδρόμους.

Στο \cite{Wang2021a} η προτεινόμενη μέθοδος εντοπισμού της στάσης ενός
οχήματος βάσει πεπερασμένης αβεβαιότητας χωρίζεται σε δύο φάσεις: μια offline
και μια online. Κατά τη διάρκεια της πρώτης ο χάρτης χωρίζεται σε ένα
δισδιάστατο πλέγμα καθορισμένης ανάλυσης. Στη συνέχεια δημιουργείται μια
υπογραφή θέσης, η οποία είναι αναλλοίωτη σε περιστροφές, για κάθε εικονική
πανοραμική σάρωση που λαμβάνεται από κάθε διασχίσιμο κελί του χάρτη. Όλες οι
προκύπτουσες υπογραφές εισάγονται στη συνέχεια σε μια αναζήτηση δένδρου ANN.
Στη δεύτερη φάση, για κάθε εισερχόμενη πραγματική μέτρηση παράγεται η υπογραφή
της με τον ίδιο τρόπο όπως και κατά την πρώτη φάση. Στη συνέχεια η υπογραφή
χρησιμοποιείται για την ανάκτηση των γειτονικών υποψήφιων θέσεων από το δένδρο
αναζήτησης: η θέση της εκτίμησης στάσης είναι εκείνη η θέση της οποίας η
υπογραφή της εικονικής μέτρησης είναι ο πλησιέστερος γείτονας της υπογραφής της
πραγματικής μέτρησης.  Για να ληφθεί ο προσανατολισμός της στάσης του ρομπότ
δημιουργείται μια εικονική σάρωση από την εκτιμώμενη θέση και η οποία
ευθυγραμμίζεται με την πανοραμική μέτρηση μετά από βήματα προεπεξεργασίας και
προ-ευθυγράμμισης. Η γωνιακή ευθυγράμμιση πραγματοποιείται σε βήματα μίας
μοίρας και ο προσανατολισμός του ρομπότ είναι αυτός που καταγράφει την ελάχιστη
σχετική εντροπία μεταξύ της εικονικής και της πραγματικής μέτρησης.
